\documentclass[10pt,a4paper]{article}
\usepackage[UTF8,fontset = windows]{ctex}
\setCJKmainfont[BoldFont=黑体,ItalicFont=楷体]{华文中宋}
\usepackage{amssymb,amsmath,amsfonts,amsthm,mathrsfs,dsfont,graphicx}
\usepackage{ifthen,indentfirst,enumerate,color,titletoc}
\usepackage{tikz}
\usepackage{multicol}
\usepackage{makecell}
\usepackage{longtable}
\usepackage{diagbox}
\usetikzlibrary{arrows,calc,intersections,patterns,decorations.pathreplacing,3d,angles,quotes,positioning}
\usepackage[bf,small,indentafter,pagestyles]{titlesec}
\usepackage[top=1in, bottom=1in,left=0.8in,right=0.8in]{geometry}
\renewcommand{\baselinestretch}{1.65}
\newtheorem{defi}{定义~}
\newtheorem{eg}{例~}
\newtheorem{ex}{~}
\newtheorem{rem}{注~}
\newtheorem{thm}{定理~}
\newtheorem{coro}{推论~}
\newtheorem{axiom}{公理~}
\newtheorem{prop}{性质~}
\newcommand{\blank}[1]{\underline{\hbox to #1pt{}}}
\newcommand{\bracket}[1]{(\hbox to #1pt{})}
\newcommand{\onech}[4]{\par\begin{tabular}{p{.9\textwidth}}
A.~#1\\
B.~#2\\
C.~#3\\
D.~#4
\end{tabular}}
\newcommand{\twoch}[4]{\par\begin{tabular}{p{.46\textwidth}p{.46\textwidth}}
A.~#1& B.~#2\\
C.~#3& D.~#4
\end{tabular}}
\newcommand{\vartwoch}[4]{\par\begin{tabular}{p{.46\textwidth}p{.46\textwidth}}
(1)~#1& (2)~#2\\
(3)~#3& (4)~#4
\end{tabular}}
\newcommand{\fourch}[4]{\par\begin{tabular}{p{.23\textwidth}p{.23\textwidth}p{.23\textwidth}p{.23\textwidth}}
A.~#1 &B.~#2& C.~#3& D.~#4
\end{tabular}}
\newcommand{\varfourch}[4]{\par\begin{tabular}{p{.23\textwidth}p{.23\textwidth}p{.23\textwidth}p{.23\textwidth}}
(1)~#1 &(2)~#2& (3)~#3& (4)~#4
\end{tabular}}
\begin{document}

0.8822 r

010923	已知集合$A=\{1,2,3,4\}$, $B=\{2,4,6\}$, 则$A\cup B=$\blank{50}.

003673	已知集合$A=\{1,2,3,4\}$, $B=\{3,4,5\}$, 则$A\cap B=$\blank{50}.

0.8518 r

010923	已知集合$A=\{1,2,3,4\}$, $B=\{2,4,6\}$, 则$A\cup B=$\blank{50}.

004552	已知集合$A=\{1,2,3,4,5\}$, $B=\{3,5,6\}$, 则$A\cap B=$\blank{50}.

0.9515 r

010924	不等式$\dfrac{x-2}{x+1}<0$的解集为\blank{50}.

000378	不等式$\dfrac{x+1}{x+2}<0$的解集为\blank{50}.

0.9390 r

010924	不等式$\dfrac{x-2}{x+1}<0$的解集为\blank{50}.

000468	不等式$\dfrac x{x+1}<0$的解是\blank{50}.

0.9510 r

010924	不等式$\dfrac{x-2}{x+1}<0$的解集为\blank{50}.

000507	不等式$\dfrac{x-1}x<0$的解为\blank{50}.

0.8929 r

010924	不等式$\dfrac{x-2}{x+1}<0$的解集为\blank{50}.

000586	不等式$\dfrac x{x+1}\le 0$的解集为\blank{50}.

0.9634 r

010924	不等式$\dfrac{x-2}{x+1}<0$的解集为\blank{50}.

000797	不等式$\dfrac x{x-1}<0$的解集为\blank{50}.

0.8699 r

010925	函数$y=\lg (x-1)+\dfrac 1{\sqrt {2-x}}$的定义域是\blank{50}.

000607	函数$y=\log_2(1-\dfrac1x)$的定义域为\blank{50}.

0.8670 r

010925	函数$y=\lg (x-1)+\dfrac 1{\sqrt {2-x}}$的定义域是\blank{50}.

004228	函数$f(x)=\sqrt{1-\dfrac 2x}$的定义域是\blank{50}.

0.8586 n

010925	函数$y=\lg (x-1)+\dfrac 1{\sqrt {2-x}}$的定义域是\blank{50}.

005311	函数$y=\sqrt {1-(\dfrac{x-1}{x+1})^2}$的定义域为\blank{50}.

1.0000 s

010926	函数$y=\sin( \omega x-\dfrac{\pi}{3})$($\omega >0$)的最小正周期是$\pi$, 则$\omega =$\blank{50}.

000347	函数$y=\sin (\omega x-\dfrac{\pi}{3})$($\omega >0$)的最小正周期是$\pi$, 则$\omega =$\blank{50}.

0.9770 r

010926	函数$y=\sin( \omega x-\dfrac{\pi}{3})$($\omega >0$)的最小正周期是$\pi$, 则$\omega =$\blank{50}.

000471	若函数$y=2\sin (\omega x-\dfrac\pi 3)+1 \ (\omega >0)$的最小正周期是$\pi$, 则$\omega=$\blank{50}.

1.0000 s

010927	若函数$f(x)=\log_2(x+1)+a$的反函数的图像经过点$(4, 1)$, 则实数$a=$\blank{50}.

000349	若函数$f(x)=\log_2 (x+1)+a$的反函数的图像经过点$(4,1)$, 则实数$a=$\blank{50}.

1.0000 s

010929	甲、乙两人从$5$门不同的选修课中各选修$2$门, 则甲、乙所选的课程中恰有$1$门相同的选法有\blank{50}种.

000351	甲、乙两人从$5$门不同的选修课中各选修$2$门, 则甲、乙所选的课程中恰有$1$门相同的选法有\blank{50}种.

0.8838 r

010945	函数$y=x^2$($x\ge 0$)的反函数为\blank{50}.

002931	函数$y=x^2$($x\le 0$)的反函数是\blank{50}.

0.8843 r

010945	函数$y=x^2$($x\ge 0$)的反函数为\blank{50}.

003896	函数$y=x^2+4x \ (x<-3)$的反函数为\blank{50}.

0.8770 r

010945	函数$y=x^2$($x\ge 0$)的反函数为\blank{50}.

004555	函数$f(x)=x^2$($x<0$)的反函数为\blank{50}.

0.8611 r

010945	函数$y=x^2$($x\ge 0$)的反函数为\blank{50}.

004704	函数$y=\log_2(x+1)$的反函数为\blank{50}.

0.8604 r

010945	函数$y=x^2$($x\ge 0$)的反函数为\blank{50}.

008079	函数$y=\log _2x(x\ge 1)$的反函数是\blank{50}.

1.0000 s

010965	已知全集$U=\mathbf{R}$, 集合$A=\{x||x-1|>1\}$, $B=\{x|\dfrac{x-3}{x+1}<0\}$, 则$\complement _UA\cap B=$\blank{50}.

000496	已知全集$U=\mathbf{R}$, 集合$A=\{x||x-1|>1\}$, $B=\{x|\dfrac{x-3}{x+1}<0\}$, 则$(\complement_U A)\cap B=$\blank{50}.

1.0000 s

010966	已知幂函数的图像过点$(2,\dfrac 14)$, 则该幂函数的单调递增区间是\blank{50}.

000498	已知幂函数的图像过点$(2,\dfrac14)$, 则该幂函数的单调递增区间是\blank{50}.

1.0000 s

010967	若$S_n$是等差数列$\{a_n\}$($n\in \mathbf{N}^*$): $-1,2,5,8,\cdots$的前$n$项和, 则$\displaystyle\lim_{n\to\infty}\dfrac{S_n}{n^2+1}=$\blank{50}.

000499	若$S_n$是等差数列$\{a_n\}\ (n\in \mathbf{N}^*)$: $-1,2,5,8,\cdots$的前$n$项和, 则$\displaystyle\lim_{n\to\infty}\dfrac{{S_n}}{{n^2}+1}=$\blank{50}.

1.0000 s

010968	某圆锥体的底面圆的半径长为$\sqrt 2$, 其侧面展开图是圆心角为$\dfrac 23\pi$的扇形, 则该圆锥体的体积是\blank{50}.

000500	某圆锥体的底面圆的半径长为$\sqrt2$, 其侧面展开图是圆心角为$\dfrac23\pi$的扇形, 则该圆锥体的体积是\blank{50}.

1.0000 s

010969	过点$P(-2,1)$作圆$x^2+y^2=5$的切线, 则该切线的点法向式方程是\blank{50}.

000501	过点$P(-2,1)$作圆$x^2+y^2=5$的切线, 则该切线的点法向式方程是\blank{50}.

1.0000 s

010970	函数$f(x)=\sqrt 3\sin x\cos x+\cos ^2x$的最大值为\blank{50}.

000559	函数$f(x)=\sqrt3\sin x\cos x+\cos^2x$的最大值为\blank{50}.

0.9457 s

010972	某高级中学欲从本校的$7$位古诗词爱好者(其中男生$2$人、女生$5$人)中随机选取3名同学作为学校诗词朗读比赛的主持人, 若要求主持人中至少有一位是男同学, 则不同选取方法的种数是\blank{50}(结果用数值表示).

000503	某高级中学欲从本校的$7$位古诗词爱好者(其中男生$2$人、女生$5$人)中随机选取$3$名同学作为学校诗词朗读比赛的主持人. 若要求主持人中至少有一位是男同学, 则不同选取方法的种数是\blank{50}(结果用数值表示).

0.8873 s

010974	已知函数$f(x)=\begin{cases}  \log_2(x+a), & -a<x\le 0,  \\ x^2-3ax+a, & x>0  \end{cases}$有三个不同的零点, 则实数$a$的取值范围是\blank{50}.

000565	已知函数$f(x)=\begin{cases} \log_2 (x+a), & x\le 0, \\ x^2-3ax+a, & x>0 \end{cases}$有三个不同的零点, 则实数$a$的取值范围是\blank{50}.

0.9367 r

010987	已知复数$z$满足$z\mathrm{i}=2+\mathrm{i}$($\mathrm{i}$为虚数单位), 则$z=$\blank{50}.

000469	若复数$z$满足$\mathrm{i}z=1+\mathrm{i}$($\mathrm{i}$为虚数单位), 则$z=$\blank{50}.

0.8564 n

010987	已知复数$z$满足$z\mathrm{i}=2+\mathrm{i}$($\mathrm{i}$为虚数单位), 则$z=$\blank{50}.

000687	已知复数$z$满足$(2-3\mathrm{i})z=3+2\mathrm{i}$($i$为虚数单位), 则$|z|=$\blank{50}.

0.9336 n

010987	已知复数$z$满足$z\mathrm{i}=2+\mathrm{i}$($\mathrm{i}$为虚数单位), 则$z=$\blank{50}.

000777	若复数$z$满足$z(1-\mathrm{i})=2 \mathrm{i}$($\mathrm{i}$是虚数单位), 则$|z|=$\blank{50}.

0.9325 n

010987	已知复数$z$满足$z\mathrm{i}=2+\mathrm{i}$($\mathrm{i}$为虚数单位), 则$z=$\blank{50}.

003656	已知复数$z$满足$(1+\mathrm{i})z=1-7\mathrm{i}$($\mathrm{i}$是虚数单位), 则$|z|=$\blank{50}.

0.8501 r

010987	已知复数$z$满足$z\mathrm{i}=2+\mathrm{i}$($\mathrm{i}$为虚数单位), 则$z=$\blank{50}.

004185	己知复数$z$满足$z(1+\mathrm{i}^{2020})=2-4\mathrm{i}$(其中, $\mathrm{i}$为虚数单位), 则$z=$\blank{50}.

0.8667 n

010987	已知复数$z$满足$z\mathrm{i}=2+\mathrm{i}$($\mathrm{i}$为虚数单位), 则$z=$\blank{50}.

004512	复数$z$满足$z\cdot \mathrm{i}=1+\mathrm{i}$($\mathrm{i}$为虚数单位), 则$|z|=$\blank{50}.

0.8504 r

010990	在$(1-2x)^6$的二项展开式中, $x^3$项的系数为\blank{50}. (用数字作答)

000410	$(1+2x)^6$展开式中$x^3$项的系数为\blank{50}(用数字作答).

0.9137 r

010990	在$(1-2x)^6$的二项展开式中, $x^3$项的系数为\blank{50}. (用数字作答)

004686	在$(1+2x)^6$的二项展开式中, $x^2$项的系数为\blank{50}.

0.8737 r

010990	在$(1-2x)^6$的二项展开式中, $x^3$项的系数为\blank{50}. (用数字作答)

004727	$(1+2x)^{10}$ 的二项展开式中, $x^2$ 项的系数为\blank{50}.

0.8505 r

010990	在$(1-2x)^6$的二项展开式中, $x^3$项的系数为\blank{50}. (用数字作答)

004747	在$(1+2x)^6$的二项展开式中, $x^5$项的系数为\blank{50}.

0.8678 r

011007	若集合$A=\{x|1\le x\}$, $B=\{-1,1,2,3\}$, 则$A\cap B=$\blank{50}.

003590	已知$A=\{x|2x\le 1\}$, $B=\{-1,0,1\}$, 则$A\cap B=$\blank{50}.

0.8554 n

011008	已知复数$z$满足$z\cdot (1-\mathrm{i})=1+3\mathrm{i}$($\mathrm{i}$为虚数单位), 则$|z|=$\blank{50}.

000469	若复数$z$满足$\mathrm{i}z=1+\mathrm{i}$($\mathrm{i}$为虚数单位), 则$z=$\blank{50}.

0.8535 r

011008	已知复数$z$满足$z\cdot (1-\mathrm{i})=1+3\mathrm{i}$($\mathrm{i}$为虚数单位), 则$|z|=$\blank{50}.

000777	若复数$z$满足$z(1-\mathrm{i})=2 \mathrm{i}$($\mathrm{i}$是虚数单位), 则$|z|=$\blank{50}.

0.9208 r

011008	已知复数$z$满足$z\cdot (1-\mathrm{i})=1+3\mathrm{i}$($\mathrm{i}$为虚数单位), 则$|z|=$\blank{50}.

000847	已知复数$z$满足$z\cdot (1-\mathrm{i})=2\mathrm{i}$, 其中$\mathrm{i}$为虚数单位, 则$|z|=$\blank{50}.

0.8795 r

011008	已知复数$z$满足$z\cdot (1-\mathrm{i})=1+3\mathrm{i}$($\mathrm{i}$为虚数单位), 则$|z|=$\blank{50}.

003656	已知复数$z$满足$(1+\mathrm{i})z=1-7\mathrm{i}$($\mathrm{i}$是虚数单位), 则$|z|=$\blank{50}.

0.9430 r

011008	已知复数$z$满足$z\cdot (1-\mathrm{i})=1+3\mathrm{i}$($\mathrm{i}$为虚数单位), 则$|z|=$\blank{50}.

004512	复数$z$满足$z\cdot \mathrm{i}=1+\mathrm{i}$($\mathrm{i}$为虚数单位), 则$|z|=$\blank{50}.

0.8859 n

011009	若$\sin \alpha =\dfrac 13$, 则$\sin (\dfrac \pi 2-2\alpha)=$\blank{50}.

000818	若$\sin\alpha =\dfrac13$, 则$\cos(\alpha -\dfrac{\pi}2)=$\blank{50}.

0.8681 n

011009	若$\sin \alpha =\dfrac 13$, 则$\sin (\dfrac \pi 2-2\alpha)=$\blank{50}.

004122	若$\sin\alpha=\dfrac 14$, 则$\sin(\pi+\alpha)=$\blank{50}.

0.8626 r

011012	若关于$x$、$y$的方程组$\begin{cases} 2x+3y=1 \\ ax-y=2 \end{cases}$无解, 则实数$a=$\blank{50}.

003558	关于$x$、$y$的方程组$\begin{cases} mx+2y=m+4, \\ 2x+my=m \end{cases}$无解, 则实数$m=$\blank{50}.

0.8536 r

011012	若关于$x$、$y$的方程组$\begin{cases} 2x+3y=1 \\ ax-y=2 \end{cases}$无解, 则实数$a=$\blank{50}.

003564	若关于$x$、$y$的方程组$\begin{cases} mx+y=-1,\\ x+my=1 \end{cases}$有解, 则实数$m$的取值范围为\blank{50}.

1.0000 s

011014	已知$A,B$分别是函数$f(x)=2\sin \omega x$($\omega >0$)在$y$轴右侧图像上的第一个最高点和第一个最低点, 且$\angle AOB=\dfrac\pi 2$, 则该函数的最小正周期是\blank{50}.

000374	已知$A,B$分别是函数$f(x)=2\sin \omega x$($\omega >0$)在$y$轴右侧图像上的第一个最高点和第一个最低点, 且$\angle AOB=\dfrac\pi 2$, 则该函数的最小正周期是\blank{50}.

0.9132 r

011028	已知集合$A=\{1,3,5,6,7\}$, $B=\{2,4,5,6,8\}$, 则$A\cap B=$\blank{50}.

004552	已知集合$A=\{1,2,3,4,5\}$, $B=\{3,5,6\}$, 则$A\cap B=$\blank{50}.

0.9012 r

011029	不等式$|3x-2|<1$的解集是\blank{50}.

000757	不等式$|1-x|>1$的解集是\blank{50}.

0.9082 r

011029	不等式$|3x-2|<1$的解集是\blank{50}.

000816	不等式$|x-3|<2$的解集为\blank{50}.

0.9047 r

011029	不等式$|3x-2|<1$的解集是\blank{50}.

002793	不等式$2<|x+1|<3$的解集是\blank{50}.

0.9130 r

011029	不等式$|3x-2|<1$的解集是\blank{50}.

002794	不等式$|x-2|>9x$的解集是\blank{50}.

0.9012 r

011029	不等式$|3x-2|<1$的解集是\blank{50}.

004312	不等式$|1-x|>1$的解集是\blank{50}.

0.9686 r

011049	已知集合$A=(-\infty ,-3)$, $B=(-4,+\infty)$, 则$A\cap B=$\blank{50}.

003631	已知集合$A=(-\infty,3)$, $B=(2,+\infty)$, 则$A\cap B=$\blank{50}.

0.9285 r

011049	已知集合$A=(-\infty ,-3)$, $B=(-4,+\infty)$, 则$A\cap B=$\blank{50}.

004724	若集合$A=(-\infty ,1)$, $B=(0,+\infty)$, 则$A\cap B=$\blank{50}.

0.8580 r

011052	函数$f(x)=\log_2(2x+4)$的反函数为$f^{-1}(x)$, 则$f^{-1}(4)=$\blank{50}.

004390	已知函数$f(x)$的反函数$f^{-1}(x)=\log_2x$, 则$f(-1)=$\blank{50}.

0.8720 r

011054	已知二项式$(2x+\dfrac 1x)^6$, 则其展开式中的常数项为\blank{50}.

000398	在二项式$(x+\dfrac2x)^6$的展开式中, 常数项是\blank{50}.

0.8884 r

011055	计算: $\displaystyle\lim_{n\to \infty} \dfrac{|4n-23|}{2n}=$\blank{50}.

000376	$\displaystyle\lim_{n\to\infty}\dfrac{2n+3}{n+1}=$\blank{50}.

0.8823 r

011055	计算: $\displaystyle\lim_{n\to \infty} \dfrac{|4n-23|}{2n}=$\blank{50}.

000516	计算: $\displaystyle\lim_{n\to\infty}(1-\dfrac n{n+1})=$\blank{50}.

0.9309 r

011055	计算: $\displaystyle\lim_{n\to \infty} \dfrac{|4n-23|}{2n}=$\blank{50}.

000546	计算: $\displaystyle\lim_{n\to\infty}\dfrac{2n}{3n-1}=$\blank{50}.

0.8736 r

011055	计算: $\displaystyle\lim_{n\to \infty} \dfrac{|4n-23|}{2n}=$\blank{50}.

000606	计算: $\displaystyle\lim_{n\to\infty}(1+\dfrac1n)^3=$\blank{50}.

0.8773 r

011055	计算: $\displaystyle\lim_{n\to \infty} \dfrac{|4n-23|}{2n}=$\blank{50}.

000796	$\displaystyle\lim_{n\to \infty}\dfrac{2n+1}{n-1}=$\blank{50}.

0.9270 r

011055	计算: $\displaystyle\lim_{n\to \infty} \dfrac{|4n-23|}{2n}=$\blank{50}.

000827	计算:$\displaystyle\lim_{n\to\infty}\dfrac{2n}{4n+1}=$\blank{50}.

0.9132 r

011055	计算: $\displaystyle\lim_{n\to \infty} \dfrac{|4n-23|}{2n}=$\blank{50}.

003611	计算: $\displaystyle\lim_{n\to\infty}\dfrac{n+1}{3n-1}=$\blank{50}.

0.8877 r

011055	计算: $\displaystyle\lim_{n\to \infty} \dfrac{|4n-23|}{2n}=$\blank{50}.

004513	$\displaystyle\lim_{n\to \infty}\dfrac{2n}{3{n^2}+1}=$\blank{50}.

0.8637 r

011055	计算: $\displaystyle\lim_{n\to \infty} \dfrac{|4n-23|}{2n}=$\blank{50}.

004553	计算: $\displaystyle\lim_{n\to\infty}\dfrac{2n^2-3n+1}{n^2-4n+1}=$\blank{50}.

0.8688 r

011055	计算: $\displaystyle\lim_{n\to \infty} \dfrac{|4n-23|}{2n}=$\blank{50}.

008484	计算: $\displaystyle\lim_{n\to\infty}(\dfrac 1{n^2}+\dfrac 2n-3)=$\blank{50}.

0.8791 r

011055	计算: $\displaystyle\lim_{n\to \infty} \dfrac{|4n-23|}{2n}=$\blank{50}.

008485	计算: $\displaystyle\lim_{n\to\infty}\dfrac{7n+4}{5-3n}=$\blank{50}.

0.8657 r

011055	计算: $\displaystyle\lim_{n\to \infty} \dfrac{|4n-23|}{2n}=$\blank{50}.

008488	计算: $\displaystyle\lim_{n\to\infty}\dfrac{(n+3)(n-4)}{(n-1)(3-2n)}=$\blank{50}.

0.8692 n

011055	计算: $\displaystyle\lim_{n\to \infty} \dfrac{|4n-23|}{2n}=$\blank{50}.

008671	$\displaystyle\lim_{n\to\infty}\dfrac{1+(-1)^n}n=$\blank{50}.

0.9225 r

011070	函数$f(x)=x^{- \frac 23}$的定义域为\blank{50}.

002907	函数$y=x^{-\frac 32}$的定义域为\blank{50}.

0.8530 n

011070	函数$f(x)=x^{- \frac 23}$的定义域为\blank{50}.

004228	函数$f(x)=\sqrt{1-\dfrac 2x}$的定义域是\blank{50}.

0.8545 n

011070	函数$f(x)=x^{- \frac 23}$的定义域为\blank{50}.

004270	函数$f(x)=\sqrt{\dfrac{1-x}{3+x}}$的定义域为\blank{50}.

0.9753 r

011070	函数$f(x)=x^{- \frac 23}$的定义域为\blank{50}.

004389	函数$f(x)=x^{- \frac 12}$的定义域为\blank{50}.

0.9506 r

011070	函数$f(x)=x^{- \frac 23}$的定义域为\blank{50}.

004661	函数$f(x)={x^{-\frac 12}}$的定义域是\blank{50}.

0.8598 n

011070	函数$f(x)=x^{- \frac 23}$的定义域为\blank{50}.

005310	函数$y=\dfrac 1{|x|-x^2}$的定义域为\blank{50}.

1.0000 s

011073	若$\triangle ABC$中, $a+b=4$, $\angle C=30^\circ$, 则$\triangle  ABC$面积的最大值是\blank{50}.

000329	若$\triangle ABC$中, $a+b=4$, $\angle C=30^\circ$, 则$\triangle ABC$面积的最大值是\blank{50}.

1.0000 s

011074	若函数$f(x)=\log_2\dfrac{x-a}{x+1}$的反函数的图像过点$(-2,3)$, 则$a=$\blank{50}.

000330	若函数$f(x)=\log_2\dfrac{x-a}{x+1}$的反函数的图像过点$(-2,3)$, 则$a=$\blank{50}.

0.9121 r

011091	不等式$\dfrac 1x<1$的解集为\blank{50}.

000459	不等式$\dfrac{x+2}{x+1}>1$的解集为\blank{50}.

0.9171 n

011091	不等式$\dfrac 1x<1$的解集为\blank{50}.

000540	不等式$\dfrac1{|x-1|}\ge 1 $的解集为\blank{50}.

0.8534 r

011091	不等式$\dfrac 1x<1$的解集为\blank{50}.

000797	不等式$\dfrac x{x-1}<0$的解集为\blank{50}.

0.8611 n

011091	不等式$\dfrac 1x<1$的解集为\blank{50}.

002802	不等式$\dfrac{1+|x|}{|x|-1}\ge 3$的解集是\blank{50}.

0.8754 n

011091	不等式$\dfrac 1x<1$的解集为\blank{50}.

002961	不等式$\log_{\frac 12}(x-1)\ge 1$的解集为\blank{50}.

0.9332 r

011091	不等式$\dfrac 1x<1$的解集为\blank{50}.

003675	不等式$\dfrac{x-1}{x}>1$的解集为\blank{50}.

0.9476 r

011091	不等式$\dfrac 1x<1$的解集为\blank{50}.

004249	不等式$\dfrac 1{x-1}>1$的解集为\blank{50}.

0.8653 r

011091	不等式$\dfrac 1x<1$的解集为\blank{50}.

004409	不等式$\dfrac 1x\le 3$的解集是\blank{50}.

0.9476 r

011091	不等式$\dfrac 1x<1$的解集为\blank{50}.

004469	不等式$\dfrac 1{x-1}>1$的解集为\blank{50}.

0.9420 r

011092	抛物线$y^2=2x$的焦点坐标为\blank{50}.

000467	抛物线$y^2=4x$的焦点坐标是\blank{50}.

0.9241 r 

011092	抛物线$y^2=2x$的焦点坐标为\blank{50}.

000728	抛物线$y=x^2$的焦点坐标是\blank{50}.

0.8550 r

011092	抛物线$y^2=2x$的焦点坐标为\blank{50}.

000806	抛物线$x^2=12y$的准线方程为\blank{50}.

0.9420 r

011092	抛物线$y^2=2x$的焦点坐标为\blank{50}.

000878	抛物线$y^2=4x$的焦点坐标是\blank{50}.

0.9588 r

011092	抛物线$y^2=2x$的焦点坐标为\blank{50}.

002405	抛物线$x^2=-32y$的焦点坐标为\blank{50}.

0.8976 r

011092	抛物线$y^2=2x$的焦点坐标为\blank{50}.

003448	抛物线$y=-4x^2$的焦点坐标是\blank{50}.

1.0000 s

011093	三阶行列式$\begin{vmatrix}
3 & -5 & 1  \\ 2 & 3 & -6  \\ -7 & 2 & 4  \end{vmatrix}$中元素$-5$的代数余子式的值为\blank{50}.

000417	三阶行列式$\begin{vmatrix}   3 & -5 & 1 \\   2 & 3 & -6 \\   -7 & 2 & 4 \\ \end{vmatrix}$中元素$-5$的代数余子式的值为\blank{50}.

0.9158 r

011093	三阶行列式$\begin{vmatrix}
3 & -5 & 1  \\ 2 & 3 & -6  \\ -7 & 2 & 4  \end{vmatrix}$中元素$-5$的代数余子式的值为\blank{50}.

000696	行列式$\begin{vmatrix} 1 & 2 & 3 \\ 4 & 5 & 6  \\ 7 & 8 & 9 \end{vmatrix}$中, 元素$5$的代数余子式的值为\blank{50}.

0.9039 r

011094	已知向量$\overrightarrow a=(1,-2)$, $\overrightarrow b=(3,4)$, 则向量$\overrightarrow a$在向量$\overrightarrow b$的方向上的投影为\blank{50}.

000382	已知向量$\overrightarrow{a}=(1,2)$, $\overrightarrow{b}=(0,3)$, 则$\overrightarrow{b}$在$\overrightarrow{a}$的方向上的投影为\blank{50}.

1.0000 s

011096	已知直线$l:x-y+b=0$被圆$C:x^2+y^2=25$所截得的弦长为$6$, 则$b=$\blank{50}.

000421	已知直线$l:x-y+b=0$被圆$C:x^2+y^2=25$所截得的弦长为$6$, 则$b=$\blank{50}.

1.0000 s

011098	函数$f(x)=(\sqrt 3\sin x+\cos x)(\sqrt 3\cos x-\sin x)$的最小正周期为\blank{50}.

000423	函数$f(x)=(\sqrt3\sin x+\cos x)(\sqrt3\cos x-\sin x)$的最小正周期为\blank{50}.

1.0000 s

011099	过双曲线$C:\dfrac{x^2}{a^2}-\dfrac{y^2}4=1$的右焦点$F$作一条垂直于$x$轴的垂线交双曲线$C$的两条渐近线于$A$、$B$两点, $O$为坐标原点, 则$\triangle OAB$的面积的最小值为\blank{50}.

000424	过双曲线$C:\dfrac{x^2}{a^2}-\dfrac{y^2}4=1$的右焦点$F$作一条垂直于$x$轴的垂线交双曲线$C$的两条渐近线于$A$、$B$两点, $O$为坐标原点, 则$\triangle OAB$的面积的最小值为\blank{50}.

0.9938 s

011100	若关于$x$的不等式$|2^x-m|-\dfrac 1{2^x}<0$在区间$[0,1]$内恒成立, 则实数$m$的范围是\blank{50}.

000425	若关于$x$的不等式$|2^x-m|-\dfrac1{2^x}<0$在区间$[0,1]$内恒成立, 则实数$m$的范围\blank{50}.

0.9219 r

011112	已知集合$A=\{1,2,m\}$, $B=\{3,4\}$, 若$A\cap B=\{4\}$, 则实数$m=$\blank{50}

000576	已知集合$A=\{1,2,m\}$, $B=\{3,4\}$.若$A\cap B=\{3\}$, 则实数$m=$\blank{50}.

0.8501 r

011112	已知集合$A=\{1,2,m\}$, $B=\{3,4\}$, 若$A\cap B=\{4\}$, 则实数$m=$\blank{50}

000836	已知集合$A=\{1,2,m\}$,$B=\{2,4\}$, 若$A\cup B=\{1,2,3,4\}$, 则实数$m=$\blank{50}.

0.8812 n

011114	函数$f(x)=\arcsin x+1$的定义域为\blank{50}.

000868	函数$f(x)=\dfrac{\sqrt{x+2}}{x-1}$的定义域为\blank{50}.

0.8818 r

011114	函数$f(x)=\arcsin x+1$的定义域为\blank{50}.

004186	函数$y=\arcsin (x+1)$的定义域是\blank{50}.

0.8545 n

011114	函数$f(x)=\arcsin x+1$的定义域为\blank{50}.

004270	函数$f(x)=\sqrt{\dfrac{1-x}{3+x}}$的定义域为\blank{50}.

0.8530 n

011114	函数$f(x)=\arcsin x+1$的定义域为\blank{50}.

004377	函数$f(x)=\sqrt{\dfrac{1-x}x}$的定义域为\blank{50}.

0.9151 r

011116	函数$f(x)=\sin^2 x-\cos^2 x$的最小正周期为\blank{50}.

000709	函数$f(x)=2\sin 4x \cos 4x$的最小正周期为\blank{50}.

0.9368 r

011116	函数$f(x)=\sin^2 x-\cos^2 x$的最小正周期为\blank{50}.

000945	函数$f(x)=(\sin x-\cos x)^2$的最小正周期为\blank{50}.

0.9155 r

011116	函数$f(x)=\sin^2 x-\cos^2 x$的最小正周期为\blank{50}.

009986	函数$f(x)=\cos^2 x-\sin ^2 x+1$的周期为\blank{50}.

0.8557 r

011117	若掷一颗质地均匀的骰子, 则出现向上的点数大于$4$的概率是\blank{50}.

000829	掷一颗均匀的骰子, 出现奇数点的概率为\blank{50}.

0.9638 r

011119	设常数$a\in \mathbf{R}$, 函数$f(x)=\ln (x+a)$. 若$f(x)$的反函数图像经过点$(3,1)$, 则$a=$\blank{50}.

003655	设常数$a\in \mathbf{R}$, 函数$f(x)=\log_2(x+a)$. 若$f(x)$的反函数的图像经过点$(3,1)$, 则$a=$\blank{50}.

0.8531 n

011120	函数$y=\sqrt{x}-\sqrt{1-x}$的值域为\blank{50}.

001242	函数$y=\sqrt{1+x}+2x$的值域为\blank{80}.

0.9093 n

011120	函数$y=\sqrt{x}-\sqrt{1-x}$的值域为\blank{50}.

001256	函数$y=\sqrt{6-x}+\sqrt{x-3}$的值域为\blank{80}.

0.8713 n

011120	函数$y=\sqrt{x}-\sqrt{1-x}$的值域为\blank{50}.

005304	函数$y=\sqrt {1-x^2}+\sqrt {x+1}$的定义域为\blank{50}.

0.8683 n

011120	函数$y=\sqrt{x}-\sqrt{1-x}$的值域为\blank{50}.

005317	函数$y=4+\sqrt {2x+1}$的值域为\blank{50}.

0.8570 n

011120	函数$y=\sqrt{x}-\sqrt{1-x}$的值域为\blank{50}.

005319	函数$y=\sqrt {-x^2+x+2}$的值域为\blank{50}.

0.8545 n

011125	已知$a,b\in \mathbf{R}$, 则``$ab>0$''是``$\dfrac ab+\dfrac ba>2$''的\bracket{20}.
\twoch{充分非必要条件}{必要非充分条件}{充要条件}{既非充分也非必要条件}

004564	已知$a,b\in\mathbf{R}$, 则``$a^2>b^2$''是``$|a|>|b|$''的\bracket{20}.
\twoch{充分非必要条件}{必要非充分条件}{充要条件}{既非充分又非必要条件}

0.9191 r

011133	函数$f(x)=\log_2(x-1)$的定义域为\blank{50}.

000486	函数$f(x)=\lg(2-x)$的定义域是\blank{50}.

0.8759 n

011133	函数$f(x)=\log_2(x-1)$的定义域为\blank{50}.

000567	函数$f(x)=\sqrt{1-\lg x}$的定义域为\blank{50}.

0.8904 n

011133	函数$f(x)=\log_2(x-1)$的定义域为\blank{50}.

000738	函数$f(x)=\lg (3^x-2^x)$的定义域为\blank{50}.

0.8633 r

011133	函数$f(x)=\log_2(x-1)$的定义域为\blank{50}.

000931	函数$y=\log_3 (x-1)$的定义域是\blank{50}.

0.8660 n

011133	函数$f(x)=\log_2(x-1)$的定义域为\blank{50}.

001324	函数$y=\log_{x^2+x-1} 2$的定义域是\blank{150}.

0.8841 r

011133	函数$f(x)=\log_2(x-1)$的定义域为\blank{50}.

004332	函数$y=\log_2(x-2)$的定义域为\blank{50}.

0.8523 n

011133	函数$f(x)=\log_2(x-1)$的定义域为\blank{50}.

004516	函数$f(x)=1+\log_2x$($x\ge 4$)的反函数的定义域为\blank{50}.

0.8560 n 

011133	函数$f(x)=\log_2(x-1)$的定义域为\blank{50}.

005699	函数$y=\log_{(2x-1)}(32-4^x)$的定义域为\blank{50}.

0.9389 r

011134	已知集合$A=\{1,2,3,4\}$, 集合$B=\{4,5\}$, 则$A\cap B=$\blank{50}.

003610	已知集合$A=\{1,2,4\}$,$B=\{2,4,5\}$, 则$A\cap B=$\blank{50}.

0.9449 r

011134	已知集合$A=\{1,2,3,4\}$, 集合$B=\{4,5\}$, 则$A\cap B=$\blank{50}.

003673	已知集合$A=\{1,2,3,4\}$, $B=\{3,4,5\}$, 则$A\cap B=$\blank{50}.

0.9677 r

011135	函数$y=2\cos ^2x-1$的最小正周期为\blank{50}.

000458	函数$y=2\cos^2(3\pi x)-1$的最小正周期为\blank{50}.

0.9183 r

011135	函数$y=2\cos ^2x-1$的最小正周期为\blank{50}.

000676	函数$y=2\sin^2(2x)-1$的最小正周期是\blank{50}.

0.8520 n

011135	函数$y=2\cos ^2x-1$的最小正周期为\blank{50}.

001534	函数$y=\tan x-\cot x$的最小正周期为\blank{50}.

0.8954 n

011136	已知球的体积为$36\pi$, 则该球大圆的面积等于\blank{50}.

000589	已知球的表面积为$16\pi$, 则该球的体积为\blank{50}.

0.9455 r

011136	已知球的体积为$36\pi$, 则该球大圆的面积等于\blank{50}.

003676	已知球的体积为$36\pi$, 则该球主视图的面积等于\blank{50}.

0.8684 r

011137	二项式$(x-\dfrac 1x)^6$的展开式中的常数项为\blank{50}. (用数字作答)

000398	在二项式$(x+\dfrac2x)^6$的展开式中, 常数项是\blank{50}.

0.9257 r

011137	二项式$(x-\dfrac 1x)^6$的展开式中的常数项为\blank{50}. (用数字作答)

000568	二项式$(x-\dfrac1{2x})^4$的展开式中的常数项为\blank{50}.

0.8825 r

011169	已知$x,y\in \mathbf{R}^+$, $x-y=1+xy$, 求$x-4y$的取值范围.

002760	已知$x,y\in \mathbf{R}^+, \ xy=x+y+1$, 求$x+y$的取值范围(试用两种方法求解).

0.8709 r

011175	不等式$\log_2(1-x)>1+\log_4x$的解集是\blank{50}.

003025	方程$\log_2(x-1)=\log_4(2-x)$的解集是\blank{50}.

0.8588 n

011177	不等式$\dfrac{3|x|-2}{|x|}\le 1$的解集是\blank{50}.

000586	不等式$\dfrac x{x+1}\le 0$的解集为\blank{50}.

0.8928 n

011177	不等式$\dfrac{3|x|-2}{|x|}\le 1$的解集是\blank{50}.

002801	不等式$\dfrac{2x}{1-x}\le 1$的解集是\blank{50}.

0.8766 r

011177	不等式$\dfrac{3|x|-2}{|x|}\le 1$的解集是\blank{50}.

002802	不等式$\dfrac{1+|x|}{|x|-1}\ge 3$的解集是\blank{50}.

0.9137 n

011177	不等式$\dfrac{3|x|-2}{|x|}\le 1$的解集是\blank{50}.

004409	不等式$\dfrac 1x\le 3$的解集是\blank{50}.

0.8899 r

011181	函数$y=f(x)$满足对于任意$x\ne 0$, 恒有$f(x-\dfrac 1x)=x^2+\dfrac 1{x^2}$. 若存在$x_0$使得$f(x_0)-x_0=2$成立, 则$x_0=$\blank{50}.

002826	函数$y=f(x)$满足对于任意$x\ne 0$, 恒有$f(x-\dfrac 1x)=x^3-\dfrac 1{x^3}$. 若存在$x_0$使得$f(x_0)=0$, 则$x_0=$\blank{50}.

0.9428 r

011184	已知$xy<0$, 且$x^2-4y^2=4$. 问: 能否由此条件将$y$表示成$x$的函数? 若能, 求出该函数的解析式、定义域; 若不能, 说明理由.

002840	已知$xy<0$, 且$4x^2-9y^2=36$. 问: 能否由此条件将$y$表示成$x$的函数? 若能, 求出该函数的解析式; 若不能, 说明理由.

0.8737 r

011191	函数$y=\dfrac 1{\sqrt {x^2-5x-6}}$的递增区间是\blank{50}.

002888	函数$y=\dfrac 1{\sqrt{x^2+2x-3}}$的递增区间是\blank{50}.

1.0000 s

011200	设集合$A=\{5, \log_2(a+3)\}$, $B=\{a,b\}$, 若$A\cap B=\{2\}$, 则$A\cup B=$\blank{50}.

002711	设集合$A=\{5,\log_2(a+3)\}$, $B=\{a,b\}$, 若$A\cap B=\{2\}$, 则$A\cup B=$\blank{50}.

0.9926 r

011201	已知函数$f(x)=\log_3(\dfrac 4x+2)$, 则方程$f^{-1}(x)=4$的解为$x=$\blank{50}.

004165	已知函数$f(x)=\log_3(\dfrac 4{x+2})$ , 则方程$f^{-1}(x)=4$的解$x=$\blank{50}.

0.8651 r

011211	已知函数$f(x)$满足: \textcircled{1} 对任意$x\in (0, +\infty)$, 恒有$f(2x)=2f(x)$成立; \textcircled{2} 当$x\in (1,2]$时, $f(x)=2-x$. 若$f(a)=f(2015)$, 则满足条件的最小的正实数$a$是\blank{50}.

004217	已知函数$f(x)$满足: \textcircled{1} 对任意$x\in (0,+\infty)$恒有$f(2x)=2f(x)$成立; \textcircled{2} $x\in (1,2]$时, $f(x)=2-x$; 若$f(a)=f(2020)$, 则满足条件的最小的正实数$a$是\blank{50}.

0.8744 r

011221	方程$4^x-2^x=0$的解集为\blank{50}.

003024	方程$4^{x+1}-13\cdot 2^x+3=0$的解集是\blank{50}.

0.8565 r

011221	方程$4^x-2^x=0$的解集为\blank{50}.

005772	方程$3^{x+1}-3^{-x}=2$的解为\blank{50}.

0.9639 s

011228	设常数$\omega >0$, $t>0$, 函数$f(x)=\begin{vmatrix}
\sqrt 3 & \sin \omega x  \\ 1 & \cos \omega x  \end{vmatrix}$的最小正周期为$2\pi$, 将$f(x)$的图像向左平移$t$个单位, 所得图像对应的函数为偶函数, 则$t$的最小值为\blank{50}.

004474	已知$\omega,t>0$, 函数$f(x)=\begin{vmatrix}
\sqrt 3 & \sin \omega x  \\ 1  & \cos \omega x  \end{vmatrix}$的最小正周期为$2\pi$, 将$f(x)$的图像向左平移$t$个单位, 所得图像对应的函数为偶函数, 则$t$的最小值为\blank{50}.

0.8522 r

010923	已知集合$A=\{1,2,3,4\}$, $B=\{2,4,6\}$, 则$A\cup B=$\blank{50}.

011134	已知集合$A=\{1,2,3,4\}$, 集合$B=\{4,5\}$, 则$A\cap B=$\blank{50}.

0.8606 s

010954	正方形$ABCD$的边长为$4$, $O$是正方形$ABCD$的中心, 过中心$O$的直线$l$与边$AB$交于点$M$, 与边$CD$交于点$N$. $P$为平面上一点, 满足$2
\overrightarrow{OP}=\lambda \overrightarrow{OB}+(1-\lambda)\overrightarrow{OC}$, 则$\overrightarrow{PM}\cdot \overrightarrow{PN}$的最小值为\blank{50}.

011016	正方形$ABCD$的边长为$4$, $O$是正方形$ABCD$的中心, 过中心$O$的直线$l$与边$AB$交于点$M$, 与边$CD$交于点$N$, $P$为平面上一点, 满足: 存在$\lambda\in \mathbf{R}$, 使得$2\overrightarrow{OP}=\lambda \overrightarrow{OB}+(1-\lambda)\overrightarrow{OC}$, 则$\overrightarrow{PM}\cdot \overrightarrow{PN}$的最小值为\blank{50}.

0.8698 n

010987	已知复数$z$满足$z\mathrm{i}=2+\mathrm{i}$($\mathrm{i}$为虚数单位), 则$z=$\blank{50}.

011008	已知复数$z$满足$z\cdot (1-\mathrm{i})=1+3\mathrm{i}$($\mathrm{i}$为虚数单位), 则$|z|=$\blank{50}.

0.8518 n

011008	已知复数$z$满足$z\cdot (1-\mathrm{i})=1+3\mathrm{i}$($\mathrm{i}$为虚数单位), 则$|z|=$\blank{50}.

011051	已知复数$z$满足$\dfrac 1{z-1}=\mathrm{i}$($\mathrm{i}$为虚数单位), 则$z=$\blank{50}.



\end{document}