\documentclass[10pt,a4paper]{article}

\usepackage[UTF8,fontset = windows]{ctex}

\setCJKmainfont[BoldFont=黑体,ItalicFont=楷体]{等线}

\usepackage{amssymb,amsmath,amsfonts,amsthm,mathrsfs,dsfont,graphicx}

\usepackage{ifthen,indentfirst,enumerate,color,titletoc}

\usepackage{tikz}

\usepackage{multicol}

\usepackage{makecell}

\usepackage{longtable}

\usetikzlibrary{arrows,calc,intersections,patterns,decorations.pathreplacing,3d,angles}

\usepackage[bf,small,indentafter,pagestyles]{titlesec}

\usepackage[top=1in, bottom=1in,left=0.8in,right=0.8in]{geometry}

\renewcommand{\baselinestretch}{1.65}

\newtheorem{defi}{定义~}

\newtheorem{eg}{例~}

\newtheorem{ex}{~}

\newtheorem{rem}{注~}

\newtheorem{thm}{定理~}

\newtheorem{coro}{推论~}

\newtheorem{axiom}{公理~}

\newtheorem{prop}{性质~}

\newcommand{\blank}[1]{\underline{\hbox to #1pt{}}}

\newcommand{\bracket}[1]{(\hbox to #1pt{})}

\newcommand{\onech}[4]{\par\begin{tabular}{p{.9\textwidth}}

A.~#1\\

B.~#2\\

C.~#3\\

D.~#4

\end{tabular}}

\newcommand{\twoch}[4]{\par\begin{tabular}{p{.46\textwidth}p{.46\textwidth}}

A.~#1& B.~#2\\

C.~#3& D.~#4

\end{tabular}}

\newcommand{\vartwoch}[4]{\par\begin{tabular}{p{.46\textwidth}p{.46\textwidth}}

(1)~#1& (2)~#2\\

(3)~#3& (4)~#4

\end{tabular}}

\newcommand{\fourch}[4]{\par\begin{tabular}{p{.23\textwidth}p{.23\textwidth}p{.23\textwidth}p{.23\textwidth}}

A.~#1 &B.~#2& C.~#3& D.~#4

\end{tabular}}

\newcommand{\varfourch}[4]{\par\begin{tabular}{p{.23\textwidth}p{.23\textwidth}p{.23\textwidth}p{.23\textwidth}}

(1)~#1 &(2)~#2& (3)~#3& (4)~#4

\end{tabular}}

\begin{document}



1.0000 相同

005877	与$-45^\circ$角终边相同的角的集合是\blank{50}.

003065	与$-45^\circ$角终边相同的角的集合是\blank{50}.



0.9344 关联

005886	设角$\alpha$的终边与$\dfrac 75\pi$的终边关于$y$轴对称, 且$\alpha \in (-2\pi ,2\pi)$, 则$\alpha =$\blank{50}.

003066	设角$\alpha$的终边与角$\dfrac{7\pi}5$的终边关于$y$轴对称, 且$\alpha\in (0,2\pi)$, 则$\alpha=$\blank{50}.



0.9427 相同

005906	$\dfrac{\sin x}{|\sin x|}+\dfrac{|\cos x|}{\cos x}+\dfrac{\tan x}{|\tan x|}+\dfrac{|\cot x|}{\cot x}$的取值范围是\blank{50}.

003061	函数$f(x)=\dfrac{\sin x}{|\sin x|}+\dfrac{|\cos x|}{\cos x}+\dfrac{\tan x}{|\tan x|}+\dfrac{|\cot x|}{\cot x}$的值域是\blank{50}.



0.9412 关联

005910	函数$y=\sqrt {\cos x}$的定义域是\blank{50}.

003148	函数$y=\sqrt{-\cos x}$的定义域为\blank{50}.



0.9697 相同

006010	函数$f(x)=\log _{\frac 12}(2\sin x)$的最小值是\blank{50}.

001506	函数$f(x)=\log_{\frac{1}{2}} (2\sin x)$的最小值是\blank{80}.



0.9949 相同

006229	化简$\dfrac{\tan (45^\circ -\alpha)}{1-\tan ^2(45^\circ -\alpha)}\cdot \dfrac{\sin \alpha \cos \alpha}{\cos ^2\alpha -\sin ^2\alpha}=$\blank{50}.

003114	化简: $\dfrac{\tan (45^\circ-\alpha)}{1-\tan^2(45^\circ-\alpha)}\cdot \dfrac{\sin \alpha \cos \alpha}{\cos^2\alpha -\sin ^2\alpha}=$\blank{50}.



0.9538 相同

006886	已知等差数列$\{a_n\}$的首项为$1$, 公差为$d$, 前$n$项和为$A_n$; 等比数列$\{b_n\}$的首项为$1$, 公比为$q$($|q|<1$), 前$n$项和为$B_n$.记$S_n=B_1+B_2+\cdots +B_n$, 若$\displaystyle \lim_{n\to \infty} (\dfrac{A_n}n-S_n)=1$, 求$d$和$q$.

003307	已知等差数列$\{a_n\}$的首项为$1$, 公差为$d$, 前$n$项的和为$A_n$; 等比数列的首项为$1$, 公比为$q$, $|q|<1$, 前$n$项的和为$B_n$, 记$S_n=B_1+B_2+\cdots+B_n$, 若$\displaystyle\lim_{n\to \infty}(\dfrac{a_n}n-S_n)=1$, 求$d$、$q$.



0.9864 相同

006894	已知$S_n=\dfrac 15+\dfrac 2{5^2}+\dfrac 1{5^3}+\dfrac 2{5^4}+\cdots +\dfrac 1{5^{2n-1}}+\dfrac 2{5^{2n}}$, 则$\displaystyle \lim_{n\to \infty} S_n=$\blank{50}.

003715	若$S_n=\dfrac 15+\dfrac {2}{5^2}+\dfrac {1}{5^3}+\dfrac{2}{5^4}+\cdots+\dfrac{1}{5^{2n-1}}+\dfrac{2}{5^{2n}}$, 则$\displaystyle\lim_{n\to \infty}S_n=$\blank{50}.



0.9238 相同

006925	利用数学归纳法证明: $1-\dfrac 12+\dfrac 13-\dfrac 14+\cdots +\dfrac 1{2n-1}-\dfrac 1{2n}=\dfrac 1{n+1}+\dfrac 1{n+2}+\cdots +\dfrac 1{2n}$($n\in \mathbf{N}^*$).

000322	用数学归纳法证明: $1-\dfrac12+\dfrac 13-\dfrac 14+\cdots +\dfrac{1}{2n-1}-\dfrac{1}{2n}=\dfrac{1}{n+1}+\dfrac{1}{n+2}+\cdots+\dfrac{1}{2n}$($n$为正整数).



1.0000 相同

007013	若复数$z=(x-1)+(2x-1)\mathrm{i}$的模小于$\sqrt {10}$, 则实数$x$的取值范围是\blank{50}.

002011	若复数$z=(x-1)+(2x-1)\mathrm{i}$的模小于$\sqrt{10}$, 则实数$x$的取值范围是\blank{50}.



0.9640 相关

007092	计算: $\mathrm{i}\cdot \mathrm{i}^2\cdot \mathrm{i}^3\cdot \cdots \cdot \mathrm{i}^{1997}=$\blank{50}.

003550	计算: $\mathrm{i}\cdot\mathrm{i}^2\cdot\mathrm{i}^3\cdot \cdots \cdot \mathrm{i}^{100}=$\blank{50}.



0.9500 相同

007123	已知复数$z$满足$|z|=2$, 求复数$w =\dfrac{z+1}z$在复平面内的对应点的轨迹.

003535	已知复数$z$满足$|z|=2$, 求复数$w=\dfrac{1+z}z$在复平面内的对应点的轨迹.



0.9832 相同

007156	记$A=\cos \dfrac{\pi }{11}+\cos \dfrac{3\pi }{11}+\cos \dfrac{5\pi }{11}+\cos \dfrac{7\pi }{11}+\cos \dfrac{9\pi }{11}$, $B=\sin \dfrac{\pi }{11}+\sin \dfrac{3\pi }{11}+\sin \dfrac{5\pi }{11}+\sin \dfrac{7\pi }{11}+\sin \dfrac{9\pi }{11}$, 求证: $A=\dfrac 12$, $B=\dfrac 12\cot \dfrac{\pi }{22}$.

002061	[选做]

记$A=\cos\dfrac{\pi}{11}+\cos\dfrac{3\pi}{11}+\cos\dfrac{5\pi}{11}+\cos\dfrac{7\pi}{11}+\cos\dfrac{9\pi}{11}$, $B=\sin\dfrac{\pi}{11}+\sin\dfrac{3\pi}{11}+\sin\dfrac{5\pi}{11}+\sin\dfrac{7\pi}{11}+\sin\dfrac{9\pi}{11}$. 证明: $A=\dfrac{1}{2}$, $B=\dfrac{1}{2}\cot\dfrac{\pi}{22}$.



0.9402 相同

007296	若实系数的一元二次方程的一个根是$\dfrac 13-\dfrac{4\sqrt 5}3\mathrm{i}$, 则这个方程为\blank{50}.

002080	若实系数一元二次方程的一个根是$\dfrac{1}{3}-\dfrac{4\sqrt{5}}{3}\mathrm{i}$, 则这个方程可以是\blank{80}.



0.9773 相同

007321	若关于$x$的实系数方程$2x^2+3ax+a^2-a=0$至少布一个模为1的根, 求实数$a$的值.

002094	若关于$x$的实系数方程$2x^2+3ax+a^2-a=0$至少有一个模为$1$的根, 求实数$a$的值.



0.9369 相同

007325	实系数方程$x^4-4x^3+9x^2-ax+b=0$的一个根是$1+\mathrm{i}$, 求$a$, $b$的值, 并解此方程.

002084	已知关于$x$的实系数方程$x^4-4x^3+9x^2-ax+b=0$的一个根是$1+\mathrm{i}$, 求$a,b$的值并解此方程.



0.9667 相同

007352	已知半径为1的定圆$O$的内接正$n$边形的顶点为$P_k$($k=1,2,\cdots n$), $P$为该圆周上任意一点, 求证: $|PP_1|^2+|PP_2|^2+\cdots +|PP_n|^2$为一定值.

002075	[选做]

已知半径为$1$的定圆$O$的内接正$n$边形的顶点为$P_k(k=1,2,\cdots,n)$, $P$为该圆周上任意一点, 求证: $|PP_1|^2+|PP_2|^2+\cdots+|PP_n|^2$是一个定值.



0.9796 相同

007447	计算: $\mathrm{C}_m^5-\mathrm{C}_{m+1}^5+\mathrm{C}_m^4=$\blank{50}.

002568	计算: $\mathrm{C}_m^5-\mathrm{C}_{m+1}^5+\mathrm{C}_m^4=$\blank{80}.



0.9701 相同

007448	计算: $\mathrm{C}_{96}^{94}+\mathrm{C}_{97}^{95}+\mathrm{C}_{98}^{96}+\mathrm{C}_{99}^{97}=$\blank{50}.

002571	计算: $\mathrm{C}_{97}^{94}+\mathrm{C}_{97}^{95}+\mathrm{C}_{98}^{96}+\mathrm{C}_{99}^{97}=$\blank{80}.



0.9697 关联

007449	计算: $\mathrm{C}_2^2+\mathrm{C}_3^2+\mathrm{C}_4^2+\cdots +\mathrm{C}_{10}^2=$\blank{50}.

002570	计算: $\mathrm{C}_2^2+\mathrm{C}_3^2+\mathrm{C}_4^2+\cdot+\mathrm{C}_{100}^2=$\blank{80}.



0.9592 相同

007456	平面内共有$17$个点, 其中有且仅有$5$个点共线, 以这些点中的$3$个点为顶点的三角形共有\blank{50}个.

002579	平面内共有$17$个点, 其中有且仅有$5$个点共线, 以这些点中的三个点为顶点的三角形共有\blank{80}个.



0.9444 关联

005989	求函数$y=\dfrac{\sec ^2x-\tan x}{\sec ^2x+\tan x}$的值域.

006078	求函数$y=\dfrac{\sec ^2x+\tan x}{\sec ^2x-\tan x}$的值域.



0.9945 相同

006107	化简$\dfrac{1+\cos \theta -\sin \theta}{1-\cos \theta -\sin \theta}+\dfrac{1-\cos \theta -\sin \theta}{1+\cos \theta -\sin \theta}$.

006218	化简: $\dfrac{1+\cos \theta -\sin \theta}{1-\cos \theta -\sin \theta}+\dfrac{1-\cos \theta -\sin \theta}{1+\cos \theta -\sin \theta}$.



1.0000 相同

006252	在$\triangle ABC$中, 求证: $\sin ^2\dfrac A2+\sin ^2\dfrac B2+\sin ^2\dfrac C2=1-2\sin \dfrac A2\sin \dfrac B2\sin \dfrac C2$.

006334	在$\triangle ABC$中, 求证: $\sin ^2\dfrac A2+\sin ^2\dfrac B2+\sin ^2\dfrac C2=1-2\sin \dfrac A2\sin \dfrac B2\sin \dfrac C2$.



0.9487 关联

006489	函数$y=\sqrt {\arcsin x}$的定义域为\blank{50}, 值域为\blank{50}.

006510	函数$y=\sqrt {\arccos x}$的定义域为\blank{50}, 值域为\blank{50}.



0.9545 关联

006496	计算: $\arcsin (\cos 2)=$\blank{50}.

006497	计算: $\arcsin (\cos 5)=$\blank{50}.



0.9748 关联

006503	求函数$f(x)=\sin (x-\dfrac{\pi }4)\cos (x+\dfrac{\pi }4), \ -\dfrac{\pi }4\le x\le \dfrac{\pi }4$的反函数.

006504	求函数$f(x)=\sin (x-\dfrac{\pi }4)\cos (x+\dfrac{\pi }4),\  \dfrac{\pi }4\le x\le \dfrac{\pi }2$的反函数.



1.0000 关联

006582	解方程$\sin 2x-12(\sin x-\cos x)+12=0$.

006625	解方程$\sin 2x-12(\sin x-\cos x)+12=0$.



0.9487 相同

006855	$\displaystyle \lim_{n\to \infty} (1-\dfrac 12)(1-\dfrac 13)(1-\dfrac 14)\cdots (1-\dfrac 1n)=$\blank{50}.

006856	$\displaystyle \lim_{n\to \infty} (1-\dfrac 1{2^2})(1-\dfrac 1{3^2})(1-\dfrac 1{4^2})\cdots (1-\dfrac 1{n^2})=$\blank{50}.



0.9639 相同

006909	用数学归纳法证明: $1+2+\cdots +2n=n(2n+1)$($n\in \mathbf{N}^*$).

006923	利用数学归纳法证明: $1+2+3+\cdots +2n=n(2n+1)$($n\in \mathbf{N}^*$).



0.9459 关联

007034	若复数$z$满足$z+\dfrac 4z\in \mathbf{R}$, 且$|z-2|=2$, 求$z$.

007115	已知复数$z$满足$z+\dfrac 4z\in \mathbf{R}$, $|z-2|=2$, 求$z$.



0.9730 关联

007070	若$z$是复数, 判断``$|z|^2=z^2$恒成立''的真假:\blank{50}.

007071	若$z$是复数, 判断``$|z|^2= z^2$恒不成立''.的真假:\blank{50}.



0.9730 关联

007070	若$z$是复数, 判断``$|z|^2=z^2$恒成立''的真假:\blank{50}.

007072	若$z$是复数, 判断``$|z|^2=|z|^2$恒成立''的真假:\blank{50}.



0.9474 关联

007071	若$z$是复数, 判断``$|z|^2= z^2$恒不成立''.的真假:\blank{50}.

007072	若$z$是复数, 判断``$|z|^2=|z|^2$恒成立''的真假:\blank{50}.



0.9231 关联

007072	若$z$是复数, 判断``$|z|^2=|z|^2$恒成立''的真假:\blank{50}.

007074	若$z$是复数, 判断``$\sqrt {|z|^2}=|z|$恒成立''的真假:\blank{50}.



0.9383 关联

007076	若$z$是复数, 判断``$z+\overline z$一定是实数''的真假:\blank{50}.

007077	若$z$是复数, 判断``$z-\overline z$一定是纯虚数''的真假:\blank{50}.



0.9697 相同

007112	已知复数$z$满足$|z|=5$, 且$(3+4\mathrm{i})z$是纯虚数, 求$z$.

007233	已知复数$z$满足$|z|=5$, 且$(3+4\mathrm{i})z$为纯虚数, 求$z$.



0.9290 关联

007129	利用$||z_1|-|z_2||\le|z_1+z_2|\le|z_2|+|z_2|$, 求函数$y=\sqrt {x^2+4}+\sqrt {x^2-8x+17}$的最小值及相应的$x$.

007130	利用$||z_1|-|z_2||\le|z_1+z_2|\le|z_2|+|z_2|$, 求函数$y=\sqrt {x^2+9}-\sqrt {x^2-2x+5}$的最大值及相应的$x$.



0.9630 关联

007140	已知$|z|=1$, 求$|z^2-z+1|$的最大值和最小值.

007141	已知$|z|=1$, 求$|z^2-z+2|$的最大值和最小值.



0.9434 相同

007140	已知$|z|=1$, 求$|z^2-z+1|$的最大值和最小值.

007234	若$|z|=1$, 求$|z^2-z+1|$的最大值和最小值.



0.9639 关联

007143	将复数$2(\cos \dfrac{\pi }5-\mathrm{i}\sin \dfrac{\pi }5)$化为三角形式.

007144	将复数$2(-\cos \dfrac{\pi }5+\mathrm{i}\sin \dfrac{\pi }5)$化为三角形式.



0.9639 关联

007143	将复数$2(\cos \dfrac{\pi }5-\mathrm{i}\sin \dfrac{\pi }5)$化为三角形式.

007145	将复数$-2(\cos \dfrac{\pi }5+\mathrm{i}\sin \dfrac{\pi }5)$化为三角形式.



0.9762 关联

007144	将复数$2(-\cos \dfrac{\pi }5+\mathrm{i}\sin \dfrac{\pi }5)$化为三角形式.

007145	将复数$-2(\cos \dfrac{\pi }5+\mathrm{i}\sin \dfrac{\pi }5)$化为三角形式.



0.9684 关联

007169	复数$2(\cos \dfrac{\pi }5-\mathrm{i}\sin \dfrac{\pi }5)$的三角形式为\blank{50}.

007171	复数$2(-\cos \dfrac{\pi }5+\mathrm{i}\sin \dfrac{\pi }5)$的三角形式为\blank{50}.



0.9684 关联

007169	复数$2(\cos \dfrac{\pi }5-\mathrm{i}\sin \dfrac{\pi }5)$的三角形式为\blank{50}.

007172	复数$-2(\cos \dfrac{\pi }5+\mathrm{i}\sin \dfrac{\pi }5)$的三角形式为\blank{50}.



0.9792 关联

007171	复数$2(-\cos \dfrac{\pi }5+\mathrm{i}\sin \dfrac{\pi }5)$的三角形式为\blank{50}.

007172	复数$-2(\cos \dfrac{\pi }5+\mathrm{i}\sin \dfrac{\pi }5)$的三角形式为\blank{50}.



0.9262 关联

007243	复平面内, 两点$A,B$分别对应于非零复数$\alpha ,\beta$, 若$\alpha =\pm \beta \mathrm{i}$, 判断$\triangle OAB$的形状($O$为原点).

007246	复平面内, 两点$A,B$分别对应于非零复数$\alpha ,\beta$, 若$\dfrac{\alpha }{\beta }=1+\mathrm{i}$, 判断$\triangle OAB$的形状($O$为原点).



0.9412 关联

007244	复平面内, 两点$A,B$分别对应于非零复数$\alpha ,\beta$, 若$\dfrac{\alpha }{\beta }=\pm \sqrt 3\mathrm{i}$, 判断$\triangle OAB$的形状($O$为原点).

007245	复平面内, 两点$A,B$分别对应于非零复数$\alpha ,\beta$, 若$\dfrac{\alpha }{\beta }=\dfrac{1+\sqrt 3\mathrm{i}}2$, 判断$\triangle OAB$的形状($O$为原点).



0.9434 关联

007244	复平面内, 两点$A,B$分别对应于非零复数$\alpha ,\beta$, 若$\dfrac{\alpha }{\beta }=\pm \sqrt 3\mathrm{i}$, 判断$\triangle OAB$的形状($O$为原点).

007246	复平面内, 两点$A,B$分别对应于非零复数$\alpha ,\beta$, 若$\dfrac{\alpha }{\beta }=1+\mathrm{i}$, 判断$\triangle OAB$的形状($O$为原点).



0.9333 关联

007245	复平面内, 两点$A,B$分别对应于非零复数$\alpha ,\beta$, 若$\dfrac{\alpha }{\beta }=\dfrac{1+\sqrt 3\mathrm{i}}2$, 判断$\triangle OAB$的形状($O$为原点).

007246	复平面内, 两点$A,B$分别对应于非零复数$\alpha ,\beta$, 若$\dfrac{\alpha }{\beta }=1+\mathrm{i}$, 判断$\triangle OAB$的形状($O$为原点).



0.9459 关联

007276	已知$z_n=(\dfrac{1+\mathrm{i}}2)^n$($n\in \mathbf{N}$). 记$a_n=|z_{n+1}|-|z_n|$($n\in \mathbf{N}$), 求数列$\{a_n\}$所有项之和.

007277	已知$z_n=(\dfrac{1+\mathrm{i}}2)^n$($n\in \mathbf{N}$). 记$b_n=|z_{n+2}-z_n|$($n\in \mathbf{N}$), 求数列$\{b_n\}$所有项之和.



0.9873 相同

007353	由$1, 2, 3, 4, 5, 6$这$6$个数字可以组成多少个数字不重复且是$6$的倍数的五位数?

007430	由$1, 2, 3, 4, 5, 6$这$6$个数字可组成多少个数字不重复且是$6$的倍数的五位数?



0.9811 相同

007357	从$1, 3, 5, 7$这$4$个数字中任取$3$个, 从$0, 2, 4$这$3$个数字中任取$2$个, 可以组成多少个无重复数字的五位数?

007520	从$1, 3, 5, 7$这$4$个数字中任取$3$个, 从$0, 2, 4$这$3$个数字中任取$2$个, 共可组成多少个无重复数字的五位数?



1.0000 相同

007530	求$(1+x+x^2)(1-x)^{10}$展开式中含$x^4$项的系数.

007595	求$(1+x+x^2)(1-x)^{10}$展开式中含$x^4$项的系数.



0.9948 相同

007537	求证$\mathrm{C}_n^0\mathrm{C}_n^1+\mathrm{C}_n^1\mathrm{C}_n^2+\cdots +\mathrm{C}_n^{n-1}\mathrm{C}_n^n=\dfrac{(2n)!}{(n-1)!(n+1)!}$.

007649	求证: $\mathrm{C}_n^0\mathrm{C}_n^1+\mathrm{C}_n^1\mathrm{C}_n^2+\cdots +\mathrm{C}_n^{n-1}\mathrm{C}_n^n=\dfrac{(2n)!}{(n-1)!(n+1)!}$.



0.9524 关联

007651	利用$k\mathrm{C}_n^k=n\mathrm{C}_{n-1}^{k-1}$, 求证: $\mathrm{C}_n^1+2\mathrm{C}_n^2+3\mathrm{C}_n^3+\cdots +n\mathrm{C}_n^n=n\cdot 2^{n-1}$.

007653	利用$k\mathrm{C}_n^k=n\mathrm{C}_{n-1}^{k-1}$, 求证: $\mathrm{C}_n^0+2\mathrm{C}_n^1+3\mathrm{C}_n^2+\cdots +(n+1)\mathrm{C}_n^n=(n+2)\cdot 2^{n-1}$.







\end{document}