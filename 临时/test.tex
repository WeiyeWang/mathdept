\documentclass[10pt,a4paper]{article}
\usepackage[UTF8,fontset = windows]{ctex}
\setCJKmainfont[BoldFont=黑体,ItalicFont=楷体]{等线}
\usepackage{amssymb,amsmath,amsfonts,amsthm,mathrsfs,dsfont,graphicx}
\usepackage{ifthen,indentfirst,enumerate,color,titletoc}
\usepackage{tikz}
\usepackage{multicol}
\usepackage{makecell}
\usetikzlibrary{arrows,calc,intersections,patterns}
\usepackage[bf,small,indentafter,pagestyles]{titlesec}
\usepackage[top=1in, bottom=1in,left=0.8in,right=0.8in]{geometry}
\renewcommand{\baselinestretch}{1.65}
\newtheorem{defi}{定义~}
\newtheorem{eg}{例~}
\newtheorem{ex}{~}
\newtheorem{rem}{注~}
\newtheorem{thm}{定理~}
\newtheorem{coro}{推论~}
\newtheorem{axiom}{公理~}
\newtheorem{prop}{性质~}
\newcommand{\blank}[1]{\underline{\hbox to #1pt{}}}
\newcommand{\bracket}[1]{(\hbox to #1pt{})}
\newcommand{\onech}[4]{\par\begin{tabular}{p{.9\textwidth}}
A.~#1\\
B.~#2\\
C.~#3\\
D.~#4
\end{tabular}}
\newcommand{\twoch}[4]{\par\begin{tabular}{p{.46\textwidth}p{.46\textwidth}}
A.~#1& B.~#2\\
C.~#3& D.~#4
\end{tabular}}
\newcommand{\vartwoch}[4]{\par\begin{tabular}{p{.46\textwidth}p{.46\textwidth}}
(1)~#1& (2)~#2\\
(3)~#3& (4)~#4
\end{tabular}}
\newcommand{\fourch}[4]{\par\begin{tabular}{p{.23\textwidth}p{.23\textwidth}p{.23\textwidth}p{.23\textwidth}}
A.~#1 &B.~#2& C.~#3& D.~#4
\end{tabular}}
\newcommand{\varfourch}[4]{\par\begin{tabular}{p{.23\textwidth}p{.23\textwidth}p{.23\textwidth}p{.23\textwidth}}
(1)~#1 &(2)~#2& (3)~#3& (4)~#4
\end{tabular}}
\begin{document}

\begin{enumerate}[1.]
    \item 求函数$y=\dfrac{3x-1}{x+1}$的值域.
    \item 求函数$y=\dfrac{4x+3}{2x-1}$的值域.
    \item 求函数$y=\dfrac{{x^2}-1}{{x^2}+2}$的值域.
    \item 求函数$y=\dfrac{{x^2}-x+1}{2{x^2}-2x+3}$的值域.
    \item 求函数$y=\dfrac{{x^2}+4x+3}{{x^2}+x-6}$的值域.
    \item 若实数$x,y$满足$x^2+4y^2=4x$, 求$S=x^2+y^2$的值域.
    \item 已知函数$y=f(x)=x^2+ax+3$在区间$x\in [ -1,1 ]$上的最小值为-3, 求实数$a$的值.
    \item 求函数$y=3{x^2}-12x+18\sqrt{4x-{x^2}}-23$的值域.
    \item 求函数$y=|x-2|-|x+1|$的值域.
    \item 若$f(x-1)=2x^2+1$, 求$f(x)$.
    \item 已知定义域为$R$的函数$f(x)$满足:
    \textcircled{1} $f(x+y)=f(x)\cdot f(y)$对任何实数$x,y$都成立;
    \textcircled{2} 存在实数$x_1,x_2$, 使$f(x_1)\ne f(x_2)$.
    求证: (1)$f(0)=1$.
    (2)$f(x)>0$.
    【训练题】
    (一)映  射
    \item 设映射$f:X\to Y$, 其中$X,Y$是非空集合, 则下列语句中正确的是\bracket{20}.
    \fourch{$Y$中每一个元素必有原像}{$Y$中的各元素只能有一个原像}{$X$中的不向元素在$Y$中的像也不同}{$Y$中至少存在一个元素, 它有原像}
    \item 集合$M=\{a,b,c\}$与$P=\{x,y,z\}$之间建立起四种对应关系(如图), 则下列结论中正确的是\bracket{20}.
    \fourch{只有$f_2,f_3$是从$M$到$P$的映射}{只有$f_2,f_4$是从$M$到$P$的映射}{只有$f_3,f_4$是从$M$到$P$的映射}{$f_1,f_2,f_3,f_4$都是从$M$到$P$的映射}
    \item 设$(x,y)$在映射$f$下的像是$(\dfrac{x+y}2,\dfrac{x-y}2)$, 则在$f$下$(-5,2)$的原像是\bracket{20}.
    \fourch{$(-10,4)$}{$(-3,-7)$}{$(-6,-4)$}{$(-\dfrac 32,-\dfrac 72)$}
    \item 在给定的映射$f:(x,y)\to (2x+y,xy)(x,y\in \mathbf{R})$下, 点$(\dfrac 16,-\dfrac 16)$的原像是\bracket{20}.
    \fourch{$(\dfrac 16,-\dfrac 1{36})$}{$(\dfrac 13,-\dfrac 12)$或$(-\dfrac 14,\dfrac 23)$}{$(\dfrac 1{36},-\dfrac 16)$}{$(\dfrac 12,-\dfrac 13)$或$(-\dfrac 23,\dfrac 14)$}
    \item 已知集合$M=\{x|0\le x\le 6\}$, $P=\{0\le y\le 3\}$, 则下列对应关系中, 不能行作从$M$到$P$的映射的是\bracket{20}.
    \fourch{$f:x\to y=\dfrac 12x$}{$f:x\to y=\dfrac 13x$}{$f:x\to y=x$}{$f:x\to y=\dfrac 16x$}
    \item 设$M=R$, 从$M$到$P$的映射$f:x\to y=\dfrac 1{{x^2}+1}$, 则像集$P$为\bracket{20}.
    \fourch{$\{y|y\in \mathbf{R}\}$}{$\{y|y\in \mathbf{R}\}$}{$\{y|0\le y\le 2\}$}{$\{y|0<y\le 1\}$}
    \item 若映射$f:A\to B$的像集是$Y$, 原像的集合是$X$, 则$X$与$A$的关系是\blank{50}, $Y$和$B$的关系是\blank{50}.
    \item (1)若$(x,y)$在映射$f$下的像是$(2x-y,x+2y)$, 则$(-1,2)$在$f$下的原像是\blank{50}.
      (2)已知$(a,b)$在映射$f$的像是$(a-b,ab)$, 则$(2,3)$的原像是\blank{50}.
      (3)已知$f:x\to y=x^2$是从集合$R$到集合$M=\{x|x\ge 0\}$的一个映射, 则$M$重的元素1在$R$中的原像是\blank{50}, $M$中的元素$t(t>0)$在$R$中的原像是\blank{50}.
    \item (1)从集合$\{a\}$到$\{b,c\}$的不同映射有    个.
      (2)从集合$\{1,2\}$到$\{5,6\}$的不同映射有    个.
    \item 已知集合$A=Z$, $B=\{x|x=2n+1,n\in \mathbf{Z}\}$, $C=R$, 且从$A$到$B$的映射是$x\to 2x-1$, 从$B$到$C$的映射是$x\to \dfrac 1{3x+1}$, 则从$A$到$C$的映射是\blank{50}.
    \item $f$是集合$X=\{a,b,c\}$到集合$Y=\{d,e\}$的一个映射, 则满足映射条件的``$f$''共有\bracket{20}.
    \fourch{5个}{6个}{7个}{8个}
    \item 若$f:y=3x+1$是从集合$A=\{1,2,3,k\}$到集合$B=\{4,7,a^4,a^2+3a\}$的一个映射, 求自然数$a,k$的值及集合$A,B$.
    (二)函数
    \item 函数$f(x)=\dfrac{\sqrt{{x^2}-5x+6}}{x-2}$的定义域是\bracket{20}.
    \fourch{$\{x|2<x<3\}$}{$\{x|x<2x>3\}$}{$\{x|x\le 2x\ge 3\}$}{$\{x|x<2x\ge 3\}$}
    \item 若函数$f(x)$的定义域是$[ -1,1 ]$, 则函数$f(x+1)$的定义域是\bracket{20}.
    \fourch{$[ -1,1 ]$}{$[ 0,2 ]$}{$[ -2,0 ]$}{$[ 0,1 ]$}
    \item 在``\textcircled{1} $y=x$与$y=\sqrt{x^2}$, \textcircled{2} $y=\sqrt{x^2}$与$y=(\sqrt x)^2$, \textcircled{3} $y=|x|$与$y=\dfrac{x^2}x$, \textcircled{4} $y=|x|$与$y=\sqrt{x^2}$, \textcircled{5} $y=x^0$与$y=1$''这五组函数中, 表示同一函数的组数是\bracket{20}.
    \fourch{0}{1}{2}{3}
    \item 函数$y=-x^2-2x+3(-5\le x\le 0)$的值域是\bracket{20}.
    \fourch{$(-\infty ,4 ]$}{$[ 3,12 ]$}{$[ -12,4 ]$}{$[ 4,12 ]$}
    \item 已知镭经过100年后剩下原来质量的95. 76%, 若质量为l克的镭经过$x$年后的剩余质量为$y$克, 则$y$与$x$之间的解析式是\bracket{20}.
    \fourch{$y=(\dfrac{0.9576}{100})^x$}{$y=(0.9576)^{100x}$}{$y={{(0.9576)}^{\dfrac x{100}}}$}{$y=1-{{(1-0.9576)}^{\dfrac x{100}}}$}
    \item 函数$y=x+\dfrac{|x|}x$的图象是\bracket{20}.
    \item 求下列函数的定义域:
    (1)$y=\sqrt{1-{x^2}}+\sqrt{x+1}$:\blank{50}.
    (2)$y=\dfrac 1{\sqrt{2{x^2}+3}}$:\blank{50}.
    (3)$y=\dfrac{x+5}{3{x^2}-2x-1}$:\blank{50}.
    (4)$y=\sqrt{6x-{x^2}-9}$:\blank{50}.
    (5)$y=\sqrt{4-{x^2}}+\dfrac 1{|x|-1}$:\blank{50}.
    (6)$y=\dfrac{{x^3}-1}{x+|x|}$:\blank{50}.
    (7)$y=\dfrac 1{|x|-{x^2}}$:\blank{50}.
    (8)$y=\sqrt{1-(\dfrac{x-1}{x+1})^2}$:\blank{50}.
    (9)$y=\dfrac{\sqrt{{x^2}-2x-15}}{|x+3|-8}$:\blank{50}.
    \item 求下列函数的值域:
    (1)$y=1-\dfrac 1{x+2}$:\blank{50}.				(2)$y=\dfrac 3{2x}$:\blank{50}.
    (3)$y=\dfrac{x+3}{x-3}$:\blank{50}.					(4)$y=\dfrac{5x+3}{x-3}$:\blank{50}.
    (5)$y=4+\sqrt{2x+1}$:\blank{50}.				(6)$y=\sqrt{x-\dfrac 12{x^2}}$:\blank{50}.
    (7)$y=\sqrt{-{x^2}+x+2}$:\blank{50}.			(8)$y=\dfrac{2{x^2}+2x+3}{{x^2}+x+1}$:\blank{50}.
    2l.(1)若函数$f(x)$满足$f(2x)=(1-\sqrt 2x)(1+\sqrt 2x)$, 则$f(x)=$\blank{50}.
    (2)若函数$f(x)$满足$f(\sqrt x+1)=x+2\sqrt x$, 则$f(x)=$\blank{50}.
    (3)若函数$f(x)$满足$f(\dfrac 1x)=\dfrac x{1-{x^2}}$, 则$f(x)=$\blank{50}.
    (4)若函数$f(x)=2x+1$, $g(x)=x^2+2$, 满足$f(g(x))=g(f(x))$, 则$x=$\blank{50}.
    (5)若函数$f(x)$满足$f(x+1)=2x^2+1$, 则$f(x-1)=$\blank{50}.
    (6)若一次函数$f(x)$满足$f(f(x))=1+2x$, 则$f(x)=$\blank{50}.
    (7)若$f(x^2-x)=x^4-2x^3+x^2+1$, 则$f(f(x))=$\blank{50}.
    (8)若函数$f(x)=\dfrac x{\sqrt{1+{x^2}}}$, 则$f(f(x))=$\blank{50}, $f(f(f(x)))=$\blank{50}.
    \item 若$-b<a<0$, 且函数$d(x)$的定义域是$[ a,b ]$, 则函数$F(x)=f(x)+f(-x)$的定义域是\bracket{20}.
    \fourch{$[ a,b ]$}{$[ -b,-a ]$}{$[ -b,b ]$}{$[ a,-a ]$}
    \item 若$f(x)$的定义域是$[ 0,1 ]$, 且$f(x+m)+f(x-m)$的定义域是$\varnothing$, 则正数$m$的取值范围是\bracket{20}.
    \fourch{$0<m<1$}{$0<m\le \dfrac 12$}{$0<m<\dfrac 12$}{$m>\dfrac 12$}
    \item 函数$y=\dfrac{{x^2}-1}{{x^2}+1}$的值域是\bracket{20}.
    \fourch{$(-1,1)$}{$[ -1,1 ]$}{$[ -1,1)$}{$(-1,1 ]$}
    \item 若$2x^2-3x\le 0$, 则函数$f(x)=x^2+x+1$\bracket{20}.
    \fourch{有最小值$\dfrac 34$, 但无最大值}{有最小值$\dfrac 34$, 有最大值1}{有最小值1有最大值$\dfrac{19}4$}{既无最小值, 也无最大值}
    \item 函数$f(x)=|1-x|-|x-3|(x\in \mathbf{R})$的值域是\bracket{20}.
    \fourch{$[ -2,2 ]$}{$[ -1,3 ]$}{$[ -3,1 ]$}{$[ 0,4 ]$}
    \item (1)若函数$f(x)$的定义域是$[ 0,1 ]$, 分别求函数$f(1-2x)$和$f(x+a)(a>0)$的定义域.
       (2)若函数$f(x+1)$的定义域是$[ -2,3)$, 求函数$f(\dfrac 1x+2)$的定义域.
    \item 求下列函数的值域:
    (1)$y=\dfrac{2x}{{x^2}+x+1}$.			(2)$y=\dfrac{{x^2}+x-1}{{x^2}+x+1}$.		(3)$y=\dfrac{{x^2}-1}{{x^2}-5x+4}$.
    \item (1)若实数$x,y$满足$3x^2+2y^2=6x$, 分别求$x$与$x^2+y^2$的取值范围.
        (2)若实数$x,y$满足$x^2+y^2=2x$, 求$x^2-y^2$的取值范围.
    \item 求下列函数的值域:
    (1)$y=3x-2+\sqrt{3-2x}$.
    (2)$y=2x+\sqrt{2x-1}$.
    (3)$y=(x-1)(x-2)(x-3)(x-4)+15$.
    \item (1)已知函数$f(x)=x^2-2x+3$在$[ 0,m ]$上有最大值3, 最小值2, 求正数$m$的取值范围.
        (2)已知函数$y=x^2+mx-1$在区间$[ 0,3 ]$上有最小值-2, 求实数$m$的值.
        (3)当$x\ge 0$时, 求函数$f(x)=x^2+2ax$的最小值.
    \item 已知函数$f(x)=\dfrac{ax}{2x+3}(x\ne -\dfrac 32)$满足$f(f(x))=x$, 求实数$a$的值.
    \item (1)已知$f(x)$是二次函数, 且满足$f(2x)+f(3x+1)=13x^2+6x-1$, 求$f(x)$的表达式.
        (2)已知函数$f(x)$的定义域是一切非零实数, 且满足$3f(x)+2f(\dfrac 1x)=4x$, 求, $f(x)$的表达式.
    \item (l)作(画)出下列函数的图象:
    \textcircled{1} $y=1+\dfrac{|x|}x$; \textcircled{2} $y=x-|1-x|$; \textcircled{3} $y=|x^2-4x+3|$; \textcircled{4} $y=\dfrac{{x^3}+x}{|x|}$; \textcircled{5} $y={{\dfrac{(x+\dfrac 12)}{|x|-x}}^0}$.
    (2)已知$f(x)=-x^2+2x+3$, 画出函数$y=\dfrac 12[ f(x)+|f(x)|]$的图象.
    (3)已知$f(x)=|x|$, $x\in [ -1,1 ]$, 作出函数$y=f(x+1)+1$的图象.
    \item (1)将进货单价为40元的商品按每件50元出售时, 每月能卖出500个, 已知这批商品在销售单价的基础上每涨价1元, 其月销售数就减少10个, 为了每月赚取最大利润, 销
    售单价应定为多少?
    (2)飞机飞行1时的耗费由两部分组成: 固定部分4900元, 变动部分$P$与飞机飞行速度$v$(千米/时)的函数关系是$P=0.01v^2$. 已知甲、乙两地相距为一常数$a$(千米), 试写出飞机从甲地飞到乙地的总耗费$y$与飞机速度$v$的函数关系式, 并写出耗费最小时飞机的飞行速度.
    二、幂函数
    \item 求证: 函数$f(x)=x^3$在$x\in \mathbf{R}$上是增函数.
    \item 已知奇函数$y=f(x)$在$x<0$时是减函数, 求证: $y=f(x)$在$x>0$时也是减函数.
    \item 已知$f(x)$是奇函数, 且当$x>0$时$f(x)=x(1-x)$, 求$f(x)$在$x<0$时的表达式.
    \item 已知函数$y=f(x)$满足$f(x)=f(4-x)(x\in \mathbf{R})$, 且$f(x)$在$x>2$时为增函数, 记$a=f(\dfrac 35)$, $b=f(\dfrac 65)$, $c=f(4)$, 则$a,b,c$之间的大小关系是\bracket{20}.
    \fourch{$c>a>b$}{$c>b>a$}{$b>a>c$}{$a>c>d$}
    \item 画出函数$y=x^2-2|x|-1$的图象.
    \item 求函数$y=\dfrac{x-2}{2x+1}$的值域.
    \item 已知函数$f(x)=(x-1)^2(x\le 1)$, 又$f(x)$和$\varphi (x)$的图象关于直线$y=x$对称, 求$\varphi (x)$的表达式.
    \item 求实数$m$的范围, 使关于$x$的方程$x^2+2(m-1)x+2m+6=0$:
    (1)有两个实数根, 且一个比2大, 另一个比2小.
    (2)有两个实数根, 且都比1大.
    (3)有两个实数根$\alpha ,\beta$, 且满足$0<\alpha <1<\beta <4$.
    (4)至少有一个正根.
    \item 就参数$m$讨论方程$x^2-2|x|-m=0$的解的情况.
    【训练题】
    (一)分数指数幂与根式
    \item 下列记数中, 符合科学记数法的是\bracket{20}.
    \fourch{$35.6\times {{10}^{-25}}$}{$0.356\times {{10}^{-23}}$}{$3.56\times {{10}^{-24}}$}{$356\times {{10}^{-26}}$}
    \item 计算${3^{-1}}\times {2^{-2}}\div {4^{-2}}$的结果是\bracket{20}.
    \fourch{$\dfrac 1{192}$}{$\dfrac 43$}{$\dfrac 1{12}$}{$-\dfrac 43$}
    \item 下列各式中, 正确的是\bracket{20}.
    \fourch{$(-1)^0=-1$}{$(-1)^{-1}=1$}{$3{a^{-2}}=\dfrac 1{3{a^2}}$}{$(-x)^5\div (-x)^3=x^2$}
    \item 下列各式中, 计算正确的是\bracket{20}.
    \fourch{$(-0.125)\div (-0.5)^{-3}=1$}{${{10}^{-4}}(\sqrt 5)^0=-10000$}{$(\dfrac 13)^0\div {3^{-1}}=3$}{$(\sqrt 3-\sqrt 2)^0-(\sqrt 3)^2-(-\sqrt 2)^2=1-3+2=0$}
    \item 化简$\dfrac 13x\sqrt{9x}-{x^2}\sqrt{\dfrac 1x}$的结果是\bracket{20}.
    \fourch{$\sqrt x$}{$x(1-{x^2})\sqrt x$}{${x^2}(1-x\sqrt x)$}{0}
    \item 化简$\dfrac{{a^{-2}}-{b^{-2}}}{{a^2}-{b^2}}$的结果是\bracket{20}.
    \fourch{-1}{$-\dfrac 1{{a^2}{b^2}}$}{${a^{-1}}+{b^{-1}}$}{$\dfrac 1{{a^2}{b^2}}$}
    \item 已知$x=1-2^s$, $y=1-{2^{-s}}$, 则$y$等于\bracket{20}.
    \fourch{$\dfrac{x-1}x$}{$\dfrac{2-x}{1-x}$}{$\dfrac x{x-1}$}{$\dfrac{x-2}{x-1}$}
    \item 计算$\sqrt{(3-\pi)^2}$的结果是\bracket{20}.
    \fourch{$3-\pi$}{$\pi -3$}{$\pi +3$}{$-\pi -3$}
    \item 若$(\sqrt[n]{-3})^n$有意义, 则$n$一定是\bracket{20}.
    \fourch{正偶数}{自然数}{正奇数}{整数}
    \item 在``\textcircled{1} $\sqrt[4]{(-4)^{2n}}$, \textcircled{2} $\sqrt[4]{(-4)^{2n+1}}$, \textcircled{3} $\sqrt[5]{-{x^2}}$, \textcircled{4} $\sqrt[5]{-{x^2}}(n\in \mathbf{N})$''这四个式子中, 有意义的\bracket{20}.
    \fourch{是\textcircled{1} \textcircled{2} \textcircled{3} \textcircled{4} }{只有\textcircled{3} \textcircled{4} }{只有\textcircled{1} \textcircled{3} \textcircled{4} }{只有\textcircled{4} }
    \item 若$\sqrt[4]{4{a^2}-4a+1}=\sqrt[3]{1-2a}$, 则实数$a$的取值范围是\bracket{20}.
    \fourch{$a<2$}{$a=\dfrac 12$或0}{$a>\dfrac 12$}{$R$}
    \item 在``\textcircled{1} ${0^{-1}}$, \textcircled{2} ${0^{-\dfrac 12}}$, \textcircled{3} $0^0$, \textcircled{4} ${0^{0.2}}$''这四个式子中, 有意义的个数是\bracket{20}.
    \fourch{0}{1}{2}{3}
    \item 下列各式中正确的是\bracket{20}.
    \fourch{$-4^0=1$}{$({5^{-\dfrac 12}})^2=5$}{$(-{3^{m-n}})^2={9^{m-n}}$}{$(-2)^{-1}=\dfrac 12$}
    \item 计算${{[ (-3)^2 ]}^{\dfrac 12}}-(-10)^0$的值等于\bracket{20}.
    \fourch{-2}{2}{-4}{4}
    \item 下列计算中正确的是\bracket{20}.
    \fourch{${a^{\dfrac 83}}\cdot {a^{\dfrac 38}}=a$}{${a^{\dfrac 83}}\cdot {a^{-\dfrac 83}}=0$}{${a^{\dfrac 83}}\div {a^{\dfrac 13}}={a^8}$}{${a^{\dfrac 12}}\div {a^{\dfrac 13}}={a^{\dfrac 16}}$}
    \item 下列计算中正确的是\bracket{20}.
    \fourch{${a^{\dfrac 34}}\cdot {a^{\dfrac 43}}=a$}{${a^{\dfrac 34}}\div {a^{\dfrac 34}}=a$}{${a^{-4}}\div a^4=0$}{${{({a^{\dfrac 34}})}^{\dfrac 43}}=a$}
    \item 化简$({a^{\dfrac 23}}{b^{\dfrac 12}})(-3{a^{\dfrac 12}}{b^{\dfrac 13}})\div (\dfrac 13{a^{\dfrac 16}}{b^{\dfrac 56}})$的结果是\bracket{20}.
    \fourch{$6a$}{$-a$}{$-9a$}{$9a$}
    \item 将$\sqrt[3]{-2\sqrt 2}$化成不含根号的式子是\bracket{20}.
    \fourch{$-{2^{\dfrac 12}}$}{$-{2^{-\dfrac 12}}$}{$-{2^{\dfrac 13}}$}{$-{2^{\dfrac 23}}$}
    \item 将${{({a^{\dfrac 1n}}+{b^{\dfrac 1n}})}^{\dfrac 13}}$表示成根式的形式是\bracket{20}.
    \fourch{$\sqrt[3]{{a^{\dfrac 1n}}+{b^{\dfrac 1n}}}$}{${{(\sqrt[n]a+\sqrt[n]b)}^{\dfrac 13}}$}{$\sqrt[3]{\sqrt[n]a+\sqrt[n]b}$}{$(\sqrt[n]a+\sqrt[n]b)^3$}
    \item 计算下列各式:
    (1)$\sqrt{12}-\sqrt 3\div (2+\sqrt 3)=$\blank{50}.
    (2)$(\sqrt{12}-\sqrt{\dfrac 12}-2\sqrt{\dfrac 13})-(\sqrt{\dfrac 18}-\sqrt{18})=$\blank{50}.
    (3)$(\sqrt 3+2)^{1997}\times (\sqrt 3-2)^{1988}=$\blank{50}.
    (4)$\dfrac{2\sqrt{10}-5}{4-\sqrt 10}=$\blank{50}.
    (5)$4\sqrt{\dfrac 25}-\sqrt{1000}+2\sqrt{10}=$\blank{50}.
    (6)$\dfrac 1{(2+\sqrt 3)^2}+\dfrac 1{(2-\sqrt 3)^2}=$\blank{50}.
    (7)$\dfrac 1{1+\sqrt 2+\sqrt 3}+\dfrac 1{1-\sqrt 2+\sqrt 3}=$\blank{50}.
    \item 将下列各式改写成不含分数指数幂的根式形式(要求分母不含有根式形式):
    (1)$3{x^{-\dfrac 32}}=$\blank{50}.				(2)${a^{\dfrac 12}}\cdot {b^{-\dfrac 12}}=$\blank{50}.
    (3)${{(a+b)}^{\dfrac 12}}\cdot {{(a-b)}^{-\dfrac 43}}=$\blank{50}.
    \item 将下列根式改写成分数指数幂的形式:
    (1)$\sqrt[4]{a^3}=$\blank{50}.			(2)$\sqrt[5]{b^8}=$\blank{50}.
    (3)$\sqrt[4]{{x^2}+{y^2}}=$\blank{50}.		(4)$\dfrac{\sqrt x}{\sqrt[3]{y^4}}=$\blank{50}.
    (5)$\sqrt{2\sqrt 2}=$\blank{50}.			(6)$-\dfrac 1{\sqrt{27x}}=$\blank{50}.
    (7)$\sqrt{\dfrac 4{3a{b^3}}}=$\blank{50}.			(8)$2\sqrt[6]{(m-n)^{-2}}=$    $(m<n)$.
    \item 判断下列命题是否正确:
    (1)${2^{\dfrac 32}}\cdot {2^{\dfrac 23}}=2$:\blank{50}.
    (2)${{(\dfrac 18)}^{-\dfrac 12}}=-2\sqrt 2$:\blank{50}.
    (3)若$a\in \mathbf{R}$, 则$(a-1)^0=1$:\blank{50}.
    (4)$a^x+a^y={a^{x+y}}$:\blank{50}.
    (5)$\sqrt[3]{-5}=\sqrt[6]{(-5)^2}=\sqrt[6]{25}$:\blank{50}.
    \item 计算下列各式:
    (1)${{(\dfrac{81}{625})}^{-\dfrac 34}}=$\blank{50}.
    (2)${{(0.064)}^{-\dfrac 13}}=$\blank{50}.
    (3)${{(2\sqrt 2)}^{-\dfrac 13}}=$\blank{50}.
    (4)${{[ (-3)^2 ]}^{\dfrac 32}}=$\blank{50}.
    (5)${{(-0.027)}^{-\dfrac 23}}=$\blank{50}.
    (6)${{(-0.001)}^{-\dfrac 43}}=$\blank{50}.
    (7)${5^{\dfrac 45}}\times 125\times {{25}^{-0.4}}=$\blank{50}.
    (8)${{(8+2\times {{15}^{\dfrac 12}})}^{\dfrac 12}}=$\blank{50}.
    (9)${{(4-{{12}^{\dfrac 12}})}^{\dfrac 12}}=$\blank{50}.
    (10)${{(0.25)}^{-0.5}}+{{(\dfrac 1{27})}^{-\dfrac 13}}-{{625}^{0.25}}=$\blank{50}.
    \item 化简下列各式:
    (1)$2{x^{-\dfrac 13}}(\dfrac 12{x^{\dfrac 13}}-2{x^{-\dfrac 23}})-(-3.5)^0=$\blank{50}.
    (2)$({x^{\dfrac 13}}+{y^{\dfrac 13}})({x^{\dfrac 23}}-{x^{\dfrac 13}}{y^{\dfrac 13}}+{y^{\dfrac 23}})=$\blank{50}.
    (3)$(\dfrac{b^3}{2{a^2}})\div (-\dfrac{4{b^3}}{{a^{-7}}})\times (-\dfrac{b^2}a)^3=$\blank{50}.
    (4)$(2{a^{\dfrac 14}}{b^{-\dfrac 13}})(-3{a^{-\dfrac 12}}{b^{\dfrac 23}})\div (-\dfrac 14{a^{-\dfrac 14}}{b^{-\dfrac 23}})=$\blank{50}.
    \item 若$a={{1.5}^{-\dfrac 12}}$, $b={{0.5}^{-\dfrac 12}}$, $c=1$, 则它们的大小顺序是\bracket{20}.
    \fourch{$a<c<b$}{$a<b<c$}{$c<b<a$}{$b<c<a$}
    \item (1)若$a=\dfrac 1{\sqrt 2}$, $b=\dfrac 1{\sqrt[3]2}$, 则${{[ {a^{-\dfrac 32}}b{{(a{b^{-2}})}^{-\dfrac 12}}{{({a^{-1}})}^{-\dfrac 23}} ]}^3}=$\blank{50}.
    (2)若${a^{\dfrac 12}}+{a^{-\dfrac 12}}=2$, 则: \textcircled{1} $a+{a^{-1}}=$    ; \textcircled{2} $a^2+{a^{-2}}=$    ; \textcircled{3} $a^4+{a^{-4}}=$\blank{50}.
    (3)若${{10}^{\alpha }}={2^{-\dfrac 12}}$, ${{10}^{\beta }}=\sqrt[3]{32}$, 则${{10}^{2\alpha -\dfrac 34\beta }}=$\blank{50}.
    \item 计算下列各式:
    (1)$(\dfrac 1{125})}^{-\dfrac 13}}+{{(-2)^{-2}+(-2)^0$.
    (2)$(2\dfrac 79)}^{\dfrac 12}}-{{(-0.027)}^{-\dfrac 13}}-{{(-\sqrt 3)^{-2}+{{\pi }^0}$.
    (3)$5-3\times [ {{(-3\dfrac 38)}^{-\dfrac 13}}+1031\times (0.25-{2^{-2}}) ]\div {9^0}$.
    (4)$(0.027)}^{\dfrac 13}}-{{(-\dfrac 16)^{-2}+{{256}^{0.75}}-|-{3^{-1}}|+(-5.555)^0$.
    (5)$(2.25)}^{0.5}}+{{(-4.3)^0-{{(3\dfrac 38)}^{-\dfrac 23}}+\dfrac{{3^{-2}}-{2^{-2}}}{{3^{-1}}-{2^{-1}}}$.
    (6)$(0.25)^{-2}+{{(\dfrac 8{27})}^{\dfrac 13}}+{{(\dfrac 18)}^{-\dfrac 23}}-{{(\dfrac 1{16})}^{-0.75}}$.
    \item 计算或化简下列各式:
    (1)$\sqrt[3]{{m^{\dfrac 92}}\cdot \sqrt{{m^{-3}}}}\div \sqrt{\sqrt[3]{{m^{-7}}}}\cdot \sqrt[3]{{m^{13}}}(m>0)$.
    (2)$(x-y)\div ({x^{\dfrac 12}}+{y^{\dfrac 12}})-(x+y-2{x^{\dfrac 12}}{y^{\dfrac 12}})\div ({x^{\dfrac 12}}-{y^{\dfrac 12}})(x>y>0)$.
    (3)${{(8{y^{-\dfrac 13}}\sqrt{{x^{-\dfrac 13}}y\sqrt{{x^{\dfrac 43}}}})}^{\dfrac 13}}$.
    (4)$\dfrac{x+y}{\sqrt x+\sqrt y}+\dfrac{2xy}{x\sqrt y+y\sqrt x}$.
    (5)$(5+\sqrt 6+\sqrt{10}+\sqrt{15})\div (\sqrt 2+\sqrt 3+\sqrt 5)$.
    (6)${{(2+{3^{\dfrac 12}})}^{\dfrac 12}}\times {{[ 2+{{(2+{3^{\dfrac 12}})}^{\dfrac 12}} ]}^{\dfrac 12}}\times {{\{2+{{[ 2+{{(2+{3^{\dfrac 12}})}^{\dfrac 12}} ]}^{\dfrac 12}}\}}^}^{\dfrac 12}$
    $\times {{\{2-{{[ 2+{{(2+{3^{\dfrac 12}})}^{\dfrac 12}} ]}^{\dfrac 12}}\}}^{\dfrac 12}}$.
    \item 化简下列各式:
    (1)$\sqrt{x+2\sqrt{x-1}}+\sqrt{x-2\sqrt{x-1}}$.
    (2)${{({x^{\dfrac{a+b}{c-a}}})}^{\dfrac 1{b-c}}}\cdot {{({x^{\dfrac{x+a}{b-c}}})}^{\dfrac 1{a-b}}}\cdot {{({x^{\dfrac{b+c}{a-b}}})}^{\dfrac 1{c-a}}}$.
    (3)$\dfrac{{a^2}-{b^2}}{{a^2}+{b^2}}{{(\dfrac{a-b}{a+b})}^{\dfrac{p+q}{p-q}}}\cdot [ {{(\dfrac{a+b}{a-b})}^{\dfrac{2p}{p-q}}}+{{(\dfrac{a+b}{a-b})}^{\dfrac{2q}{p-q}}} ]$.
    \item 当$a=0.001$时, 求$\dfrac{{a^{\dfrac 43}}-8{a^{\dfrac 13}}b}{{a^{\dfrac 23}}+2\sqrt[3]{ab}+4{b^{\dfrac 23}}}\div (1-2\sqrt[3]{\dfrac ba})$的值.
    \item 求证: $\dfrac 1{1+{x^{a-b}}+{x^{a-c}}}+\dfrac 1{1+{x^{b-c}}+{x^{b-a}}}+\dfrac 1{1+{x^{c-a}}+{x^{c-b}}}=1$.
    (二)幂函数
    \item 已知幂函数$f(x)$的图象经过点$(2,\dfrac{\sqrt 2}2)$, 则$f(4)$的值等于\bracket{20}.
    \fourch{16}{$\dfrac 1{16}$}{$\dfrac 12$}{2}
    \item 下列幂函数中, 定义域为$\{x|x>0\}$的是\bracket{20}.
    \fourch{$y={x^{\dfrac 23}}$}{$y={x^{\dfrac 32}}$}{$y={x^{-\dfrac 23}}$}{$y={x^{-\dfrac 32}}$}
    \item 幂函数$y=x^n(n\in \mathbf{Z})$的图象一定不经过\bracket{20}.
    \fourch{第一象限}{第二象限}{第三象限}{第四象限}
    *71.函数$f(x)={x^{\dfrac 23}}$的图象是\bracket{20}.
    \item 幂函数$y=x^m$和$y=x^n$在第一象限内的图象$C_1$和$C_2$图象所示, 则$m,n$之间的关系是\bracket{20}.
    \fourch{$n<m<0$}{$m<n<0$}{$n>m>0$}{$m>n>0$}
    *73.图中, $C_1,C_2,C_3$为幂函数$y=x^a$在第一象限的图象, 则解析式中的指数$\alpha$依次可以取\bracket{20}.
    \fourch{$\dfrac 43,-2,\dfrac 34$}{$-2,\dfrac 34,\dfrac 43$}{$-2,\dfrac 43,\dfrac 34$}{$\dfrac 34,\dfrac 43,-2$}
    *74.求下列函数的定义域与值域:
    (1)$y={x^{\dfrac 56}}x\in$\blank{50}, $u\in$\blank{50}.
    (2)$y={x^{\dfrac 35}}x\in$\blank{50}, $u\in$\blank{50}.
    (3)$y={x^{\dfrac 85}}x\in$\blank{50}, $u\in$\blank{50}.
    (4)$y={x^{-\dfrac 54}}x\in$\blank{50}, $u\in$\blank{50}.
    (5)$y={x^{-\dfrac 53}}x\in$\blank{50}, $u\in$\blank{50}.
    (6)$y={x^{-\dfrac 23}}x\in$\blank{50}, $u\in$\blank{50}.
    (7)$y=-2{{(x+5)}^{-\dfrac 14}}x\in$\blank{50}, $u\in$\blank{50}.
    (8)$y=5{{(2x-1)}^{\dfrac 34}}x\in$\blank{50}, $u\in$\blank{50}.
    \item 将下列函数图象的标号, 填在相应函数后面的横线上:
    *(1)$y={x^{\dfrac 23}}$:\blank{50}.	(2)$y={x^{-2}}$:\blank{50}.
    (3)$y={x^{\dfrac 12}}$:\blank{50}.				(4)$y={x^{-1}}$:\blank{50}.
    (5)$y={x^{\dfrac 13}}$:\blank{50}.				*(6)$y={x^{\dfrac 32}}$:\blank{50}.
    *(7)$y={x^{\dfrac 43}}$:\blank{50}.			(8)$y={x^{-\dfrac 12}}$:\blank{50}.
    *(9)$y={x^{\dfrac 53}}$:\blank{50}.
    \fourch{}{}{}{(E) (F)}
    (G) (H) (I)
    (第75题)
    \item (1)若幂函数$y=x^n$的图象在$0<x<1$时位于直线$y=x$的下方, 则$n$的取值范围是\blank{50}.
    (2)若幂函数$y=x^n$的图象在$0<x<1$时位于直线$y=x$的上方, 则$n$的取值范围是\blank{50}.
    *(3)函数$f(x)={x^{k^2-2k-3}}$($k\in \mathbf{Z}$)的图象如图所示, 则$k=$\blank{50}.
    (第76(3)题)
    \item 幂函数$y=x^p$与$y=x^q$的图象都通过定点\blank{50}, 它们在第一象限部分关于直线$y=x$对称, 则$p,q$应满足的条件是\blank{50}.
    \item 确定实数$a$的取值范围:
    (1)${{2.4}^a}>{{2.5}^a}.$				(2)$(\dfrac 34)^{-a}>(\dfrac 43)^{-a}.$
    (3)${a^{-2}}>{3^{-2}}.$				(4)${{0.01}^{-3}}>{a^{-3}}.$
    \item 将下列各组数从小到大排列:
    (1)${{2.5}^{\dfrac 23}}$, ${{(-1.4)}^{\dfrac 23}}$, ${{(-3)}^{\dfrac 13}}$:\blank{50}.
    (2)${{4.1}^{\dfrac 25}}$, ${{3.8}^{-\dfrac 23}}$, ${{(-1.9)}^{\dfrac 35}}$:\blank{50}.
    (3)${{0.16}^{-\dfrac 34}}$, ${{0.5}^{-\dfrac 32}}$, ${{6.25}^{\dfrac 38}}$:\blank{50}.
    \item 已知函数$y={x^{n^2-2n-3}}$($n\in \mathbf{Z}$)的图象与两坐标轴都无公共点, 且其图象关于$y$轴对称, 求$n$的值, 并画出相应的函数图象.
    (三)函数的单调性
    \item 函数$y=\sqrt{{x^2}+2x-3}$为减函数的区间是()
    \fourch{$(-\infty ,-3 ].$}{$[ -1,+\infty).$}{$(-\infty ,-1 ].$}{$[ 1,+\infty).$}
    \item 若函数$y=(2k+1)x+b$在$(-\infty ,+\infty)$上是减函数, 则()
    \fourch{$k>\dfrac 12.$}{$k<\dfrac 12.$}{$k>-\dfrac 12.$}{$k<-\dfrac 12.$}
    \item 若函数$f(x)=4x^2-mx+5$在区间$[ -2,+\infty)$上是增函数, 在区间$(-\infty ,-2 ]$上是减函数, 则$f(1)$等于()
    \fourch{$-7.$}{1}{17}{25}
    \item 若函数$y=x^2+2(a-2)x+5$在区间$(4,+\infty)$上是增函数, 则实数$a$的取值范围是()
    \fourch{$a\le -2.$}{$a\ge -2.$}{$a\le -6.$}{$a\ge -6.$}
    \item 下列函数中, 在区间$(0,2)$上为增函数的是()
    \fourch{$y=-3x+1.$}{$y=\sqrt[3]x.$}{$y=x^2-4x+3.$}{$y=\dfrac 4x.$}
    \item 若函数$f(x)$在定义域$R$上为增函数, 且$f(x)<0$, 则下列函数在$R$上为增函数的是()
    \fourch{$y=|f(x)|.$}{$y=\dfrac 1{f(x)}$}{$y={{[ f(x) ]}^2}.$}{$y={{[ f(x) ]}^3}.$}
    \item (1)函数$y=\dfrac 1{\sqrt{{x^2}-4x+5}}$为增函数的区间是\blank{50}, 为减函数的区间是\blank{50}.
    (2)函数$y=\dfrac 1{\sqrt{3+2x-{x^2}}}$为增函数的区间是\blank{50}.
    (3)函数$y=|3x-5|$为减函数的区间是\blank{50}.
    (4)函数$y=|x^2-2x-3|$为增函数的区间是\blank{50}.
    (5)函数$y=\dfrac{1-x}{1+x}$为减函数的区间是\blank{50}.
    \item 定义在[1, 3]上的函数$f(x)$为减函数, 求满足不等式$f(1-a)-f(3-a^2)>0$的解集.
    \item (1)已知$f(x)=-x^3-x+1$($x\in \mathbf{R}$), 求证$y=f(x)$在定义域上为减函数.
    (2)求证: 函数$f(x)=x+\dfrac 1x$在(0, 1)上是减函数, 在$(1,+\infty)$上是增函数.
    (3)求证: $f(x)=\sqrt x-\dfrac 1x$在定义域上是增函数.
    (4)已知常数$m,n$满足$mn<2$, 求证: 函数$f(x)=\dfrac{mx+1}{2x+n}$在$(-\dfrac n2,+\infty)$上为减函数.
    \item 已知$f(x)=x^2+1$, $g(x)=x^4+2x^2+2$, 是否存在实数$\lambda$, 使得$F(x)=g(x)-\lambda f(x)$在$(-\infty ,-1)$上是减函数, 在(-1, 0)上是增函数?
    \item 已知函数$f(x)$在区间$(-\infty ,+\infty)$上是增函数, 又实数$a,b$满足$a+b\ge 0$, 求证: $f(a)+f(b)\ge f(-a)+f(-b)$.
    \item $f(x)$是定义在$\mathbf{R}^+$的增函数, 且$f(\dfrac xy)=f(x)-f(y)$.
    (1)求$f(1)$的值.
    (2)若$f(6)=1$, 解不等式$f(x+3)-f(\dfrac 1x)<2$.
    (四)函数的奇偶性
    \item 若$f(x)=(m-1)x^2+3mx+3$为偶函数, 则$f(x)$在区间(-4, 2)上()
    \fourch{是增函数}{是减函数}{先是增函数后是减函数}{先是减函数后是增函数}
    \item 函数$f(x)=\begin{array}{*{35}l}
       1-x  (x>0),  \\0  (x=0),  \\1+x  (x<0),  \end{array}$则该函数()
    \fourch{是奇函数, 但不是偶函数}{是偶函数, 但不是奇函数}{既是奇函数, 也是偶函数}{既不是奇函数, 也不是偶函数}
    \item 下列函数中既是奇函数, 又在定义域上为增函数的是()
    \fourch{$f(x)=3x+1.$}{$f(x)=\dfrac 1x.$}{$f(x)=1-\dfrac 1x.$}{$f(x)=x^3.$}
    \item 若$f(x)$为定义在区间[-6, 6]上的偶函数, 且满足$f(3)>f(1)$, 则恒成立的是()
    \fourch{$f(-1)<f(3).$}{$f(0)<f(6).$}{$f(3)>f(2).$}{$f(2)>f(0).$}
    \item 函数$f(x)=\dfrac{\sqrt{1-{x^2}}}{2-|x+2|}$()
    \fourch{是奇函数, 但不是偶函数}{是偶函数, 但不是奇函数}{既是奇函数, 又是偶函数}{既不是奇函数, 也不是偶函数}
    \item 已知$f(x)$是奇函数, 则下列各点中在函数$y=f(x)$的图象上的点的是()
    \fourch{$(a,f(-a)).$}{$(-a,-f(a)).$}{$(\dfrac 1a,-f(\dfrac 1a)).$}{$(-\sin a,-f(-\sin a)).$}
    \item (1)若$f(x)$是定义在$R$上的偶函数, 且当$x<0$时, $f(x)=2x-3$, 则当$x>0$时, $f(x)=$\blank{50}.
    (2)若奇函数$f(x)$的定义域是$R$, 则$f(0)=$\blank{50}.
    \item (1)若奇函数$f(x)$在区间[-3, -1]上是增函数, 且有最大值-2, 则$f(x)$在[1, 3]上是\blank{50}函数(填``增''或``减''), 且最小值等于\blank{50}.
    (2)设$f(x)$为定义在$R$上的偶函数, 且$f(x)$在$[ 0,+\infty)$上是增函数, 则$f(-4)$, $f(-2)$, $f(3)$由小到大的排列顺序为\blank{50}.
    
    
\end{enumerate}
\end{document}