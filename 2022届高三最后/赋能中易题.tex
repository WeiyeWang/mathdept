\documentclass[10pt,a4paper]{article}
\usepackage[UTF8,fontset = windows]{ctex}
\setCJKmainfont[BoldFont=黑体,ItalicFont=楷体]{华文中宋}
\usepackage{amssymb,amsmath,amsfonts,amsthm,mathrsfs,dsfont,graphicx}
\usepackage{ifthen,indentfirst,enumerate,color,titletoc}
\usepackage{tikz}
\usetikzlibrary{arrows,calc}
\usepackage[bf,small,indentafter,pagestyles]{titlesec}
\usepackage[top=1in, bottom=1in,left=0.8in,right=0.8in]{geometry}
\renewcommand{\baselinestretch}{1.65}
\newtheorem{defi}{定义~}
\newtheorem{eg}{例~}
\newtheorem{ex}{~}
\newtheorem{rem}{注~}
\newtheorem{thm}{定理~}
\newtheorem{coro}{推论~}
\newtheorem{axiom}{公理~}
\newtheorem{prop}{性质~}

\newcommand{\blank}[1]{\underline{\hbox to #1pt{}}}
\newcommand{\bracket}[1]{(\hbox to #1pt{})}

\begin{document}

\begin{center}
    赋能正确率介于$0.85$至$0.9$的题目
\end{center}

{\tiny 3,5,0.884} 已知$(a+3b)^n$的展开式中, 各项系数的和与各项二项式系数的和之比为$64$, 则$n=$\blank{50}.

{\tiny 3,9,0.884} 如图, 在$\triangle ABC$中, $\angle B=45^\circ$, $D$是$BC$边上的一点, $AD=5$, $AC=7$, $DC=3$, 则$AB$的长为\blank{50}.
\begin{center}
    \begin{tikzpicture}[scale = 0.5]
        \draw  (-6.830127018922193,0.)-- (3.,0.) node [below right] {$C$};
        \draw  (3.,0.)-- (-2.5,4.330127018922193) node [above] {$A$};
        \draw  (-2.5,4.330127018922193)-- (-6.830127018922193,0.) node [below left] {$B$};
        \draw  (-2.5,4.330127018922193)-- (0.,0.) node [below] {$D$};
    \end{tikzpicture}
\end{center}

{\tiny 8,10,0.886} 已知点$A$是圆$O: x^2+y^2=4$上的一个定点, 点$B$是圆$O$上的一个动点, 若满足$|\overrightarrow{AO}+\overrightarrow{BO}|=|\overrightarrow{AO}-\overrightarrow{BO}|$, 则$\overrightarrow{AO}\cdot \overrightarrow{AB}=$\blank{50}.

% 赋能9


{\tiny 9,8,0.872} 集合$\{x|\cos (\pi \cos x)=0,x\in [0,\pi]\}=$\blank{50}(用列举法表示).

{\tiny 9,10,0.897} 已知$x$、$y$满足曲线方程$x^2+\dfrac1{y^2}=2$, 则$x^2+y^2$的取值范围是\blank{50}.

% 赋能10


{\tiny 12,6,0.886} 已知$\alpha$为锐角, 且$\cos (\alpha +\dfrac\pi 4)=\dfrac35$, 则$\sin \alpha =$\blank{50}.

{\tiny 12,7,0.886} 已知正四棱柱$ABCD-A_1B_1C_1D_1$, $AB=a$, $AA_1=2a$, $E$、$F$分别是棱$AD$、$CD$的中点,则异面直线$BC_1$与$EF$所成角是\blank{50}.

{\tiny 13,7,0.886} 如果实数$x$、$y$满足$\begin{cases} 2x-y\le 0, \\ x+y\le 3, \\  x\ge 0, \\ \end{cases}$, 则$2x+y$的最大值是\blank{50}.

{\tiny 13,9,0.886} 方程$x^2+y^2-4tx-2ty+3t^2-4=0$($t$为参数)所表示的圆的圆心轨迹方程是\blank{50}(结果化为普通方程).

{\tiny 15,8,0.884} 将一个正方形绕着它的一边所在的直线旋转一周, 所得几何体的体积为$27\pi\text{cm}^3$, 则该几何体的侧面积为\blank{50}$\text{cm}^3$.

{\tiny 17,5,0.884} 已知复数$z=a+b\mathrm{i}(a,b\in \mathbf{R})$满足$|z|=1$, 则$a\cdot b$范围是\blank{50}.

{\tiny 17,6,0.860} 某学生要从物理、化学、生物、政治、历史、地理这六门学科中选三门参加等级考, 要求是物理、化学、生物这三门至少要选一门, 政治、历史、地理这三门也至少要选一门, 则该生的可能选法总数是\blank{50}.

{\tiny 20,9,0.884} 已知圆锥的轴截面是等腰直角三角形, 该圆锥的体积为$\dfrac83\pi$, 则该圆锥的侧面积等于\blank{50}.

{\tiny 21,5,0.886} 已知直线$l$的一个法向量是$\overrightarrow{n}=(\sqrt3,-1)$, 则$l$的倾斜角的大小是\blank{50}.

{\tiny 22,7,0.857} 已知$\mathrm{i}$是虚数单位, $\overline z$是复数$z$的共轭复数, 若$\begin{vmatrix} z & 1+\mathrm{i}  \\ 1 & 2\mathrm{i} \end{vmatrix}=0$, 则$\overline z$在复平面内所对应的点所在的象限为第\blank{50}象限.

{\tiny 22,9,0.881} 若直线$l:x+y=5$与曲线$C:x^2+y^2=16$交于两点$A(x_1,y_1)$, $B(x_2,y_2)$, 则$x_1y_2+x_2y_1$的值为\blank{50}.

{\tiny 23,6,0.886} 若存在$x\in [0,+\infty)$使$\begin{vmatrix}2^x & 2^x \\ m & x \end{vmatrix}<1$成立, 则实数$m$的取值范围是\blank{50}.

{\tiny 24,3,0.864} 不等式$2^{x^2-4x-3}>(\dfrac12 )^{3(x-1)}$的解集为\blank{50}.

{\tiny 25,9,0.884} 著名的斐波那契数列$\{a_n\}:1,1,2,3,5,8,\cdots$, 满足$a_1=a_2=1,a_{n+2}=a_{n+1}+a_n \ (n\in \mathbf{N}^*)$, 那么$1+a_3+a_5+a_7+a_9+\cdots+a_{2017}$是斐波那契数列中的第\blank{50}项.

{\tiny 28,2,0.884} 参数方程为$\begin{cases} x=t^2, \\ y=2t, \end{cases}$ ($t$为参数)的曲线的焦点坐标为\blank{50}.

{\tiny 30,7,0.860} 若函数$f(x)=2^x(x+a)-1$在区间$[0,1]$上有零点, 则实数$a$的取值范围是\blank{50}.

{\tiny 31,5,0.884} 已知正四棱锥的底面边长是$2$, 侧棱长是$\sqrt3$, 则该正四棱锥的体积为\blank{50}.

{\tiny 34,9,0.884} 设$a>0$, 若对于任意的$x>0$, 都有$\dfrac1a-\dfrac1x\le 2x$, 则$a$的取值范围是\blank{50}.

{\tiny 35,9,0.884} 已知等差数列$\{a_n\}$的公差为$2$, 前$n$项和为$S_n$, 则$\displaystyle\lim_{n\to\infty}\dfrac{S_n}{{a_n}{a_{n+1}}}$=\blank{50}.

{\tiny 37,4,0.884} 已知双曲线$\dfrac{x^2}{a^2}-\dfrac{y^2}{(a+3)^2}=1 \ (a>0)$的一条渐近线方程为$y=\pm 2x$, 则$a=$\blank{50}.

{\tiny 38,10,0.884} 设$A$是椭圆$\dfrac{x^2}{a^2}+\dfrac{y^2}{a^2-4}=1 \ (a>0)$上的动点, 点$F$的坐标为$(-2,0)$, 若满足$|AF|=10$的点$A$有且仅有两个, 则实数$a$的取值范围为\blank{50}.

% 赋能39


{\tiny 39,7,0.884} 在报名的$8$名男生和$5$名女生中, 选取$6$人参加志愿者活动, 要求男、女生都有, 则不同的选取方式的种数为\blank{50}(结果用数值表示).

{\tiny 40,4,0.860} 若$\begin{vmatrix} \log_2 x & -1  \\ -4 & 2  \end{vmatrix}=0$, 则$x=$\blank{50}.

{\tiny 40,7,0.860} 若二项式$(2x+\dfrac ax)^7$的展开式中一次项的系数是$-70$, 则$\displaystyle\lim_{n\to\infty}(a+a^2+a^3+\cdots+a^n)=$\blank{50}.

{\tiny 42,3,0.860} 函数$f(x)=\lg (3^x-2^x)$的定义域为\blank{50}.

{\tiny 42,9,0.860} 将两颗质地均匀的骰子抛掷一次, 记第一颗骰子出现的点数是$m$, 记第二颗骰子出现的点数是$n$, 向量$\overrightarrow a=(m-2,2-n)$, 向量$\overrightarrow b=(1,1)$, 则向量$\overrightarrow a\perp \overrightarrow b$的概率是\blank{50}.

{\tiny 44,1,0.860} 已知集合$A=\{1,2,3\}B=\{1,m\}$, 若$3-m\in A$, 则非零实数$m$的数值是\blank{50}.

{\tiny 46,8,0.884} 已知抛物线的顶点在坐标原点, 焦点在$y$轴上, 抛物线上一点$M(a,-4) \ (a>0)$到焦点F的距离为$5$. 则该抛物线的标准方程为\blank{50}.

{\tiny 48,10,0.860} 一个四面体的顶点在空间直角坐标系$O-xyz$中的坐标分别是$(0,0,0)$, $(1,0,1)$, $(0,1,1)$, $(1,1,0)$, 则该四面体的体积为\blank{50}.

% 赋能49


{\tiny 49,1,0.884} 抛物线$x^2=12y$的准线方程为\blank{50}.

{\tiny 49,3,0.860} 若函数$f(x)=\sqrt{2x+3}$的反函数为$g(x)$, 则函数$g(x)$的零点为\blank{50}.

{\tiny 49,8,0.860} 在平面直角坐标系$xOy$中, 直线$l$的参数方程为$\begin{cases} x=\dfrac{\sqrt2}2t-\sqrt2, \\ y=\dfrac{\sqrt2}4t, \end{cases}$($t$为参数), 椭圆$C$的参数方程为$\begin{cases} x=\cos \theta,  \\ y=\dfrac12\sin \theta,  \end{cases}$($\theta$为参数), 则直线$l$与椭圆$C$的公共点坐标为\blank{50}.

{\tiny 50,10,0.884} 已知曲线$C:y=-\sqrt{9-x^2}$, 直线$l:y=2$, 若对于点$A(0,m)$, 存在$C$上的点$P$和$l$上的点$Q$, 使得$\overrightarrow{AP}+\overrightarrow{AQ}=\overrightarrow 0$, 则$m$取值范围是\blank{50}.

% 赋能51


{\tiny 52,2,0.884} $(x+\dfrac1x)^n$的展开式中的第$3$项为常数项, 则正整数$n=$\blank{50}.

{\tiny 53,7,0.860} 在$\triangle ABC$中, 边$a,b,c$所对角分别为$A,B,C$, 若$\begin{vmatrix} a & \sin (\dfrac{\pi}2+B)  \\ b & \cos A  \end{vmatrix}=0$, 则$\triangle ABC$的形状为\blank{50}.

{\tiny 55,9,0.860} 已知双曲线$x^2-\dfrac{y^2}4=1$的右焦点为$F$, 过点$F$且平行于双曲线的一条渐近线的直线与双曲线交于点$P$, $M$在直线$PF$上, 且满足$\overrightarrow{OM}\cdot \overrightarrow{PF}=0$, 则$\dfrac{|\overrightarrow{PM}|}{|\overrightarrow{PF}|}=$\blank{50}.

{\tiny 56,10,0.860} 已知实数$x,y$满足$\begin{cases}x+y\ge 2, \\ x-y\le 2, \\ 0 \le y\le 3, \end{cases}$ 则目标函数$z=-\dfrac32x-y$的最大值为\blank{50}.

{\tiny 56,12,0.860} 从集合$A=\{1,2,3,4,5,6,7,8,9,10\}$中任取两个数, 欲使取到的一个数大于$k$, 另一个数小于$k$(其中$k\in A$)的概率是$\dfrac25$, 则$k=$\blank{50}.

% 赋能57







\end{document}