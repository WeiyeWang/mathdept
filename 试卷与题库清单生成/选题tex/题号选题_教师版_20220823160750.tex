\documentclass[10pt,a4paper]{article}
\usepackage[UTF8,fontset = windows]{ctex}
\setCJKmainfont[BoldFont=黑体,ItalicFont=楷体]{华文中宋}
\usepackage{amssymb,amsmath,amsfonts,amsthm,mathrsfs,dsfont,graphicx}
\usepackage{ifthen,indentfirst,enumerate,color,titletoc}
\usepackage{tikz}
\usepackage{multicol}
\usepackage{makecell}
\usepackage{longtable}
\usetikzlibrary{arrows,calc,intersections,patterns,decorations.pathreplacing,3d,angles,quotes,positioning}
\usepackage[bf,small,indentafter,pagestyles]{titlesec}
\usepackage[top=1in, bottom=1in,left=0.8in,right=0.8in]{geometry}
\renewcommand{\baselinestretch}{1.65}
\newtheorem{defi}{定义~}
\newtheorem{eg}{例~}
\newtheorem{ex}{~}
\newtheorem{rem}{注~}
\newtheorem{thm}{定理~}
\newtheorem{coro}{推论~}
\newtheorem{axiom}{公理~}
\newtheorem{prop}{性质~}
\newcommand{\blank}[1]{\underline{\hbox to #1pt{}}}
\newcommand{\bracket}[1]{(\hbox to #1pt{})}
\newcommand{\onech}[4]{\par\begin{tabular}{p{.9\textwidth}}
A.~#1\\
B.~#2\\
C.~#3\\
D.~#4
\end{tabular}}
\newcommand{\twoch}[4]{\par\begin{tabular}{p{.46\textwidth}p{.46\textwidth}}
A.~#1& B.~#2\\
C.~#3& D.~#4
\end{tabular}}
\newcommand{\vartwoch}[4]{\par\begin{tabular}{p{.46\textwidth}p{.46\textwidth}}
(1)~#1& (2)~#2\\
(3)~#3& (4)~#4
\end{tabular}}
\newcommand{\fourch}[4]{\par\begin{tabular}{p{.23\textwidth}p{.23\textwidth}p{.23\textwidth}p{.23\textwidth}}
A.~#1 &B.~#2& C.~#3& D.~#4
\end{tabular}}
\newcommand{\varfourch}[4]{\par\begin{tabular}{p{.23\textwidth}p{.23\textwidth}p{.23\textwidth}p{.23\textwidth}}
(1)~#1 &(2)~#2& (3)~#3& (4)~#4
\end{tabular}}
\begin{document}

\begin{enumerate}[1.]

\item { (004770)}已知$a$是实常数, 集合$A=\{x|x^2-5x+4\le 0\}$与$B=\{x|x^2-2ax+a+2\le 0\}$满足$B\subseteq A$, 求$a$的取值范围.


关联目标:

K0101002B|D01001B|理解有限集、无限集、空集的含义.

K0221002B|D02003B|会运用最值的定义, 解决函数的最值问题, 以及含参数的函数最值问题(函数对应关系含参数或者定义域含参数)的数学问题.



标签: 第一单元

答案: 暂无答案

解答或提示: 暂无解答与提示

使用记录:

20220822	2023届高三10班	\fcolorbox[rgb]{0,0,0}{1.000,0.626,0}{0.687}

20220822	2023届高三2班	\fcolorbox[rgb]{0,0,0}{1.000,0.848,0}{0.576}

20220823	2023届高三8班	\fcolorbox[rgb]{0,0,0}{1.000,0.688,0}{0.656}

20220823	2023届高三11班	\fcolorbox[rgb]{0,0,0}{0.800,1.000,0}{0.400}


出处: 代数精编第一章集合与命题
\item { (000067)}设常数$a>0$且$a\ne 1$, 若函数$y=\log_a(x+1)$在区间$[0, 1]$上的最大值为$1$, 最小值为$0$, 求实数$a$的值.


关联目标:

K0214002B|D02002B|会利用对数函数的单调性解决其他相关不等式等数学问题和生活中的实际问题.

K0221002B|D02003B|会运用最值的定义, 解决函数的最值问题, 以及含参数的函数最值问题(函数对应关系含参数或者定义域含参数)的数学问题.



标签: 第二单元

答案: 暂无答案

解答或提示: 暂无解答与提示

使用记录:

暂无使用记录


出处: 教材复习题
\item { (000087)}已知函数$y=-x^2+2ax+1-a$, $x\in [0, 1]$的最大值为$2$. 求实数$a$的值.


关联目标:

K0221002B|D02003B|会运用最值的定义, 解决函数的最值问题, 以及含参数的函数最值问题(函数对应关系含参数或者定义域含参数)的数学问题.



标签: 第二单元

答案: 暂无答案

解答或提示: 暂无解答与提示

使用记录:

暂无使用记录


出处: 教材复习题
\item { (000884)}函数$y=\sqrt{x^2+2}+\dfrac1{\sqrt{x^2+2}}$的最小值为\blank{50}.


关联目标:

K0221002B|D02003B|会运用最值的定义, 解决函数的最值问题, 以及含参数的函数最值问题(函数对应关系含参数或者定义域含参数)的数学问题.



标签: 第二单元

答案: $\frac{3\sqrt 2}2$

解答或提示: 暂无解答与提示

使用记录:

20220602	2022届高三1班	\fcolorbox[rgb]{0,0,0}{1.000,0.326,0}{0.837}


出处: 赋能练习
\item { (001226)}(1) 函数$y=1-x^2, \ x\in [-1,1]$的最大值为\blank{50}, 最小值为\blank{50}, 最大值点为\blank{50}, 最小值点为\blank{50};\\ 
(2) 函数$y=2x^2-8x, \ x\in [-1,4]$的最大值为\blank{50}, 最小值为\blank{50}, 最大值点为\blank{50}, 最小值点为\blank{50};\\ 
(3) 函数$y=6x-x^2, \ x\in [-3,0]$的最大值为\blank{50}, 最小值为\blank{50}, 最大值点为\blank{50}, 最小值点为\blank{50};\\ 
(4) 函数$y=2x^2-4x+5, \ x\in [2,4]$的最大值为\blank{50}, 最小值为\blank{50}, 最大值点为\blank{50}, 最小值点为\blank{50}.


关联目标:

K0221002B|D02003B|会运用最值的定义, 解决函数的最值问题, 以及含参数的函数最值问题(函数对应关系含参数或者定义域含参数)的数学问题.



标签: 第二单元

答案: 暂无答案

解答或提示: 暂无解答与提示

使用记录:

2016届11班	\fcolorbox[rgb]{0,0,0}{1.000,0.000,0}{1.000}	\fcolorbox[rgb]{0,0,0}{1.000,0.102,0}{0.949}	\fcolorbox[rgb]{0,0,0}{1.000,0.052,0}{0.974}	\fcolorbox[rgb]{0,0,0}{1.000,0.206,0}{0.897}

2016届12班	\fcolorbox[rgb]{0,0,0}{1.000,0.054,0}{0.973}	\fcolorbox[rgb]{0,0,0}{1.000,0.108,0}{0.946}	\fcolorbox[rgb]{0,0,0}{1.000,0.000,0}{1.000}	\fcolorbox[rgb]{0,0,0}{1.000,0.162,0}{0.919}


出处: 2016届创新班作业	1142-函数的最值
\item { (001227)}(1) 函数$y=x+\dfrac{4}{x}, \ x\in [1,5]$的最大值为\blank{50}, 最小值为\blank{50}, 最大值点为\blank{50}, 最小值点为\blank{50};\\ 
(2) 函数$y=x-\dfrac{4}{x}, \ x\in [1,5]$的最大值为\blank{50}, 最小值为\blank{50}, 最大值点为\blank{50}, 最小值点为\blank{50};\\ 
(3) 函数$y=\dfrac{x-5}{3x+2}, \ x\in [0,3]$的最大值为\blank{50}, 最小值为\blank{50}, 最大值点为\blank{50}, 最小值点为\blank{50};\\ 
(4) 函数$y=x^2+\dfrac{16}{x}, \ x\in [1,4]$的最大值为\blank{50}, 最小值为\blank{50}, 最大值点为\blank{50}, 最小值点为\blank{50}.


关联目标:

K0221002B|D02003B|会运用最值的定义, 解决函数的最值问题, 以及含参数的函数最值问题(函数对应关系含参数或者定义域含参数)的数学问题.



标签: 第二单元

答案: 暂无答案

解答或提示: 暂无解答与提示

使用记录:

2016届11班	\fcolorbox[rgb]{0,0,0}{1.000,0.102,0}{0.949}	\fcolorbox[rgb]{0,0,0}{1.000,0.052,0}{0.974}	\fcolorbox[rgb]{0,0,0}{1.000,0.102,0}{0.949}	\fcolorbox[rgb]{0,0,0}{1.000,0.000,0}{1.000}

2016届12班	\fcolorbox[rgb]{0,0,0}{1.000,0.054,0}{0.973}	\fcolorbox[rgb]{0,0,0}{1.000,0.108,0}{0.946}	\fcolorbox[rgb]{0,0,0}{1.000,0.000,0}{1.000}	\fcolorbox[rgb]{0,0,0}{1.000,0.054,0}{0.973}


出处: 2016届创新班作业	1142-函数的最值
\item { (001231)}已知函数$y=\dfrac{1}{2}x^2-x+\dfrac{3}{2}$的定义域为$[1,b]$, 最大值为$b$, 最小值为$1$. 求$b$.


关联目标:

K0221002B|D02003B|会运用最值的定义, 解决函数的最值问题, 以及含参数的函数最值问题(函数对应关系含参数或者定义域含参数)的数学问题.



标签: 第二单元

答案: 暂无答案

解答或提示: 暂无解答与提示

使用记录:

2016届11班	\fcolorbox[rgb]{0,0,0}{1.000,0.206,0}{0.897}

2016届12班	\fcolorbox[rgb]{0,0,0}{1.000,0.108,0}{0.946}


出处: 2016届创新班作业	1142-函数的最值
\item { (001276)}已知$a$是实数, 函数$y=-x^2+2ax+1-a, \ x \in [0,1]$的最大值为$2$. 求$a$.


关联目标:

K0221002B|D02003B|会运用最值的定义, 解决函数的最值问题, 以及含参数的函数最值问题(函数对应关系含参数或者定义域含参数)的数学问题.



标签: 第二单元

答案: 暂无答案

解答或提示: 暂无解答与提示

使用记录:

2016届11班	\fcolorbox[rgb]{0,0,0}{1.000,0.526,0}{0.737}

2016届12班	\fcolorbox[rgb]{0,0,0}{1.000,0.270,0}{0.865}


出处: 2016届创新班作业	1147-二次函数
\item { (001277)}已知$a,b$是实数, 函数$y=ax^2-2ax+2+b$在$[2,3]$上的最大值和最小值分别为$5$和$2$, 求$a,b$.


关联目标:

K0221002B|D02003B|会运用最值的定义, 解决函数的最值问题, 以及含参数的函数最值问题(函数对应关系含参数或者定义域含参数)的数学问题.



标签: 第二单元

答案: 暂无答案

解答或提示: 暂无解答与提示

使用记录:

2016届11班	\fcolorbox[rgb]{0,0,0}{1.000,0.264,0}{0.868}

2016届12班	\fcolorbox[rgb]{0,0,0}{1.000,0.594,0}{0.703}


出处: 2016届创新班作业	1147-二次函数
\item { (002955)}设常数$a>0,\ a\ne 1$. 函数$f(x)=a^x$在$[0,1]$上的最大值和最小值之和为$a^2$, 则$a=$\blank{50}.


关联目标:

K0221002B|D02003B|会运用最值的定义, 解决函数的最值问题, 以及含参数的函数最值问题(函数对应关系含参数或者定义域含参数)的数学问题.



标签: 第二单元

答案: 暂无答案

解答或提示: 暂无解答与提示

使用记录:

暂无使用记录


出处: 2022届高三第一轮复习讲义
\item { (002959)}已知函数$y=(\log_2\dfrac x{2^a})(\log_2\dfrac x4)$, $x\in [\sqrt 2,4]$, 试求该函数的最大值$g(a)$.


关联目标:

K0221002B|D02003B|会运用最值的定义, 解决函数的最值问题, 以及含参数的函数最值问题(函数对应关系含参数或者定义域含参数)的数学问题.



标签: 第二单元

答案: 暂无答案

解答或提示: 暂无解答与提示

使用记录:

暂无使用记录


出处: 2022届高三第一轮复习讲义
\item { (002966)}*已知常数$a>1$, 函数$y=|\log_ax|$的定义域为区间$[m,n]$, 值域为区间$[0,1]$. 若$n-m$的最小值为$\dfrac 56$, 则$a$=\blank{50}.


关联目标:

K0221002B|D02003B|会运用最值的定义, 解决函数的最值问题, 以及含参数的函数最值问题(函数对应关系含参数或者定义域含参数)的数学问题.



标签: 第二单元

答案: 暂无答案

解答或提示: 暂无解答与提示

使用记录:

暂无使用记录


出处: 2022届高三第一轮复习讲义
\item { (002975)}设常数$a\in \mathbf{R}$. 若函数$y=-x^2+2ax$($0\le x\le 1$)的最小值用$g(a)$表示, 则$g(a)=$\blank{50}.


关联目标:

K0221002B|D02003B|会运用最值的定义, 解决函数的最值问题, 以及含参数的函数最值问题(函数对应关系含参数或者定义域含参数)的数学问题.



标签: 第二单元

答案: 暂无答案

解答或提示: 暂无解答与提示

使用记录:

暂无使用记录


出处: 2022届高三第一轮复习讲义
\item { (002986)}设常数$m\in \mathbf{R}$. 若函数$f(x)=x^2-(m-2)x+m-4$的图像与x轴交于$A$, $B$两点, 且$|AB|=2$, 则函数$y=f(x)$的最小值为\blank{50}.


关联目标:

K0221002B|D02003B|会运用最值的定义, 解决函数的最值问题, 以及含参数的函数最值问题(函数对应关系含参数或者定义域含参数)的数学问题.



标签: 第二单元

答案: 暂无答案

解答或提示: 暂无解答与提示

使用记录:

暂无使用记录


出处: 2022届高三第一轮复习讲义
\item { (002991)}设常数$a\in \mathbf{R}$, 并将函数$f(x)=1-2a-2a\cos x-2\sin^2 x$的最小值记为$g(a)$.\\
(1) 写出$g(a)$的表达式;\\
(2) 是否存在$a$的值, 使得$g(a)=\dfrac 12$? 若存在, 求出$a$的值以及此时函数$y=f(x)$的最大值; 若不存在, 说明理由.


关联目标:

K0221002B|D02003B|会运用最值的定义, 解决函数的最值问题, 以及含参数的函数最值问题(函数对应关系含参数或者定义域含参数)的数学问题.



标签: 第二单元

答案: 暂无答案

解答或提示: 暂无解答与提示

使用记录:

暂无使用记录


出处: 2022届高三第一轮复习讲义
\item { (004439)}函数$f(x)=|x^2-a|$在区间$[-1,1]$上的最大值是$a$, 那么实数$a$的取值范围是\bracket{20}.
\fourch{$[0,+\infty)$}{$[\dfrac 12,1]$}{$[\dfrac 12,+\infty)$}{$[1,+\infty)$}


关联目标:

K0221002B|D02003B|会运用最值的定义, 解决函数的最值问题, 以及含参数的函数最值问题(函数对应关系含参数或者定义域含参数)的数学问题.



标签: 第二单元

答案: 暂无答案

解答或提示: 暂无解答与提示

使用记录:

20211026	2022届高三1班	\fcolorbox[rgb]{0,0,0}{1.000,0.094,0}{0.953}


出处: 2022届高三上学期测验卷05第15题
\item { (005344)}已知函数$f(x)=x^2-2x+3$在$[0,m]$上有最大值$3$, 最小值$2$, 求正数$m$的取值范围.


关联目标:

K0221002B|D02003B|会运用最值的定义, 解决函数的最值问题, 以及含参数的函数最值问题(函数对应关系含参数或者定义域含参数)的数学问题.



标签: 第二单元

答案: 暂无答案

解答或提示: 暂无解答与提示

使用记录:

暂无使用记录


出处: 代数精编第三章函数
\item { (000555)}已知函数$f(x)=x|2x-a|-1$有三个零点, 则实数$a$的取值范围为\blank{50}.


关联目标:

K0223001B|D02004B|知道函数零点的定义.



标签: 第二单元

答案: $(2\sqrt 2,+\infty)$

解答或提示: 暂无解答与提示

使用记录:

20220309	2022届高三1班	\fcolorbox[rgb]{0,0,0}{1.000,0.864,0}{0.568}

20220622	2022届高三1班  	\fcolorbox[rgb]{0,0,0}{1.000,0.418,0}{0.791}


出处: 赋能练习
\item { (000622)}若函数$f(x)=2^x(x+a)-1$在区间$[0,1]$上有零点, 则实数$a$的取值范围是\blank{50}.


关联目标:

K0223001B|D02004B|知道函数零点的定义.



标签: 第二单元

答案: $[-\frac 12,1]$

解答或提示: 暂无解答与提示

使用记录:

20220325	2022届高三1班	\fcolorbox[rgb]{0,0,0}{1.000,0.280,0}{0.860}


出处: 赋能练习
\item { (003013)}函数$f(x)=3ax-2a+1$在$[-1,1]$上存在一个零点, 则实数$a$的取值范围是\blank{50}.


关联目标:

K0223001B|D02004B|知道函数零点的定义.



标签: 第二单元

答案: 暂无答案

解答或提示: 暂无解答与提示

使用记录:

暂无使用记录


出处: 2022届高三第一轮复习讲义
\item { (003648)}已知$f(x)=ax+\dfrac{1}{x+1}, \ a\in \mathbf{R}$.\\
(1) 已知$a=1$时, 求不等式$f(x)+1<f(x+1)$的解集;\\
(2) 若$f(x)$在$x\in [1,2]$时有零点, 求$a$的取值范围.


关联目标:

K0223001B|D02004B|知道函数零点的定义.



标签: 第二单元

答案: 暂无答案

解答或提示: 暂无解答与提示

使用记录:

暂无使用记录


出处: 上海2019年秋季高考试题18
\item { (004720)}已知函数$f(x)=x^2+mx+3$, 其中$m\in \mathbf{R}$.\\
(1) 若不等式$f(x)<5$的解集是$(-1,2)$, 求$m$的值;\\
(2) 若函数$y=f(x)$在区间$[0,3]$上有且仅有一个零点, 求$m$的取值范围.


关联目标:

K0223001B|D02004B|知道函数零点的定义.

K0223004B|D02004B|会用函数的观点求解较为复杂的方程.



标签: 第二单元

答案: 暂无答案

解答或提示: 暂无解答与提示

使用记录:

20220414	2022届高三	\fcolorbox[rgb]{0,0,0}{1.000,0.042,0}{0.979}	\fcolorbox[rgb]{0,0,0}{1.000,0.670,0}{0.665}


出处: 2022届高三下期中区统考第18题
\item { (003032)}设常数$a\in \mathbf{R}$.已知函数$f(x)=4^x-a\cdot 2^x+a+3$.\\
(1) 若函数$y=f(x)$有且仅有一个零点, 求$a$的取值范围;\\
(2) 若函数$y=f(x)$有零点, 求$a$的取值范围.


关联目标:

K0223004B|D02004B|会用函数的观点求解较为复杂的方程.



标签: 第二单元

答案: 暂无答案

解答或提示: 暂无解答与提示

使用记录:

暂无使用记录


出处: 2022届高三第一轮复习讲义
\item { (010196)}证明: 方程$\lg x+2x=16$没有整数解.


关联目标:

K0223004B|D02004B|会用函数的观点求解较为复杂的方程.



标签: 第二单元

答案: 暂无答案

解答或提示: 暂无解答与提示

使用记录:

暂无使用记录


出处: 新教材必修第一册习题
\item { (009530)}用函数的观点解不等式: $2^x+\log_2x>2$.


关联目标:

K0223005B|D02004B|会用函数的观点求解较为复杂的不等式.



标签: 第二单元

答案: 暂无答案

解答或提示: 暂无解答与提示

使用记录:

暂无使用记录


出处: 新教材必修第一册课堂练习
\item { (005236)}解不等式: $|x+2|-|x-3|<4$.


关联目标:

K0117002B|D01004B|会用分类讨论的思想求解一些基本的含绝对值的不等式.

K0223005B|D02004B|会用函数的观点求解较为复杂的不等式.



标签: 第一单元

答案: 暂无答案

解答或提示: 暂无解答与提示

使用记录:

暂无使用记录


出处: 代数精编第二章不等式
\item { (010197)}解不等式: $\dfrac 2{x^2}\ge 3x-1$.


关联目标:

K0223005B|D02004B|会用函数的观点求解较为复杂的不等式.



标签: 第二单元

答案: 暂无答案

解答或提示: 暂无解答与提示

使用记录:

暂无使用记录


出处: 新教材必修第一册习题
\item { (009531)}对于在区间$[a, b]$上的图像是一段连续曲线的函数$y=f(x)$, 如果$f(a)\cdot f(b)>0$, 那么是否该函数在区间$(a, b)$上一定无零点? 说明理由.


关联目标:

K0224001B|D02004B|知道零点存在定理, 会用零点存在定理判断连续函数在区间上存在零点.



标签: 第二单元

答案: 暂无答案

解答或提示: 暂无解答与提示

使用记录:

暂无使用记录


出处: 新教材必修第一册课堂练习
\item { (009532)}已知函数$y=2x^3-3x^2-18x+28$在区间$(1, 2)$上有且仅有一个零点. 试用二分法求出该零点的近似值. (结果精确到$0.1$)


关联目标:

K0224002B|D02004B|理解并会运用二分法寻求连续函数在某个区间上的零点的近似值.



标签: 第二单元

答案: 暂无答案

解答或提示: 暂无解答与提示

使用记录:

暂无使用记录


出处: 新教材必修第一册课堂练习
\item { (010192)}已知函数$y=x^3+x^2+x-1$在区间$(0, 1)$上有且仅有一个零点, 用二分法求该零点的近似值. (结果精确到$0.1$)


关联目标:

K0224002B|D02004B|理解并会运用二分法寻求连续函数在某个区间上的零点的近似值.



标签: 第二单元

答案: 暂无答案

解答或提示: 暂无解答与提示

使用记录:

暂无使用记录


出处: 新教材必修第一册习题
\end{enumerate}



\end{document}