\documentclass[10pt,a4paper]{article}
\usepackage[UTF8,fontset = windows]{ctex}
\setCJKmainfont[BoldFont=黑体,ItalicFont=楷体]{华文中宋}
\usepackage{amssymb,amsmath,amsfonts,amsthm,mathrsfs,dsfont,graphicx}
\usepackage{ifthen,indentfirst,enumerate,color,titletoc}
\usepackage{tikz}
\usepackage{multicol}
\usepackage{makecell}
\usepackage{longtable}
\usetikzlibrary{arrows,calc,intersections,patterns,decorations.pathreplacing,3d,angles,quotes,positioning}
\usepackage[bf,small,indentafter,pagestyles]{titlesec}
\usepackage[top=1in, bottom=1in,left=0.8in,right=0.8in]{geometry}
\renewcommand{\baselinestretch}{1.65}
\newtheorem{defi}{定义~}
\newtheorem{eg}{例~}
\newtheorem{ex}{~}
\newtheorem{rem}{注~}
\newtheorem{thm}{定理~}
\newtheorem{coro}{推论~}
\newtheorem{axiom}{公理~}
\newtheorem{prop}{性质~}
\newcommand{\blank}[1]{\underline{\hbox to #1pt{}}}
\newcommand{\bracket}[1]{(\hbox to #1pt{})}
\newcommand{\onech}[4]{\par\begin{tabular}{p{.9\textwidth}}
A.~#1\\
B.~#2\\
C.~#3\\
D.~#4
\end{tabular}}
\newcommand{\twoch}[4]{\par\begin{tabular}{p{.46\textwidth}p{.46\textwidth}}
A.~#1& B.~#2\\
C.~#3& D.~#4
\end{tabular}}
\newcommand{\vartwoch}[4]{\par\begin{tabular}{p{.46\textwidth}p{.46\textwidth}}
(1)~#1& (2)~#2\\
(3)~#3& (4)~#4
\end{tabular}}
\newcommand{\fourch}[4]{\par\begin{tabular}{p{.23\textwidth}p{.23\textwidth}p{.23\textwidth}p{.23\textwidth}}
A.~#1 &B.~#2& C.~#3& D.~#4
\end{tabular}}
\newcommand{\varfourch}[4]{\par\begin{tabular}{p{.23\textwidth}p{.23\textwidth}p{.23\textwidth}p{.23\textwidth}}
(1)~#1 &(2)~#2& (3)~#3& (4)~#4
\end{tabular}}
\begin{document}

\begin{enumerate}[1.]

\item {\tiny (000004)}已知方程$x^2+px+4=0$的所有解组成的集合为$A$, 方程$x^2+x+q=0$的所有解组成的集合为$B$, 且$A\cap B=\{4\}$. 求集合$A\cup B$的所有子集.
\item {\tiny (000020)}设一元二次方程$2x^2-6x-3=0$的两个实根为$x_1,x_2$, 求下列各式的值:\\
(1) $(x_1+1)(x_2+1)$;\\
(2) $(x_1^2-1)(x_2^2-1)$.
\item {\tiny (000023)}若关于$x$的不等式$(a+1)x-a<0$的解集为$(2,+\infty)$, 求实数$a$的值, 并求不等式$(a-1)x+3-a>0$的解集.
\item {\tiny (000025)}试写出一个二次项系数为$1$的一元二次不等式, 使它的解集分别为:\\
(1) $(-\infty, \sqrt 2)\cup  (\sqrt 2, +\infty)$;\\
(2) $[2-\sqrt 3, 2+\sqrt 3]$.
\item {\tiny (000028)}设关于$x$的不等式$a_1x^2+b_1x+c_1>0$与$a_2x^2+b_2x+c_2>0$的解集分别为$A$、$B$,
试用集合运算表示下列不等式组的解集:\\
(1) $\begin{cases} a_1x^2+b_1x+c_1>0, \\ a_2x^2+b_2x+c_2>0;\end{cases}$\\
(2) $\begin{cases} a_1x^2+b_1x+c_1\le 0, \\ a_2x^2+b_2x+c_2>0;\end{cases}$\\
(3) $\begin{cases} a_1x^2+b_1x+c_1\le 0, \\ a_2x^2+b_2x+c_2\le 0.\end{cases}$
\item {\tiny (000033)}已知一元二次方程$x^2+px+p=0$的两个实根分别为$\alpha$、$\beta$, 且$\alpha^2$+$\beta^2=3$, 求实数$p$的值.
\item {\tiny (000034)}已知一元二次方程$2x^2-4x+m+3=0$有两个同号实根, 求实数$m$的取值范围.
\item {\tiny (000035)}设$a,b\in \mathbf{R}$, 已知关于$x$的不等式$(a+b)x+(b-2a)<0$的解集为$(1, +\infty)$, 求不等式$(a-b)x+3b-a>0$的解集.
\item {\tiny (000047)}方程$(x-1)(x-2)(x-3)=0$的三个根$1$、$2$、$3$将数轴划分为四个区间, 即$(-\infty, 1)$, $(1, 2)$, $(2, 3)$, $(3, +\infty)$. 试在这四个区间上分别考察$(x-1)(x-2)(x-3)$的
符号, 从而得出不等式$(x-1)(x-2)(x-3)>0$与$(x-1)(x-2)(x-3)<0$的解集.\\
一般地, 对$x_1$、$x_2$、$x_3\in \mathbf{R}$, 且$x_1\le x_2\le x_3$, 试分别求不等式$(x-x_1)(x-x_2)(x-x_3)>0$与$(x-x_1)(x-x_2)(x-x_3)<0$的解集(提示: $x_1$、$x_2$、$x_3$相互之间可能相等, 需要分情况讨论).
\item {\tiny (000378)}不等式$\dfrac{x+1}{x+2}<0$的解集为\blank{50}.
\item {\tiny (000389)}不等式$x|x-1|>0$的解集为\blank{50}.
\item {\tiny (000407)}若关于$x$的不等式$\dfrac{x-a}{x-b}>0$($a,b\in \mathbf{R}$)的解集为$(-\infty ,1)\cup (4,+\infty)$, 则$a+b=$\blank{50}.
\item {\tiny (000415)}已知$x$、$y$满足曲线方程$x^2+\dfrac1{y^2}=2$, 则$x^2+y^2$的取值范围是\blank{50}.
\item {\tiny (000459)}不等式$\dfrac{x+2}{x+1}>1$的解集为\blank{50}.
\item {\tiny (000540)}不等式$\dfrac1{|x-1|}\ge 1 $的解集为\blank{50}.
\item {\tiny (000586)}不等式$\dfrac x{x+1}\le 0$的解集为\blank{50}.
\item {\tiny (000639)}若方程组$\begin{cases} ax+2y=3, \\ 2x+ay=2 \end{cases}$ 无解, 则实数$a=$\blank{50}.
\item {\tiny (000647)}若关于$x,y$的方程组$\begin{cases} ax+y-1=0,  \\ 4x+ay-2=0  \end{cases}$有无数多组解, 则实数$a=$\blank{50}.
\item {\tiny (000757)}不等式$|1-x|>1$的解集是\blank{50}.
\item {\tiny (000797)}不等式$\dfrac x{x-1}<0$的解集为\blank{50}.
\item {\tiny (000816)}不等式$|x-3|<2$的解集为\blank{50}.
\item {\tiny (000976)}在下列各命题的右边写出其否定形式(否定命题).\\ 
(1) $2 \times 2 =5$; \blank{150}.\\ 
(2) $\sqrt{3-\pi}$有意义; \blank{150}.\\ 
(3) $a$不是非负数; \blank{150}.\\ 
(4) $\sqrt{a}$不是无理数; \blank{150}.(本小题中已知$a\ge 0$)\\ 
(5) $x=1$不是方程$x(x+1)=0$的根; \blank{150}.
\item {\tiny (000977)}下列各组命题是否互为否定形式? ($\checkmark$ or $\times$).\\ 
\blank{30}(1) 所有直角三角形都不是等边三角形; / 所有直角三角形都是等边三角形.\\ 
\blank{30}(2) 对一切实数$x$, $x^2+1 \ne 0$; / 存在实数$x$, 使得$x^2+1=0$.\\ 
\blank{30}(3) 所有一元二次方程都没有实数根; / 有些一元二次方程没有实数根.\\ 
\blank{30}(4) 所有自然数都不是$0$; / 所有自然数都是$0$.\\ 
\blank{30}(5) 存在实数$x$, 使得$x^2-5x+6=0$; / 所有实数$x$, 都使得$x^2-5x+6\ne 0$.\\ 
\blank{30}(6) 对于一些实数$x$, $x^3+1=0$; / 对于一些实数$x$, $x^3+1\ne 0$.\\ 
\blank{30}(7) 有些三角形两边的平方和等于第三边的平方; / 所有三角形两边的平方和不等于第三边的平方.\\ 
\blank{30}(8) 对于某些实数$x$, $x=x+1$; / 对于任意实数$x$, $x \ne x+1$.\\ 
\blank{30}(9) 负实数没有平方根; / 负实数有平方根.
\item {\tiny (000987)}已知实数$t\ne 0$. 证明: ``$x=t$是方程$a x^3+b x^2+cx+d=0$的根''的充分必要条件是``$x=\dfrac{1}{t}$是方程$d x^3+c x^2+ b x+a=0$的根''.
\item {\tiny (000990)}用描述法或列举法(自行择其一种)表示下列集合.\\ 
(1) 大于$0$且小于$3$的实数的全体.\\ 
(2) 方程$x^3-x=0$的解的全体.\\ 
(3) 一次函数$y=2x+1$图像上所有点的全体.\\ 
(4) 被$3$除余$2$的整数的全体.
\item {\tiny (001040)}解方程: $x+\sqrt{2+x}=0$.
\item {\tiny (001041)}解方程: $\dfrac{3}{4x^2+20x+25}=\dfrac{5}{4x^2+8x-5}-\dfrac{2}{4x^2-4x+1}$.
\item {\tiny (001042)}设常数$b\geq 0$, 求证: 方程$\sqrt{f(x)}=b$与方程$f(x)=b^2$同解.
\item {\tiny (001043)}解方程: $\sqrt{1+x}=\sqrt{2x-5}+1$.
\item {\tiny (001044)}(1) 求证: 方程``$\sqrt{f(x)}\sqrt{g(x)}=h(x)$''与
``$f(x)g(x)=(h(x))^2$且$h(x)\geq 0$且$f(x)\geq 0$且$g(x)\geq 0$''同解.\\ 
(2) 试举一例并分析, 说明: 方程``$\sqrt{f(x)}\sqrt{g(x)}=h(x)$''和
``$f(x)g(x)=(h(x))^2$且$h(x)\geq 0$且$f(x)\geq 0$''有时会不同解.
\item {\tiny (001045)}(1) 求证: 方程``$\sqrt{f(x)}+\sqrt{g(x)}=\sqrt{h(x)}$''与方程``$f(x)+g(x)+2\sqrt{f(x)}\sqrt{g(x)}=h(x)$''同解.\\ 
(2) 试举一例并分析, 说明: 方程``$\sqrt{f(x)}+\sqrt{g(x)}=\sqrt{h(x)}$''与方程``$f(x)+g(x)+2\sqrt{f(x)g(x)}=h(x)$''有时会不同解.
\item {\tiny (001046)}解方程: $111x^2+83x-28=0$.
\item {\tiny (001047)}解方程: $x^2+x=\sqrt{5}+5$.
\item {\tiny (001048)}求实数$a,b$, 使得关于$x$的方程$x^2+2(1+a)x+(3a^2+4ab+4b^2+2)=0$有实根.
\item {\tiny (001049)}解关于$x$的方程: $ax-1=x+ab$.
\item {\tiny (001050)}解关于$x$的方程: $m^2(x-1)+m(x+3)=6x+2$.
\item {\tiny (001051)}已知实数$a,b,c \ne 0$. 解关于$x$的方程: $\dfrac{x-b-c}{a}+\dfrac{x-c-a}{b}+\dfrac{x-a-b}{c}=3$.
\item {\tiny (001052)}若关于$x$的方程$2ax=(a+1)x+6$的解集真包含于$\mathbf{Z}^+$, 求$a$.
\item {\tiny (001053)}[选做]
解关于$x$的方程: $\dfrac{(x-a)^2}{(x-b)(x-c)}+\dfrac{(x-b)^2}{(x-c)(x-a)}+\dfrac{(x-c)^2}{(x-a)(x-b)}=3$.
\item {\tiny (001054)}解方程: $x^4+x^3-7x^2-x+6=0$.
\item {\tiny (001055)}解方程: $2x^5-x^4-15x^3+9x^2+16x+4=0$.
\item {\tiny (001056)}解方程: $(9-16x^2)^3+(16-9x^2)^3+(25x^2-25)^3=0$.
\item {\tiny (001057)}解方程: $2(x^2+6x+1)^2+5(x^2+6x+1)(x^2+1)+2(x^2+1)^2=0$
\item {\tiny (001058)}解方程: $(x+1)(x+3)(x+5)(x+7)=-12$.
\item {\tiny (001059)}解方程: $6x^4+5x^3-38x^2+5x+6=0$.
\item {\tiny (001060)}解方程: $6x^4-25x^3+12x^2+25x+6=0$.
\item {\tiny (001061)}[选做]
解方程: $x^4+8x^3+24x^2+32x+12=0$.
\item {\tiny (001062)}已知关于$x$的方程$x^2+2x-1=0$的两个实根为$x_1,x_2$, 则$x_1+x_2=$\blank{50}, $x_1x_2=$\blank{50}.
\item {\tiny (001063)}已知关于$x$的方程$ax^2+bx+1=0$有两个实根$\dfrac{1}{2}, \dfrac{1}{3}$, 则$b=$\blank{50}.
\item {\tiny (001064)}已知关于$x$的方程$x^2+bx-2=0$的一个实根为$2$, 则另一实根为\blank{50}.
\item {\tiny (001065)}已知关于$x$的方程$-x^2-3x+3=0$的两个实根为$x_1,x_2$, 则$\dfrac{x_1}{x_2}+\dfrac{x_2}{x_1}=$\blank{50}.
\item {\tiny (001066)}已知关于$x$的二次方程$ax^2+bx+c=0$的两实根为$x_1,x_2$, 则$|x_1-x_2|=$\blank{50}.
\item {\tiny (001067)}已知关于$x$的方程$x^2+2mx+6=0$的两实根的倒数之和为$1$, 则实数$m=$\blank{50}.
\item {\tiny (001068)}关于$y$的方程$4y^2+(b^2-3b-10)y+4b=0$的两个实根互为相反数, 则实数$b=$\blank{50}.
\item {\tiny (001069)}若关于$x$的方程$x^2-mx+2m-2=0$的两实根的平方和为$1$, 则实数$m=$\blank{50}.
\item {\tiny (001070)}方程组$\left\{
\begin{array}{l}
x+y+xy=5,\\
x^2y+xy^2=6
\end{array}
\right.$的解为$(x,y)=$\blank{200}.
\item {\tiny (001071)}方程组$\left\{
\begin{array}{l}
x-y=3,\\
xy=-2
\end{array}
\right.$的解为$(x,y)=$\blank{200}.
\item {\tiny (001072)}关于$x$的方程$x^2+px+q=0$的两个实根之比为$1:2$, 判别式的值为$1$, 求实数$p,q$.
\item {\tiny (001073)}已知$\alpha,\beta$是关于$x$的二次方程$x^2+(p-2)x+1=0$的两根. 试求$(1+p\alpha+\alpha^2)(1+p\beta+\beta^2)$的值.
\item {\tiny (001074)}设$\alpha,\beta$是方程$2x^2+x-7=0$的两根, 试以$\dfrac{1}{\alpha^2-1},\dfrac{1}{\beta^2-1}$为根作一个新的二次方程.
\item {\tiny (001075)}设常数$k\in\mathbf{N}$, 若关于$x$的方程$x^2=2(k+1)x-(k^2+4k-3)$的两个实根符号相反, 求$k$的值,
并解此方程.
\item {\tiny (001076)}设常数$a>0,m>0$, 若方程组$\left\{
\begin{array}{l}
y^2=4a(x+a),\\
x+y+m=0
\end{array}
\right.$有两组不同的解$(x_1,y_1)$,$(x_2,y_2)$,\\ 
(1) 求$a,m$所满足的条件;\\ 
(2) 用$a,m$表示$\sqrt{(x_1-x_2)^2+(y_1-y_2)^2}$.
\item {\tiny (001077)}[选做]
解方程组: $\left\{
\begin{array}{l}
x+y+z=15,\\
x^2+y^2+z^2=83,\\
x^3+y^3+z^3=495.
\end{array}\right.$
\item {\tiny (001078)}解方程: $1+\dfrac{1}{1+\dfrac{1}{1+\dfrac{1}{1+\dfrac{1}{x}}}}=2$.
\item {\tiny (001079)}解方程: $\dfrac{x^4-(x-1)^2}{(x^2+1)^2-x^2}+\dfrac{x^2-(x^2-1)^2}{x^2(x+1)^2-1}+\dfrac{x^2(x-1)^2-1}{x^4-(x+1)^2}=x^2$.
\item {\tiny (001080)}[选做]
解方程: $\dfrac{1}{(x-5)(x-4)}+\dfrac{1}{(x-4)(x-3)}+\cdots+\dfrac{1}{(x+4)(x+5)}=\dfrac{10}{11}$.
\item {\tiny (001081)}解方程: $\sqrt[3]{3-\sqrt{x+1}}+\sqrt[3]{2}=0$.
\item {\tiny (001082)}解方程: $\sqrt{3x+4}+2=3\sqrt[4]{3x+4}$.
\item {\tiny (001083)}已知$a>b$, $a,b\in \mathbf{R}$. 解关于$y$的方程: $\sqrt{a-y}+\sqrt{y-b}=\sqrt{a-b}$.
\item {\tiny (001084)}[选做]
解方程: $\sqrt[4]{97-x}+\sqrt[4]{x}=5$.
\item {\tiny (001101)}求不等式$3x-1>2-\dfrac{x+1}{3}\ge 1-\dfrac{2x-3}{2}$的解集.
\item {\tiny (001105)}关于$x$的不等式$ax^2+bx+c>0$的解集为$(-\infty,1)\cup (3,+\infty)$, 求$a:b:c$. 在你求出的这个比值下, 不等式的解集一定如题中所说吗? 为什么?
\item {\tiny (001106)}解不等式组: $x^2-2x-3\le 0<x^2-3x+2$.
\item {\tiny (001115)}设$a,m$是实常数, 且关于$x$的不等式$\sqrt{x}>ax+\dfrac{3}{2}$的解集为$(4,m)$, 求$a,m$的值.
\item {\tiny (001117)}已知关于$x$的不等式$|ax+1|\leq b$的解集为$[2,3]$, 求实常数$a,b$的值.
\item {\tiny (001118)}若关于$x$的不等式$|x-1|-|x-2|<a$的解集为$\mathbf{R}$, 求实数$a$的取值范围.
\item {\tiny (001122)}在解不等式时, 有时我们可以用不等式的性质来求解. 例如解不等式$x^2+x+1\ge 0$, 我们可以利用不等式的基本性质, 得到$x^2+x+1=\left(x+\dfrac{1}{2}\right)^2+\dfrac{3}{4}\ge\dfrac{3}{4}>0$恒成立, 因此解集为$\mathbf{R}$. 请你用基本不等式的观点解以下两个不等式:\\ 
(1) $x+\dfrac{1}{x}>1$; \hfill (2) $x+\dfrac{1}{x}>2$. \hfill
\item {\tiny (002703)}设全集$U=\mathbf{R}$, 函数$y=f(x)$, $y=g(x)$, $y=h(x)$的定义域均为$\mathbf{R}$. 设集合$A=\{x|f(x)=0\}$, $B=\{x|g(x)=0\}$, $C=\{x|h(x)=0, \ x\in \mathbf{R}\}$, 则方程$\dfrac{f^2(x)+g^2(x)}{h(x)}=0$的解集是\blank{50}(用$A,B,C$表示).
\item {\tiny (002739)}填空: (填``充分不必要''、``必要不充分''、``充要''、``既不充分也不必要'')\\ 
(1) 对于实数$x,y$, $p$: $xy>1$且$x+y>2$是$q$: $x>1$且$y>1$的\blank{50}条件;\\
(2) 对于实数$x,y$, $p$: $x+y\ne 8$是$q$: $x\ne 2$或$y\ne 6$的\blank{50}条件;\\
(3) 已知$x,y\in \mathbf{R}$, $p$: $(x-1)^2+(y-2)^2=0$是$q$: $(x-1)(y-2)=0$的\blank{50}条件;\\
*(4) 设$x,y\in \mathbf{R}$, 则``$x^2+y^2<2$''是``$|x|+|y|\le \sqrt2$''的\blank{50}条件; 又是``$|x|+|y|<2$''的\blank{50}条件; 又是``$|x|<\sqrt2$且$|y|<\sqrt2$''的\blank{50}条件.\\
(5) 设$a_1,b_1,c_1,a_2,b_2,c_2$均为非零实数, 方程$a_1x^2+b_1x+c_1=0$和方程$a_2x^2+b_2x+c_2=0$的实数解集分别为$M$和$N$, 则``$\dfrac{a_1}{a_2}=\dfrac{b_1}{b_2}=\dfrac{c_1}{c_2}$''是``$M=N$''的\blank{50}条件.
\item {\tiny (002741)}已知关于$x$的实系数二次方程$a x^2 +bx+c=0\ (a>0)$, 分别求下列命题的一个充要条件:\\
(1) 方程有一正根, 一根是零;\\
(2) 两根都比$2$小.
\item {\tiny (002747)}已知$m$是实常数. 命题甲: 关于$x$的方程$x^2+x+m=0$有两个相异的负根; 命题乙: 关于$x$的方程$4x^2+x+m=0$无实根, 若这两个命题有且只有一个是真命题, 求实数$m$的取值范围.
*
\vspace*{16ex}
\item {\tiny (002770)}下列不等式中解集为$\mathbf{R}$的是\bracket{20}.
\fourch{$x^2-6x+9>0$}{$4x^2+12x+9<0$}{$3x^2-x+2>0$}{$3x^2-x+2<0$}
\item {\tiny (002771)}不等式$(x-1)^2(2-x)\le 0$的解集是\blank{50}; $(x-1)^2(2-x)>0$的解集是\blank{50}.
\item {\tiny (002772)}已知关于$x$的不等式$x^2+ax+b<0$的解集为$(-1,2)$, 则$a+b=$\blank{50}.
\item {\tiny (002773)}不等式$-1<x^2+2x-1\le 2$的解集是\blank{50}.
\item {\tiny (002775)}已知关于$x$的不等式$ax^2-bx+c>0$的解集是$(-\dfrac 12,2)$, 对于$a,b,c$有以下结论: \textcircled{1} $a>0$; \textcircled{2} $b>0$; \textcircled{3} $c>0$; \textcircled{4} $a+b+c>0$; \textcircled{5} $a-b+c>0$. 其中正确的序号有\blank{50}.
\item {\tiny (002777)}已知关于$x$的不等式$(2a-b)x+a-5b>0$的解集是$(-\infty,\dfrac{10}7)$, 则关于$x$的不等式$ax>b$的解集是\blank{50}.
\item {\tiny (002778)}已知关于$x$的不等式$ax^2+bx+c>0$的解集为$\{x|2<x<4\}$, 求关于$x$的不等式$cx^2+bx+a<0$的解集.
\item {\tiny (002781)}不等式$-6x^2-x+2\le 0$的解集是\blank{50}.
\item {\tiny (002784)}若关于$x$的不等式$ax^2+bx+c>0$的解集为$(-1,2)$, 求关于$x$的不等式$a(x^2+1)+b(x-1)+c>2ax$的解集.
\item {\tiny (002785)}若关于$x$的不等式$(a^2-4)x^2+(a+2)x-1\ge 0$的解集为$\varnothing$, 求实数$a$的取值范围.
\item {\tiny (002790)}不等式$\dfrac{3x+4}{5-x}\ge 6$的解集是\blank{50}.
\item {\tiny (002791)}若不等式$\dfrac{2x+a}{x+b}\le 1$的解集为$\{x|1<x\le 3\}$, 则$a+b$的值是\blank{50}.
\item {\tiny (002792)}不等式$(x-1)^2(2-x)(x+1)\le 0$的解集是\blank{50}.
\item {\tiny (002793)}不等式$2<|x+1|<3$的解集是\blank{50}.
\item {\tiny (002794)}不等式$|x-2|>9x$的解集是\blank{50}.
\item {\tiny (002795)}不等式$4^{x-\frac 5x+1}\le 2$的解集是\blank{50}.
\item {\tiny (002796)}不等式$\log_{\frac 14}4{x^2}>\log_{\frac 14}(3-x)$的解集是\blank{50}.
\item {\tiny (002798)}(1) 关于$x$的不等式$|x-1|-|x-2|<a^2+a-1$的解集是$\mathbf{R}$, 求实数$a$取值范围;\\
(2) 关于$x$的不等式$|x-1|-|x-2|<a^2+a-1$有实数解, 求实数$a$的取值范围.
\item {\tiny (002799)}*设全集$U=\mathbf{R}$, 已知关于x的不等式$|x-1|+a-1>0$($a\in \mathbf{R}$)的解集为$A$, 若$\complement_U A\cap \mathbf{Z}$恰有$3$个元素, 求$a$的取值范围.
\item {\tiny (002800)}不等式$|\dfrac x{1+x}|>\dfrac x{1+x}$的解集是\blank{50}.
\item {\tiny (002801)}不等式$\dfrac{2x}{1-x}\le 1$的解集是\blank{50}.
\item {\tiny (002802)}不等式$\dfrac{1+|x|}{|x|-1}\ge 3$的解集是\blank{50}.
\item {\tiny (002804)}已知$a>0$且$a\ne 1$, 关于$x$的不等式$a^x>\dfrac 12$的解集是$(-\infty ,1)$, 则$a=$\blank{50}.
\item {\tiny (002805)}关于$x$的不等式$\log_{\frac 12}(x-\dfrac 1x)>0$的解集是\blank{50}.
\item {\tiny (002806)}若不等式$|3x-b|<4$的解集中的整数有且仅有$1$, $2$, $3$, 则$b$的取值范围为\blank{50}.
\item {\tiny (002807)}已知关于$x$的不等式$\dfrac{ax-5}{x^2-a}<0$的解集为$M$.\\
(1) 当$a=5$时, 求集合$M$;\\
(2) 若$2\in M$且$5\notin M$, 求实数$a$的取值范围.
\item {\tiny (002809)}(1) 若关于$x$的不等式$x^2-kx+1>0$的解集为$\mathbf{R}$, 求实数$k$的取值范围;\\
(2) *若关于$x$的不等式$x^2-kx+1>0$在$[1,2]$上有解, 求实数$k$的取值范围.
\item {\tiny (003675)}不等式$\dfrac{x-1}{x}>1$的解集为\blank{50}.
\item {\tiny (003716)}若函数$f(x)=ax^2+bx+c\ (a>0)$, 不等式$ax^2+bx+c<0$的解集为$\{x|-2<x<0\}$, 当$0<n<m$时, $f(n),f(m),f(\sqrt{mn}),f\left(\dfrac{m+n}2\right)$这四个值中最大的一个是\blank{50}.
\item {\tiny (003754)}定义区间$(c,d),(c,d],[c,d),[c,d]$的长度均为$d-c\ (d>c)$. 若$a\ne 0$, 关于$x$的不等式$x^2-\left(2a+\dfrac 1a\right)x-1<0$的非空解集(用区间表示)记为$I(a)$, 则当区间$I(a)$的长度取得最小值时, 实数$a$的值为\blank{50}.
\item {\tiny (003758)}已知$a\in\mathbf{R}$, 命题$P:$``实系数一元二次方程$x^2+ax+2=0$的两根都是虚数''; 命题$Q:$``存在复数$z$同时满足$|z|=2$且$|z+a|=1$''.
是判断命题$P$和命题$Q$之间是否存在推出关系? 说明你的理由.
\item {\tiny (003774)}已知集合$A=\left\{x\left|\dfrac{2x+1}{x+2}<1, \ x\in \mathbf{R}\right.\right\}$, 函数$f(x)=|mx+1| \ (m\in \mathbf{R})$. 函数$g(x)=x^2+ax+b \ (a,b\in \mathbf{R})$的值域为$[0,+\infty)$.\\
(1) 若不等式$f(x)<3$的解集为$A$, 求$m$的值;\\
(2) 在(1)的条件下, 若$\left|f(x)-2f\left(\dfrac x 2\right)\right|\le k$恒成立, 求$k$的取值范围;\\
(3) 若关于$x$的不等式$g(x)<c$的解集为$(m,m+6)$, 求实数$c$的值.
\item {\tiny (003777)}若存在实数$a$, 使得关于$x$的不等式$ax+b>x+1$的解集为$\{x|x<1\}$, 则实数$b$的取值范围为\blank{50}.
\item {\tiny (003861)}设$A(-1,0)$, $B(1,0)$, 条件甲: $A,B,C$是以$C$为直角顶点的三角形的三个顶点; 条件乙: $C$的坐标是方程$x^2+y^2=1$的解, 则甲是乙的\blank{30}.
\fourch{充分非必要条件}{必要非充分条件}{充要条件}{既不充分又不必要条件}
\item {\tiny (004125)}关于$x$的不等式$\dfrac{1}{x}>1$的解集为\blank{50}.
\item {\tiny (004249)}不等式$\dfrac 1{x-1}>1$的解集为\blank{50}.
\item {\tiny (004312)}不等式$|1-x|>1$的解集是\blank{50}.
\item {\tiny (004409)}不等式$\dfrac 1x\le 3$的解集是\blank{50}.
\item {\tiny (004422)}已知$a_1$、$a_2$与$b_1$、$b_2$是$4$个不同的实数, 关于x的方程$|x-a_1|+|x-a_2|=|x-b_1|+|x-b_2|$的解集为$A$, 则集合$A$中元素的个数为\bracket{20}.
\twoch{$1$个}{$0$个或$1$个或$2$个}{$0$个或$1$个或$2$个或无限个}{$1$个或无限个}
\item {\tiny (004469)}不等式$\dfrac 1{x-1}>1$的解集为\blank{50}.
\item {\tiny (004502)}已知两条直线$l_1$、$l_2$的方程分别为$l_1:ax+y-1=0$和$l_2:x-y+1=0$, 则``$a=1$''是``直线$l_1\perp l_2$''的\bracket{20}.
\fourch{充分不必要条件}{必要不充分条件}{充要条件}{既不充分也不必要条件}
\item {\tiny (004554)}不等式$|x+1|<5$的解集为\blank{50}.
\item {\tiny (004557)}已知二元线性方程组$\begin{cases}
    2x+2y=-1, \\ 4x+a^2y=a
\end{cases}$有无穷多解, 则实数$a=$\blank{50}.
\item {\tiny (004636)}已知$a$是常数, 设函数$f(x)=(a-2)x^2+2(a-2)x-4$.\\
(1) 解不等式: $f(x)>-4$;\\
(2) 求实数$a$的取值范围, 使得$f(x)<0$对任意$x\in [1,3]$恒成立;
\item {\tiny (004650)}已知常数$k,b,t\in \mathbf{R}$直线$f(x)=kx+b$与曲线$g(x)=\dfrac{t^2}x$交于点$M(m,-1)$, $N(n,2)$, 则不等式$f^{-1}(x)\ge g^{-1}(x)$的解集为\blank{50}.
\item {\tiny (004697)}已知非空集合$A,B$满足: $A\cup B=R$, $A\cap B=\varnothing$, 函数$f(x)=\begin{cases}
x^2, &  x\in A,  \\ 2x-1, &  x\in B.  \end{cases}$ 对于下列两个命题: \textcircled{1} 存在唯一的非空集合对$(A,B)$, 使得$f(x)$为偶函数; \textcircled{2} 存在无穷多非空集合对$(A,B)$, 使得方程$f(x)=2$无解. 下面判断正确的是\bracket{20}.
\fourch{\textcircled{1} 正确, \textcircled{2} 错误}{\textcircled{1} 错误, \textcircled{2} 正确}{\textcircled{1} 、\textcircled{2} 都正确}{\textcircled{1} 、\textcircled{2} 都错误}
\item {\tiny (004772)}在``\textcircled{1} 难解的题目, \textcircled{2} 方程$x^2+1=0$在实数集内的解, \textcircled{3} 直角坐标平面内第四象限的一些点, \textcircled{4} 很多多项式''中, 能够组成集合的是\bracket{20}.
\fourch{\textcircled{2}}{\textcircled{1}\textcircled{3}}{\textcircled{2}\textcircled{4}}{\textcircled{1}\textcircled{2}\textcircled{4}}
\item {\tiny (004775)}方程组$\begin{cases} 2x+y=0, \\ x-y+3=0 \end{cases}$的解集是\bracket{20}.
\fourch{$\{-1,2\}$}{$(-1,2)$}{$\{(-1,2)\}$}{$\{(x,y)|x=-1, \ y=2\}$}
\item {\tiny (004777)}用列举法表示下列各集合.\\
(1) 不大于$6$的非负数整数所组成的集合:\blank{50};\\
(2) 方程$x^3-x^2-x+1=0$的解所组成的集合:\blank{50};\\
(3) $\{y|y=x^2-1, \ |x|\le 2, \ x\in \mathbf{Z}\}$:\blank{50};\\
(4) $\{(x,y)|y =x^2-1, \  |x|\le 2,\ x\in \mathbf{Z}\}$:\blank{50};\\
(5) $\{(x,y)|x+y=5, \ x\in \mathbf{N},\ y\in \mathbf{Z}\}$:\blank{50}.
\item {\tiny (004818)}设集合$M=\{x|a_1x^2+b_1x+c_1=0\}$, $N=\{x|a_2x^2+b_2x+c_2=0\}$, 方程$(a_1x^2+b_1x+c_1)(a_2x^2+b_2x+c_2)=0$的解集是\bracket{20}.
\item {\tiny (004830)}设方程$x^2+px-12=0$的解集为$A$, 方程$x^2+qx+r=0$的解集为$B$, 且$A\ne B$, $A\cup B=\{-3,4\}$, $A\cap B=\{-3\}$, 求$p,q,r$的值.
\item {\tiny (004843)}下列语句哪些不是命题? 哪些是命题? 如果是命题, 那么它们是真命题还是假命题? 为什么?\\
(1) 你到过北京吗?\\
(2) 当$x=4$时, $2x<0$;\\
(3) 若$x\in \mathbf{R}$, 则方程$x^2-x+1=0$无实数根;\\
(4) $1+2=5$或$3\ge 3$;\\
(5) $x<-2$或$x>2$;\\
\item {\tiny (004848)}已知命题$\alpha$: 方程$x^2+mx+1=0$有两个相异负实数根, 命题$\beta$: $4x^2+4(m-2)x+1=0$无实数根, 命题$\alpha,\beta$有且只有一个为真命题, 求实数$m$的取值范围.
\item {\tiny (004865)}已知$\alpha :|a-1|<2$, $\beta:$方程$x^2+(a+2)x+1=0(x\in \mathbf{R})$没有正根, 求实数$a$的取值范围, 使$\alpha,\beta$有且只有一个为真命题.
\item {\tiny (004866)}已知关于$x$的方程$(x^2-1)^2-|x^2-1|+k=0$. 判断下列命题的真假:\\
(1) 存在实数$k$, 使得方程恰有$2$个不同的实数根;\\
(2) 存在实数$k$, 使得方程恰有$4$个不同的实数根;\\
(3) 存在实数$k$, 使得方程恰有$5$个不同的实数根;\\
(4) 存在实数$k$, 使得方程恰有$8$个不同的实数根.
\item {\tiny (004867)}如果$a,b,c$都是实数, 那么``$ac<0$''是``关于$x$的方程$ax^2+bx+c=0$有一个正根和一个负根''的\bracket{20}.
\twoch{必要不充分条件}{充分不必要条件}{充要条件}{既不充分也不必要条件}
\item {\tiny (004869)}设$\alpha ,\beta$是方程$x^2-ax+b=0$的两个实数根, 试分析``$a>2$且$b>1$''是``两根$\alpha ,\beta$均大于$1$''的什么条件.
\item {\tiny (004873)}已知$\triangle ABC$的三边为$a,b,c$求证: 关于$x$的方程$x^2+2ax+b^2=0$与$x^2+2cx-b^2=0$有公共根的充要条件是$A=90^\circ$.
\vspace*{16ex}
\item {\tiny (004880)}方程$ax^2+2x+1=0$至少有一个负实数根的充要条件是\bracket{20}.
\fourch{$0<a\le 1$}{$a>1$}{$a\le 1$}{$0<a\le 1$或$a<0$}
\item {\tiny (004913)}已知关于$x$的不等式$ax^2+bx+c<0$的解集是$\{x|x<-2\text{或}x>-\dfrac 12\}$, 求$ax^2-bx+c>0$的解集.
\item {\tiny (004918)}不等式$|x|-3<0$的解集是\bracket{20}.
\fourch{$\{x|x<\pm 3\}$}{$\{x|-3<x<3\}$}{$\{x|x>3\}$}{$\{x|x<-3\}$}
\item {\tiny (004922)}不等式$2x+3-x^2>0$的解集是\bracket{20}.
\fourch{$\{x|-\dfrac 32\le x<1\}$}{$\{x|-1<x<3\}$}{$\{x|1\le x<3\}$}{$\{x|-\dfrac 32\le x<3\}$}
\item {\tiny (004923)}不等式$6x^2+5x<4$的解集是\bracket{20}.
\fourch{$\{x|x<-\dfrac 43\text{或}x>\dfrac 12\}$}{$\{x|-\dfrac 43<x<\dfrac 12\}.$}{$\{x|-\dfrac 12<x<\dfrac 43\}.$}{$\{x|x<-\dfrac 12\text{或}x>\dfrac 43\}$}
\item {\tiny (004924)}当$a<0$时, 关于$x$的不等式$x^2-4ax-5a^2>0$的解集是\bracket{20}.
\fourch{$\{x|x>5a\text{或}x<-a\}$}{$\{x|x<5a\text{或}x>-a\}$}{$\{x|-a<x<5a\}$}{$\{x|5a<x<-a\}$}
\item {\tiny (004925)}若$x$为实数, 则下列命题正确的是\bracket{20}.
\onech{$x^2\ge 2$的解集是$\{x|x\ge \pm \sqrt 2\}$}{$(x-1)^2<2$的解集是$\{x|1-\sqrt 2<x<1+\sqrt 2\}$}{$x^2-9<0$的解集是$\{x|x<3\}$}{设$x_1,x_2$为$ax^2+bx+c=0$的两个实根, 且$x_1>x_2$, 则$ax^2+bx+c>0$的解集是$\{x|x_2<x<x_1\}$}
\item {\tiny (004926)}在\textcircled{1} $x^2-2x-3<0$与$\dfrac{x^2-2x}{x-1}<\dfrac 3{x-1}$; \textcircled{2} $x^2+3x-4>0$与$x^2+3x+\sqrt x>4+\sqrt x$; \textcircled{3} $\dfrac{(x+2)(x^2-1)}{x+2}>0$与$x^2-1>0$''三组不等式中, 解集相同的组数是\bracket{20}.
\fourch{$0$}{$1$}{$2$}{$3$}
\item {\tiny (004928)}直接写出下列不等式的解集:\\
(1) $(x-1)^2>0$:\blank{50};\\
(2) $(2-x)(3x+1)>0$:\blank{50};\\
(3) $1-3x^2>2x$:\blank{50};\\
(4) $1-2x-x^2\ge 0$:\blank{50};\\
(5) $x+\sqrt x-6<0$:\blank{50}.
\item {\tiny (004929)}直接写出下列不等式的解集:\\
(1) $\dfrac{3x+4}{x-2}\ge 0$:\blank{50};\\
(2) $\dfrac{4-2x}{1+3x}>0$:\blank{50};\\
(3) $\dfrac 1x>x$:\blank{50};\\	
(4) $x^2-2|x|-3>0$:\blank{50};\\
(5) $x^2-x-5>|2x-1|$:\blank{50}.
\item {\tiny (004933)}不等式$4\le x^2-3x<18$的整数解集是\blank{50}.
\item {\tiny (004935)}已知$a>b$, 直接写出下列不等式的解集:\\
(1) $\dfrac{x-a}{x-b}\ge 0$:\blank{50};\\
(2) $\dfrac{x-a}{x-b}<0$:\blank{50};\\
(3) $x^2-(a-b)x+ab>0$:\blank{50};\\
(4) $x^2-(a-b)x+ab<0$:\blank{50}.
\item {\tiny (004936)}若关于$x$的方程$2kx^2+(8k+1)x+8k=0$有两个不等实根, 则实数$k$的取值范围是\blank{50}.
\item {\tiny (004938)}不等式$\dfrac{x-1}{2x}\le 1$的解集是\bracket{20}.
\fourch{$\{x|x\ge -1\}$	}{$\{x|x\le -1\}$}{$\{x|-1\le x<0\}$}{$\{x|x\le -1\text{或}x>0\}$}
\item {\tiny (004939)}若关于$x$的二次不等式$mx^2+8mx+21<0$的解集是$\{x|-1<x<-1\}$, 则实数$m$的值等于\bracket{20}.
\fourch{$1$}{$2$}{$3$}{$4$}
\item {\tiny (004940)}若关于$x$的不等式$(a^2-3)x^2+5x-2>0$的解集是$\{x|\dfrac 12<x<2\}$, 则实数$a$的值等于\bracket{20}.
\fourch{$1$}{$-1$}{$\pm 1$}{$0$}
\item {\tiny (004941)}若关于$x$的不等式$ax^2+bx+c<0(a\ne 0)$的解集是空集, 则\bracket{20}.
\fourch{$a<0$且$b^2-4ac>0$}{$a<0$且$b^2-4ac\le 0$}{$a>0$且$b^2-4ac\le 0$}{$a>0$且$b^2-4ac>0$}
\item {\tiny (004944)}若关于$x$的二次方程$2(k+1)x^2+4kx+3k-2=0$的两根同号, 则$k$的取值范围是\bracket{20}.
\twoch{$-2<k<1$}{$-2\le k<-1$或$\dfrac 23<k\le 1$}{$k<-1$或$k>\dfrac 23$}{$-2<k<1$或$\dfrac 23<k<1$}
\item {\tiny (004945)}已知关于$x$的方程$(m+3)x^2-4mx+2m-1=0$的两根异号, 且负根的绝对值比正根大, 那么实数$m$的取值范围是\bracket{20}.
\fourch{$-3<m<0$}{$0<m<3$}{$m<-3$或$m>0$}{$m<0$或$m>3$}
\item {\tiny (004946)}若$\alpha ,\beta$是关于$x$的方程$x^2-(k-2)x+k^2+3k+5=0$($k$为实数)的两个实根, 则${{\alpha }^2}+{{\beta }^2}$的最大值等于\bracket{20}.
\fourch{$19$}{$18$}{$\dfrac{50}9$}{$-6$}
\item {\tiny (004948)}在三个关于$x$的方程$x^2-ax+4=0$, $x^2+(a-1)x+16=0$和$x^2+2ax+3a+10=0$中, 已知至少有一个方程有实根, 则实数$a$的取值范围是\bracket{20}.
\fourch{$-4\le a\le 4$}{$-2<a<4$}{$a\le -2$或$a\ge 4$}{$a<0$}
\item {\tiny (004949)}若关于$x$的二次方程$x^2-2mx+4x+2m^2-4m-2=0$有实根, 则其两根之积的最大值等于\blank{50}.
\item {\tiny (004950)}使关于$x$的方程$x^2-kx+2k-3=0$的两实根的平方和取最小值, 实数$k$的值等于\blank{50}.
\item {\tiny (004951)}若关于$x$的不等式$x^2-mx+n\le 0$的解集是$\{x|-5\le x\le 1\}$, 则实数$m=$\blank{50}, $n=$\blank{50}.
\item {\tiny (004952)}若关于$x$的不等式$ax^2+bx+1\ge 0$的解集是$\{x|-5\le x\le 1\}$, 则实数$a=$\blank{50}, $b=$\blank{50}.
\item {\tiny (004953)}若关于$x$的不等式$ax^2+bx+2>0$的解集是$\{x|-\dfrac 12<x<\dfrac 13\}$, 则实数$a=$\blank{50}, $b=$\blank{50}.
\item {\tiny (004954)}若关于$x$的不等式$ax^2+bx-6>0$的解集是$\{x|2<x<3\}$, 则实数$a=$\blank{50}, $b=$\blank{50}.
\item {\tiny (004955)}若关于$x$的不等式$(a+b)x+(2a-3b)<0$的解集是$\{x|x>3\}$, 则不等式$(a-3b)x+b-2a>0$的解集是\blank{50}.
\item {\tiny (004956)}若关于$x$的不等式$ax^2+bx+c<0$的解集是$\{x|x<-2\text{或}x>-\dfrac 12\}$, 则关于$x$的不等式$ax^2-bx+c>0$的解集是\blank{50}.
\item {\tiny (004957)}解不等式$x^4-2x^2+1>x^2-1$.
\item {\tiny (004958)}已知关于$x$的不等式$kx^2-2x+6k<0(k\ne 0)$.\\
(1) 若不等式的解集是$\{x|x<-3\text{或}x>-2\}$, 求实数$k$的值;\\
(2) 若不等式的解集是$\{x|x\ne \dfrac 1k\}$, 求实数$k$的值;\\
(3) 若不等式的解集是实数集, 求实数$k$的值.
\item {\tiny (004959)}已知关于$x$的方程$m(x-1)=3(x+2)$的解是正实数, 求实数$m$的取值范围.
\item {\tiny (004960)}已知关于$x$的方程$\dfrac 14x^2-kx+5k-6=0$无实数解, 求实数$k$的取值范围.
\item {\tiny (004961)}已知关于$x$的方程$kx^2-(3k-1)x+k=0$有两个正实数根, 求实数$k$的取值范围.
\item {\tiny (004968)}已知关于$x$的不等式$(a^2-4)x^2+(a+2)x-1\ge 0$的解集是空集, 求实数$a$的取值范围.
\item {\tiny (004969)}若关于$x$的不等式$\dfrac{x^2-8x+20}{mx^2+2(m+1)x+9m+4}<0$的解集为$\mathbf{R}$, 求实数$m$的取值范围.
\item {\tiny (004971)}既要使关于$x$的不等式$x^2+(m-\dfrac 12)x-\dfrac 7{16}\le 0$有实数解, 又要使关于$x$的方程$(2m+3)x^2+mx+\dfrac{m-2}4=0$有实数解, 求实数$m$的取值范围.
\item {\tiny (005079)}己知$\tan \alpha,\tan \beta$是关于$x$的方程$mx^2+(2m-3)x+(m-2)=0(m\ne 0)$的两根, 求证: $\tan (\alpha +\beta)\ge -\dfrac 34$.
\item {\tiny (005102)}若$a>0$, $b>0$, 且$a^3+b^3=2$, 试分别利用$x^3+y^3+z^3\ge 3xyz$($x,y,z\ge 0$)构造方程, 并利用判别式以及反证法证明: $a+b\le 2$.
\item {\tiny (005139)}已知关于$x$的不等式$ax^2+bx+c>0$的解集是$\{x|\alpha<x<\beta\}$, 其中$0<\alpha<\beta$, 求$cx^2+bx+a<0$的解集.
\item {\tiny (005140)}解不等式$(x+1)^2(x-1)(x-4)^3>0$.
\item {\tiny (005141)}解不等式$\dfrac{3x^2-14x+14}{x^2-6x+8}\ge 1$.
\item {\tiny (005142)}解不等式$\sqrt{x^2-3x+2}>x-3$.
\item {\tiny (005143)}解不等式$\sqrt{2x-1}<x-2$.
\item {\tiny (005144)}解不等式$|x^2-4|\le x+2$.
\item {\tiny (005145)}解不等式$|x^2-\dfrac 12|>2x$.
\item {\tiny (005147)}若关于$x$的不等式$2x-1>a(x-2)$的解集是$\mathbf{R}$, 则实数$a$的取值范围是\bracket{20}.
\fourch{$a>2$}{$a=2$}{$a<2$}{$a$不存在}
\item {\tiny (005148)}若关于$x$的不等式$ax^2+bx-2>0$的解集是$(-\infty ,-\dfrac 12)\cup (\dfrac 13,+\infty)$, 则$ab$等于\bracket{20}.
\fourch{$-24$}{$24$}{$14$}{$-14$}
\item {\tiny (005150)}若$q<0<p$, 则不等式$q<\dfrac 1x<p$的解集为\bracket{20}.
\twoch{$\{x|\dfrac 1q<x<\dfrac 1p,\  x\ne 0\}$}{$\{x|x<\dfrac 1q\text{或}x>\dfrac 1p\}$}{$\{x|-\dfrac 1p<x<-\dfrac 1q, \ x\ne 0\}$}{$\{x|\dfrac 1p<x<-\dfrac 1q\}$}
\item {\tiny (005151)}若关于$x$的不等式$(a+b)x+2a-3b<0$的解集是$\{x|x<-\dfrac 13\}$, 则$(a-3b)x+b-2a>0$的解集是\blank{50}.
\item {\tiny (005153)}若关于$x$的不等式$ax^2+bx+c>0$的解集是$\{x|3<x<5\}$, 则不等式$cx^2+bx+a<0$的解集是\blank{50}.
\item {\tiny (005154)}若关于$x$的不等式$\dfrac{x-a}{x^2-3x+2}\ge 0$的解集是$\{x|1<x\le ax>2\}$, 则实数$a$的取值范围是\blank{50}.
\item {\tiny (005155)}不等式$(x+2)(x+1)^2(x-1)^3(x-3)>0$的解集为:\blank{50}.
\item {\tiny (005156)}不等式$\dfrac{(x-1)^2(x+2)}{(x-3)(x-4)}\le 0$的解集为:\blank{50}.
\item {\tiny (005157)}不等式$x+1\le \dfrac 4{x+1}$的解集为:\blank{50}.
\item {\tiny (005158)}若不等式$f(x)\ge 0$的解集为$[1,2]$, 不等式$g(x)\ge 0$的解集为$\varnothing$, 则不等式$\dfrac{f(x)}{g(x)}$的解集是\bracket{20}.
\fourch{$\varnothing$}{$(-\infty ,1)\cup (2,+\infty)$}{$[1,2)$}{$\mathbf{R}$}
\item {\tiny (005159)}若关于$x$的不等式$ax^2-bx+c<0$的解集为$(-\infty ,\alpha)\cup (\beta ,+\infty)$, 其中$\alpha <\beta <0$, 则不等式$cx^2+bx+a>0$的解集为\bracket{20}.
\fourch{$(\dfrac 1{\beta},\dfrac 1{\alpha})$}{$(\dfrac 1{\alpha},\dfrac 1{\beta})$}{$(-\dfrac 1{\beta},-\dfrac 1{\alpha})$}{$(-\dfrac 1{\alpha},-\dfrac 1{\beta})$}
\item {\tiny (005162)}已知关于$x$的不等式$\sqrt x>ax+\dfrac 32$的解集是$\{x|4<x<b\}$, 求$a,b$的值.
\item {\tiny (005163)}已知$x=3$是不等式$ax>b$解集中的元素, 求实数$a,b$应满足的条件.
\item {\tiny (005168)}已知关于$x$的方程$3x^2+x\log_{\frac 12}^2a+2\log_{\frac 12}a=0$的两根$x_1,x_2$满足条件$-1<x_1<0<x_2<1$, 求实数$a$的取值范围.
\item {\tiny (005169)}已知关于$x$的方程$x^2+(m^2-1)x+m-2=0$的一个根比$-1$小, 另一个根比$1$大, 求参数$m$的取值范围.
\item {\tiny (005171)}不等式$\sqrt{x+3}>-1$的解集是\bracket{20}.
\fourch{$\{x|x>-2\}$}{$\{x|x\ge -3\}$}{$\varnothing$}{$\mathbf{R}$}
\item {\tiny (005172)}不等式$(x-1)\sqrt{x+2}\ge 0$的解集是\bracket{20}.
\fourch{$\{x|x>1\}$}{$\{x|x\ge 1\}$}{$\{x|x\ge 1\text{或}x=-2\}$}{$\{x|x>1\text{或}x=-2\}$}
\item {\tiny (005174)}解不等式: $\sqrt{x-5}+4x-3>3x+1+\sqrt{x-5}$.
\item {\tiny (005175)}解不等式: $\sqrt{x^2+1}>\sqrt{x^2-x+3}$.
\item {\tiny (005176)}解不等式: $(x-4)\sqrt{x^2-3x-4}\ge 0$.
\item {\tiny (005177)}解不等式: $\dfrac{x+1}{x+4}\sqrt{\dfrac{x+3}{1-x}}<0$.
\item {\tiny (005178)}解不等式: $\sqrt{x+2}+\sqrt{x-5}\ge \sqrt{5-x}$.
\item {\tiny (005179)}解不等式: $\sqrt{x-6}+\sqrt{x-3}\ge \sqrt{3-x}$.
\item {\tiny (005180)}解不等式: $\sqrt{2-x}<x$.
\item {\tiny (005181)}解不等式: $\sqrt{4-x^2}<x+1$.
\item {\tiny (005182)}解不等式: $\sqrt{3-2x}>x$.
\item {\tiny (005183)}解不等式: $\sqrt{(x-1)(2-x)}>4-3x$.
\item {\tiny (005184)}不等式$\sqrt{4-x^2}+\dfrac{|x|}x\ge 0$的解集是\bracket{20}.
\fourch{$[-2,2]$}{$[-\sqrt 3,0)\cup (0,2]$}{$[-2,0]\cup (0,2]$}{$[-\sqrt 3,0)\cup (0,\sqrt 3]$}
\item {\tiny (005185)}已知关于$x$的不等式$\sqrt{2x-x^2}>kx$的解集是$\{x|0<x\le 2\}$, 则实数$k$的取值范围是\bracket{20}.
\fourch{$k<0$}{$k\ge 0$}{$0<k<2$}{$-\dfrac 12<k<0$}
\item {\tiny (005186)}解不等式: $\sqrt{2x-4}-\sqrt{x+5}<1$.
\item {\tiny (005187)}解不等式: $\sqrt{x^2-5x-6}<|x-3|$.
\item {\tiny (005188)}解不等式: $|2\sqrt{x+3}-x+1|<1$.
\item {\tiny (005223)}不等式$|x|<\dfrac 1x$的解集为\bracket{20}.
\fourch{$\varnothing$}{$\{x|x<0\}$}{$\{x|0<x<1\}$}{$\{x|x<0\text{或}x\ge 1\}$}
\item {\tiny (005226)}不等式$|\dfrac x{1+x}|>\dfrac x{1+x}$的解集是\bracket{20}.
\fourch{$\{x|x\ne -1\}$}{$\{x|x>-1\}$}{$\{x|x<0\text{且}x\ne -1\}$}{$\{x|-1<x<0\}$}
\item {\tiny (005227)}解不等式: $x^2+|x|-6<0$.
\item {\tiny (005228)}解不等式: $x^2-2|x|-15>0$.
\item {\tiny (005229)}解不等式: $4<|1-3x|\le 7$.
\item {\tiny (005230)}解不等式: $|x-3|<x-1$
\item {\tiny (005233)}解不等式: $|x^2-5x+10|>x^2-8$.
\item {\tiny (005234)}解不等式: $|x^2-4|\le x+2$.
\item {\tiny (005235)}解不等式: $|x+1|<\dfrac 1{x-1}$.
\item {\tiny (005236)}解不等式: $|x+2|-|x-3|<4$.
\item {\tiny (005237)}解不等式: $|x+3|-|2x-1|<\dfrac x2+1$.
\item {\tiny (005239)}已知关于$x$的不等式$|x-4|+|x-3|<a$在实数集$\mathbf{R}$上的解集不是空集, 求正数$a$的取值范围.
\item {\tiny (005266)}解不等式: $\dfrac x{\sqrt{1+x^2}}+\dfrac{1-x^2}{1+x^2}>0$.
\item {\tiny (005269)}已知关于$x$的方程$a\sin^2x+\dfrac 12\cos x+\dfrac 12-a=0$在$0\le x<2\pi$内有两个相异的实根, 求实数$a$的取值范围.
\item {\tiny (007684)}用适当的方法表示下列集合:\\
(1) 方程$x^2-2=0$的实数解组成的集合;\\
(2) 两直线$y=2x+1$和$y=x-2$的交点组成的集合.
\item {\tiny (007719)}判断下列命题的真假, 并在相应的横线上填入``真命题''或``假命题''.\\
(1) 若$A\cap B\ne \varnothing$, $B\subset C$, 则$A\cap C\ne \varnothing$\blank{20};\\
(2) 方程$(a+1)x+b=0$($a$、$b\in \mathbf{R}$)的解为$x=-\dfrac b{a+1}$\blank{20};\\
(3)若命题$\alpha$、$\beta$、$\gamma$满足$\alpha \Rightarrow \beta$, $\beta \Rightarrow \gamma$, $\gamma \Rightarrow \alpha$, 则$\alpha \Leftrightarrow \gamma$\blank{20}.
\item {\tiny (007739)}如果命题$p$: $m<-3$, 命题$q$: 方程$x^2-x-m=0$无实数根, 那么$p$是$q$的什么条件?
\item {\tiny (007742)}已知$a$为实数, 写出关于$x$的方程$ax^2+2x+1=0$至少有一个实数根的一个充要条件、一个充分条件、一个必要条件.
\item {\tiny (007750)}若方程$x^2+px+4=0$的解集为$A$, 方程$x^2+x+q=0$的解集为$B$, 且$A\cap B=\{4\}$, 则集合$A\cup B$的所有子集是\blank{50}.
\item {\tiny (007762)}解不等式: $2(x+1)-3(x-2)>8$.
\item {\tiny (007763)}解不等式组: $\begin{cases} 3x-2(5-3x)>8, \\ 2x\le 2(2x+3). \end{cases}$
\item {\tiny (007774)}已知$a>2$, 解关于$x$的方程$ax+4<2x+a^2$.
\item {\tiny (007775)}已知$m<1$, 解关于$x$的方程$mx+1<x+m^3$.
\item {\tiny (007776)}已知$p\ne q$, 解关于$x$的方程$(p-q)x<p^2-q^2$.
\item {\tiny (007777)}解关于$x$的方程$mx+4<m^2+2x$.
\item {\tiny (007782)}解不等式: $2x^2-3x+1<0$.
\item {\tiny (007783)}解不等式: $(x+1)^2-6>0$.
\item {\tiny (007784)}解不等式: $x(x-1)<x(2x-3)+1$.
\item {\tiny (007785)}解不等式: $-x^2+2x+35>0$.
\item {\tiny (007786)}解不等式: $(x-2)(3-x)\le 0$.
\item {\tiny (007787)}解不等式: $2x-1\ge x^2$.
\item {\tiny (007790)}写出一个解集只含一个元素的一元二次不等式.
\item {\tiny (007791)}解不等式组: $\begin{cases} 6-x-x^2\le 0, \\ x^2+3x-4<0. \end{cases}$.
\item {\tiny (007792)}解不等式组: $\begin{cases} 4x^2-27x+18>0, \\ x^2-6x+4<0. \end{cases}$.
\item {\tiny (007794)}已知不等式$x^2+ax+b<0$的解集为$(-3,-1)$, 求实数$a$、$b$的值.
\item {\tiny (007795)}已知关于$x$的二次方程$2x^2+ax+1=0$无实数解, 求实数$a$的取值范围.
\item {\tiny (007798)}解不等式组: $\begin{cases} 3x^2+x-2\ge 0, \\ 4x^2-15x+9>0. \end{cases}$
\item {\tiny (007801)}已知关于$x$的不等式$ax^2+bx+c>0$的解集是$\{x|x>2$或$x<\dfrac 12\}$, 求关于$x$的不等式$ax^2-bx+c\le 0$的解集.
\item {\tiny (007803)}解不等式: $\dfrac 1x<1$.
\item {\tiny (007804)}解不等式: $\dfrac{4x+3}{x-1}>5$.
\item {\tiny (007805)}解不等式: $\dfrac 2x<\dfrac 2{x-3}$.
\item {\tiny (007806)}解不等式: $\dfrac 1{x-4}\le 1-\dfrac x{4-x}$.
\item {\tiny (007807)}求当$k$为何值时, 关于$x$的方程$\dfrac{4k-3x}{k+2}=2x$的解分别是:\\
(1) 正数;\\
(2) 负数.
\item {\tiny (007808)}解不等式: $|x^2-3|<2$.
\item {\tiny (007809)}解不等式: $|\dfrac 1{2-x}|\ge 2$.
\item {\tiny (007810)}解不等式: $|x^2-3x+2|\le 0$.
\item {\tiny (007811)}解不等式: $|\dfrac x{x+1}|>\dfrac x{x+1}$.
\item {\tiny (007812)}解不等式: $|x-3|<x-1$.
\item {\tiny (007813)}若$a<b<0$, 则不等式$\dfrac{x+a}{x+b}>0$的解集是\blank{50}.
\item {\tiny (007814)}解不等式: $4\le|x^2-4x|<5$.
\item {\tiny (007815)}解不等式: $\dfrac 1{|x|}>x$.
\item {\tiny (007816)}已知不等式$|ax+1|\le b$的解集是$[-1,3]$, 求$a$、$b$的值.
\item {\tiny (007836)}不等式$1+|x+1|<0$的解集是\bracket{20}.
\fourch{$(-\infty ,-2)$}{$(-2,0)$}{$\mathbf{R}$}{$\varnothing$}
\item {\tiny (007840)}解不等式: $2(x+1)(x+2)>(x+3)(x+4)$.
\item {\tiny (007841)}解不等式: $-3x^25x-4<0$.
\item {\tiny (007842)}解不等式: $4x^2-20x+25\le 0$.
\item {\tiny (007843)}解不等式: $x^2-16x+64>0$.
\item {\tiny (007844)}解不等式组: $\begin{cases} x^2-16<0, \\ x^2-4x+3\ge 0. \end{cases}$.
\item {\tiny (007845)}解不等式组: $4<x^2-x-2<10$.
\item {\tiny (007846)}解不等式: $|\dfrac{3x-9}2|\le 6$.
\item {\tiny (007847)}解不等式: $3<|x-2|<5$.
\item {\tiny (007848)}解不等式: $|\dfrac 1x|<\dfrac 45$.
\item {\tiny (007849)}下列四对不等式(组)中, 哪几对具有相同的解集?\\
(1) $-\dfrac 12x^2+3x+\dfrac{27}2>0$与$x^2-6x-27>0$;\\
(2) $4<x^2-x+2<10$与$\begin{cases} x^2-x+2<10, \\ x^2-x+2>4; \end{cases}$\\
(3) $|2x+1|<5$与$2x+1<5$或$2x+1>-5$;\\
(4) $\dfrac{x-1}{x+1}<2$与$x-1<2(x+1)$.
\item {\tiny (007850)}已知关于$x$的不等式$2x^2-2(a-1)x+(a+3)>0$的解集是$\mathbf{R}$, 求实数$a$的取值范围.
\item {\tiny (007852)}当$k$是什么实数时, 关于$x$的方程$2x+k(x+3)=4$的解是正数?
\item {\tiny (007857)}当$k$为什么实数时, 方程组$\begin{cases} 3x-6y=1, \\ 5x-ky=2 \end{cases}$的解满足$x<0$且$y<0$的条件?
\item {\tiny (007858)}当$k$为什么实数时, 方程组$\begin{cases} 4x+3y=60, \\ kx+(k+2)y=60 \end{cases}$的解满足$x>y>0$的条件?
\item {\tiny (007859)}已知$m<n$, 试写出一个形如$ax^2+bx+c>0$的一元二次不等式, 使它的解集分别为:\\
(1) $(-\infty ,m)\cup (n,+\infty)$;\\
(2) $(m,n)$.
\item {\tiny (007991)}已知关于$x$的不等式$ax^2+3ax-2<0$的解集为$\mathbf{R}$, 求实数$a$的取值范围.
\item {\tiny (009445)}设$a\in \mathbf{R}$, 求关于$x$的方程$ax=a^2+x-1$的解集.
\item {\tiny (009446)}设$k\in \mathbf{R}$, 求关于$x$与$y$的二元一次方程组$\begin{cases}y=kx+1, \\ y=2kx+3 \end{cases}$的解集.
\item {\tiny (009447)}求一元二次方程$ax^2-4x+2=0$($a\ne 0$)的解集.
\item {\tiny (009448)}已知方程$2x^2+4x-3=0$的两个根为$x_1$、$x_2$, 求下列各式的值:\\
(1) $x_1^2x_2+x_2^2x_1$;\\
(2) $\dfrac1{x_1}+\dfrac1{x_2}$;\\
(3) $x_1^2+x_2^2$;\\
(4) $x_1^3+x_2^3$.
\item {\tiny (009454)}填空题:\\
(1) $(x-2)(x+3)<0$的解集是\blank{50};\\
(2) $(2-x)(x+3)<0$的解集是\blank{50};\\
(3) $(x-2)(x+3)\ge 0$的解集是\blank{50}.
\item {\tiny (009455)}求下列不等式的解集:\\
(1) $-8x\le 3x^2+4$;\\
(2) $-x^2<2x-4$.
\item {\tiny (009457)}写出一个一元二次不等式, 使它的解集分别为:\\
(1) $(3-\sqrt 2, 3+\sqrt 2)$;\\
(2) $(-\infty, 3-\sqrt 2]\cup [3+\sqrt 2, +\infty)$;\\
(3) $\mathbf{R}$;\\
(4) $\varnothing$.
\item {\tiny (009458)}求下列不等式组的解集:\\
(1) $\begin{cases} x^2-2x-3>0, \\ x-1>0; \end{cases}$\\
(2) $\begin{cases} x^2-2x-15\ge 0, \\ x^2-4x-12<0. \end{cases}$\\
\item {\tiny (009459)}若关于$x$的不等式$x^2-x+m<0$的解集为$\varnothing$, 求实数$m$的取值范围.
\item {\tiny (009460)}已知一元二次不等式$x^2-ax-b<0$的解集为$(2, 3)$, 求实数$a$、$b$的值及不等式$bx^2-ax-1>0$的解集.
\item {\tiny (010036)}判断下列命题的真假, 并说明理由:\\
(1) 若$A\cap B=\varnothing$, $C\subset B$, 则$A\cap C=\varnothing$;\\
(2) 若$a$、$b\in \mathbf{R}$, 则关于$x$的方程$(a+1)x+b=0$的解为$x=- \dfrac b{a+1}$.
\item {\tiny (010037)}已知$a$为实数. 写出关于$x$的方程$ax^2+2x+1=0$至少有一个实根的一个充要条件、一个充分非必要条件和一个必要非充分条件.
\item {\tiny (010040)}设$a\in \mathbf{R}$, 求关于$x$的方程$ax=2$的解集.
\item {\tiny (010041)}设$k\in \mathbf{R}$, 求关于$x$与$y$的二元一次方程组$\begin{cases}y=-2x+1,\\  y=kx-3\end{cases}$的解集.
\item {\tiny (010042)}设$a\in \mathbf{R}$, 求一元二次方程$x^2-2ax+a^2-4=0$的解集.
\item {\tiny (010044)}已知一元二次方程$ax^2+bx+c=0$($a\ne 0$)的两实根为$x_1$、$x_2$, 求证: $|x_2-x_1| = \dfrac{\sqrt{b^2-4 ac}}{|a|}$.
\item {\tiny (010045)}已知一元二次方程$x^2+3x-3=0$的两个实根分别为$x_1$、$x_2$, 求作二次项系数是$1$, 且分别以下列数值为根的一元二次方程:\\
(1) $-x_1, -x_2$;\\
(2) $2x_1+1, 2x_2+1$;\\
(3) $\dfrac 1{x_1}, \dfrac 1{x_2}$;\\
(4) $x_1^2, x_2^2$.
\item {\tiny (010057)}设$a$为实数, 求关于$x$的方程$2x+a^2=ax+4$的解集.
\item {\tiny (010058)}设$m$为实数, 求关于$x$的方程$(m+1)x^2+6mx+9m=1$的解集.
\item {\tiny (010060)}对一元二次方程$ax^2+bx+c=0$($a\ne 0$), 证明: $ac<0$是该方程有两个异号实根的充要条件.
\item {\tiny (010061)}已知一元二次方程$2x^2+x-3=0$的两个实根分别为$x_1$、$x_2$, 求作二次项系数是$1$, 且分别以下列数值为根的一元二次方程:\\
(1) $x_1+x_2, x_1x_2$;\\
(2) $2x_1^2+1, 2x_2^2+1$;\\
(3) $\dfrac{x_2}{x_1}$, $\dfrac{x_1}{x_2}$;\\
(4) $x_1^4$, $x_2^4$.
\item {\tiny (010062)}已知一元二次方程$x^2-2mx+m-1=0$的两实根为$x_1$、$x_2$, 且$x_1^2+x_2^2=4$. 求实数$m$的值.
\item {\tiny (010070)}已知下列关于$x$的方程有两个不同实根, 求实数$k$的取值范围:\\
(1) $x^2+(k+3)x+k^2=0$;\\
(2) $3x^2+2kx+k=0$.
\item {\tiny (010071)}若下列关于$x$的方程有实数解, 求实数$k$的取值范围:\\
(1) $x^2+kx-k+3=0$;\\
(2) $x^2+2\sqrt 2x+k(k-1)=0$.
\item {\tiny (010074)}已知关于$x$的一元二次方程$2x^2+ax+1=0$无实数解, 求实数$a$的取值范围.
\item {\tiny (010075)}已知关于$x$的一元二次不等式$x^2+ax+b<0$的解集为$(-3, -1)$, 求实数$a$及$b$的值.
\item {\tiny (010078)}当关于$x$的方程$4k-3x=2(k+2)x$的解分别满足以下条件时, 求实数$k$的取值范围.\\
(1) 正数;\\
(2) 负数.
\item {\tiny (010083)}已知关于$x$的不等式$x^2+bx+c>0$的解集是$(-\infty, \dfrac 12)\cup(2, +\infty)$, 求实数$b$及$c$的值, 并求$x^2-bx+c\le 0$的解集.
\item {\tiny (010096)}设$x\in \mathbf{R}$, 求方程$|x-2|+|2x-3|=|3x-5|$的解集.
\item {\tiny (020001)}判断下列各组对象能否组成集合, 若能组成集合, 指出是有限集还是无限集.\\
(1) 上海市控江中学$2022$年入学的全体高一年级新生;\\
(2) 中国现有各省的名称;\\
(3) 太阳、$2$、上海市;\\
(4) 大于$10$且小于$15$的有理数;\\
(5) 末位是$3$的自然数;\\
(6) 影响力比较大的中国数学家;\\
(7) 方程$x^2+x+3=0$的所有实数解;\\ 
(8) 函数$y=\dfrac 1x$图像上所有的点;\\ 
(9) 在平面直角坐标系中, 到定点$(0, 0)$的距离等于$1$的所有点;\\
(10) 不等式$3x-10<0$的所有正整数解;\\
(11) 所有的平面四边形.
\item {\tiny (020004)}已知关于$x$的方程$\sqrt {x^2+4x+a}=x+2$, 若以该方程的所有解为元素组成的集合是无限集, 求实数$a$满足的条件.
\item {\tiny (020007)}用区间表示下列集合:\\
(1) $\{x|-2<x<7\}$;\\
(2) $\{x|-2\le\ x\le7\}$;\\
(3) $\{x|-2\le\ x<7\}$;\\
(4) 不等式$2x<5$的解集;\\
(5) 不等式$-x<5$的解集; \\
(6) 非负实数集.
\item {\tiny (020008)}用适当的方法表示下列集合:\\
(1) 能被$10$整除的所有正整数组成的集合;\\
(2) 能整除$10$的所有正整数组成的集合;\\
(3) 方程$x^2+2=0$的实数解组成的集合;\\
(4) 方程组$\begin{cases}2x+y=0, \\ x-y+3=0\end{cases}$的所有解组成的集合;\\
(5) 两直线$y=2x+1$和$y=x-2$的交点组成的集合.
\item {\tiny (020071)}判断下列命题的真假, 并在相应的括号内填入``真''或``假''.\\
(1) $2\sqrt 3>3\sqrt 2$或$1\le 1$;\blank{50};\\
(2) $2\sqrt 3>3\sqrt 2$且$1\le1$;\blank{50};\\
(3) 如果$a$、$b$都是奇数, 那么$ab$也是奇数;\blank{50};\\
(4) $\{1\}$是$\{0, 1, 2\}$的真子集;\blank{50};\\
(5) $1$是$\{0, 1, 2\}$的真子集;\blank{50};\\
(6) 若$x<-2$或$x>2$, 则$x^2>1$;\blank{50};\\
(7) 如果$|a|<2$, 那么$a<2$;\blank{50};\\
(8) 对任意实数$a,b$, 方程$(a+1)x+b=0$的解为$x=-\dfrac b{a+1}$;\blank{50};\\
(9) 若命题$\alpha$、$\beta$、$\gamma$满足$\alpha\Rightarrow \beta$, $\beta\Rightarrow \gamma$, $\gamma\Rightarrow \alpha$, 则$\alpha\Leftrightarrow \gamma$;\blank{50};\\
(10) 若关于$x$的方程$ax^2+bx+c=0$($a\ne 0$)的两实数根之积是正数, 则$ac>0$;\blank{50};\\
(11) 若某个整数不是偶数, 则这个数不能被$4$整除;\blank{50};\\
(12) 合数一定是偶数;\blank{50};\\
(13) 所有的偶数都是素数或合数;\blank{50};\\
(14) 所有的偶数都是素数或所有的偶数都是合数;\blank{50};\\
(15) 如果$A\subset B$, $B\supset C$, 那么$A=C$;\blank{50};\\
(16) 空集是任何集合的真子集;\blank{50};\\
(17) 若$x\in \mathbf{R}$, 则方程$x^2-x+1=0$不成立;\blank{50};\\
(18) 若$A\cap B\ne \varnothing$, $B\subset C$, 则$A\cap C\ne \varnothing$;\blank{50};\\
(19) 存在一个三角形, 它的任意两边的平方和小于第三边的平方;\blank{50};\\
(20) 对于任意一个三角形, 存在一组两边的平方和不等于第三边的平方;\blank{50}.
\item {\tiny (020075)}已知$a$是常数, 命题$\alpha :-1<a<3$, $\beta$: 关于$x$的方程$x+a=0$($x\in \mathbf{R}$)没有正根, 若命题$\alpha$、$\beta$有且只有一个是真命题, 求实数$a$的取值范围.
\item {\tiny (020080)}关于$x$的方程$ax^2=0$至少有一个实数根的一个充要条件是\blank{50}.
\item {\tiny (020085)}设$\alpha,\beta$是方程$x^2-ax+b=0$的两个实数根. 试分析$a>2$且$b>1$是``两个实数根$\alpha,\beta$均大于$1$''的什么条件? 并证明你的结论.
\item {\tiny (020088)}在横线上写出下列命题的否定形式, 并判断命题真假, 在相应的位置中填入``真''或``假''.\\
(1) $\pi$是无理数; \blank{20}; \blank{150}; \blank{20};\\
(2) $2+1=4$;  \blank{20}; \blank{150}; \blank{20};\\
(3) 任何实数是正数或负数;  \blank{20}; \blank{150}; \blank{20};\\
(4) 任何实数是正数或任何实数是负数;  \blank{20}; \blank{150}; \blank{20};\\
(5) 对一切实数$x, x^3+1=0$;  \blank{20}; \blank{150}; \blank{20};\\
(6) 存在实数$x, x^3+1=0$;  \blank{20}; \blank{150}; \blank{20};\\
(7) 对于任意实数$k$, 关于$x$的方程$x^2+x+k=0$都有实数根;  \blank{20}; \blank{250}; \blank{20};\\
(8) 任何三角形中至多有一个钝角;  \blank{20}; \blank{150}; \blank{20};\\
(9) 若$a>1$, $b>1$, 则$ab>1$;  \blank{20}; \blank{150}; \blank{20};\\
(10) 能被$2$整除的整数是质数;  \blank{20}; \blank{150}; \blank{20}.\\
\end{enumerate}



\end{document}