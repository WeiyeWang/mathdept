\documentclass[10pt,a4paper]{article}
\usepackage[UTF8,fontset = windows]{ctex}
\setCJKmainfont[BoldFont=黑体,ItalicFont=楷体]{华文中宋}
\usepackage{amssymb,amsmath,amsfonts,amsthm,mathrsfs,dsfont,graphicx}
\usepackage{ifthen,indentfirst,enumerate,color,titletoc}
\usepackage{tikz}
\usepackage{multicol}
\usepackage{makecell}
\usepackage{longtable}
\usetikzlibrary{arrows,calc,intersections,patterns,decorations.pathreplacing,3d,angles,quotes,positioning}
\usepackage[bf,small,indentafter,pagestyles]{titlesec}
\usepackage[top=1in, bottom=1in,left=0.8in,right=0.8in]{geometry}
\renewcommand{\baselinestretch}{1.65}
\newtheorem{defi}{定义~}
\newtheorem{eg}{例~}
\newtheorem{ex}{~}
\newtheorem{rem}{注~}
\newtheorem{thm}{定理~}
\newtheorem{coro}{推论~}
\newtheorem{axiom}{公理~}
\newtheorem{prop}{性质~}
\newcommand{\blank}[1]{\underline{\hbox to #1pt{}}}
\newcommand{\bracket}[1]{(\hbox to #1pt{})}
\newcommand{\onech}[4]{\par\begin{tabular}{p{.9\textwidth}}
A.~#1\\
B.~#2\\
C.~#3\\
D.~#4
\end{tabular}}
\newcommand{\twoch}[4]{\par\begin{tabular}{p{.46\textwidth}p{.46\textwidth}}
A.~#1& B.~#2\\
C.~#3& D.~#4
\end{tabular}}
\newcommand{\vartwoch}[4]{\par\begin{tabular}{p{.46\textwidth}p{.46\textwidth}}
(1)~#1& (2)~#2\\
(3)~#3& (4)~#4
\end{tabular}}
\newcommand{\fourch}[4]{\par\begin{tabular}{p{.23\textwidth}p{.23\textwidth}p{.23\textwidth}p{.23\textwidth}}
A.~#1 &B.~#2& C.~#3& D.~#4
\end{tabular}}
\newcommand{\varfourch}[4]{\par\begin{tabular}{p{.23\textwidth}p{.23\textwidth}p{.23\textwidth}p{.23\textwidth}}
(1)~#1 &(2)~#2& (3)~#3& (4)~#4
\end{tabular}}
\begin{document}

\begin{enumerate}[1.]

\item {\tiny (001085)}判断题: (如果正确请在题目前面的横线上写``\checkmark'', 错误请在题目前面的横线上写``$\times$'')\\ 
\blank{30}(1) 若$a>b$, $c=d$, 则$ac>bd$;\\ 
\blank{30}(2) 若$\dfrac{a}{c^2}<\dfrac{b}{c^2}$, 则$a<b$;\\ 
\blank{30}(3) 若$ac<bc$, 则$a<b$;\\ 
\blank{30}(4) 若$a>b$, 则$ac^2>bc^2$;\\ 
\blank{30}(5) 若$a>b,c<d$, 则$ac>bd$;\\ 
\blank{30}(6) 若$a>b>0$, $c>d>0$, 则$\dfrac{a}{c}>\dfrac{b}{d}$;\\ 
\blank{30}(7) 若$a>b$, $c\geq d$, 则$a+c>b+d$;\\ 
\blank{30}(8) 若$a>b$, $c\geq d$, 则$a+c\geq b+d$;\\ 
\blank{30}(9) 若$\sqrt[3]{a}>\sqrt[3]{b}$, 则$a>b$.\\ 
\blank{30}(10) 若$ab^2\geq 0$, 则$a\geq 0$.
\item {\tiny (001095)}证明或否定: ``$\sqrt{f(x)}>g(x)$''和
``$\left\{\begin{array}{l}f(x)>g^2(x),\\g(x)\ge 0,\end{array}\right.\ \text{或} \ \left\{\begin{array}{l}f(x)\ge 0,\\g(x)<0\end{array}\right.$''
同解.
\item {\tiny (001143)}已知$a,b,c$是{\bf 不全相等}的正数, 求证: $(ab+a+b+1)(ab+bc+ca+c^2)> 16abc$.
\item {\tiny (002750)}命题\textcircled{1} $a>b\Rightarrow ac^2>bc^2$; \textcircled{2} $ac^2>bc^2\Rightarrow a>b$; \textcircled{3} $a>b\Rightarrow \dfrac 1a<\dfrac 1b$; \textcircled{4} $a<b<0, \ c<d<0\Rightarrow ac>bd$;   \textcircled{5} $\sqrt[n]a>\sqrt[n]b\Rightarrow a>b \ (n\in \mathbf{N}^*)$; \textcircled{6} $a+c<b+d\Leftrightarrow \begin{cases} a<b, \\ c<d; \end{cases}$ \textcircled{7} $a<b<0\Rightarrow a^2>ab>b^2$. 其中真命题的序号是\blank{50}.
\item {\tiny (004985)}已知实数$a,b,c$满足$a+b+c=0$和$abc=2$, 求证: $a,b,c$中至少有一个不小于2.
\item {\tiny (007766)}如果$a<b<0$, 那么下列不等式中正确的是\bracket{20}.
\fourch{$\dfrac{-a}{-b}<1$}{$a^2>ab$}{$\dfrac 1{b^2}<\dfrac 1{a^2}$}{$\dfrac 1a<\dfrac 1b$}
\item {\tiny (007767)}如果$a<0<b$, 那么下列不等式中正确的是\bracket{20}.
\fourch{$\sqrt {-a}<\sqrt b$}{$a^2<b^2$}{$a^3<b^3$}{$ab>b^2$}
\item {\tiny (007770)}用``$>$''或``$<$''号填空: 如果$a<b<0$, 那么\\
(1) $\sqrt [n]{-a}$\blank{50}$\sqrt [n]{-b}(n\ge 2,n\in \mathbf{N}^*)$;\\
(2) $\dfrac 1{a^{2n}}$\blank{50}$\dfrac 1{b^{2n}}(n\in \mathbf{N}^*)$.
\end{enumerate}



\end{document}