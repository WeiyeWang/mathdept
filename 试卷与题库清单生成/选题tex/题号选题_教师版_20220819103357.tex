\documentclass[10pt,a4paper]{article}
\usepackage[UTF8,fontset = windows]{ctex}
\setCJKmainfont[BoldFont=黑体,ItalicFont=楷体]{华文中宋}
\usepackage{amssymb,amsmath,amsfonts,amsthm,mathrsfs,dsfont,graphicx}
\usepackage{ifthen,indentfirst,enumerate,color,titletoc}
\usepackage{tikz}
\usepackage{multicol}
\usepackage{makecell}
\usepackage{longtable}
\usetikzlibrary{arrows,calc,intersections,patterns,decorations.pathreplacing,3d,angles,quotes,positioning}
\usepackage[bf,small,indentafter,pagestyles]{titlesec}
\usepackage[top=1in, bottom=1in,left=0.8in,right=0.8in]{geometry}
\renewcommand{\baselinestretch}{1.65}
\newtheorem{defi}{定义~}
\newtheorem{eg}{例~}
\newtheorem{ex}{~}
\newtheorem{rem}{注~}
\newtheorem{thm}{定理~}
\newtheorem{coro}{推论~}
\newtheorem{axiom}{公理~}
\newtheorem{prop}{性质~}
\newcommand{\blank}[1]{\underline{\hbox to #1pt{}}}
\newcommand{\bracket}[1]{(\hbox to #1pt{})}
\newcommand{\onech}[4]{\par\begin{tabular}{p{.9\textwidth}}
A.~#1\\
B.~#2\\
C.~#3\\
D.~#4
\end{tabular}}
\newcommand{\twoch}[4]{\par\begin{tabular}{p{.46\textwidth}p{.46\textwidth}}
A.~#1& B.~#2\\
C.~#3& D.~#4
\end{tabular}}
\newcommand{\vartwoch}[4]{\par\begin{tabular}{p{.46\textwidth}p{.46\textwidth}}
(1)~#1& (2)~#2\\
(3)~#3& (4)~#4
\end{tabular}}
\newcommand{\fourch}[4]{\par\begin{tabular}{p{.23\textwidth}p{.23\textwidth}p{.23\textwidth}p{.23\textwidth}}
A.~#1 &B.~#2& C.~#3& D.~#4
\end{tabular}}
\newcommand{\varfourch}[4]{\par\begin{tabular}{p{.23\textwidth}p{.23\textwidth}p{.23\textwidth}p{.23\textwidth}}
(1)~#1 &(2)~#2& (3)~#3& (4)~#4
\end{tabular}}
\begin{document}

\begin{enumerate}[1.]

\item { (002802)}不等式$\dfrac{1+|x|}{|x|-1}\ge 3$的解集是\blank{50}.


关联目标:

暂未关联目标



标签: 第一单元

答案: 暂无答案

解答或提示: 暂无解答与提示

使用记录:

暂无使用记录


出处: 2022届高三第一轮复习讲义
\item { (002803)}设函数$f(x)=\begin{cases} 2^{-x}-1, & x\le 0, \\ x^{\frac 12}, & x>0, \end{cases}$ 若$f(x_0)>1$, 则$x_0$的取值范围是\blank{50}.


关联目标:

暂未关联目标



标签: 第一单元

答案: 暂无答案

解答或提示: 暂无解答与提示

使用记录:

暂无使用记录


出处: 2022届高三第一轮复习讲义
\item { (002804)}已知$a>0$且$a\ne 1$, 关于$x$的不等式$a^x>\dfrac 12$的解集是$(-\infty ,1)$, 则$a=$\blank{50}.


关联目标:

暂未关联目标



标签: 第一单元

答案: 暂无答案

解答或提示: 暂无解答与提示

使用记录:

暂无使用记录


出处: 2022届高三第一轮复习讲义
\item { (002805)}关于$x$的不等式$\log_{\frac 12}(x-\dfrac 1x)>0$的解集是\blank{50}.


关联目标:

暂未关联目标



标签: 第一单元

答案: 暂无答案

解答或提示: 暂无解答与提示

使用记录:

暂无使用记录


出处: 2022届高三第一轮复习讲义
\item { (002806)}若不等式$|3x-b|<4$的解集中的整数有且仅有$1$, $2$, $3$, 则$b$的取值范围为\blank{50}.


关联目标:

暂未关联目标



标签: 第一单元

答案: 暂无答案

解答或提示: 暂无解答与提示

使用记录:

暂无使用记录


出处: 2022届高三第一轮复习讲义
\item { (002808)}(1) 对任意实数$x$, $|x-1|-|x+3|>a$恒成立, 求实数$a$的取值范围;\\
(2) *对任意实数$x$, $|x-1|-|x+3|>a$恒不成立, 求实数$a$的取值范围.


关联目标:

暂未关联目标



标签: 第一单元

答案: 暂无答案

解答或提示: 暂无解答与提示

使用记录:

暂无使用记录


出处: 2022届高三第一轮复习讲义
\item { (002809)}(1) 若关于$x$的不等式$x^2-kx+1>0$的解集为$\mathbf{R}$, 求实数$k$的取值范围;\\
(2) *若关于$x$的不等式$x^2-kx+1>0$在$[1,2]$上有解, 求实数$k$的取值范围.


关联目标:

暂未关联目标



标签: 第一单元

答案: 暂无答案

解答或提示: 暂无解答与提示

使用记录:

暂无使用记录


出处: 2022届高三第一轮复习讲义
\item { (002810)}已知$a,b\in \mathbf{R}^+$, 求证: $\dfrac a{\sqrt b}+\dfrac b{\sqrt a}\ge \sqrt a+\sqrt b$.


关联目标:

暂未关联目标



标签: 第一单元

答案: 暂无答案

解答或提示: 暂无解答与提示

使用记录:

暂无使用记录


出处: 2022届高三第一轮复习讲义
\item { (002811)}已知$x,y\in \mathbf{R}$, 求证: $x^2+y^2+1\ge x+y+xy$.


关联目标:

暂未关联目标



标签: 第一单元

答案: 暂无答案

解答或提示: 暂无解答与提示

使用记录:

暂无使用记录


出处: 2022届高三第一轮复习讲义
\item { (002813)}已知$0<a<1$ ,$0<b<1$, $0<c<1$, 求证: $(1-a)b,(1-b)c,(1-c)a$中至少有一个小于等于$\dfrac 14$.


关联目标:

暂未关联目标



标签: 第一单元

答案: 暂无答案

解答或提示: 暂无解答与提示

使用记录:

暂无使用记录


出处: 2022届高三第一轮复习讲义
\item { (002814)}$a$、$b$、$c$是互不相等的正数, 则下列不等式中不正确的序号是\blank{50}.\\
(1) $|a-b|\le |a-c|+|c-b|$; (2) ${a^2}+\dfrac 1{a^2}\ge a+\dfrac 1a$; (3) $|a-b|+\dfrac 1{a-b}\ge 2$; (4) $\sqrt{a+3}-\sqrt{a+1}\le \sqrt{a+2}-\sqrt a$.


关联目标:

暂未关联目标



标签: 第一单元

答案: 暂无答案

解答或提示: 暂无解答与提示

使用记录:

暂无使用记录


出处: 2022届高三第一轮复习讲义
\item { (002815)}已知$a>b>c>0$, 试比较$\dfrac{a-c}b$与$\dfrac{b-c}a$的大小.


关联目标:

暂未关联目标



标签: 第一单元

答案: 暂无答案

解答或提示: 暂无解答与提示

使用记录:

暂无使用记录


出处: 2022届高三第一轮复习讲义
\item { (002816)}已知$a>0$, 试比较$a$与$\dfrac 1a$的大小.


关联目标:

暂未关联目标



标签: 第一单元

答案: 暂无答案

解答或提示: 暂无解答与提示

使用记录:

暂无使用记录


出处: 2022届高三第一轮复习讲义
\item { (002817)}若$x,y,m,n$均为正数, 求证: $\sqrt{(m+n)(x+y)}\ge \sqrt{mx}+\sqrt{ny}$.


关联目标:

暂未关联目标



标签: 第一单元

答案: 暂无答案

解答或提示: 暂无解答与提示

使用记录:

暂无使用记录


出处: 2022届高三第一轮复习讲义
\item { (002818)}已知$a,b,c\in \mathbf{R}^+$, 求证: $a^2b^2+b^2c^2+c^2a^2\ge a^2bc+ab^2c+abc^2$.


关联目标:

暂未关联目标



标签: 第一单元

答案: 暂无答案

解答或提示: 暂无解答与提示

使用记录:

暂无使用记录


出处: 2022届高三第一轮复习讲义
\item { (002819)}设$f(x)=\sqrt{1+x}\ (x>0)$. 若$x_1\ne x_2$, 求证: $|f(x_1)-f(x_2)|<|x_1-x_2|$.


关联目标:

暂未关联目标



标签: 第一单元

答案: 暂无答案

解答或提示: 暂无解答与提示

使用记录:

暂无使用记录


出处: 2022届高三第一轮复习讲义
\item { (002820)}若实数$x$、$y$、$m$满足$|x-m|>|y-m|$, 则称$x$比$y$远离$m$.\\
(1) 若$x^2-1$比$1$远离$0$, 求$x$的取值范围;\\
(2)定义: 在$\mathbf{R}$上的函数$f(x)$等于$x^2$和$x+2$中远离$0$的那个值. 求证: $f(x)\ge 1$在$\mathbf{R}$上恒成立.


关联目标:

暂未关联目标



标签: 第一单元

答案: 暂无答案

解答或提示: 暂无解答与提示

使用记录:

暂无使用记录


出处: 2022届高三第一轮复习讲义
\item { (005030)}利用公式$\dfrac{a+b+c}3\le \sqrt{\dfrac{a^2+b^2+c^2}3}$, 求证: $\sqrt{a^2}+{b^2}+\sqrt{b^2}+{c^2}+\sqrt{c^2}+{a^2}\ge \sqrt 2(a+b+c)$.


关联目标:

暂未关联目标



标签: 第一单元

答案: 暂无答案

解答或提示: 暂无解答与提示

使用记录:

暂无使用记录


出处: 代数精编第二章不等式
\item { (005031)}利用公式$\dfrac{a+b}2\le \sqrt{\dfrac{a^2+b^2}2}$, 求证: 若$a+b=1(a,b\ge 0)$, 则$\sqrt{2a+1}+\sqrt{2b+1}\le 2\sqrt 2$.


关联目标:

暂未关联目标



标签: 第一单元

答案: 暂无答案

解答或提示: 暂无解答与提示

使用记录:

暂无使用记录


出处: 代数精编第二章不等式
\item { (005032)}利用公式$\dfrac{a+b+c}3\le \sqrt{\dfrac{a^2+b^2+c^2}3}$, 求证: 若$a+b+c=1(a,b,c\ge 0)$, 则$\sqrt{13a+1}+\sqrt{13b+1}+\sqrt{13c+1}\le 4\sqrt 3$.


关联目标:

暂未关联目标



标签: 第一单元

答案: 暂无答案

解答或提示: 暂无解答与提示

使用记录:

暂无使用记录


出处: 代数精编第二章不等式
\item { (005033)}利用公式$\dfrac{a+b}2\le \sqrt{\dfrac{a^2+b^2}2}$, 求证: $a\cos \varphi +b\sin \varphi +c\le \sqrt{2(a^2+b^2+c^2)}$.


关联目标:

暂未关联目标



标签: 第一单元|第三单元

答案: 暂无答案

解答或提示: 暂无解答与提示

使用记录:

暂无使用记录


出处: 代数精编第二章不等式
\item { (005034)}利用$a^2+b^2+c^2\ge ab+bc+ca(a,b,c\in \mathbf{R})$, 证明: 若$a>0$, $b>0$, $c>0$, 则$\dfrac{a^2}{b^2}+{b^2}{c^2}+{c^2}{a^2}{a+b+c}\ge abc$.


关联目标:

暂未关联目标



标签: 第一单元

答案: 暂无答案

解答或提示: 暂无解答与提示

使用记录:

暂无使用记录


出处: 代数精编第二章不等式
\item { (005035)}利用$a^2+b^2+c^2\ge ab+bc+ca(a,b,c\in \mathbf{R})$, 证明: 若半径为$1$的圆内接$\triangle ABC$的而积为$\dfrac 14$, 二边长分别为$a,b,c$, 则\\(1) $abc=1$;\\
(2) $\sqrt b+\sqrt c<\dfrac 1a+\dfrac 1b+\dfrac 1c$.


关联目标:

暂未关联目标



标签: 第一单元

答案: 暂无答案

解答或提示: 暂无解答与提示

使用记录:

暂无使用记录


出处: 代数精编第二章不等式
\item { (005036)}利用$a^2+b^2+c^2\ge ab+bc+ca(a,b,c\in \mathbf{R})$, 证明: 若$a,b,c>0$, $n\in \mathbf{N}$, $f(n)=\lg \dfrac{a^n+b^n+c^n}3$, 则$2f(n)\le f(2n)$.


关联目标:

暂未关联目标



标签: 第一单元|第二单元

答案: 暂无答案

解答或提示: 暂无解答与提示

使用记录:

暂无使用记录


出处: 代数精编第二章不等式
\item { (005037)}利用放缩法并结合公式$ab\le (\dfrac{a+b}2)^2$, 证明: $\lg 9\cdot \lg 11<1$.


关联目标:

暂未关联目标



标签: 第一单元

答案: 暂无答案

解答或提示: 暂无解答与提示

使用记录:

暂无使用记录


出处: 代数精编第二章不等式
\item { (005038)}利用放缩法并结合公式$ab\le (\dfrac{a+b}2)^2$, 证明: $\log_a(a-1)\cdot \log_a(a+1)<1$($a>1$).


关联目标:

暂未关联目标



标签: 第一单元|第二单元

答案: 暂无答案

解答或提示: 暂无解答与提示

使用记录:

暂无使用记录


出处: 代数精编第二章不等式
\item { (005039)}利用放缩法并结合公式$ab\le (\dfrac{a+b}2)^2$, 证明: 若$a>b>c$, 则$\dfrac 1{a-b}+\dfrac 1{b-c}+\dfrac 4{c-a}\ge 0$.


关联目标:

暂未关联目标



标签: 第一单元

答案: 暂无答案

解答或提示: 暂无解答与提示

使用记录:

暂无使用记录


出处: 代数精编第二章不等式
\item { (005040)}利用放缩法证明: $\dfrac 1n+\dfrac 1{n+1}+\dfrac 1{n+2}+\dfrac 1{n+3}+\dfrac 1{n+4}+\cdots +\dfrac 1{n^2}>1$($n\in \mathbf{N}$, $n\ge 2$).


关联目标:

暂未关联目标



标签: 第一单元

答案: 暂无答案

解答或提示: 暂无解答与提示

使用记录:

暂无使用记录


出处: 代数精编第二章不等式
\item { (005041)}利用放缩法证明: $\dfrac 12\le \dfrac 1{n+1}+\dfrac 1{n+2}+\cdots +\dfrac 1{2n}<1$($n\in \mathbf{N}$).


关联目标:

暂未关联目标



标签: 第一单元

答案: 暂无答案

解答或提示: 暂无解答与提示

使用记录:

暂无使用记录


出处: 代数精编第二章不等式
\item { (005042)}利用放缩法证明: 已知$a>0$, $b>0$, $c>0$, 且$a^2+b^2=c^2$, 求证: $a^n+b^n<c^n$($n\ge 3$, $n\in \mathbf{N}$).


关联目标:

暂未关联目标



标签: 第一单元

答案: 暂无答案

解答或提示: 暂无解答与提示

使用记录:

暂无使用记录


出处: 代数精编第二章不等式
\item { (005043)}利用拆项法证明: 若$x>y$, $xy=1$, 则$\dfrac{x^2+y^2}{x-y}\ge 2\sqrt 2$.


关联目标:

暂未关联目标



标签: 第一单元

答案: 暂无答案

解答或提示: 暂无解答与提示

使用记录:

暂无使用记录


出处: 代数精编第二章不等式
\item { (005044)}利用拆项法证明: $\dfrac 12({a^2}+{b^2})+1\ge \sqrt{{a^2}+1}\cdot \sqrt{{b^2}+1}$.


关联目标:

暂未关联目标



标签: 第一单元

答案: 暂无答案

解答或提示: 暂无解答与提示

使用记录:

暂无使用记录


出处: 代数精编第二章不等式
\item { (005045)}利用拆项法证明: 若$a>0$, $b>0$, $c>0$, 则$2(\dfrac{a+b}2-\sqrt{ab})\le 3(\dfrac{a+b+c}3-\sqrt[3]{abc})$.


关联目标:

暂未关联目标



标签: 第一单元

答案: 暂无答案

解答或提示: 暂无解答与提示

使用记录:

暂无使用记录


出处: 代数精编第二章不等式
\item { (005046)}利用拆项法证明: $2(\sqrt{n+1}-1)<1+\dfrac 1{\sqrt 2}+\dfrac 1{\sqrt 3}+\cdots +\dfrac 1{\sqrt n}<2\sqrt n$($n\in \mathbf{N}$).


关联目标:

暂未关联目标



标签: 第一单元

答案: 暂无答案

解答或提示: 暂无解答与提示

使用记录:

暂无使用记录


出处: 代数精编第二章不等式
\item { (005047)}利用逆代法证明: 若正数$x,y$满足$x+2y=1$, 则$\dfrac 1x+\dfrac 1y\ge 3+2\sqrt 2$.


关联目标:

暂未关联目标



标签: 第一单元

答案: 暂无答案

解答或提示: 暂无解答与提示

使用记录:

暂无使用记录


出处: 代数精编第二章不等式
\item { (005048)}利用逆代法证明: $\dfrac 1{\sin ^2\alpha}+\dfrac 3{\cos^2\alpha}\ge 4+2\sqrt 3$.


关联目标:

暂未关联目标



标签: 第一单元|第三单元

答案: 暂无答案

解答或提示: 暂无解答与提示

使用记录:

暂无使用记录


出处: 代数精编第二章不等式
\item { (005049)}利用逆代法证明: 若$x,y>0$, $a,b$为正常数, 且$\dfrac ax+\dfrac ay=1$, 则$x+y\ge (\sqrt a+\sqrt b)^2$.


关联目标:

暂未关联目标



标签: 第一单元

答案: 暂无答案

解答或提示: 暂无解答与提示

使用记录:

暂无使用记录


出处: 代数精编第二章不等式
\item { (005050)}利用判别式法证明: $\dfrac 13\le \dfrac{x^2-x+1}{x^2+x+1}\le 3$.


关联目标:

暂未关联目标



标签: 第一单元

答案: 暂无答案

解答或提示: 暂无解答与提示

使用记录:

暂无使用记录


出处: 代数精编第二章不等式
\item { (005051)}利用判别式法证明: 若关于$x$的不等式$(a^2-1)x^2-(a-1)x-1<0(a\in \mathbf{R})$对仟意实数$x$恒成立, 则$-\dfrac 35<a\le 1$.


关联目标:

暂未关联目标



标签: 第一单元

答案: 暂无答案

解答或提示: 暂无解答与提示

使用记录:

暂无使用记录


出处: 代数精编第二章不等式
\item { (005052)}利用函数的单调性证明: 若$x>0$, $y>0$, $x+y=1$, 则$(x+\dfrac 1x)(y+\dfrac 1y)\ge \dfrac{25}4$.


关联目标:

暂未关联目标



标签: 第一单元|第二单元

答案: 暂无答案

解答或提示: 暂无解答与提示

使用记录:

暂无使用记录


出处: 代数精编第二章不等式
\item { (005053)}利用函数的单调性证明: 若$0<a<\dfrac 1k(k\ge 2,k\in \mathbf{N})$, 且$a^2<a-b$, 则$b<\dfrac 1{k+1}$.


关联目标:

暂未关联目标



标签: 第一单元|第二单元

答案: 暂无答案

解答或提示: 暂无解答与提示

使用记录:

暂无使用记录


出处: 代数精编第二章不等式
\item { (005054)}利用三角换元法证明: 若$a^2+b^2=1$, 则$a\sin x+b\cos x\le 1$.


关联目标:

暂未关联目标



标签: 第一单元|第三单元

答案: 暂无答案

解答或提示: 暂无解答与提示

使用记录:

暂无使用记录


出处: 代数精编第二章不等式
\item { (005055)}利用三角换元法证明: 若$|a|<1$, $|b|<1$, 则$|ab\pm \sqrt{(1-{a^2})(1-{b^2})}|\le 1$.


关联目标:

暂未关联目标



标签: 第一单元|第三单元

答案: 暂无答案

解答或提示: 暂无解答与提示

使用记录:

暂无使用记录


出处: 代数精编第二章不等式
\item { (005056)}利用三角换元法证明: 若$x^2+y^2\le 1$, 则$-\sqrt 2\le x^2+2xy-y^2\le \sqrt 2$.


关联目标:

暂未关联目标



标签: 第一单元|第三单元

答案: 暂无答案

解答或提示: 暂无解答与提示

使用记录:

暂无使用记录


出处: 代数精编第二章不等式
\item { (005057)}利用三角换元法证明: 若$|x|\le 1$, 则$(1+x)^n+(1-x)^n\le 2^n$.


关联目标:

暂未关联目标



标签: 第一单元|第三单元

答案: 暂无答案

解答或提示: 暂无解答与提示

使用记录:

暂无使用记录


出处: 代数精编第二章不等式
\item { (005058)}利用三角换元法证明: 若$a>0$, $b>0$, 且$a-b=1$, 则$0<\dfrac 1a(\sqrt a-\dfrac 1{\sqrt a})(\sqrt b+\dfrac 1{\sqrt b})<1$.


关联目标:

暂未关联目标



标签: 第一单元|第三单元

答案: 暂无答案

解答或提示: 暂无解答与提示

使用记录:

暂无使用记录


出处: 代数精编第二章不等式
\item { (005059)}利用三角换元法证明: $0<\sqrt{1+x}-\sqrt x\le 1$.


关联目标:

暂未关联目标



标签: 第一单元|第三单元

答案: 暂无答案

解答或提示: 暂无解答与提示

使用记录:

暂无使用记录


出处: 代数精编第二章不等式
\item { (005060)}试构造几何图形证明: 若$f(x)=\sqrt{1+x^2}$, $x>b>0$, 则$|f(a)-f(b)|<|a-b|$.


关联目标:

暂未关联目标



标签: 第一单元

答案: 暂无答案

解答或提示: 暂无解答与提示

使用记录:

暂无使用记录


出处: 代数精编第二章不等式
\item { (005061)}试构造几何图形证明: 若$x,y,z>0$, 则$\sqrt{x^2+y^2+xy}+\sqrt{y^2+z^2+yz}>\sqrt{z^2+x^2+zx}$.


关联目标:

暂未关联目标



标签: 第一单元

答案: 暂无答案

解答或提示: 暂无解答与提示

使用记录:

暂无使用记录


出处: 代数精编第二章不等式
\item { (005062)}利用均值换元证明: 若$a>0$, $b>0$, 且$a+b=1$, 则$\dfrac 43\le \dfrac 1{a+1}+\dfrac 1{b+1}<\dfrac 32$.


关联目标:

暂未关联目标



标签: 第一单元

答案: 暂无答案

解答或提示: 暂无解答与提示

使用记录:

暂无使用记录


出处: 代数精编第二章不等式
\item { (005063)}利用均值换元证明: 若$a+b+c=1$, 则${a^2}+{b^2}+{c^2}\ge \dfrac 13$.


关联目标:

暂未关联目标



标签: 第一单元

答案: 暂无答案

解答或提示: 暂无解答与提示

使用记录:

暂无使用记录


出处: 代数精编第二章不等式
\item { (005064)}利用设差换元证明: 若$x\ge y\ge 0$, 则$\sqrt{2xy-{y^2}}+\sqrt{x^2-y^2}\ge x$.


关联目标:

暂未关联目标



标签: 第一单元

答案: 暂无答案

解答或提示: 暂无解答与提示

使用记录:

暂无使用记录


出处: 代数精编第二章不等式
\item { (005065)}已知$a,b,c$都是正数, 求证: $a^ab^bc^c\ge (abc)^{\frac{a+b+c}3}$.


关联目标:

暂未关联目标



标签: 第一单元

答案: 暂无答案

解答或提示: 暂无解答与提示

使用记录:

暂无使用记录


出处: 代数精编第二章不等式
\item { (005066)}已知正数$a,b$满足$a+b=1$, 求证: $(ax+by)(ay+bx)\ge xy$.


关联目标:

暂未关联目标



标签: 第一单元

答案: 暂无答案

解答或提示: 暂无解答与提示

使用记录:

暂无使用记录


出处: 代数精编第二章不等式
\item { (005067)}已知正数$a,b$满足$a+b=1$, 求证: $(a+\dfrac 1a)^2+(b+\dfrac 1b)^2\ge \dfrac{25}2$.


关联目标:

暂未关联目标



标签: 第一单元

答案: 暂无答案

解答或提示: 暂无解答与提示

使用记录:

暂无使用记录


出处: 代数精编第二章不等式
\item { (005068)}已知正数$a,b$满足$a+b=1$, 求证: $(a+\dfrac 1a)(b+\dfrac 1b)\ge \dfrac{25}4$.


关联目标:

暂未关联目标



标签: 第一单元

答案: 暂无答案

解答或提示: 暂无解答与提示

使用记录:

暂无使用记录


出处: 代数精编第二章不等式
\item { (005069)}已知正数$a,b,c$满足$a+b+c=1$, 求证: $(a+\dfrac 1a)+(b+\dfrac 1b)+(c+\dfrac 1c)\ge 10$.


关联目标:

暂未关联目标



标签: 第一单元

答案: 暂无答案

解答或提示: 暂无解答与提示

使用记录:

暂无使用记录


出处: 代数精编第二章不等式
\item { (005070)}已知正数$a,b,c$满足$a+b+c=1$, 求证: $(a+\dfrac 1a)^2+(b+\dfrac 1b)^2+(c+\dfrac 1c)^2\ge \dfrac{100}3$


关联目标:

暂未关联目标



标签: 第一单元

答案: 暂无答案

解答或提示: 暂无解答与提示

使用记录:

暂无使用记录


出处: 代数精编第二章不等式
\item { (005071)}已知正数$a,b,c$满足$a+b+c=1$, 求证: $\dfrac 1{\sqrt a}+\dfrac 1{\sqrt b}+\dfrac 1{\sqrt c}\ge 3\sqrt 3$.


关联目标:

暂未关联目标



标签: 第一单元

答案: 暂无答案

解答或提示: 暂无解答与提示

使用记录:

暂无使用记录


出处: 代数精编第二章不等式
\item { (005072)}已知$a^2+b^2+c^2=1$, 求证: $-\dfrac 12\le ab+bc+ca\le 1$.


关联目标:

暂未关联目标



标签: 第一单元

答案: 暂无答案

解答或提示: 暂无解答与提示

使用记录:

暂无使用记录


出处: 代数精编第二章不等式
\item { (005073)}已知$a^2+b^2+c^2=1$, 求证: $|abc|\le \dfrac{\sqrt 3}9$.


关联目标:

暂未关联目标



标签: 第一单元

答案: 暂无答案

解答或提示: 暂无解答与提示

使用记录:

暂无使用记录


出处: 代数精编第二章不等式
\item { (005074)}已知$x>1$, 求证: $\sqrt x-\sqrt{x-1}>\sqrt{x+1}-\sqrt x$.


关联目标:

暂未关联目标



标签: 第一单元

答案: 暂无答案

解答或提示: 暂无解答与提示

使用记录:

暂无使用记录


出处: 代数精编第二章不等式
\item { (005075)}已知$a>0$, $b>0$, $c>0$, 求证: $\dfrac 1a+\dfrac 1b+\dfrac 1c\ge 2(\dfrac 1{a+b}+\dfrac 1{b+c}+\dfrac 1{c+a})$.


关联目标:

暂未关联目标



标签: 第一单元

答案: 暂无答案

解答或提示: 暂无解答与提示

使用记录:

暂无使用记录


出处: 代数精编第二章不等式
\item { (005076)}已知$a>0$, $b>0$, $c>0$, 求证: $\dfrac c{a+b}+\dfrac a{b+c}+\dfrac b{c+a}\ge \dfrac 32$.


关联目标:

暂未关联目标



标签: 第一单元

答案: 暂无答案

解答或提示: 暂无解答与提示

使用记录:

暂无使用记录


出处: 代数精编第二章不等式
\item { (005077)}已知$\alpha ,\beta \in (0,\dfrac{\pi}2)$, 求证: $\dfrac 1{\cos^2\alpha}+\dfrac 1{\sin^2\alpha \sin^2\beta\cos^2\beta}\ge 9$.


关联目标:

暂未关联目标



标签: 第一单元

答案: 暂无答案

解答或提示: 暂无解答与提示

使用记录:

暂无使用记录


出处: 代数精编第二章不等式
\item { (005078)}已知$a>0$, $b>0$, $c>0$, 求证: $\dfrac 1{a+b}+\dfrac 1{b+c}+\dfrac 1{c+a}\ge \dfrac 9{2(a+b+c)}$.


关联目标:

暂未关联目标



标签: 第一单元

答案: 暂无答案

解答或提示: 暂无解答与提示

使用记录:

暂无使用记录


出处: 代数精编第二章不等式
\item { (005079)}己知$\tan \alpha,\tan \beta$是关于$x$的方程$mx^2+(2m-3)x+(m-2)=0(m\ne 0)$的两根, 求证: $\tan (\alpha +\beta)\ge -\dfrac 34$.


关联目标:

暂未关联目标



标签: 第一单元|第三单元

答案: 暂无答案

解答或提示: 暂无解答与提示

使用记录:

暂无使用记录


出处: 代数精编第二章不等式
\item { (005080)}已知长方体的对角线长为定长$l$, 求证: 它的体积$V\le \dfrac{\sqrt 3l^3}9$.


关联目标:

暂未关联目标



标签: 第一单元

答案: 暂无答案

解答或提示: 暂无解答与提示

使用记录:

暂无使用记录


出处: 代数精编第二章不等式
\item { (005088)}求证: $\dfrac{x+b+c+abc}{1+ab+bc+ca}\le 1$, 其中$0\le a\le 1$, $0\le b\le 1$, $0\le c\le 1$.


关联目标:

暂未关联目标



标签: 第一单元

答案: 暂无答案

解答或提示: 暂无解答与提示

使用记录:

暂无使用记录


出处: 代数精编第二章不等式
\item { (005091)}求证: 若$a>b>0$, $c>d>0$, 则$\sqrt{ac}-\sqrt{bd}>\sqrt{(a-b)(c-d)}$.


关联目标:

暂未关联目标



标签: 第一单元

答案: 暂无答案

解答或提示: 暂无解答与提示

使用记录:

暂无使用记录


出处: 代数精编第二章不等式
\item { (005092)}求证: $ac+bd\le \sqrt{a^2+b^2}\cdot \sqrt{c^2+d^2}$.


关联目标:

暂未关联目标



标签: 第一单元

答案: 暂无答案

解答或提示: 暂无解答与提示

使用记录:

暂无使用记录


出处: 代数精编第二章不等式
\item { (005094)}求证: 若$-1<x<1$, $-1<y<1$, 则$|\dfrac{x+y}{1+xy}|<1$.


关联目标:

暂未关联目标



标签: 第一单元

答案: 暂无答案

解答或提示: 暂无解答与提示

使用记录:

暂无使用记录


出处: 代数精编第二章不等式
\item { (005097)}求证: 若$a>0$, $b>0$, $a+b=1$, 则$3^a+3^b<4$.


关联目标:

暂未关联目标



标签: 第一单元

答案: 暂无答案

解答或提示: 暂无解答与提示

使用记录:

暂无使用记录


出处: 代数精编第二章不等式
\item { (005098)}利用反证法证明: 若$0<a<1$, $0<b<1$, $0<c<1$, 则$(1-a)b$, $(1-b)c$, $(1-c)a$不能都大于$\dfrac 14$.


关联目标:

暂未关联目标



标签: 第一单元

答案: 暂无答案

解答或提示: 暂无解答与提示

使用记录:

暂无使用记录


出处: 代数精编第二章不等式
\item { (005099)}利用反证法证明: 若$0<a<2$, $0<b<2$, $0<c<2$, 则$a(2-b)$, $b(2-c)$, $c(2-a)$不可能都大于$1$.


关联目标:

暂未关联目标



标签: 第一单元

答案: 暂无答案

解答或提示: 暂无解答与提示

使用记录:

暂无使用记录


出处: 代数精编第二章不等式
\item { (005100)}利用反证法证明: 若$x,y>0$, 且$x+y>2$, 则$\dfrac{1+y}x$和$\dfrac{1+x}y$中至少有一个小于$2$.


关联目标:

暂未关联目标



标签: 第一单元

答案: 暂无答案

解答或提示: 暂无解答与提示

使用记录:

暂无使用记录


出处: 代数精编第二章不等式
\item { (005101)}利用反证法证明: 若$0<a<1$, $b>0$, 且$a^b=b^a$, 则$a=b$.


关联目标:

暂未关联目标



标签: 第一单元

答案: 暂无答案

解答或提示: 暂无解答与提示

使用记录:

暂无使用记录


出处: 代数精编第二章不等式
\item { (005102)}若$a>0$, $b>0$, 且$a^3+b^3=2$, 试分别利用$x^3+y^3+z^3\ge 3xyz$($x,y,z\ge 0$)构造方程, 并利用判别式以及反证法证明: $a+b\le 2$.


关联目标:

暂未关联目标



标签: 第一单元

答案: 暂无答案

解答或提示: 暂无解答与提示

使用记录:

暂无使用记录


出处: 代数精编第二章不等式
\item { (005103)}下列函数中, 最小值为$2$的是\bracket{20}.
\twoch{$x+\dfrac 1x$}{$\dfrac{x^2+2}{\sqrt{x^2+1}}$}{$\log_ax+\log_xa$($a>0$, $x>0$, $a\ne 1$, $x\ne 1$)}{$3^x+3^{-x}$($x>0$)}


关联目标:

暂未关联目标



标签: 第一单元|第二单元

答案: 暂无答案

解答或提示: 暂无解答与提示

使用记录:

暂无使用记录


出处: 代数精编第二章不等式
\item { (005104)}若$\log_{\sqrt 2}x+\log_{\sqrt 2}y=4$, 则$x+y$的最小值是\bracket{20}.
\fourch{$8$}{$4\sqrt 2$}{$4$}{$2$}


关联目标:

暂未关联目标



标签: 第一单元|第二单元

答案: 暂无答案

解答或提示: 暂无解答与提示

使用记录:

暂无使用记录


出处: 代数精编第二章不等式
\item { (005105)}若$a,b$均为大于$1$的正数, 且$ab=100$, 则$\lg a\cdot \lg b$的最大值是\bracket{20}.
\fourch{$0$}{$1$}{$2$}{$\dfrac 52$}


关联目标:

暂未关联目标



标签: 第一单元|第二单元

答案: 暂无答案

解答或提示: 暂无解答与提示

使用记录:

暂无使用记录


出处: 代数精编第二章不等式
\item { (005106)}若实数$x$与$y$满足$x+y-4=0$, 则$x^2+y^2$的最小值是\bracket{20}.
\fourch{$4$}{$6$}{$8$}{$10$}


关联目标:

暂未关联目标



标签: 第一单元

答案: 暂无答案

解答或提示: 暂无解答与提示

使用记录:

暂无使用记录


出处: 代数精编第二章不等式
\item { (005107)}若非负实数$a,b$满足$2a+3b=10$, 则$\sqrt{3b}+\sqrt{2a}$的最大值是\bracket{20}.
\fourch{$\sqrt{10}$}{$2\sqrt 5$}{$5$}{$10$}


关联目标:

暂未关联目标



标签: 第一单元

答案: 暂无答案

解答或提示: 暂无解答与提示

使用记录:

暂无使用记录


出处: 代数精编第二章不等式
\item { (005108)}若$x>1$, 则$\dfrac{x^2-2x+2}{2x-2}$有\bracket{20}.
\fourch{最小值$1$}{最大值$1$}{最小值$-1$}{最大值$-1$}


关联目标:

暂未关联目标



标签: 第一单元

答案: 暂无答案

解答或提示: 暂无解答与提示

使用记录:

暂无使用记录


出处: 代数精编第二章不等式
\item { (005109)}若$x,y\in \mathbf{R}^+$, 且$x^2+y^2=1$, 则$x+y$的最大值是\blank{50}.


关联目标:

暂未关联目标



标签: 第一单元

答案: 暂无答案

解答或提示: 暂无解答与提示

使用记录:

暂无使用记录


出处: 代数精编第二章不等式
\item { (005110)}若$x+2y=2\sqrt 2a$($x>0$, $y>0$, $a>1$), 则$\log_ax+\log_ay$的最大值是\blank{50}.


关联目标:

暂未关联目标



标签: 第一单元|第二单元

答案: 暂无答案

解答或提示: 暂无解答与提示

使用记录:

暂无使用记录


出处: 代数精编第二章不等式
\item { (005111)}若$x>1$, 则$2+3x+\dfrac 4{x-1}$的最小值\blank{50}, 此时$x=$\blank{50}.


关联目标:

暂未关联目标



标签: 第一单元

答案: 暂无答案

解答或提示: 暂无解答与提示

使用记录:

暂无使用记录


出处: 代数精编第二章不等式
\item { (005112)}若$x>0$, 则$x+\dfrac 1x+\dfrac{16x}{x^2+1}$的最小值是\blank{50}, 此时$x=$\blank{50}.


关联目标:

暂未关联目标



标签: 第一单元

答案: 暂无答案

解答或提示: 暂无解答与提示

使用记录:

暂无使用记录


出处: 代数精编第二章不等式
\item { (005113)}若正数$a,b$满足$a^2+\dfrac{b^2}2=1$, 则$a\sqrt{1+b^2}$的最大值为\blank{50}, 此时$a=$\blank{50}, $b=$\blank{50}.


关联目标:

暂未关联目标



标签: 第一单元

答案: 暂无答案

解答或提示: 暂无解答与提示

使用记录:

暂无使用记录


出处: 代数精编第二章不等式
\item { (005114)}若$x>0$, 则$3x+\dfrac{12}{x^2}$的最小值是\blank{50}, 此时$x=$\blank{50}.


关联目标:

暂未关联目标



标签: 第一单元

答案: 暂无答案

解答或提示: 暂无解答与提示

使用记录:

暂无使用记录


出处: 代数精编第二章不等式
\item { (005115)}若$0<x<\dfrac 13$, 则$x^2(1-3x)$的最大值是\blank{50}, 此时$x=$\blank{50}.


关联目标:

暂未关联目标



标签: 第一单元

答案: 暂无答案

解答或提示: 暂无解答与提示

使用记录:

暂无使用记录


出处: 代数精编第二章不等式
\item { (005116)}若$xy>0$, 且$x^2y=2$, 则$xy+x^2$的最小值是\blank{50}.


关联目标:

暂未关联目标



标签: 第一单元

答案: 暂无答案

解答或提示: 暂无解答与提示

使用记录:

暂无使用记录


出处: 代数精编第二章不等式
\item { (005118)}若正数$x,y,z$满足$5x+2y+z=100$, 则$\lg x+\lg y+\lg z$的最大值是\blank{50}.


关联目标:

暂未关联目标



标签: 第一单元|第二单元

答案: 暂无答案

解答或提示: 暂无解答与提示

使用记录:

暂无使用记录


出处: 代数精编第二章不等式
\item { (005119)}若$\dfrac{x^2}4+{y^2}=x$, 则$x^2+y^2$有\bracket{20}.
\fourch{最小值$0$, 最大值$16$}{最小值$-\dfrac 13$, 最大值$0$}{最小值$0$, 最大值$1$}{最小值$1$, 最大值$2$}


关联目标:

暂未关联目标



标签: 第一单元

答案: 暂无答案

解答或提示: 暂无解答与提示

使用记录:

暂无使用记录


出处: 代数精编第二章不等式
\item { (005121)}若$x>0$, 则$\dfrac x{x^3+2}$的最大值是\bracket{20}.
\fourch{$5$}{$3$}{$1$}{$\dfrac 13$}


关联目标:

暂未关联目标



标签: 第一单元

答案: 暂无答案

解答或提示: 暂无解答与提示

使用记录:

暂无使用记录


出处: 代数精编第二章不等式
\item { (005122)}若正数$a,b$满足$ab-(a+b)=1$, 则$a+b$的最小值是\bracket{20}.
\fourch{$2+2\sqrt 2$}{$2\sqrt 2-2$}{$\sqrt 5+2$}{$\sqrt 5-2$}


关联目标:

暂未关联目标



标签: 第一单元

答案: 暂无答案

解答或提示: 暂无解答与提示

使用记录:

暂无使用记录


出处: 代数精编第二章不等式
\item { (005127)}若$x,y>0$, 求$\dfrac{\sqrt x+\sqrt y}{\sqrt{x+y}}$的最大值.


关联目标:

暂未关联目标



标签: 第一单元

答案: 暂无答案

解答或提示: 暂无解答与提示

使用记录:

暂无使用记录


出处: 代数精编第二章不等式
\item { (005128)}已知正常数$a,b$和正变数$x,y$满足$a+b=10$, $\dfrac ax+\dfrac by=1$, $x+y$的最小值为$18$, 求$a,b$的值.


关联目标:

暂未关联目标



标签: 第一单元

答案: 暂无答案

解答或提示: 暂无解答与提示

使用记录:

暂无使用记录


出处: 代数精编第二章不等式
\item { (005129)}已知$x^2+y^2=1$, 求$(1+xy)(1-xy)$的最大值和最小值.


关联目标:

暂未关联目标



标签: 第一单元

答案: 暂无答案

解答或提示: 暂无解答与提示

使用记录:

暂无使用记录


出处: 代数精编第二章不等式
\item { (005130)}已知$x^2+y^2=3$, $a^2+b^2=4$, 求$ax+by$的最大值和最小值.


关联目标:

暂未关联目标



标签: 第一单元

答案: 暂无答案

解答或提示: 暂无解答与提示

使用记录:

暂无使用记录


出处: 代数精编第二章不等式
\item { (005131)}已知$\sqrt{1-y^2}+y\sqrt{1-x^2}=1$, 求$x+y$的最大值和最小值.


关联目标:

暂未关联目标



标签: 第一单元

答案: 暂无答案

解答或提示: 暂无解答与提示

使用记录:

暂无使用记录


出处: 代数精编第二章不等式
\item { (005132)}已知函数$f(x)=\dfrac{2^{x+3}}{{4^x}+8}$.\\
(1) 求$f(x)$的最大值;\\
(2) 对于任意实数$a,b$, 求证: $f(a)<b^2-4b+\dfrac{11}2$.


关联目标:

暂未关联目标



标签: 第一单元|第二单元

答案: 暂无答案

解答或提示: 暂无解答与提示

使用记录:

暂无使用记录


出处: 代数精编第二章不等式
\item { (005133)}若直角三角形的周长为$1$, 求它的面积的最大值.


关联目标:

暂未关联目标



标签: 第一单元

答案: 暂无答案

解答或提示: 暂无解答与提示

使用记录:

暂无使用记录


出处: 代数精编第二章不等式
\item { (005134)}若直角三角形的内切圆半径为$1$, 求它的面积的最小值.


关联目标:

暂未关联目标



标签: 第一单元

答案: 暂无答案

解答或提示: 暂无解答与提示

使用记录:

暂无使用记录


出处: 代数精编第二章不等式
\item { (005135)}若球半径为$R$, 试求它的内接圆柱的最大体积. 请指出下向解法的错误, 并给出正确的解答.\\
解: 设圆柱底面半径为$r$, 则$4r^2=4R^2-h^2$, 而$V_=\pi {r^2}h=\dfrac{\pi}4(4{R^2}-{h^2})h=\dfrac{\pi }4(2R+h)(2R-h)=\dfrac{\pi}8(2R+h)(4R-2h)h\le \dfrac{\pi}8(\dfrac{2R+h+4R-2h+h}3)^3=\dfrac{\pi}8(2R)^3=\pi R^3$. 所以所求最大体积为$\pi R^3$.


关联目标:

暂未关联目标



标签: 第一单元

答案: 暂无答案

解答或提示: 暂无解答与提示

使用记录:

暂无使用记录


出处: 代数精编第二章不等式
\item { (005136)}在$\triangle ABC$中, 已知$BC=a$, $CA=b$, $AB=c$, $\angle ACB=\theta$. 现将$\triangle ABC$分别以$BC,CA,AB$所在直线为轴旋转一周, 设所得三个旋转体的体积依次为$V_1,V_2,V_3$.\\
(1) 设$T=\dfrac{V_3}{V_1+V_2}$, 试用$a,b,c$表示$T$;\\
(2) 若$\theta$为定值, 并令$\dfrac{a+b}c=x$, 将$T=\dfrac{V_3}{V_1+V_2}$表示为$x$的函数, 写出这个函数的定义域, 并求这个函数的最大值$M$;\\
(3) 若$\theta \in [\dfrac{\pi }3,\pi)$, 求(2)中$M$的最大值.


关联目标:

暂未关联目标



标签: 第一单元|第二单元

答案: 暂无答案

解答或提示: 暂无解答与提示

使用记录:

暂无使用记录


出处: 代数精编第二章不等式
\item { (005137)}已知$A(0,\sqrt 3a)$, $B(-a,0)$, $C(a,0)$是等边$\triangle ABC$的顶点, 点$M,N$分别在边$AB,BC$上, 且将$\triangle ABC$的面积两等分, 记$N$的横坐标为$x$, $|MN|=y$.\\
(1) 写出$y=f(x)$的表达式;\\
(2) 求$y=f(x)$的最小值.


关联目标:

暂未关联目标



标签: 第一单元

答案: 暂无答案

解答或提示: 暂无解答与提示

使用记录:

暂无使用记录


出处: 代数精编第二章不等式
\item { (005139)}已知关于$x$的不等式$ax^2+bx+c>0$的解集是$\{x|\alpha<x<\beta\}$, 其中$0<\alpha<\beta$, 求$cx^2+bx+a<0$的解集.


关联目标:

暂未关联目标



标签: 第一单元

答案: 暂无答案

解答或提示: 暂无解答与提示

使用记录:

暂无使用记录


出处: 代数精编第二章不等式
\item { (005140)}解不等式$(x+1)^2(x-1)(x-4)^3>0$.


关联目标:

暂未关联目标



标签: 第一单元

答案: 暂无答案

解答或提示: 暂无解答与提示

使用记录:

暂无使用记录


出处: 代数精编第二章不等式
\item { (005141)}解不等式$\dfrac{3x^2-14x+14}{x^2-6x+8}\ge 1$.


关联目标:

暂未关联目标



标签: 第一单元

答案: 暂无答案

解答或提示: 暂无解答与提示

使用记录:

暂无使用记录


出处: 代数精编第二章不等式
\item { (005142)}解不等式$\sqrt{x^2-3x+2}>x-3$.


关联目标:

暂未关联目标



标签: 第一单元

答案: 暂无答案

解答或提示: 暂无解答与提示

使用记录:

暂无使用记录


出处: 代数精编第二章不等式
\item { (005143)}解不等式$\sqrt{2x-1}<x-2$.


关联目标:

暂未关联目标



标签: 第一单元

答案: 暂无答案

解答或提示: 暂无解答与提示

使用记录:

暂无使用记录


出处: 代数精编第二章不等式
\item { (005144)}解不等式$|x^2-4|\le x+2$.


关联目标:

暂未关联目标



标签: 第一单元

答案: 暂无答案

解答或提示: 暂无解答与提示

使用记录:

暂无使用记录


出处: 代数精编第二章不等式
\item { (005145)}解不等式$|x^2-\dfrac 12|>2x$.


关联目标:

暂未关联目标



标签: 第一单元

答案: 暂无答案

解答或提示: 暂无解答与提示

使用记录:

暂无使用记录


出处: 代数精编第二章不等式
\item { (005146)}解关于$x$的不等式$|\log_ax|<|\log_a(ax^2)|-2$($0<a<1$).


关联目标:

暂未关联目标



标签: 第一单元|第二单元

答案: 暂无答案

解答或提示: 暂无解答与提示

使用记录:

暂无使用记录


出处: 代数精编第二章不等式
\item { (005147)}若关于$x$的不等式$2x-1>a(x-2)$的解集是$\mathbf{R}$, 则实数$a$的取值范围是\bracket{20}.
\fourch{$a>2$}{$a=2$}{$a<2$}{$a$不存在}


关联目标:

暂未关联目标



标签: 第一单元

答案: 暂无答案

解答或提示: 暂无解答与提示

使用记录:

暂无使用记录


出处: 代数精编第二章不等式
\item { (005148)}若关于$x$的不等式$ax^2+bx-2>0$的解集是$(-\infty ,-\dfrac 12)\cup (\dfrac 13,+\infty)$, 则$ab$等于\bracket{20}.
\fourch{$-24$}{$24$}{$14$}{$-14$}


关联目标:

暂未关联目标



标签: 第一单元

答案: 暂无答案

解答或提示: 暂无解答与提示

使用记录:

暂无使用记录


出处: 代数精编第二章不等式
\item { (005149)}若关于$x$的不等式$(a-2)x^2+2(a-2)x-4<0$对一切实数$x$恒成立, 则实数$a$的取值范围是\bracket{20}.
\fourch{$(-\infty ,2]$}{$(-\infty,-2)$}{$(-2,2]$}{$(-2,2)$}


关联目标:

暂未关联目标



标签: 第一单元

答案: 暂无答案

解答或提示: 暂无解答与提示

使用记录:

暂无使用记录


出处: 代数精编第二章不等式
\item { (005151)}若关于$x$的不等式$(a+b)x+2a-3b<0$的解集是$\{x|x<-\dfrac 13\}$, 则$(a-3b)x+b-2a>0$的解集是\blank{50}.


关联目标:

暂未关联目标



标签: 第一单元

答案: 暂无答案

解答或提示: 暂无解答与提示

使用记录:

暂无使用记录


出处: 代数精编第二章不等式
\item { (005152)}若不等式$\dfrac{2x^2+2kx+k}{4x^2+6x+3}<1$对一切$x\in \mathbf{R}$恒成立, 则实数$k$的取值范围是\blank{50}.


关联目标:

暂未关联目标



标签: 第一单元

答案: 暂无答案

解答或提示: 暂无解答与提示

使用记录:

暂无使用记录


出处: 代数精编第二章不等式
\item { (005153)}若关于$x$的不等式$ax^2+bx+c>0$的解集是$\{x|3<x<5\}$, 则不等式$cx^2+bx+a<0$的解集是\blank{50}.


关联目标:

暂未关联目标



标签: 第一单元

答案: 暂无答案

解答或提示: 暂无解答与提示

使用记录:

暂无使用记录


出处: 代数精编第二章不等式
\item { (005154)}若关于$x$的不等式$\dfrac{x-a}{x^2-3x+2}\ge 0$的解集是$\{x|1<x\le ax>2\}$, 则实数$a$的取值范围是\blank{50}.


关联目标:

暂未关联目标



标签: 第一单元

答案: 暂无答案

解答或提示: 暂无解答与提示

使用记录:

暂无使用记录


出处: 代数精编第二章不等式
\item { (005155)}不等式$(x+2)(x+1)^2(x-1)^3(x-3)>0$的解集为:\blank{50}.


关联目标:

暂未关联目标



标签: 第一单元

答案: 暂无答案

解答或提示: 暂无解答与提示

使用记录:

暂无使用记录


出处: 代数精编第二章不等式
\item { (005156)}不等式$\dfrac{(x-1)^2(x+2)}{(x-3)(x-4)}\le 0$的解集为:\blank{50}.


关联目标:

暂未关联目标



标签: 第一单元

答案: 暂无答案

解答或提示: 暂无解答与提示

使用记录:

暂无使用记录


出处: 代数精编第二章不等式
\item { (005157)}不等式$x+1\le \dfrac 4{x+1}$的解集为:\blank{50}.


关联目标:

暂未关联目标



标签: 第一单元

答案: 暂无答案

解答或提示: 暂无解答与提示

使用记录:

暂无使用记录


出处: 代数精编第二章不等式
\item { (005158)}若不等式$f(x)\ge 0$的解集为$[1,2]$, 不等式$g(x)\ge 0$的解集为$\varnothing$, 则不等式$\dfrac{f(x)}{g(x)}$的解集是\bracket{20}.
\fourch{$\varnothing$}{$(-\infty ,1)\cup (2,+\infty)$}{$[1,2)$}{$\mathbf{R}$}


关联目标:

暂未关联目标



标签: 第一单元

答案: 暂无答案

解答或提示: 暂无解答与提示

使用记录:

暂无使用记录


出处: 代数精编第二章不等式
\item { (005159)}若关于$x$的不等式$ax^2-bx+c<0$的解集为$(-\infty ,\alpha)\cup (\beta ,+\infty)$, 其中$\alpha <\beta <0$, 则不等式$cx^2+bx+a>0$的解集为\bracket{20}.
\fourch{$(\dfrac 1{\beta},\dfrac 1{\alpha})$}{$(\dfrac 1{\alpha},\dfrac 1{\beta})$}{$(-\dfrac 1{\beta},-\dfrac 1{\alpha})$}{$(-\dfrac 1{\alpha},-\dfrac 1{\beta})$}


关联目标:

暂未关联目标



标签: 第一单元

答案: 暂无答案

解答或提示: 暂无解答与提示

使用记录:

暂无使用记录


出处: 代数精编第二章不等式
\item { (005160)}解关于$x$的不等式: $m^2x-1<x+m$.


关联目标:

暂未关联目标



标签: 第一单元

答案: 暂无答案

解答或提示: 暂无解答与提示

使用记录:

暂无使用记录


出处: 代数精编第二章不等式
\item { (005161)}解关于$x$的不等式: $x^2-ax-2a^2<0$.


关联目标:

暂未关联目标



标签: 第一单元

答案: 暂无答案

解答或提示: 暂无解答与提示

使用记录:

暂无使用记录


出处: 代数精编第二章不等式
\item { (005162)}已知关于$x$的不等式$\sqrt x>ax+\dfrac 32$的解集是$\{x|4<x<b\}$, 求$a,b$的值.


关联目标:

暂未关联目标



标签: 第一单元

答案: 暂无答案

解答或提示: 暂无解答与提示

使用记录:

暂无使用记录


出处: 代数精编第二章不等式
\item { (005163)}已知$x=3$是不等式$ax>b$解集中的元素, 求实数$a,b$应满足的条件.


关联目标:

暂未关联目标



标签: 第一单元

答案: 暂无答案

解答或提示: 暂无解答与提示

使用记录:

暂无使用记录


出处: 代数精编第二章不等式
\item { (005164)}已知集合$\{x|x<-2\text{或}x>3\}$是集合$\{x|2ax^2+(2-ab)x-b>0\}$的子集, 求实数$a,b$的取值范围.


关联目标:

暂未关联目标



标签: 第一单元

答案: 暂无答案

解答或提示: 暂无解答与提示

使用记录:

暂无使用记录


出处: 代数精编第二章不等式
\item { (005165)}已知集合$A=\{x|\dfrac{2x-1}{x^2+3x+2}>0\}$, $B=\{x|x^2+ax+b\le 0\}$, 且$A\cap B=\{x|\dfrac 12<x\le 3\}$, 求实数$a,b$的取值范围.


关联目标:

暂未关联目标



标签: 第一单元

答案: 暂无答案

解答或提示: 暂无解答与提示

使用记录:

暂无使用记录


出处: 代数精编第二章不等式
\item { (005166)}已知集合$A=\{x|(x+2)(x+1)(2x-1)>0\}$, $B=\{x|x^2+ax+b\le 0\}$, 且$A\cup B=\{x|x+2 >0\}$, $A\cap B=\{x|\dfrac 12<x\le 3\}$, 求实数$a,b$的值.


关联目标:

暂未关联目标



标签: 第一单元

答案: 暂无答案

解答或提示: 暂无解答与提示

使用记录:

暂无使用记录


出处: 代数精编第二章不等式
\item { (005167)}已知关于$x$的不等式$x^2-ax-6a\le 0$有解, 且解$x_1,x_2$满足$|x_1-x_2|\le 5$, 求实数$a$的取值范围.


关联目标:

暂未关联目标



标签: 第一单元

答案: 暂无答案

解答或提示: 暂无解答与提示

使用记录:

暂无使用记录


出处: 代数精编第二章不等式
\item { (005168)}已知关于$x$的方程$3x^2+x\log_{\frac 12}^2a+2\log_{\frac 12}a=0$的两根$x_1,x_2$满足条件$-1<x_1<0<x_2<1$, 求实数$a$的取值范围.


关联目标:

暂未关联目标



标签: 第一单元

答案: 暂无答案

解答或提示: 暂无解答与提示

使用记录:

暂无使用记录


出处: 代数精编第二章不等式
\item { (005169)}已知关于$x$的方程$x^2+(m^2-1)x+m-2=0$的一个根比$-1$小, 另一个根比$1$大, 求参数$m$的取值范围.


关联目标:

暂未关联目标



标签: 第一单元

答案: 暂无答案

解答或提示: 暂无解答与提示

使用记录:

暂无使用记录


出处: 代数精编第二章不等式
\item { (005170)}已知集合$A=\{x|x-a>0\}$, $B=\{x|x^2-2ax-3a^2<0\}$, 求$A\cap B$与$A\cup B$.


关联目标:

暂未关联目标



标签: 第一单元

答案: 暂无答案

解答或提示: 暂无解答与提示

使用记录:

暂无使用记录


出处: 代数精编第二章不等式
\item { (005171)}不等式$\sqrt{x+3}>-1$的解集是\bracket{20}.
\fourch{$\{x|x>-2\}$}{$\{x|x\ge -3\}$}{$\varnothing$}{$\mathbf{R}$}


关联目标:

暂未关联目标



标签: 第一单元

答案: 暂无答案

解答或提示: 暂无解答与提示

使用记录:

暂无使用记录


出处: 代数精编第二章不等式
\item { (005172)}不等式$(x-1)\sqrt{x+2}\ge 0$的解集是\bracket{20}.
\fourch{$\{x|x>1\}$}{$\{x|x\ge 1\}$}{$\{x|x\ge 1\text{或}x=-2\}$}{$\{x|x>1\text{或}x=-2\}$}


关联目标:

暂未关联目标



标签: 第一单元

答案: 暂无答案

解答或提示: 暂无解答与提示

使用记录:

暂无使用记录


出处: 代数精编第二章不等式
\item { (005173)}与不等式$\sqrt{(x-4)(x+3)}\le 1$的解完全相同的不等式是\bracket{20}.
\fourch{$|(x-4)(x+3)|\le 1$}{$(x-4)(x+3)\le 1$}{$\lg [ (x-4)(x+3) ]\le 0$}{$0\le (x-4)(x+3)\le 1$}


关联目标:

暂未关联目标



标签: 第一单元

答案: 暂无答案

解答或提示: 暂无解答与提示

使用记录:

暂无使用记录


出处: 代数精编第二章不等式
\item { (005174)}解不等式: $\sqrt{x-5}+4x-3>3x+1+\sqrt{x-5}$.


关联目标:

暂未关联目标



标签: 第一单元

答案: 暂无答案

解答或提示: 暂无解答与提示

使用记录:

暂无使用记录


出处: 代数精编第二章不等式
\item { (005175)}解不等式: $\sqrt{x^2+1}>\sqrt{x^2-x+3}$.


关联目标:

暂未关联目标



标签: 第一单元

答案: 暂无答案

解答或提示: 暂无解答与提示

使用记录:

暂无使用记录


出处: 代数精编第二章不等式
\item { (005176)}解不等式: $(x-4)\sqrt{x^2-3x-4}\ge 0$.


关联目标:

暂未关联目标



标签: 第一单元

答案: 暂无答案

解答或提示: 暂无解答与提示

使用记录:

暂无使用记录


出处: 代数精编第二章不等式
\item { (005177)}解不等式: $\dfrac{x+1}{x+4}\sqrt{\dfrac{x+3}{1-x}}<0$.


关联目标:

暂未关联目标



标签: 第一单元

答案: 暂无答案

解答或提示: 暂无解答与提示

使用记录:

暂无使用记录


出处: 代数精编第二章不等式
\item { (005178)}解不等式: $\sqrt{x+2}+\sqrt{x-5}\ge \sqrt{5-x}$.


关联目标:

暂未关联目标



标签: 第一单元

答案: 暂无答案

解答或提示: 暂无解答与提示

使用记录:

暂无使用记录


出处: 代数精编第二章不等式
\item { (005179)}解不等式: $\sqrt{x-6}+\sqrt{x-3}\ge \sqrt{3-x}$.


关联目标:

暂未关联目标



标签: 第一单元

答案: 暂无答案

解答或提示: 暂无解答与提示

使用记录:

暂无使用记录


出处: 代数精编第二章不等式
\item { (005180)}解不等式: $\sqrt{2-x}<x$.


关联目标:

暂未关联目标



标签: 第一单元

答案: 暂无答案

解答或提示: 暂无解答与提示

使用记录:

暂无使用记录


出处: 代数精编第二章不等式
\item { (005181)}解不等式: $\sqrt{4-x^2}<x+1$.


关联目标:

暂未关联目标



标签: 第一单元

答案: 暂无答案

解答或提示: 暂无解答与提示

使用记录:

暂无使用记录


出处: 代数精编第二章不等式
\item { (005182)}解不等式: $\sqrt{3-2x}>x$.


关联目标:

暂未关联目标



标签: 第一单元

答案: 暂无答案

解答或提示: 暂无解答与提示

使用记录:

暂无使用记录


出处: 代数精编第二章不等式
\item { (005183)}解不等式: $\sqrt{(x-1)(2-x)}>4-3x$.


关联目标:

暂未关联目标



标签: 第一单元

答案: 暂无答案

解答或提示: 暂无解答与提示

使用记录:

暂无使用记录


出处: 代数精编第二章不等式
\item { (005184)}不等式$\sqrt{4-x^2}+\dfrac{|x|}x\ge 0$的解集是\bracket{20}.
\fourch{$[-2,2]$}{$[-\sqrt 3,0)\cup (0,2]$}{$[-2,0]\cup (0,2]$}{$[-\sqrt 3,0)\cup (0,\sqrt 3]$}


关联目标:

暂未关联目标



标签: 第一单元

答案: 暂无答案

解答或提示: 暂无解答与提示

使用记录:

暂无使用记录


出处: 代数精编第二章不等式
\item { (005185)}已知关于$x$的不等式$\sqrt{2x-x^2}>kx$的解集是$\{x|0<x\le 2\}$, 则实数$k$的取值范围是\bracket{20}.
\fourch{$k<0$}{$k\ge 0$}{$0<k<2$}{$-\dfrac 12<k<0$}


关联目标:

暂未关联目标



标签: 第一单元

答案: 暂无答案

解答或提示: 暂无解答与提示

使用记录:

暂无使用记录


出处: 代数精编第二章不等式
\item { (005186)}解不等式: $\sqrt{2x-4}-\sqrt{x+5}<1$.


关联目标:

暂未关联目标



标签: 第一单元

答案: 暂无答案

解答或提示: 暂无解答与提示

使用记录:

暂无使用记录


出处: 代数精编第二章不等式
\item { (005187)}解不等式: $\sqrt{x^2-5x-6}<|x-3|$.


关联目标:

暂未关联目标



标签: 第一单元

答案: 暂无答案

解答或提示: 暂无解答与提示

使用记录:

暂无使用记录


出处: 代数精编第二章不等式
\item { (005188)}解不等式: $|2\sqrt{x+3}-x+1|<1$.


关联目标:

暂未关联目标



标签: 第一单元

答案: 暂无答案

解答或提示: 暂无解答与提示

使用记录:

暂无使用记录


出处: 代数精编第二章不等式
\item { (005189)}解关于$x$的不等式: $\sqrt{a(a-x)}>a-2x$($a>0$).


关联目标:

暂未关联目标



标签: 第一单元

答案: 暂无答案

解答或提示: 暂无解答与提示

使用记录:

暂无使用记录


出处: 代数精编第二章不等式
\item { (005190)}解关于$x$的不等式: $\sqrt{4x-x^2}>ax$($a<0$).


关联目标:

暂未关联目标



标签: 第一单元

答案: 暂无答案

解答或提示: 暂无解答与提示

使用记录:

暂无使用记录


出处: 代数精编第二章不等式
\item { (005191)}解关于$x$的不等式: $\sqrt{1-ax}<x-1$($a>0$).


关联目标:

暂未关联目标



标签: 第一单元

答案: 暂无答案

解答或提示: 暂无解答与提示

使用记录:

暂无使用记录


出处: 代数精编第二章不等式
\item { (005192)}解关于$x$的不等式: $\sqrt{a^2-x^2}>2x-a$.


关联目标:

暂未关联目标



标签: 第一单元

答案: 暂无答案

解答或提示: 暂无解答与提示

使用记录:

暂无使用记录


出处: 代数精编第二章不等式
\item { (005817)}已知实数集$\mathbf{R}$的子集$P$满足两个条件: \textcircled{1} $1\notin P$; \textcircled{2} 若实数$a\in P$, 则$\dfrac 1{1-a}\in P$. 求证:\\
(1) 若$2\in P$, 则$P$中必含有其他两个数, 并求出这两个数;\\
(2) 集合$P$不可能是单元素集.


关联目标:

暂未关联目标



标签: 第一单元

答案: 暂无答案

解答或提示: 暂无解答与提示

使用记录:

暂无使用记录


出处: 代数精编第三章函数
\item { (005818)}已知集合$A,B,C$满足$A\cap B=A$, $B\cap C=B$, 求证: $A\subseteq C$.


关联目标:

暂未关联目标



标签: 第一单元

答案: 暂无答案

解答或提示: 暂无解答与提示

使用记录:

暂无使用记录


出处: 代数精编第三章函数
\item { (005819)}已知集合$A=\{x|x=a^2+1,\ a\in \mathbf{N}\}$, $B=\{y|y=b^2-4b+5,\ b\in \mathbf{N}\}$, 求证: $A\subset B$.


关联目标:

暂未关联目标



标签: 第一单元

答案: 暂无答案

解答或提示: 暂无解答与提示

使用记录:

暂无使用记录


出处: 代数精编第三章函数
\item { (005820)}已知集合$A=\{x|x=12a+8b,\ a,b\in \mathbf{Z}\}$, $B=\{x|x=20c+16d,\ c,d\in \mathbf{Z}\}$, 求证: $A=B$.


关联目标:

暂未关联目标



标签: 第一单元

答案: 暂无答案

解答或提示: 暂无解答与提示

使用记录:

暂无使用记录


出处: 代数精编第三章函数
\item { (005821)}某班学生期中考试数学得优秀的有$18$人, 物理得优秀的有$14$人, 其中数学、物理两科中至少有一科得优秀的有$22$人, 求两科都得优秀的学生人数.


关联目标:

暂未关联目标



标签: 第一单元

答案: 暂无答案

解答或提示: 暂无解答与提示

使用记录:

暂无使用记录


出处: 代数精编第三章函数
\item { (005822)}由某班学生组成的篮球队、排球队、乒乓球队分别有$14, 15, 13$名队员.已知同时参加这三个队的有$3$人, 既参加篮球队又参加排球队的有$5$人, 仅参加乒乓球队的有$4$人, 仅参加排球队的有$5$人, 问: 仅参加篮球队的有几人.


关联目标:

暂未关联目标



标签: 第一单元

答案: 暂无答案

解答或提示: 暂无解答与提示

使用记录:

暂无使用记录


出处: 代数精编第三章函数
\item { (005823)}某地区先后举行中学生数、理、化三科竞赛, 参加竞赛的学生人数依次是$807$人、$739$人、$437$人, 其中参加数学、物理两科竞赛的有$513$人, 参加物理、化学竞赛的有$267$人, 参加数学、化学竞赛的有$371$人, 三科竞赛都参加的有$213$人, 求参加竞赛的学生总人数.


关联目标:

暂未关联目标



标签: 第一单元

答案: 暂无答案

解答或提示: 暂无解答与提示

使用记录:

暂无使用记录


出处: 代数精编第三章函数
\item { (007783)}解不等式: $(x+1)^2-6>0$.


关联目标:

暂未关联目标



标签: 第一单元

答案: 暂无答案

解答或提示: 暂无解答与提示

使用记录:

暂无使用记录


出处: 二期课改练习册高一第一学期
\item { (007787)}解不等式: $2x-1\ge x^2$.


关联目标:

暂未关联目标



标签: 第一单元

答案: 暂无答案

解答或提示: 暂无解答与提示

使用记录:

暂无使用记录


出处: 二期课改练习册高一第一学期
\item { (007788)}解关于$x$的不等式: $(x-a)(x-1)<0(a>1)$.


关联目标:

暂未关联目标



标签: 第一单元

答案: 暂无答案

解答或提示: 暂无解答与提示

使用记录:

暂无使用记录


出处: 二期课改练习册高一第一学期
\item { (007789)}解关于$x$的不等式: $(x-a)(x-2a)<0(a>0)$.


关联目标:

暂未关联目标



标签: 第一单元

答案: 暂无答案

解答或提示: 暂无解答与提示

使用记录:

暂无使用记录


出处: 二期课改练习册高一第一学期
\item { (007790)}写出一个解集只含一个元素的一元二次不等式.


关联目标:

暂未关联目标



标签: 第一单元

答案: 暂无答案

解答或提示: 暂无解答与提示

使用记录:

暂无使用记录


出处: 二期课改练习册高一第一学期
\item { (007791)}解不等式组: $\begin{cases} 6-x-x^2\le 0, \\ x^2+3x-4<0. \end{cases}$.


关联目标:

暂未关联目标



标签: 第一单元

答案: 暂无答案

解答或提示: 暂无解答与提示

使用记录:

暂无使用记录


出处: 二期课改练习册高一第一学期
\item { (007792)}解不等式组: $\begin{cases} 4x^2-27x+18>0, \\ x^2-6x+4<0. \end{cases}$.


关联目标:

暂未关联目标



标签: 第一单元

答案: 暂无答案

解答或提示: 暂无解答与提示

使用记录:

暂无使用记录


出处: 二期课改练习册高一第一学期
\item { (007793)}已知集合$U=\mathbf{R}$, 且集合$A=\{x|x^2-16<0\}$, 集合$B=\{x|x^2-4x+3\ge 0\}$, 求:\\
(1) $A\cap B$;\\
(2) $A\cup B$;\\
(3) $\complement _U(A\cap B)$;\\
(4) $\complement _UA\cup \complement _UB$.


关联目标:

暂未关联目标



标签: 第一单元

答案: 暂无答案

解答或提示: 暂无解答与提示

使用记录:

暂无使用记录


出处: 二期课改练习册高一第一学期
\item { (007794)}已知不等式$x^2+ax+b<0$的解集为$(-3,-1)$, 求实数$a$、$b$的值.


关联目标:

暂未关联目标



标签: 第一单元

答案: 暂无答案

解答或提示: 暂无解答与提示

使用记录:

暂无使用记录


出处: 二期课改练习册高一第一学期
\item { (007795)}已知关于$x$的二次方程$2x^2+ax+1=0$无实数解, 求实数$a$的取值范围.


关联目标:

暂未关联目标



标签: 第一单元

答案: 暂无答案

解答或提示: 暂无解答与提示

使用记录:

暂无使用记录


出处: 二期课改练习册高一第一学期
\item { (007796)}已知$P(a,b)$为正比例函数$y=2x$的图像上的点, 且$P$与$B(2,-1)$之间的距离不超过$3$, 求$a$的取值范围.


关联目标:

暂未关联目标



标签: 第一单元

答案: 暂无答案

解答或提示: 暂无解答与提示

使用记录:

暂无使用记录


出处: 二期课改练习册高一第一学期
\item { (007797)}某船从甲码头沿河顺流航行$75$千米到达乙码头, 停留$30$分钟后再逆流航行$126$千米到达丙码头.如果水流的速度为每小时$4$千米, 该船要在$5$小时内完成航行任务, 那么船的速度每小时至少为多少千米?


关联目标:

暂未关联目标



标签: 第一单元

答案: 暂无答案

解答或提示: 暂无解答与提示

使用记录:

暂无使用记录


出处: 二期课改练习册高一第一学期
\item { (007798)}解不等式组: $\begin{cases} 3x^2+x-2\ge 0, \\ 4x^2-15x+9>0. \end{cases}$


关联目标:

暂未关联目标



标签: 第一单元

答案: 暂无答案

解答或提示: 暂无解答与提示

使用记录:

暂无使用记录


出处: 二期课改练习册高一第一学期
\item { (007799)}已知关于$x$的不等式组$\begin{cases} (2x-3)(3x+2)\le 0, \\ x-a>0 \end{cases}$无实数解, 求实数$a$的取值范围.


关联目标:

暂未关联目标



标签: 第一单元

答案: 暂无答案

解答或提示: 暂无解答与提示

使用记录:

暂无使用记录


出处: 二期课改练习册高一第一学期
\item { (007837)}证明: 如果$a>b>0$, $c>d>0$, 那么$a^2c>b^2d$.


关联目标:

暂未关联目标



标签: 第一单元

答案: 暂无答案

解答或提示: 暂无解答与提示

使用记录:

暂无使用记录


出处: 二期课改练习册高一第一学期
\item { (007838)}证明: $a^2+b^2+2\ge 2(a+b)$.


关联目标:

暂未关联目标



标签: 第一单元

答案: 暂无答案

解答或提示: 暂无解答与提示

使用记录:

暂无使用记录


出处: 二期课改练习册高一第一学期
\item { (007839)}证明: 如果$a$、$b$、$c$都是正数, 那么$(a+b)(b+c)(c+a)\ge 8abc$.


关联目标:

暂未关联目标



标签: 第一单元

答案: 暂无答案

解答或提示: 暂无解答与提示

使用记录:

暂无使用记录


出处: 二期课改练习册高一第一学期
\item { (007840)}解不等式: $2(x+1)(x+2)>(x+3)(x+4)$.


关联目标:

暂未关联目标



标签: 第一单元

答案: 暂无答案

解答或提示: 暂无解答与提示

使用记录:

暂无使用记录


出处: 二期课改练习册高一第一学期
\item { (007841)}解不等式: $-3x^25x-4<0$.


关联目标:

暂未关联目标



标签: 第一单元

答案: 暂无答案

解答或提示: 暂无解答与提示

使用记录:

暂无使用记录


出处: 二期课改练习册高一第一学期
\item { (007842)}解不等式: $4x^2-20x+25\le 0$.


关联目标:

暂未关联目标



标签: 第一单元

答案: 暂无答案

解答或提示: 暂无解答与提示

使用记录:

暂无使用记录


出处: 二期课改练习册高一第一学期
\item { (007843)}解不等式: $x^2-16x+64>0$.


关联目标:

暂未关联目标



标签: 第一单元

答案: 暂无答案

解答或提示: 暂无解答与提示

使用记录:

暂无使用记录


出处: 二期课改练习册高一第一学期
\item { (007844)}解不等式组: $\begin{cases} x^2-16<0, \\ x^2-4x+3\ge 0. \end{cases}$.


关联目标:

暂未关联目标



标签: 第一单元

答案: 暂无答案

解答或提示: 暂无解答与提示

使用记录:

暂无使用记录


出处: 二期课改练习册高一第一学期
\item { (007845)}解不等式组: $4<x^2-x-2<10$.


关联目标:

暂未关联目标



标签: 第一单元

答案: 暂无答案

解答或提示: 暂无解答与提示

使用记录:

暂无使用记录


出处: 二期课改练习册高一第一学期
\item { (007846)}解不等式: $|\dfrac{3x-9}2|\le 6$.


关联目标:

暂未关联目标



标签: 第一单元

答案: 暂无答案

解答或提示: 暂无解答与提示

使用记录:

暂无使用记录


出处: 二期课改练习册高一第一学期
\item { (007847)}解不等式: $3<|x-2|<5$.


关联目标:

暂未关联目标



标签: 第一单元

答案: 暂无答案

解答或提示: 暂无解答与提示

使用记录:

暂无使用记录


出处: 二期课改练习册高一第一学期
\item { (007848)}解不等式: $|\dfrac 1x|<\dfrac 45$.


关联目标:

暂未关联目标



标签: 第一单元

答案: 暂无答案

解答或提示: 暂无解答与提示

使用记录:

暂无使用记录


出处: 二期课改练习册高一第一学期
\item { (007849)}下列四对不等式(组)中, 哪几对具有相同的解集?\\
(1) $-\dfrac 12x^2+3x+\dfrac{27}2>0$与$x^2-6x-27>0$;\\
(2) $4<x^2-x+2<10$与$\begin{cases} x^2-x+2<10, \\ x^2-x+2>4; \end{cases}$\\
(3) $|2x+1|<5$与$2x+1<5$或$2x+1>-5$;\\
(4) $\dfrac{x-1}{x+1}<2$与$x-1<2(x+1)$.


关联目标:

暂未关联目标



标签: 第一单元

答案: 暂无答案

解答或提示: 暂无解答与提示

使用记录:

暂无使用记录


出处: 二期课改练习册高一第一学期
\item { (007850)}已知关于$x$的不等式$2x^2-2(a-1)x+(a+3)>0$的解集是$\mathbf{R}$, 求实数$a$的取值范围.


关联目标:

暂未关联目标



标签: 第一单元

答案: 暂无答案

解答或提示: 暂无解答与提示

使用记录:

暂无使用记录


出处: 二期课改练习册高一第一学期
\item { (007851)}已知函数$y=(m-1)x^2+(m-3)x+(m-1)$, $m$取什么实数时, 函数图像与$x$轴\\
(1) 没有公共点?\\
(2) 只有一个公共点?\\
(3) 有两个不同的公共点?


关联目标:

暂未关联目标



标签: 第一单元

答案: 暂无答案

解答或提示: 暂无解答与提示

使用记录:

暂无使用记录


出处: 二期课改练习册高一第一学期
\item { (007852)}当$k$是什么实数时, 关于$x$的方程$2x+k(x+3)=4$的解是正数?


关联目标:

暂未关联目标



标签: 第一单元

答案: 暂无答案

解答或提示: 暂无解答与提示

使用记录:

暂无使用记录


出处: 二期课改练习册高一第一学期
\item { (007853)}已知直角三角形的周长为$4$, 求这个直角三角形面积的最大值, 并求此时各边的长.


关联目标:

暂未关联目标



标签: 第一单元

答案: 暂无答案

解答或提示: 暂无解答与提示

使用记录:

暂无使用记录


出处: 二期课改练习册高一第一学期
\item { (007854)}求证: $(\dfrac{a+b}2)^2\le \dfrac{a^2+b^2}2$.


关联目标:

暂未关联目标



标签: 第一单元

答案: 暂无答案

解答或提示: 暂无解答与提示

使用记录:

暂无使用记录


出处: 二期课改练习册高一第一学期
\item { (007855)}求不等式$5\le x^2-2x+2<26$的正整数解.


关联目标:

暂未关联目标



标签: 第一单元

答案: 暂无答案

解答或提示: 暂无解答与提示

使用记录:

暂无使用记录


出处: 二期课改练习册高一第一学期
\item { (007856)}已知$x$、$y\in [a,b]$.\\
(1) 求$x+y$的范围;\\
(2) 若$x<y$, 求$x-y$的范围.


关联目标:

暂未关联目标



标签: 第一单元

答案: 暂无答案

解答或提示: 暂无解答与提示

使用记录:

暂无使用记录


出处: 二期课改练习册高一第一学期
\item { (007857)}当$k$为什么实数时, 方程组$\begin{cases} 3x-6y=1, \\ 5x-ky=2 \end{cases}$的解满足$x<0$且$y<0$的条件?


关联目标:

暂未关联目标



标签: 第一单元

答案: 暂无答案

解答或提示: 暂无解答与提示

使用记录:

暂无使用记录


出处: 二期课改练习册高一第一学期
\item { (007858)}当$k$为什么实数时, 方程组$\begin{cases} 4x+3y=60, \\ kx+(k+2)y=60 \end{cases}$的解满足$x>y>0$的条件?


关联目标:

暂未关联目标



标签: 第一单元

答案: 暂无答案

解答或提示: 暂无解答与提示

使用记录:

暂无使用记录


出处: 二期课改练习册高一第一学期
\item { (007859)}已知$m<n$, 试写出一个形如$ax^2+bx+c>0$的一元二次不等式, 使它的解集分别为:\\
(1) $(-\infty ,m)\cup (n,+\infty)$;\\
(2) $(m,n)$.


关联目标:

暂未关联目标



标签: 第一单元

答案: 暂无答案

解答或提示: 暂无解答与提示

使用记录:

暂无使用记录


出处: 二期课改练习册高一第一学期
\item { (007985)}若集合$A=\{x|0.1<\dfrac 1x<0.3,\ x\in \mathbf{N}\}$, 集合$B=\{x||x|\le 5,\ x\in \mathbf{Z}\}$, 则$A\cup B$中的元素个数是\bracket{20}.
\fourch{$11$}{$13$}{$15$}{$17$}


关联目标:

暂未关联目标



标签: 第一单元

答案: 暂无答案

解答或提示: 暂无解答与提示

使用记录:

暂无使用记录


出处: 二期课改练习册高一第一学期
\item { (007986)}``$x\ne 1$且$y\ne 2$''是``$x+y\ne 3$''的\bracket{20}.
\twoch{充分非必要条件}{必要非充分条件}{充要条件}{既非充分又非必要条件}


关联目标:

暂未关联目标



标签: 第一单元

答案: 暂无答案

解答或提示: 暂无解答与提示

使用记录:

暂无使用记录


出处: 二期课改练习册高一第一学期
\item { (007988)}已知集合$A=\{x|3x^2+x-2\ge 0,\  x\in \mathbf{R}\}$, 集合$B=\{x|\dfrac{4x-3}{x-3}>0,\ x\in \mathbf{R}\}$, 求$A\cap B$.


关联目标:

暂未关联目标



标签: 第一单元

答案: 暂无答案

解答或提示: 暂无解答与提示

使用记录:

暂无使用记录


出处: 二期课改练习册高一第一学期
\item { (007990)}已知集合$A=(-2,-1)\cup (0,+\infty)$, 集合$B=\{x|x^2+ax+b\le 0\}$, 且$A\cap B=(0,2]$, $A\cup B=(-2,+\infty)$, 求实数$a$、$b$的值.


关联目标:

暂未关联目标



标签: 第一单元

答案: 暂无答案

解答或提示: 暂无解答与提示

使用记录:

暂无使用记录


出处: 二期课改练习册高一第一学期
\item { (007995)}已知集合$A=\{x||x-a|<2\}$, 集合$B=\{x|\dfrac{2x-1}{x-2}<1\}$, 且$A\subseteq B$, 求实数$a$的取值范围.


关联目标:

暂未关联目标



标签: 第一单元

答案: 暂无答案

解答或提示: 暂无解答与提示

使用记录:

暂无使用记录


出处: 二期课改练习册高一第一学期
\item { (007996)}已知全集$U=\mathbf{R}$, 集合$A=\{x|x^2+px+12=0\}$, 集合$B=\{x|x-5x-q=0\}$, 满足$(\complement _UA)\cap B=\{2\}$.求实数$p$与$q$的值.


关联目标:

暂未关联目标



标签: 第一单元

答案: 暂无答案

解答或提示: 暂无解答与提示

使用记录:

暂无使用记录


出处: 二期课改练习册高一第一学期
\item { (009426)}判断下列各组对象能否组成集合. 若能组成集合, 指出是有限集还是无限集; 若不能组成集合, 请说明理由.\\
(1) 上海市现有各区的名称;\\
(2) 末位是$3$的自然数;\\
(3) 比较大的苹果.


关联目标:

暂未关联目标



标签: 第一单元

答案: 暂无答案

解答或提示: 暂无解答与提示

使用记录:

暂无使用记录


出处: 新教材必修第一册课堂练习
\item { (009427)}用符号``$\in$''或``$\not\in$''填空:\\
(1) $\dfrac12$\blank{50}$\mathbf{N}$;\\
(2) $5$\blank{50}$\mathbf{Z}$;\\
(3) $-2$\blank{50}$\mathbf{Q}$;\\
(4) $\pi$\blank{50}$\mathbf{R}$.


关联目标:

暂未关联目标



标签: 第一单元

答案: 暂无答案

解答或提示: 暂无解答与提示

使用记录:

暂无使用记录


出处: 新教材必修第一册课堂练习
\item { (009428)}用列举法表示下列集合:\\
(1) 能整除$10$的所有正整数组成的集合;\\
(2) 绝对值小于$4$的所有整数组成的集合.


关联目标:

暂未关联目标



标签: 第一单元

答案: 暂无答案

解答或提示: 暂无解答与提示

使用记录:

暂无使用记录


出处: 新教材必修第一册课堂练习
\item { (009429)}用描述法表示下列集合:\\
(1) 全体偶数组成的集合;\\
(2) 平面直角坐标系中$x$轴上所有点组成的集合.


关联目标:

暂未关联目标



标签: 第一单元

答案: 暂无答案

解答或提示: 暂无解答与提示

使用记录:

暂无使用记录


出处: 新教材必修第一册课堂练习
\item { (009430)}用区间表示下列集合:\\
(1) $\{x|-1<x\le 5\}$;\\
(2) 不等式$-2x>6$的所有解组成的集合.


关联目标:

暂未关联目标



标签: 第一单元

答案: 暂无答案

解答或提示: 暂无解答与提示

使用记录:

暂无使用记录


出处: 新教材必修第一册课堂练习
\item { (009431)}判断下列说法是否正确, 并简要说明理由:\\
(1) 若$a\in A$且$A\subseteq B$, 则$a\in B$;\\
(2) 若$A\subseteq B$且$A\subseteq C$, 则$B=C$;\\
(3) 若$A\subset B$且$B\subseteq C$, 则$A\subset C$.


关联目标:

暂未关联目标



标签: 第一单元

答案: 暂无答案

解答或提示: 暂无解答与提示

使用记录:

暂无使用记录


出处: 新教材必修第一册课堂练习
\item { (009432)}用符号``$\supset$''``$=$''或``$\subset$''填空:\\
(1) $\{a\}$\blank{50}$\{a, b, c\}$;\\
(2) $\{a, b, c\}$\blank{50}$\{a, c\}$;\\
(3) $\{1, 2\}$\blank{50}$\{x|x^2-3x+2=0\}$.


关联目标:

暂未关联目标



标签: 第一单元

答案: 暂无答案

解答或提示: 暂无解答与提示

使用记录:

暂无使用记录


出处: 新教材必修第一册课堂练习
\item { (009433)}写出所有满足$\{a\}\subset M\subset \{a, b, c, d\}$的集合$M$.


关联目标:

暂未关联目标



标签: 第一单元

答案: 暂无答案

解答或提示: 暂无解答与提示

使用记录:

暂无使用记录


出处: 新教材必修第一册课堂练习
\item { (009434)}设$A$为全集$U$的任一子集, 则
(1) $\overline{\overline{A}}=$\blank{50}; (A表示A的补集A的补集)\\
(2) $A\cap \overline A=$\blank{50};\\
(3) $A\cup \overline A=$\blank{50}.


关联目标:

暂未关联目标



标签: 第一单元

答案: 暂无答案

解答或提示: 暂无解答与提示

使用记录:

暂无使用记录


出处: 新教材必修第一册课堂练习
\item { (009435)}已知全集为$\mathbf{R}$, 集合$A=\{x|-2<x\le 1\}$. 求$A$.


关联目标:

暂未关联目标



标签: 第一单元

答案: 暂无答案

解答或提示: 暂无解答与提示

使用记录:

暂无使用记录


出处: 新教材必修第一册课堂练习
\item { (009436)}已知集合$A=\{1, 2, 3, 4, 5\}$, $B=\{2, 4, 6, 8\}$, $C=\{3, 4, 5, 6\}$. 求:\\
(1) $(A\cap B)\cup C$, $(A\cup C)\cap (B\cup C)$;\\
(2) $(A\cup B)\cap C$, $(A\cap C)\cup (B\cap C)$.


关联目标:

暂未关联目标



标签: 第一单元

答案: 暂无答案

解答或提示: 暂无解答与提示

使用记录:

暂无使用记录


出处: 新教材必修第一册课堂练习
\item { (009437)}举几个生活中的命题的例子, 并判断其真假.


关联目标:

暂未关联目标



标签: 第一单元

答案: 暂无答案

解答或提示: 暂无解答与提示

使用记录:

暂无使用记录


出处: 新教材必修第一册课堂练习
\item { (009438)}判断下列命题的真假, 并说明理由:\\
(1) 所有偶数都不是素数;\\
(2) $\{1\}$是$\{0, 1, 2\}$的真子集;\\
(3) $0$是$\{0, 1, 2\}$的真子集;\\
(4) 如果集合$A$是集合$B$的子集, 那么$B$不是$A$的子集.


关联目标:

暂未关联目标



标签: 第一单元

答案: 暂无答案

解答或提示: 暂无解答与提示

使用记录:

暂无使用记录


出处: 新教材必修第一册课堂练习
\item { (009439)}用``$\Rightarrow$''表示下列陈述句$\alpha$与$\beta$之间的推出关系:\\
(1) $\alpha: \triangle ABC$是等边三角形, $\beta: \triangle ABC$是轴对称图形;\\
(2) $\alpha: x^2=4$, $\beta: x=2$.


关联目标:

暂未关联目标



标签: 第一单元

答案: 暂无答案

解答或提示: 暂无解答与提示

使用记录:

暂无使用记录


出处: 新教材必修第一册课堂练习
\item { (009440)}已知$\alpha$: 四边形$ABCD$的两组对边分别平行, $\beta$: 四边形$ABCD$为矩形, $\gamma$: 四边形$ABCD$的两组对边分别相等. 用``充分非必要''``必要非充分''``充要''或``既非充分又非必要''填空:\\
(1) $\alpha$是$\beta$的\blank{50}条件;\\
(2) $\beta$是$\gamma$的\blank{50}条件;\\
(3) $\alpha$是$\gamma$的\blank{50}条件.


关联目标:

暂未关联目标



标签: 第一单元

答案: 暂无答案

解答或提示: 暂无解答与提示

使用记录:

暂无使用记录


出处: 新教材必修第一册课堂练习
\item { (009441)}设$\alpha: 1\le x<4$, $\beta: x<m$, $\alpha$是$\beta$的充分条件. 求实数$m$的取值范围.


关联目标:

暂未关联目标



标签: 第一单元

答案: 暂无答案

解答或提示: 暂无解答与提示

使用记录:

暂无使用记录


出处: 新教材必修第一册课堂练习
\item { (009442)}设$n\in \mathbf{Z}$. 证明: 若$n^3$是奇数, 则$n$是奇数.


关联目标:

暂未关联目标



标签: 第一单元

答案: 暂无答案

解答或提示: 暂无解答与提示

使用记录:

暂无使用记录


出处: 新教材必修第一册课堂练习
\item { (009443)}证明: 对于三个实数$a$、$b$、$c$, 若$a\ne c$, 则$a\ne b$或$b\ne c$.


关联目标:

暂未关联目标



标签: 第一单元

答案: 暂无答案

解答或提示: 暂无解答与提示

使用记录:

暂无使用记录


出处: 新教材必修第一册课堂练习
\item { (009444)}设$a$、$b$、$c$、$d$是实数, 判断下列命题的真假, 并说明理由:\\
(1) 若$a^2=b^2$, 则$a=b$;\\
(2) 若$a(c^2+1)=b(c^2+1)$, 则$a=b$;\\
(3) 若$ab=0$, 则$a=0$或$b=0$;\\
(4) 若$\dfrac ac=\dfrac bd$, 且$c+d\ne 0$, 则$\dfrac{a+b}{c+d}=\dfrac ac$.


关联目标:

暂未关联目标



标签: 第一单元

答案: 暂无答案

解答或提示: 暂无解答与提示

使用记录:

暂无使用记录


出处: 新教材必修第一册课堂练习
\item { (009445)}设$a\in \mathbf{R}$, 求关于$x$的方程$ax=a^2+x-1$的解集.


关联目标:

暂未关联目标



标签: 第一单元

答案: 暂无答案

解答或提示: 暂无解答与提示

使用记录:

暂无使用记录


出处: 新教材必修第一册课堂练习
\item { (009447)}求一元二次方程$ax^2-4x+2=0$($a\ne 0$)的解集.


关联目标:

暂未关联目标



标签: 第一单元

答案: 暂无答案

解答或提示: 暂无解答与提示

使用记录:

暂无使用记录


出处: 新教材必修第一册课堂练习
\item { (009448)}已知方程$2x^2+4x-3=0$的两个根为$x_1$、$x_2$, 求下列各式的值:\\
(1) $x_1^2x_2+x_2^2x_1$;\\
(2) $\dfrac1{x_1}+\dfrac1{x_2}$;\\
(3) $x_1^2+x_2^2$;\\
(4) $x_1^3+x_2^3$.


关联目标:

暂未关联目标



标签: 第一单元

答案: 暂无答案

解答或提示: 暂无解答与提示

使用记录:

暂无使用记录


出处: 新教材必修第一册课堂练习
\item { (009449)}设$a$、$b$、$c$、$d$为实数, 判断下列命题的真假, 并说明理由:\\
(1) 如果$a>b$, $c>d$, 那么$a+d>b+c$;\\
(2) 如果$ab>ac$, 那么$b>c$;\\
(3) 如果$a\ge b$且$a\le b$, 那么$a=b$;\\
(4) 如果$a>b$, $\dfrac 1c>\dfrac 1d$, 那么$ac>bd$;\\
(5) 如果$\dfrac ba>\dfrac dc$, 那么$bc>ad$.


关联目标:

暂未关联目标



标签: 第一单元

答案: 暂无答案

解答或提示: 暂无解答与提示

使用记录:

暂无使用记录


出处: 新教材必修第一册课堂练习
\item { (009450)}设$ab>0$, 求证: $a>b$是$\dfrac 1a<\dfrac 1b$的充要条件.


关联目标:

暂未关联目标



标签: 第一单元

答案: 暂无答案

解答或提示: 暂无解答与提示

使用记录:

暂无使用记录


出处: 新教材必修第一册课堂练习
\end{enumerate}



\end{document}