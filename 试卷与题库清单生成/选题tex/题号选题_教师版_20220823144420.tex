\documentclass[10pt,a4paper]{article}
\usepackage[UTF8,fontset = windows]{ctex}
\setCJKmainfont[BoldFont=黑体,ItalicFont=楷体]{华文中宋}
\usepackage{amssymb,amsmath,amsfonts,amsthm,mathrsfs,dsfont,graphicx}
\usepackage{ifthen,indentfirst,enumerate,color,titletoc}
\usepackage{tikz}
\usepackage{multicol}
\usepackage{makecell}
\usepackage{longtable}
\usetikzlibrary{arrows,calc,intersections,patterns,decorations.pathreplacing,3d,angles,quotes,positioning}
\usepackage[bf,small,indentafter,pagestyles]{titlesec}
\usepackage[top=1in, bottom=1in,left=0.8in,right=0.8in]{geometry}
\renewcommand{\baselinestretch}{1.65}
\newtheorem{defi}{定义~}
\newtheorem{eg}{例~}
\newtheorem{ex}{~}
\newtheorem{rem}{注~}
\newtheorem{thm}{定理~}
\newtheorem{coro}{推论~}
\newtheorem{axiom}{公理~}
\newtheorem{prop}{性质~}
\newcommand{\blank}[1]{\underline{\hbox to #1pt{}}}
\newcommand{\bracket}[1]{(\hbox to #1pt{})}
\newcommand{\onech}[4]{\par\begin{tabular}{p{.9\textwidth}}
A.~#1\\
B.~#2\\
C.~#3\\
D.~#4
\end{tabular}}
\newcommand{\twoch}[4]{\par\begin{tabular}{p{.46\textwidth}p{.46\textwidth}}
A.~#1& B.~#2\\
C.~#3& D.~#4
\end{tabular}}
\newcommand{\vartwoch}[4]{\par\begin{tabular}{p{.46\textwidth}p{.46\textwidth}}
(1)~#1& (2)~#2\\
(3)~#3& (4)~#4
\end{tabular}}
\newcommand{\fourch}[4]{\par\begin{tabular}{p{.23\textwidth}p{.23\textwidth}p{.23\textwidth}p{.23\textwidth}}
A.~#1 &B.~#2& C.~#3& D.~#4
\end{tabular}}
\newcommand{\varfourch}[4]{\par\begin{tabular}{p{.23\textwidth}p{.23\textwidth}p{.23\textwidth}p{.23\textwidth}}
(1)~#1 &(2)~#2& (3)~#3& (4)~#4
\end{tabular}}
\begin{document}

\begin{enumerate}[1.]

\item { (001211)}求下列各函数的单调区间, 并证明.\\ 
(1) $f(x)=2x+3$;\\ 
(2) $f(x)=\dfrac{1}{x}$;\\ 
(3) $f(x)=x^2+2x$;\\ 
(4) $f(x)=x-\dfrac{1}{x}$;\\ 
(5) $f(x)=ax+\dfrac{b}{x}$, 其中$a>0, \ b>0$;


关联目标:

K0219002B|D02003B|会运用函数单调性的定义证明一次函数、二次函数、反比例函数的单调性.

K0219003B|D02003B|会运用函数单调性的定义以及已知的基本初等函数的单调性, 判断较为复杂的函数单调性.

K0220001B|D02003B|理解单调函数、单调区间的定义.

K0220002B|D02003B|会求函数的单调区间.



标签: 第二单元

答案: 暂无答案

解答或提示: 暂无解答与提示

使用记录:

2016届11班	\fcolorbox[rgb]{0,0,0}{1.000,0.000,0}{1.000}	\fcolorbox[rgb]{0,0,0}{1.000,0.206,0}{0.897}	\fcolorbox[rgb]{0,0,0}{1.000,0.052,0}{0.974}	\fcolorbox[rgb]{0,0,0}{1.000,0.052,0}{0.974}	\fcolorbox[rgb]{0,0,0}{1.000,0.154,0}{0.923}

2016届12班	\fcolorbox[rgb]{0,0,0}{1.000,0.000,0}{1.000}	\fcolorbox[rgb]{0,0,0}{1.000,0.158,0}{0.921}	\fcolorbox[rgb]{0,0,0}{1.000,0.052,0}{0.974}	\fcolorbox[rgb]{0,0,0}{1.000,0.158,0}{0.921}	\fcolorbox[rgb]{0,0,0}{1.000,0.474,0}{0.763}


出处: 2016届创新班作业	1139-函数的单调性
\item { (009518)}证明: 函数$y=\dfrac2{x^3}$在区间$(-\infty, 0)$上是严格减函数.


关联目标:

K0219002B|D02003B|会运用函数单调性的定义证明一次函数、二次函数、反比例函数的单调性.

K0219003B|D02003B|会运用函数单调性的定义以及已知的基本初等函数的单调性, 判断较为复杂的函数单调性.



标签: 第二单元

答案: 暂无答案

解答或提示: 暂无解答与提示

使用记录:

暂无使用记录


出处: 新教材必修第一册课堂练习
\item { (001218)}判断下列各函数的单调性, 并证明.\\ 
(1) $f(x)=\sqrt{1+x}$;\\ 
(2) $f(x)=x+x^5,x\in[0,+\infty)$;\\ 
(3) $f(x)=(\sqrt{x}+1)(x^2+1)$;


关联目标:

K0219003B|D02003B|会运用函数单调性的定义以及已知的基本初等函数的单调性, 判断较为复杂的函数单调性.



标签: 第二单元

答案: 暂无答案

解答或提示: 暂无解答与提示

使用记录:

2016届11班	\fcolorbox[rgb]{0,0,0}{1.000,0.000,0}{1.000}	\fcolorbox[rgb]{0,0,0}{1.000,0.102,0}{0.949}	\fcolorbox[rgb]{0,0,0}{1.000,0.666,0}{0.667}

2016届12班	\fcolorbox[rgb]{0,0,0}{1.000,0.052,0}{0.974}	\fcolorbox[rgb]{0,0,0}{1.000,0.210,0}{0.895}	\fcolorbox[rgb]{0,0,0}{1.000,0.526,0}{0.737}


出处: 2016届创新班作业	1140-函数的运算复合与奇偶性单调性的关系
\item { (010178)}证明:函数$y=\lg (1-x)$在其定义域上是严格减函数.


关联目标:

K0219003B|D02003B|会运用函数单调性的定义以及已知的基本初等函数的单调性, 判断较为复杂的函数单调性.



标签: 第二单元

答案: 暂无答案

解答或提示: 暂无解答与提示

使用记录:

暂无使用记录


出处: 新教材必修第一册习题
\item { (009521)}判断函数$y=|x+1|$, $x\in [-2, 2]$的单调性, 并求出其单调区间.


关联目标:

K0219003B|D02003B|会运用函数单调性的定义以及已知的基本初等函数的单调性, 判断较为复杂的函数单调性.



标签: 第二单元

答案: 暂无答案

解答或提示: 暂无解答与提示

使用记录:

暂无使用记录


出处: 新教材必修第一册课堂练习
\item { (002894)}设函数$f(x)=\mathrm{e}^x+\dfrac 1{\mathrm{e}^x}$.\\
(1) 求证: $y=f(x)$在$\mathbf{R}$上不是增函数;\\
(2) 求证: $y=f(x)$在$[0,+\infty)$上是增函数.


关联目标:

K0219003B|D02003B|会运用函数单调性的定义以及已知的基本初等函数的单调性, 判断较为复杂的函数单调性.



标签: 第二单元

答案: 暂无答案

解答或提示: 暂无解答与提示

使用记录:

暂无使用记录


出处: 2022届高三第一轮复习讲义
\item { (000092)}作出函数$y=(x^2-1)^2-1$的大致图像, 写出它的单调区间, 并证明你的结论.


关联目标:

K0219001B|D02003B|理解单调函数、单调区间的定义.

K0219003B|D02003B|会运用函数单调性的定义以及已知的基本初等函数的单调性, 判断较为复杂的函数单调性.

K0220001B|D02003B|理解单调函数、单调区间的定义.



标签: 第二单元

答案: 暂无答案

解答或提示: 暂无解答与提示

使用记录:

暂无使用记录


出处: 教材复习题
\item { (002884)}下列函数中, 在其定义域上是单调函数的序号为\blank{50}.\\
\textcircled{1} $y=\dfrac{2-x}x$; \textcircled{2} $y=x-\dfrac 1x$; \textcircled{3} $y={3^{x-1}}$; \textcircled{4} $y=ln\dfrac 1x$; \textcircled{5} $y=tanx$.


关联目标:

K0220001B|D02003B|理解单调函数、单调区间的定义.



标签: 第二单元

答案: 暂无答案

解答或提示: 暂无解答与提示

使用记录:

暂无使用记录


出处: 2022届高三第一轮复习讲义
\item { (002885)}函数$y=|x-1|$递减区间的是\blank{50}.


关联目标:

K0220001B|D02003B|理解单调函数、单调区间的定义.



标签: 第二单元

答案: 暂无答案

解答或提示: 暂无解答与提示

使用记录:

暂无使用记录


出处: 2022届高三第一轮复习讲义
\item { (007911)}画出函数$y=x^2-2|x|$的图像, 并写出它的定义域、奇偶性、单调区间、最小值.


关联目标:

K0220001B|D02003B|理解单调函数、单调区间的定义.



标签: 第二单元

答案: 暂无答案

解答或提示: 暂无解答与提示

使用记录:

暂无使用记录


出处: 二期课改练习册高一第一学期
\item { (007931)}作出函数$y=|x^2-4x|$的图像, 并指出其单调区间.


关联目标:

K0220001B|D02003B|理解单调函数、单调区间的定义.



标签: 第二单元

答案: 暂无答案

解答或提示: 暂无解答与提示

使用记录:

暂无使用记录


出处: 二期课改练习册高一第一学期
\item { (007932)}作出函数$y=2|x|-3$的图像, 并指出其单调区间.


关联目标:

K0220001B|D02003B|理解单调函数、单调区间的定义.



标签: 第二单元

答案: 暂无答案

解答或提示: 暂无解答与提示

使用记录:

暂无使用记录


出处: 二期课改练习册高一第一学期
\item { (007941)}已知函数$y=f(x)$具有如下性质:\\
\textcircled{1} 定义在$\mathbf{R}$上的偶函数; \textcircled{2} 在$(-\infty ,0)$上为增函数; \textcircled{3} $f(0)=1$; \textcircled{4} $f(-2)=-7$; \textcircled{5} 不是二次函数.\\
求$y=f(x)$的一个可能的解析式.


关联目标:

K0220001B|D02003B|理解单调函数、单调区间的定义.



标签: 第二单元

答案: 暂无答案

解答或提示: 暂无解答与提示

使用记录:

暂无使用记录


出处: 二期课改练习册高一第一学期
\item { (007950)}已知函数$f(x)=\dfrac{ax+1}{x+2},\ a\in \mathbf{Z}$. 是否存在整数$a$, 使函数$f(x)$在$x\in [-1,+\infty)$上递减, 并且$f(x)$不恒为负? 若存在, 找出一个满足条件的$a$; 若不存在, 请说出理由.


关联目标:

K0220001B|D02003B|理解单调函数、单调区间的定义.



标签: 第二单元

答案: 暂无答案

解答或提示: 暂无解答与提示

使用记录:

暂无使用记录


出处: 二期课改练习册高一第一学期
\item { (009517)}小明说: ``如果当$x>0$时, 总有$f(x)>f(0)$, 那么函数$y=f(x)$在区间$[0, +\infty)$上是严格增函数.''他的说法是否正确? 说明理由.


关联目标:

K0220001B|D02003B|理解单调函数、单调区间的定义.



标签: 第二单元

答案: 暂无答案

解答或提示: 暂无解答与提示

使用记录:

暂无使用记录


出处: 新教材必修第一册课堂练习
\item { (010187)}如果函数$y=x^2-2mx+1$在区间$(-\infty, 2]$上是严格减函数, 那么实数$m$的取值范围为\blank{50}.


关联目标:

K0220001B|D02003B|理解单调函数、单调区间的定义.



标签: 第二单元

答案: 暂无答案

解答或提示: 暂无解答与提示

使用记录:

暂无使用记录


出处: 新教材必修第一册习题
\item { (002895)}设常数$a\in \mathbf{R}$. 若$y=\log_{\frac 12}(x^2-ax+2)$在$[-1,+\infty)$上是减函数, 求$a$的取值范围.


关联目标:

K0220001B|D02003B|理解单调函数、单调区间的定义.



标签: 第二单元

答案: 暂无答案

解答或提示: 暂无解答与提示

使用记录:

暂无使用记录


出处: 2022届高三第一轮复习讲义
\item { (001270)}写出下列函数的单调减区间:\\ 
(1) $y=x^2$; \blank{80}\\ 
(2) $y=x^2+2x+3$; \blank{80}\\ 
(3) $y=-x^2+2x+3$; \blank{80}\\ 
(4) $y=\sqrt{-x^2+2x+3}$. \blank{80}


关联目标:

K0220002B|D02003B|会求函数的单调区间.



标签: 第二单元

答案: 暂无答案

解答或提示: 暂无解答与提示

使用记录:

2016届11班	\fcolorbox[rgb]{0,0,0}{1.000,0.000,0}{1.000}	\fcolorbox[rgb]{0,0,0}{1.000,0.000,0}{1.000}	\fcolorbox[rgb]{0,0,0}{1.000,0.000,0}{1.000}	\fcolorbox[rgb]{0,0,0}{1.000,0.316,0}{0.842}

2016届12班	\fcolorbox[rgb]{0,0,0}{1.000,0.000,0}{1.000}	\fcolorbox[rgb]{0,0,0}{1.000,0.000,0}{1.000}	\fcolorbox[rgb]{0,0,0}{1.000,0.162,0}{0.919}	\fcolorbox[rgb]{0,0,0}{1.000,0.378,0}{0.811}


出处: 2016届创新班作业	1147-二次函数
\item { (002887)}函数$y=(\dfrac 12)^{x^2}$的递减区间是\blank{50}.


关联目标:

K0220002B|D02003B|会求函数的单调区间.



标签: 第二单元

答案: 暂无答案

解答或提示: 暂无解答与提示

使用记录:

暂无使用记录


出处: 2022届高三第一轮复习讲义
\item { (002888)}函数$y=\dfrac 1{\sqrt{x^2+2x-3}}$的递增区间是\blank{50}.


关联目标:

K0220002B|D02003B|会求函数的单调区间.



标签: 第二单元

答案: 暂无答案

解答或提示: 暂无解答与提示

使用记录:

暂无使用记录


出处: 2022届高三第一轮复习讲义
\item { (002977)}若函数$f(x)=x+\dfrac 4x$($1\le x\le 5$), 则函数$y=f(x)$的递减区间是\blank{50}, 递增区间是\blank{50}, 最小值是\blank{50}, 最大值是\blank{50}.


关联目标:

K0220002B|D02003B|会求函数的单调区间.



标签: 第二单元

答案: 暂无答案

解答或提示: 暂无解答与提示

使用记录:

暂无使用记录


出处: 2022届高三第一轮复习讲义
\item { (002982)}函数$y=2x+\dfrac 1x$($x<0$)的递增区间是\blank{50}.


关联目标:

K0220002B|D02003B|会求函数的单调区间.



标签: 第二单元

答案: 暂无答案

解答或提示: 暂无解答与提示

使用记录:

暂无使用记录


出处: 2022届高三第一轮复习讲义
\item { (004265)}已知a为实数, 函数$f(x)=x|x-a|-a$, $x\in \mathbf{R}$.\\
(1) 当$a=2$时, 求函数$f(x)$的单调递增区间;\\
(2) 若对任意$x\in (0,1)$, $f(x)<0$恒成立, 求a的取值范围.


关联目标:

K0220002B|D02003B|会求函数的单调区间.



标签: 第二单元

答案: 暂无答案

解答或提示: 暂无解答与提示

使用记录:

20220517	2022届高三1班	\fcolorbox[rgb]{0,0,0}{1.000,0.140,0}{0.930}	\fcolorbox[rgb]{0,0,0}{1.000,0.396,0}{0.802}


出处: 2022届高三下学期测验卷10第18题
\item { (001278)}试分析函数$y=x+\sqrt{4-x^2}$的单调性. (提示, 分$x\le0$和$x \ge 0$讨论, 有一部分比较容易)


关联目标:

K0220002B|D02003B|会求函数的单调区间.



标签: 第二单元

答案: 暂无答案

解答或提示: 暂无解答与提示

使用记录:

2016届11班	\fcolorbox[rgb]{0,0,0}{1.000,0.948,0}{0.526}

2016届12班	\fcolorbox[rgb]{0,0,0}{1.000,0.810,0}{0.595}


出处: 2016届创新班作业	1147-二次函数
\item { (001331)}函数$y=\log_{x^2+x-1} 2$的递增区间是\blank{150}.


关联目标:

K0220002B|D02003B|会求函数的单调区间.



标签: 第二单元

答案: 暂无答案

解答或提示: 暂无解答与提示

使用记录:

2016届11班	\fcolorbox[rgb]{0,0,0}{0.106,1.000,0}{0.053}

2016届12班	\fcolorbox[rgb]{0,0,0}{0.368,1.000,0}{0.184}


出处: 2016届创新班作业	1155-对数函数
\item { (002889)}设常数$a\in \mathbf{R}$.若$y=\dfrac{ax}{x+1}$在区间$(-1,+\infty)$上递增, 则$a$的取值范围是\blank{50}.


关联目标:

K0220002B|D02003B|会求函数的单调区间.



标签: 第二单元

答案: 暂无答案

解答或提示: 暂无解答与提示

使用记录:

暂无使用记录


出处: 2022届高三第一轮复习讲义
\item { (002893)}设常数$a\in \mathbf{R}$. 若函数$f(x)=\begin{cases} x+a,& x<1, \\ x^2,& x\ge 1 \end{cases}$在$\mathbf{R}$上递增, 则$a$的取值范围为\blank{50}.


关联目标:

K0220002B|D02003B|会求函数的单调区间.



标签: 第二单元

答案: 暂无答案

解答或提示: 暂无解答与提示

使用记录:

暂无使用记录


出处: 2022届高三第一轮复习讲义
\item { (007939)}已知$y=f(x)$是定义在$(-1,1)$上的奇函数, 在区间$[0,1)$上是减函数, 且$f(1-a)+f(1-a^2)<0$, 求实数$a$的取值范围.


关联目标:

K0220003B|D02003B|能直观地感知奇偶性可用于分析单调性并能说理.



标签: 第二单元

答案: 暂无答案

解答或提示: 暂无解答与提示

使用记录:

暂无使用记录


出处: 二期课改练习册高一第一学期
\item { (008392)}定义在$\mathbf{R}$上的偶函数$f(x)$在$[0,+\infty)$上是增函数, 且$f(\dfrac 12)=0$, 则满足$f(\log _{\dfrac 14}x)>0$的$x$的值范围是\blank{50}.


关联目标:

K0220003B|D02003B|能直观地感知奇偶性可用于分析单调性并能说理.



标签: 第二单元

答案: 暂无答案

解答或提示: 暂无解答与提示

使用记录:

暂无使用记录


出处: 二期课改练习册高一第二学期
\item { (009522)}设$y=f(x)$是奇函数, 且它在区间$(-3, 0]$上是严格增函数.\\
(1) 求证: 它在区间$[0, 3)$上是严格增函数;\\
(2) $y=f(x)$是否在区间$(-3, 3)$上是严格增函数? 说明理由.


关联目标:

K0220003B|D02003B|能直观地感知奇偶性可用于分析单调性并能说理.



标签: 第二单元

答案: 暂无答案

解答或提示: 暂无解答与提示

使用记录:

暂无使用记录


出处: 新教材必修第一册课堂练习
\item { (002899)}已知$y=f(x)$是偶函数, 且在区间$[0,4]$上递减. 记$a=f(2)$, $b=f(-3)$, $c=f(-4)$, 则将$a,b,c$按从小到大的顺序排列是	\blank{50}.


关联目标:

K0220003B|D02003B|能直观地感知奇偶性可用于分析单调性并能说理.



标签: 第二单元

答案: 暂无答案

解答或提示: 暂无解答与提示

使用记录:

暂无使用记录


出处: 2022届高三第一轮复习讲义
\end{enumerate}



\end{document}