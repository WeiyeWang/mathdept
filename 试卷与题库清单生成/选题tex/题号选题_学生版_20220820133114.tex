\documentclass[10pt,a4paper]{article}
\usepackage[UTF8,fontset = windows]{ctex}
\setCJKmainfont[BoldFont=黑体,ItalicFont=楷体]{华文中宋}
\usepackage{amssymb,amsmath,amsfonts,amsthm,mathrsfs,dsfont,graphicx}
\usepackage{ifthen,indentfirst,enumerate,color,titletoc}
\usepackage{tikz}
\usepackage{multicol}
\usepackage{makecell}
\usepackage{longtable}
\usetikzlibrary{arrows,calc,intersections,patterns,decorations.pathreplacing,3d,angles,quotes,positioning}
\usepackage[bf,small,indentafter,pagestyles]{titlesec}
\usepackage[top=1in, bottom=1in,left=0.8in,right=0.8in]{geometry}
\renewcommand{\baselinestretch}{1.65}
\newtheorem{defi}{定义~}
\newtheorem{eg}{例~}
\newtheorem{ex}{~}
\newtheorem{rem}{注~}
\newtheorem{thm}{定理~}
\newtheorem{coro}{推论~}
\newtheorem{axiom}{公理~}
\newtheorem{prop}{性质~}
\newcommand{\blank}[1]{\underline{\hbox to #1pt{}}}
\newcommand{\bracket}[1]{(\hbox to #1pt{})}
\newcommand{\onech}[4]{\par\begin{tabular}{p{.9\textwidth}}
A.~#1\\
B.~#2\\
C.~#3\\
D.~#4
\end{tabular}}
\newcommand{\twoch}[4]{\par\begin{tabular}{p{.46\textwidth}p{.46\textwidth}}
A.~#1& B.~#2\\
C.~#3& D.~#4
\end{tabular}}
\newcommand{\vartwoch}[4]{\par\begin{tabular}{p{.46\textwidth}p{.46\textwidth}}
(1)~#1& (2)~#2\\
(3)~#3& (4)~#4
\end{tabular}}
\newcommand{\fourch}[4]{\par\begin{tabular}{p{.23\textwidth}p{.23\textwidth}p{.23\textwidth}p{.23\textwidth}}
A.~#1 &B.~#2& C.~#3& D.~#4
\end{tabular}}
\newcommand{\varfourch}[4]{\par\begin{tabular}{p{.23\textwidth}p{.23\textwidth}p{.23\textwidth}p{.23\textwidth}}
(1)~#1 &(2)~#2& (3)~#3& (4)~#4
\end{tabular}}
\begin{document}

\begin{enumerate}[1.]

\item {\tiny (007783)}解不等式: $(x+1)^2-6>0$.
\item {\tiny (007787)}解不等式: $2x-1\ge x^2$.
\item {\tiny (007788)}解关于$x$的不等式: $(x-a)(x-1)<0(a>1)$.
\item {\tiny (007789)}解关于$x$的不等式: $(x-a)(x-2a)<0(a>0)$.
\item {\tiny (007790)}写出一个解集只含一个元素的一元二次不等式.
\item {\tiny (007791)}解不等式组: $\begin{cases} 6-x-x^2\le 0, \\ x^2+3x-4<0. \end{cases}$.
\item {\tiny (007792)}解不等式组: $\begin{cases} 4x^2-27x+18>0, \\ x^2-6x+4<0. \end{cases}$.
\item {\tiny (007793)}已知集合$U=\mathbf{R}$, 且集合$A=\{x|x^2-16<0\}$, 集合$B=\{x|x^2-4x+3\ge 0\}$, 求:\\
(1) $A\cap B$;\\
(2) $A\cup B$;\\
(3) $\complement _U(A\cap B)$;\\
(4) $\complement _UA\cup \complement _UB$.
\item {\tiny (007794)}已知不等式$x^2+ax+b<0$的解集为$(-3,-1)$, 求实数$a$、$b$的值.
\item {\tiny (007795)}已知关于$x$的二次方程$2x^2+ax+1=0$无实数解, 求实数$a$的取值范围.
\item {\tiny (007796)}已知$P(a,b)$为正比例函数$y=2x$的图像上的点, 且$P$与$B(2,-1)$之间的距离不超过$3$, 求$a$的取值范围.
\item {\tiny (007797)}某船从甲码头沿河顺流航行$75$千米到达乙码头, 停留$30$分钟后再逆流航行$126$千米到达丙码头.如果水流的速度为每小时$4$千米, 该船要在$5$小时内完成航行任务, 那么船的速度每小时至少为多少千米?
\item {\tiny (007798)}解不等式组: $\begin{cases} 3x^2+x-2\ge 0, \\ 4x^2-15x+9>0. \end{cases}$
\item {\tiny (007799)}已知关于$x$的不等式组$\begin{cases} (2x-3)(3x+2)\le 0, \\ x-a>0 \end{cases}$无实数解, 求实数$a$的取值范围.
\item {\tiny (007837)}证明: 如果$a>b>0$, $c>d>0$, 那么$a^2c>b^2d$.
\item {\tiny (007838)}证明: $a^2+b^2+2\ge 2(a+b)$.
\item {\tiny (007839)}证明: 如果$a$、$b$、$c$都是正数, 那么$(a+b)(b+c)(c+a)\ge 8abc$.
\item {\tiny (007840)}解不等式: $2(x+1)(x+2)>(x+3)(x+4)$.
\item {\tiny (007841)}解不等式: $-3x^25x-4<0$.
\item {\tiny (007842)}解不等式: $4x^2-20x+25\le 0$.
\item {\tiny (007843)}解不等式: $x^2-16x+64>0$.
\item {\tiny (007844)}解不等式组: $\begin{cases} x^2-16<0, \\ x^2-4x+3\ge 0. \end{cases}$.
\item {\tiny (007845)}解不等式组: $4<x^2-x-2<10$.
\item {\tiny (007846)}解不等式: $|\dfrac{3x-9}2|\le 6$.
\item {\tiny (007847)}解不等式: $3<|x-2|<5$.
\item {\tiny (007848)}解不等式: $|\dfrac 1x|<\dfrac 45$.
\item {\tiny (007849)}下列四对不等式(组)中, 哪几对具有相同的解集?\\
(1) $-\dfrac 12x^2+3x+\dfrac{27}2>0$与$x^2-6x-27>0$;\\
(2) $4<x^2-x+2<10$与$\begin{cases} x^2-x+2<10, \\ x^2-x+2>4; \end{cases}$\\
(3) $|2x+1|<5$与$2x+1<5$或$2x+1>-5$;\\
(4) $\dfrac{x-1}{x+1}<2$与$x-1<2(x+1)$.
\item {\tiny (007850)}已知关于$x$的不等式$2x^2-2(a-1)x+(a+3)>0$的解集是$\mathbf{R}$, 求实数$a$的取值范围.
\item {\tiny (007851)}已知函数$y=(m-1)x^2+(m-3)x+(m-1)$, $m$取什么实数时, 函数图像与$x$轴\\
(1) 没有公共点?\\
(2) 只有一个公共点?\\
(3) 有两个不同的公共点?
\item {\tiny (007852)}当$k$是什么实数时, 关于$x$的方程$2x+k(x+3)=4$的解是正数?
\item {\tiny (007853)}已知直角三角形的周长为$4$, 求这个直角三角形面积的最大值, 并求此时各边的长.
\item {\tiny (007854)}求证: $(\dfrac{a+b}2)^2\le \dfrac{a^2+b^2}2$.
\item {\tiny (007855)}求不等式$5\le x^2-2x+2<26$的正整数解.
\item {\tiny (007856)}已知$x$、$y\in [a,b]$.\\
(1) 求$x+y$的范围;\\
(2) 若$x<y$, 求$x-y$的范围.
\item {\tiny (007857)}当$k$为什么实数时, 方程组$\begin{cases} 3x-6y=1, \\ 5x-ky=2 \end{cases}$的解满足$x<0$且$y<0$的条件?
\item {\tiny (007858)}当$k$为什么实数时, 方程组$\begin{cases} 4x+3y=60, \\ kx+(k+2)y=60 \end{cases}$的解满足$x>y>0$的条件?
\item {\tiny (007859)}已知$m<n$, 试写出一个形如$ax^2+bx+c>0$的一元二次不等式, 使它的解集分别为:\\
(1) $(-\infty ,m)\cup (n,+\infty)$;\\
(2) $(m,n)$.
\item {\tiny (007985)}若集合$A=\{x|0.1<\dfrac 1x<0.3,\ x\in \mathbf{N}\}$, 集合$B=\{x||x|\le 5,\ x\in \mathbf{Z}\}$, 则$A\cup B$中的元素个数是\bracket{20}.
\fourch{$11$}{$13$}{$15$}{$17$}
\item {\tiny (007986)}``$x\ne 1$且$y\ne 2$''是``$x+y\ne 3$''的\bracket{20}.
\twoch{充分非必要条件}{必要非充分条件}{充要条件}{既非充分又非必要条件}
\item {\tiny (007988)}已知集合$A=\{x|3x^2+x-2\ge 0,\  x\in \mathbf{R}\}$, 集合$B=\{x|\dfrac{4x-3}{x-3}>0,\ x\in \mathbf{R}\}$, 求$A\cap B$.
\item {\tiny (007990)}已知集合$A=(-2,-1)\cup (0,+\infty)$, 集合$B=\{x|x^2+ax+b\le 0\}$, 且$A\cap B=(0,2]$, $A\cup B=(-2,+\infty)$, 求实数$a$、$b$的值.
\item {\tiny (007995)}已知集合$A=\{x||x-a|<2\}$, 集合$B=\{x|\dfrac{2x-1}{x-2}<1\}$, 且$A\subseteq B$, 求实数$a$的取值范围.
\item {\tiny (007996)}已知全集$U=\mathbf{R}$, 集合$A=\{x|x^2+px+12=0\}$, 集合$B=\{x|x-5x-q=0\}$, 满足$(\complement _UA)\cap B=\{2\}$.求实数$p$与$q$的值.
\item {\tiny (009426)}判断下列各组对象能否组成集合. 若能组成集合, 指出是有限集还是无限集; 若不能组成集合, 请说明理由.\\
(1) 上海市现有各区的名称;\\
(2) 末位是$3$的自然数;\\
(3) 比较大的苹果.
\item {\tiny (009427)}用符号``$\in$''或``$\not\in$''填空:\\
(1) $\dfrac12$\blank{50}$\mathbf{N}$;\\
(2) $5$\blank{50}$\mathbf{Z}$;\\
(3) $-2$\blank{50}$\mathbf{Q}$;\\
(4) $\pi$\blank{50}$\mathbf{R}$.
\item {\tiny (009428)}用列举法表示下列集合:\\
(1) 能整除$10$的所有正整数组成的集合;\\
(2) 绝对值小于$4$的所有整数组成的集合.
\item {\tiny (009429)}用描述法表示下列集合:\\
(1) 全体偶数组成的集合;\\
(2) 平面直角坐标系中$x$轴上所有点组成的集合.
\item {\tiny (009430)}用区间表示下列集合:\\
(1) $\{x|-1<x\le 5\}$;\\
(2) 不等式$-2x>6$的所有解组成的集合.
\item {\tiny (009431)}判断下列说法是否正确, 并简要说明理由:\\
(1) 若$a\in A$且$A\subseteq B$, 则$a\in B$;\\
(2) 若$A\subseteq B$且$A\subseteq C$, 则$B=C$;\\
(3) 若$A\subset B$且$B\subseteq C$, 则$A\subset C$.
\item {\tiny (009432)}用符号``$\supset$''``$=$''或``$\subset$''填空:\\
(1) $\{a\}$\blank{50}$\{a, b, c\}$;\\
(2) $\{a, b, c\}$\blank{50}$\{a, c\}$;\\
(3) $\{1, 2\}$\blank{50}$\{x|x^2-3x+2=0\}$.
\item {\tiny (009433)}写出所有满足$\{a\}\subset M\subset \{a, b, c, d\}$的集合$M$.
\item {\tiny (009434)}设$A$为全集$U$的任一子集, 则
(1) $\overline{\overline{A}}=$\blank{50}; (A表示A的补集A的补集)\\
(2) $A\cap \overline A=$\blank{50};\\
(3) $A\cup \overline A=$\blank{50}.
\item {\tiny (009435)}已知全集为$\mathbf{R}$, 集合$A=\{x|-2<x\le 1\}$. 求$A$.
\item {\tiny (009436)}已知集合$A=\{1, 2, 3, 4, 5\}$, $B=\{2, 4, 6, 8\}$, $C=\{3, 4, 5, 6\}$. 求:\\
(1) $(A\cap B)\cup C$, $(A\cup C)\cap (B\cup C)$;\\
(2) $(A\cup B)\cap C$, $(A\cap C)\cup (B\cap C)$.
\item {\tiny (009437)}举几个生活中的命题的例子, 并判断其真假.
\item {\tiny (009438)}判断下列命题的真假, 并说明理由:\\
(1) 所有偶数都不是素数;\\
(2) $\{1\}$是$\{0, 1, 2\}$的真子集;\\
(3) $0$是$\{0, 1, 2\}$的真子集;\\
(4) 如果集合$A$是集合$B$的子集, 那么$B$不是$A$的子集.
\item {\tiny (009439)}用``$\Rightarrow$''表示下列陈述句$\alpha$与$\beta$之间的推出关系:\\
(1) $\alpha: \triangle ABC$是等边三角形, $\beta: \triangle ABC$是轴对称图形;\\
(2) $\alpha: x^2=4$, $\beta: x=2$.
\item {\tiny (009440)}已知$\alpha$: 四边形$ABCD$的两组对边分别平行, $\beta$: 四边形$ABCD$为矩形, $\gamma$: 四边形$ABCD$的两组对边分别相等. 用``充分非必要''``必要非充分''``充要''或``既非充分又非必要''填空:\\
(1) $\alpha$是$\beta$的\blank{50}条件;\\
(2) $\beta$是$\gamma$的\blank{50}条件;\\
(3) $\alpha$是$\gamma$的\blank{50}条件.
\item {\tiny (009441)}设$\alpha: 1\le x<4$, $\beta: x<m$, $\alpha$是$\beta$的充分条件. 求实数$m$的取值范围.
\item {\tiny (009442)}设$n\in \mathbf{Z}$. 证明: 若$n^3$是奇数, 则$n$是奇数.
\item {\tiny (009443)}证明: 对于三个实数$a$、$b$、$c$, 若$a\ne c$, 则$a\ne b$或$b\ne c$.
\item {\tiny (009444)}设$a$、$b$、$c$、$d$是实数, 判断下列命题的真假, 并说明理由:\\
(1) 若$a^2=b^2$, 则$a=b$;\\
(2) 若$a(c^2+1)=b(c^2+1)$, 则$a=b$;\\
(3) 若$ab=0$, 则$a=0$或$b=0$;\\
(4) 若$\dfrac ac=\dfrac bd$, 且$c+d\ne 0$, 则$\dfrac{a+b}{c+d}=\dfrac ac$.
\item {\tiny (009445)}设$a\in \mathbf{R}$, 求关于$x$的方程$ax=a^2+x-1$的解集.
\item {\tiny (009447)}求一元二次方程$ax^2-4x+2=0$($a\ne 0$)的解集.
\item {\tiny (009448)}已知方程$2x^2+4x-3=0$的两个根为$x_1$、$x_2$, 求下列各式的值:\\
(1) $x_1^2x_2+x_2^2x_1$;\\
(2) $\dfrac1{x_1}+\dfrac1{x_2}$;\\
(3) $x_1^2+x_2^2$;\\
(4) $x_1^3+x_2^3$.
\item {\tiny (009449)}设$a$、$b$、$c$、$d$为实数, 判断下列命题的真假, 并说明理由:\\
(1) 如果$a>b$, $c>d$, 那么$a+d>b+c$;\\
(2) 如果$ab>ac$, 那么$b>c$;\\
(3) 如果$a\ge b$且$a\le b$, 那么$a=b$;\\
(4) 如果$a>b$, $\dfrac 1c>\dfrac 1d$, 那么$ac>bd$;\\
(5) 如果$\dfrac ba>\dfrac dc$, 那么$bc>ad$.
\item {\tiny (009450)}设$ab>0$, 求证: $a>b$是$\dfrac 1a<\dfrac 1b$的充要条件.
\end{enumerate}



\end{document}