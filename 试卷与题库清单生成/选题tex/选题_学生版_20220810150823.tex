\documentclass[10pt,a4paper]{article}
\usepackage[UTF8,fontset = windows]{ctex}
\setCJKmainfont[BoldFont=黑体,ItalicFont=楷体]{华文中宋}
\usepackage{amssymb,amsmath,amsfonts,amsthm,mathrsfs,dsfont,graphicx}
\usepackage{ifthen,indentfirst,enumerate,color,titletoc}
\usepackage{tikz}
\usepackage{multicol}
\usepackage{makecell}
\usepackage{longtable}
\usetikzlibrary{arrows,calc,intersections,patterns,decorations.pathreplacing,3d,angles,quotes,positioning}
\usepackage[bf,small,indentafter,pagestyles]{titlesec}
\usepackage[top=1in, bottom=1in,left=0.8in,right=0.8in]{geometry}
\renewcommand{\baselinestretch}{1.65}
\newtheorem{defi}{定义~}
\newtheorem{eg}{例~}
\newtheorem{ex}{~}
\newtheorem{rem}{注~}
\newtheorem{thm}{定理~}
\newtheorem{coro}{推论~}
\newtheorem{axiom}{公理~}
\newtheorem{prop}{性质~}
\newcommand{\blank}[1]{\underline{\hbox to #1pt{}}}
\newcommand{\bracket}[1]{(\hbox to #1pt{})}
\newcommand{\onech}[4]{\par\begin{tabular}{p{.9\textwidth}}
A.~#1\\
B.~#2\\
C.~#3\\
D.~#4
\end{tabular}}
\newcommand{\twoch}[4]{\par\begin{tabular}{p{.46\textwidth}p{.46\textwidth}}
A.~#1& B.~#2\\
C.~#3& D.~#4
\end{tabular}}
\newcommand{\vartwoch}[4]{\par\begin{tabular}{p{.46\textwidth}p{.46\textwidth}}
(1)~#1& (2)~#2\\
(3)~#3& (4)~#4
\end{tabular}}
\newcommand{\fourch}[4]{\par\begin{tabular}{p{.23\textwidth}p{.23\textwidth}p{.23\textwidth}p{.23\textwidth}}
A.~#1 &B.~#2& C.~#3& D.~#4
\end{tabular}}
\newcommand{\varfourch}[4]{\par\begin{tabular}{p{.23\textwidth}p{.23\textwidth}p{.23\textwidth}p{.23\textwidth}}
(1)~#1 &(2)~#2& (3)~#3& (4)~#4
\end{tabular}}
\begin{document}

\begin{enumerate}[1.]

\item {\tiny (000003)}(1) 若$\alpha$: $x^2-5x+6=0$, $\beta$: $x=2$, 则$\alpha$是$\beta$的\blank{50}条件;
(2) 若$\alpha$: 四边形$ABCD$是正方形, $\beta$: 四边形$ABCD$的两条对角线互相垂直平分, 则$\alpha$是$\beta$的\blank{50}条件.
\item {\tiny (000008)}设$a$是实数. 若$x=1$是$x>a$的一个充分条件, 则$a$的取值范围为\blank{50}.
\item {\tiny (000009)}已知陈述句$\alpha$是$\beta$的充分非必要条件. 若集合$M=\{x|x\text{满足}\alpha\}$, $N=\{x|x\text{满足}\beta\}$, 则$M$与$N$的关系为\bracket{20}.
\fourch{$M\subset N$}{$M\supset N$}{$M=N$}{$M\cap N=\varnothing$}
\item {\tiny (000010)}证明: 若梯形的对角线不相等, 则该梯形不是等腰梯形.
\item {\tiny (000012)}若$\alpha$是$\beta$的必要非充分条件, $\beta$是$\gamma$的充要条件, $\gamma$是$\delta$的必要非充分条件, 则$\delta$是$\alpha$的\blank{50}条件, $\gamma$是$\alpha$的\blank{50}条件.
\item {\tiny (000017)}证明: $\sqrt[3]{2}$是无理数.
\item {\tiny (000018)}设$a,b$是正整数. 求证: 若$ab-1$是$3$的倍数, 则$a$与$b$被$3$除的余数相同.
\item {\tiny (000019)}已知非空数集$S$满足: 对任意给定的$x,y\in S$($x,y$可以相同), 有$x+y\in S$且$x-y\in S$.\\
(1) 哪个数一定是$S$中的元素? 说明理由;\\
(2) 若$S$是有限集, 求$S$;\\
(3) 若$S$中最小的正数为$5$, 求$S$.
\item {\tiny (000326)}若``$a>b$'', 则``$a^3>b^3$''是\blank{50}命题(填: 真、假).
\item {\tiny (000431)}已知角$A$是$\triangle ABC$的内角, 则``$\cos A=\dfrac12$''是``$\sin A=\dfrac{\sqrt3}2$''的\blank{50}条件(填``充分非必要''、``必要非充分''、``充要条件''、``既非充分又非必要''之一).
\item {\tiny (000436)}``$x<0$''是``$x<a$''的充分非必要条件, 则$a$的取值范围是\blank{50}.
\item {\tiny (000648)}若``$x^2-2x-3>0$''是``$x<a$''的必要不充分条件, 则$a$的最大值为\blank{50}.
\item {\tiny (000975)}下列各句是否是命题? (T or F)\\ 
\blank{30} (1) $1$是偶数;\\ 
\blank{30} (2) 线段$AB$太长;\\ 
\blank{30} (3) 所有有理数都大于零;\\ 
\blank{30} (4) $2>5$;\\ 
\blank{30} (5) 存在实数$a$使$|a|=-a$不成立.
\item {\tiny (000976)}在下列各命题的右边写出其否定形式(否定命题).\\ 
(1) $2 \times 2 =5$; \blank{150}.\\ 
(2) $\sqrt{3-\pi}$有意义; \blank{150}.\\ 
(3) $a$不是非负数; \blank{150}.\\ 
(4) $\sqrt{a}$不是无理数; \blank{150}.(本小题中已知$a\ge 0$)\\ 
(5) $x=1$不是方程$x(x+1)=0$的根; \blank{150}.
\item {\tiny (000977)}下列各组命题是否互为否定形式(否定命题)? (T or F).\\ 
\blank{30}(1) 所有直角三角形都不是等边三角形; / 所有直角三角形都是等边三角形.\\ 
\blank{30}(2) 对一切实数$x$, $x^2+1 \ne 0$; / 存在实数$x$, 使得$x^2+1=0$.\\ 
\blank{30}(3) 所有一元二次方程都没有实数根; / 有些一元二次方程没有实数根.\\ 
\blank{30}(4) 所有自然数都不是$0$; / 所有自然数都是$0$.\\ 
\blank{30}(5) 存在实数$x$, 使得$x^2-5x+6=0$; / 所有实数$x$, 都使得$x^2-5x+6\ne 0$.\\ 
\blank{30}(6) 对于一些实数$x$, $x^3+1=0$; / 对于一些实数$x$, $x^3+1\ne 0$.\\ 
\blank{30}(7) 有些三角形两边的平方和等于第三边的平方; / 所有三角形两边的平方和不等于第三边的平方.\\ 
\blank{30}(8) 对于某些实数$x$, $x=x+1$; / 对于任意实数$x$, $x \ne x+1$.\\ 
\blank{30}(9) 负实数没有平方根; / 负实数有平方根.
\item {\tiny (000978)}在下列各命题的右边写出其否定命题.\\ 
(1) $a=0$且$b=0$; \blank{150}.\\ 
(2) $x>0$或$x \le -3$; \blank{150}.\\ 
(3*) 平面上的点$P$在第一象限或第二象限; \blank{150}.
\item {\tiny (000979)}下列各组命题是否互为否定形式(否定命题)? (T or F).\\ 
\blank{30}(1) $a,b$都是偶数; / $a,b$都不是偶数.\\ 
\blank{30}(2) $a,b$不都是偶数; / $a,b$都是偶数.\\ 
\blank{30}(3) $a,b$中至少有一个是偶数; / $a,b$中至多有两个是偶数.\\ 
\blank{30}(4) $a,b$都不是偶数; / $a,b$都是奇数.
\item {\tiny (000981)}在下列各命题的右边写出其否定形式.\\ 
(1) 若$x$是实数, 则$x^2+x+1>0$; \blank{30}$x$是实数, 使得$x^2+x+1$\blank{10}$0$.\\ 
(2) 若$a>0$, 则$|a|\le a$; \blank{150}.\\ 
(3) 若实数$x$满足$x^2-x=0$, 则$x=1$或$x=0$; \blank{150}.\\ 
(4) 若实数$x$满足$x^2-x<0$, 则$0<x<1$; \blank{150}.
\item {\tiny (000983)}用反证法证明如下命题:\\ 
(1) 已知$n$是整数. 如果$3$整除$n^3$, 则$3$整除$n$(提示: 讨论$n=3k,3k+1,3k+2$, 其中$k$是整数);\\ 
(2) 如果实数$x$满足$x^{101}-4x^2+8x-1=0$, 则$x>0$;\\ 
(3) $\sqrt[3]{3}$是无理数(提示: 可借鉴讲义上$\sqrt{6}$是无理数的证明方法);\\ 
(4*) $\sqrt{2}+\sqrt{3}$是无理数.
\item {\tiny (000985)}写出下列各命题的逆命题, 否命题, 逆否命题, 并判断真假.\\ 
(1) (已知$a,b$均为实数) 若$a^2+b^2=0$, 则$a=0$. 原命题的真值: \blank{30};\\ 
逆命题: \blank{250}; 逆命题的真值: \blank{30};\\ 
否命题: \blank{250}; 否命题的真值: \blank{30};\\ 
逆否命题: \blank{250}; 逆否命题的真值: \blank{30}.\\ 
(2) 若$ab=0$, 则$a=0$或$b=0$. 原命题的真值: \blank{30};\\ 
逆命题: \blank{250}; 逆命题的真值: \blank{30};\\ 
否命题: \blank{250}; 否命题的真值: \blank{30};\\ 
逆否命题: \blank{250}; 逆否命题的真值: \blank{30}.\\ 
(3) (已知$a,b$均为整数) 若$a,b$都是偶数, 则$a+b$是偶数. 原命题的真值: \blank{30};\\ 
逆命题: \blank{250}; 逆命题的真值: \blank{30};\\ 
否命题: \blank{250}; 否命题的真值: \blank{30};\\ 
逆否命题: \blank{250}; 逆否命题的真值: \blank{30}.\\ 
(4) (已知$a,b$均为整数) 若$ab$是奇数, 则$a,b$中至少有一个是奇数. 原命题的真值: \blank{30};\\ 
逆命题: \blank{250}; 逆命题的真值: \blank{30};\\ 
否命题: \blank{250}; 否命题的真值: \blank{30};\\ 
逆否命题: \blank{250}; 逆否命题的真值: \blank{30}.
\item {\tiny (000986)}在下列横线上填写A, B, C 或 D. \\ 
\twoch{充分不必要条件}{必要不充分条件}{充分必要条件}{既不充分又不必要条件}\\ 
(1) ``$b=0$''是``直线$y=kx+b$过原点''的\blank{30};\\ 
(2) ``$x^2-1=0$''是``$x-1=0$''的\blank{30};\\ 
(3) ``$m$是正整数''是``$m$是有理数''的\blank{30};\\ 
(4) ``$x<5$''是``$x<3$''的\blank{30};\\ 
(5) ``一个自然数的末位数是$0$''是``这个自然数可被$5$整除''的\blank{30};\\ 
(6) ``$x+y+z>0$''是``$x,y,z$均大于零''的\blank{30};\\ 
(7) ``一个自然数的末位数是$3,6$或$9$''是``这个自然数可被$3$整除''的\blank{30};\\ 
(8) ``一个三角形中存在两个角相等''是``这个三角形是等腰三角形''的\blank{30};\\ 
(9) 已知$x$是实数, ``$x=\sqrt{2}$''是``$x^2=2$''的\blank{30};\\ 
(10) ``$x+y=0$且$xy=0$''是''$x=y=0$''的\blank{30};\\ 
(11) 已知$a,b,c$是实数, $c \ne 0$. ``$ac>bc$''是``$a>b$''的\blank{30};\\ 
(12) ``$x>y>0$''是``$x>0$且$y>0$''的\blank{30};\\ 
(13) 已知$x,y$均为实数. ``$|x|=y$''是``$x=\pm y$''的\blank{30}.
\item {\tiny (000987)}已知实数$t\ne 0$. 证明: ``$x=t$是方程$a x^3+b x^2+cx+d=0$的根''的充分必要条件是``$x=\dfrac{1}{t}$是方程$d x^3+c x^2+ b x+a=0$的根''.
\item {\tiny (000988)}已知$a,b,c$均为实数. 证明: 这三个数中``任意两数之和大于第三个数''的充分必要条件是``任意两数之差小于第三个数''.
\item {\tiny (000996)}以下各命题中, 真命题有: \blank{80}(可能多选).
\fourch{$\varnothing \in \varnothing$}{$\varnothing \in \{\varnothing\}$}{$\varnothing \subseteq \varnothing$}{$\varnothing \subseteq \{\varnothing\}$}
\item {\tiny (000997)}以下各命题中, 真命题有: \blank{80}(可能多选).
\fourch{$5\in \{x|x\le 10\}$}{$\{5\} \in \{x|x\le 10\}$}{$\varnothing \in \{1,2,3,4\}$}{$\varnothing \subseteq \{1,2,3,4\}$}
\item {\tiny (001029)}设$f(x)$是$m$次多项式, $g(x)$是$n$次多项式, $m,n$均为正整数. 判断下列命题的真假(T or F).\\ 
\blank{30} (1) 多项式$-2f(x)$的次数为$m$;\\ 
\blank{30} (2) 多项式$f(x)+g(x)$的次数为$\max\{m,n\}$($\max$表示集合中较大的那个数);\\ 
\blank{30} (3) 多项式$f(x)\times g(x)$的次数为$m+n$;\\ 
\blank{30} (4) 多项式$[f(x)]^2+f(x)+1$的次数为$2m$;
\item {\tiny (001491)}判断下列命题的真假, 真命题用``{\rm T}''表示, 假命题用``{\rm F}''表示.\\ 
\begin{enumerate}[\blank{30}(1)]
\item 设函数$y=f(x)$的定义域为$\mathbf{R}$, 若$1$是它的一个周期, 则$2$也是它的一个周期;\\ 
\item 设函数$y=f(x)$的定义域为$D$, 若$1$是它的一个周期, 则$2$也是它的一个周期;\\ 
\item 设函数$y=f(x)$的定义域为$\mathbf{R}$, 若$1$是它的一个周期, 则$-1$也是它的一个周期;\\ 
\item 设函数$y=f(x)$的定义域为$D$, 若$1$是它的一个周期, 则$-1$也是它的一个周期;\\ 
\item 设函数$f(x)$的定义域为$\mathbf{R}$, 若$1$是它的一个周期, 则$\sqrt{2}$一定不是它的周期;\\ 
\item 设函数$f(x)$的定义域为$\mathbf{R}$, 且$f(x)$不是常数函数, 若$1$是它的一个周期, 则$\sqrt{2}$一定不是它的周期;\\ 
\item 定义在$\mathbf{R}$上的常数函数是周期函数;\\ 
\item 奇函数一定是周期函数;\\ 
\item 奇函数一定不是周期函数;\\ 
\item 偶函数一定是周期函数;\\ 
\item 偶函数一定不是周期函数;\\ 
\item 单调函数一定不是周期函数;\\ 
\item 一定不存在正实数$M$, 使得周期函数$y=f(x)$的定义域包含于区间$[-M,M]$;\\ 
\item 如果$1$是函数$y=f(x)$, $y=g(x)$的周期, 且$f(x)$与$g(x)$定义域的交集非空, 那么$1$也是$y=f(x)+g(x)$的周期;\\ 
\item 设$f(x),g(x)$的定义域均为$\mathbf{R}$, 若$1$是函数$y=f(x)$的周期, 则$1$是函数$y=f(g(x))$的周期;\\ 
\item 设$f(x),g(x)$的定义域均为$\mathbf{R}$, 若$1$是函数$y=g(x)$的周期, 则$1$是函数$y=f(g(x))$的周期;\\ 
\item $y=\sin x,\ x\in (-\infty,0)\cup (0,+\infty)$是周期函数;\\ 
\item $y=\sin x,\ x\in (0,+\infty)$是周期函数;\\ 
\item 周期函数一定有最大值和最小值;\\ 
\item 定义域为$\mathbf{R}$的周期函数一定有最大值和最小值.\\ 
\end{enumerate}
\item {\tiny (001595)}判断下列命题的真假, 在横线上用``T''或``F''表示.
\begin{enumerate}[\blank{30}(1)]
\item 空间任意三点确定一个平面;\\ 
\item 空间任意两条直线确定一个平面;\\ 
\item 空间两条平行直线确定一个平面;\\ 
\item 空间一条直线和不在该直线上的一个点确定一个平面;\\ 
\item 空间一个点和不通过该点的一条直线确定一个平面;\\ 
\item 空间两条没有交点的直线必平行;\\ 
\item 若空间四边形$ABCD$若满足$AB=BC=CD=DA$, 则它一定是菱形;\\ 
\item 若空间的一条直线如果和一对平行直线之一相交, 则一定与另一条也相交;\\ 
\item 若空间三点$A,B,C$若满足$AB^2+BC^2=CA^2$, 则$\triangle ABC$是以$B$为直角顶点的直角三角形;\\ 
\item 若空间三条直线两两相交, 则通过它们中至少两条的平面有且仅有$1$个;\\ 
\item 若空间三条直线两两相交, 则通过它们中至少两条的平面有且仅有$3$个.\\ 
\end{enumerate}
\item {\tiny (001598)}判断下列命题的真假, 在横线上用``T''或``F''表示.
\begin{enumerate}[\blank{30}(1)]
\item 已知$\alpha$, $\beta$是两个平面, $l,m$是两条直线, 若$l\subsetneqq \alpha$, $m\subsetneqq \beta$, 则$l,m$异面;\\ 
\item 已知平面$\alpha,\beta$相交于直线$l$. 若直线$m\subsetneqq \alpha$, $l \parallel m$, 直线$n\subsetneqq \beta$, $l$与$n$相交, 则$m$与$n$异面;\\ 
\item 已知$l,m$是异面直线, 若直线$n\parallel l$, 则$m,n$异面;\\ 
\item 已知$l,m$是异面直线, 若直线$n$和$l$异面, 则$m,n$异面;\\ 
\item 已知$l,m$是异面直线, 若直线$n$和$l$异面, 则$m,n$共面;\\ 
\item 分别和两异面直线都相交的两直线一定是异面直线;\\ 
\item 分别和两异面直线相交的两直线不可能是平行直线;\\ 
\item 正方体的任意两条对角线(指相对顶点, 不同时出现在六个表面的任何一个上的顶点的连线)相交.\\ 
\end{enumerate}
\item {\tiny (001605)}判断下列命题的真假, 在横线上用``T''或``F''表示.\\ 
\begin{enumerate}[\blank{30}(1)]
\item 平行四边形一定在一个平面上;\\ 
\item 若直线$a,b,c$满足$a\perp b$, $a\perp c$, 则$b,c$重合或平行;\\ 
\item 存在一个空间四边形$ABCD$, 它的任意两条邻边的夹角均等于$60^\circ$;\\ 
\item 和两条异面直线都平行的直线不存在;\\ 
\item 过空间一点, 与已知直线垂直的直线有且只有一条;\\ 
\item 若$a,b$是异面直线, $b,c$是异面直线, 则$a,c$也是异面直线;\\ 
\item 若$a,b$是相交直线, $b,c$是相交直线, 则$a,c$也是相交直线;\\ 
\item 有三个角是直角的四边形是矩形;\\ 
\item 异面直线$a,b$和另一直线$c$分别所成的角的大小一定不相等.\\ 
\end{enumerate}
\item {\tiny (001610)}判断下列命题的真假, 并用``{\rm T}''或``{\rm F}''表示:
\begin{enumerate}[\blank{30}(1)]
\item 如果两条直线和同一平面平行, 那么这两直线平行;\\ 
\item 如果两直线和同一平面平行, 那么这两直线平行或相交;\\ 
\item 同时和两异面直线平行的平面有无数个;\\ 
\item 若直线$a\subsetneqq$平面$\alpha$, 直线$b$不在平面$\alpha$内, $a\cap b=\varnothing$, 则$b \parallel \alpha$;\\ 
\item 直线$a\parallel$直线$b$, 直线$b\parallel$平面$\alpha$. 若直线$a$不在$\alpha$内, 则$a\parallel \alpha$;\\ 
\item 直线$a\parallel$平面$\alpha$, 直线$a\subsetneqq$平面$\beta$. 若$\alpha\cap\beta=b$, 则$a\parallel b$;\\ 
\item 过异面直线$a,b$外一点有且仅有一个平面和$a,b$平行;\\ 
\end{enumerate}
\item {\tiny (001629)}判断下列命题的真假, 并用``{\rm T}''或``{\rm F}''表示.\\ 
\begin{enumerate}[\blank{30}(1)]
\item 在正方体$ABCD-A'B'C'D'$中, $BC'$与对角面$BB'D'D$所成的角是$\angle C'BB'$.\\ 
\item 两条异面直线在同一个平面上的射影不可能是两个点.\\ 
\item 已知$P$是三角形$ABC$所在平面外一点, 且$PA=PB$, 则$P$点在平面$ABC$上的射影一定在$AB$的中垂线(在平面$ABC$内)上.\\ 
\item 已知$P$是三角形$ABC$所在平面外一点, 且$PA=PB=PC$, 则$P$点在平面$ABC$上的射影一定在三角形$ABC$内部.\\ 
\item 已知$P$是三角形$ABC$所在平面外一点, 且$PA=PB=PC$, 则$P$点在平面$ABC$上的射影一定不与$A$重合.\\ 
\item 若两直线分别与一平面所成角相等,则两直线平行.\\ 
\item 平面$\alpha$的斜线$a$在平面$\alpha$内的射影是直线$b$, 如果直线$c\perp b$,那么$c\perp a$.\\ 
\item 若平面$\alpha$外两直线$a,b$在$\alpha$上的射影是两相交直线, 则$a$与$b$相交.\\ 
\item 两条异面直线在同一平面上的射影是两条相交或平行直线.\\ 
\item 已知平面$\alpha$有一条斜线$l$, 过平面上一点$A$, 在平面$\alpha$内有且只有一条直线与斜线$l$垂直.\\ 
\end{enumerate}
\item {\tiny (001642)}判断下列命题的真假, 并用``{\rm T}''或``{\rm F}''表示.\\ 
\begin{enumerate}[\blank{30}(1)]
\item 过平面$\alpha$外一点, 有且仅有一个平面与平面$\alpha$平行.\\ 
\item 已知直线$l$平行于平面$\alpha$, 过$l$有且仅有一个平面与平面$\alpha$平行.\\ 
\item 已知直线$l$不在平面$\alpha$内, 过$l$有且仅有一个平面与平面$\alpha$平行.\\ 
\item 平面$\alpha$平行于平面$\beta$, $l\subsetneqq \alpha$, $m\subsetneqq \beta$, 则$l,m$平行.\\ 
\item 已知$l,m$是两异面直线, 存在平面$\alpha,\beta$, 满足$l\subsetneqq \alpha$, $m\subsetneqq\beta$, 并且$\alpha\parallel \beta$.\\ 
\item 已知$l,m$是两平行直线, $l\subsetneqq \alpha$, $m\subsetneqq \beta$. 若$l\parallel \beta$, $m\parallel \alpha$, 则$\alpha\parallel \beta$.\\ 
\item 平面$\alpha$与平面$\beta$平行, 当且仅当在$\alpha$内有无穷多条直线与$\beta$ 平行.\\ 
\end{enumerate}
\item {\tiny (001682)}判断下列命题的真假, 真命题用``{\textrm T}''表示, 假命题用``{\textrm F}''表示.\\ 
\begin{enumerate}[\blank{30}(1)]
\item 有两个面互相平行, 其余的面都是四边形的多面体是棱柱.\\ 
\item 有两个面互相平行, 其余的面都是平行四边形的多面体(未必是凸多面体)是棱柱.\\ 
\item 有两个侧面是矩形的棱柱是直棱柱.\\ 
\item 棱柱被平行于侧棱的平面所截, 截面(若存在的话)是平行四边形.\\ 
\item 直平行六面体是长方体.\\ 
\item 正四棱柱是正方体.\\ 
\item 棱柱成为直棱柱的一个必要不充分的条件是棱柱有一个侧面与底面的一条边垂直.\\ 
\item 若直平行六面体的底面既有内切圆又有外接圆,则它必是正四棱柱.\\ 
\end{enumerate}
\item {\tiny (001695)}判断下列命题的真假, 真命题用``{\textrm T}'', 假命题用``{\textrm F}''表示.\\ 
\begin{enumerate}[\blank{30}(1)]
\item 有一个面是多边形,其余各面都是三角形的多面体是棱锥.\\ 
\item 侧面都是全等等腰三角形的棱锥是正棱锥.\\ 
\item 相邻两条侧棱间的夹角都相等的棱锥是正棱锥.\\ 
\item 各条侧棱与底面的所成角都相等的棱锥是正棱锥.\\ 
\item 各侧棱在底面内的射影都相等的棱锥是正棱锥.\\ 
\item 各侧棱都相等且底面多边形的各边也都相等的棱锥是正棱锥.\\ 
\item 底面三角形的各边分别与相对的侧棱垂直的三棱锥是正三棱锥.\\ 
\item 一个三棱锥的底面是直角三角形,则它的三个侧面至多有两个直角三角形.
\end{enumerate}
\item {\tiny (001832)}判断下列命题的真假, 其中假命题用``F''表示, 真命题用``T''表示.\\ 
\begin{enumerate}[\blank{20}(1)]
\item 递增数列都有极限;\\ 
\item 如果数列$\{a_n\}$有极限, 那么数列$\{|a_n|\}$也有极限;\\ 
\item 如果数列$\{|a_n|\}$有极限, 那么数列$\{a_n\}$也有极限;\\ 
\item 如果数列$\displaystyle\lim_{n\rightarrow \infty} a_n=A$, 那么$\displaystyle\lim_{n\rightarrow \infty} na_n=nA$;\\ 
\item 如果数列$\{a_n\}$有极限, 那么$\displaystyle\lim_{n\rightarrow \infty} a_n=\displaystyle\lim_{n\rightarrow \infty} a_{n+1}$;\\ 
\item 如果数列$\{a_n\}$有极限, 且其前$n$项和为$S_n$,那么$\displaystyle\lim_{n\rightarrow \infty} S_n=\displaystyle\lim_{n\rightarrow \infty} a_{1}+\displaystyle\lim_{n\rightarrow \infty} a_{2}+\cdots+\displaystyle\lim_{n\rightarrow \infty} a_{n}$;\\ 
\item 如果$2011$个数列的极限均为零, 那么这$2011$个数列之和的极限也为零;\\ 
\item 如果数列$\{a_n\}$和$\{b_n\}$使得数列$\{a_n\cdot b_n\}$的极限存在, 那么$\{a_n\}$和$\{b_n\}$的极限都存在;\\ 
\item 如果数列$\{a_n\}$的极限存在, 数列$\{b_n\}$使得数列$\{a_n\cdot b_n\}$的极限存在, 那么$\{b_n\}$的极限存在;\\ 
\item 如果数列$\{a_n\}$和$\{b_n\}$使得数列$\{a_n\cdot b_n\}$的极限为$0$, 那么$\displaystyle\lim_{n\rightarrow \infty} a_n=0$或$\displaystyle\lim_{n\rightarrow \infty} b_n=0$;\\ 
\item 如果数列$\{a_n\}$的极限是$0$, 那么对任意数列$\{b_n\}$, 均成立$\displaystyle\lim_{n\rightarrow \infty} a_n\cdot b_n=0$;\\ 
\item 如果数列$\{a_n\}$和$\{b_n\}$有极限, 且$a_n>b_n$, 那么$\displaystyle\lim_{n\rightarrow \infty} a_n\geq\displaystyle\lim_{n\rightarrow \infty} b_n$.
\end{enumerate}
\item {\tiny (001856)}判断下列命题的真假, 如果是假命题则在命题前的横线上写上``F'', 如果是真命题则写上``T''.\\ 
\begin{enumerate}[\blank{30}(1)]
\item 与非零向量$\overrightarrow{a}$平行的单位向量一定是$\dfrac{1}{|\overrightarrow{a}|}\overrightarrow{a}$.\\ 
\item 若两个非零向量互相平行, 则这两个向量所在的直线平行或重合.\\ 
\item 若非零向量$\overrightarrow{a},\overrightarrow{b},\overrightarrow{c}$满足$\overrightarrow{a}+\overrightarrow{b}+\overrightarrow{c}=\overrightarrow{0}$, 则$\overrightarrow{a},\overrightarrow{b},\overrightarrow{c}$可以依次首尾相接构成三角形.\\ 
\item 若$\overrightarrow{a}$与$\overrightarrow{b}$平行, 则存在实数$\lambda$, 使得$\overrightarrow{b}=\lambda\overrightarrow{a}$.\\ 
\item 若存在实数$\lambda$, 使得$\overrightarrow{b}=\lambda\overrightarrow{a}$, 则$\overrightarrow{a}$与$\overrightarrow{b}$平行.\\ 
\item 若$\overrightarrow{a}$与$\overrightarrow{b}$平行, 则存在实数$\lambda,\mu$, 使得$\lambda\overrightarrow{a}+\mu\overrightarrow{b}=\overrightarrow{0}$.\\ 
\item 若$\overrightarrow{a}$与$\overrightarrow{b}$平行, 则存在不全为零的实数$\lambda,\mu$, 使得$\lambda\overrightarrow{a}+\mu\overrightarrow{b}=\overrightarrow{0}$.\\ 
\item 若存在不全为零的实数$\lambda,\mu$, 使得$\lambda\overrightarrow{a}+\mu\overrightarrow{b}=\overrightarrow{0}$, 则$\overrightarrow{a}$与$\overrightarrow{b}$平行.\\ 
\end{enumerate}
\item {\tiny (002709)}设函数$f(x)=\lg (\dfrac2{x+1}-1)$的定义域为集合$A$, 函数$g(x)=\sqrt{1-|x+a|}$的定义域为集合$B$.\\
(1) 当$a=1$时, 求集合$B$.\\
(2) 问: $a\ge 2$是$A\cap B=\varnothing$的什么条件(在``充分非必要条件、必要非充分条件、充要条件、既非充分也非必要条件''中选一)?并证明你的结论.
\item {\tiny (002731)}填写下列命题的否定形式:\\
(1) $m\le 0$或$n>0$: \blank{200};\\
(2) 空间三条直线$l,m,n$两两相交: \blank{200};\\
(3) 复数$z_1,z_2,z_3$中至多一个为纯虚数: \blank{200}.
\item {\tiny (002732)}已知$a,b$是整数, 写出命题``若$ab$为偶数, 则$a+b$为偶数''的逆命题、否命题、逆否命题, 并判断所写命题的真假.\\
逆命题:\blank{200}, 真假: \blank{20};\\
否命题:\blank{200}, 真假: \blank{20};\\
逆否命题:\blank{200}, 真假: \blank{20}.
\item {\tiny (002733)}设甲是乙的充分非必要条件, 乙是丙的充要条件, 丁是丙的必要非充分条件, 则丁是甲的\bracket{20}.
\twoch{充分非必要条件}{必要非充分条件}{充要条件}{既非充分又非必要条件}
\item {\tiny (002734)}若$A$是$B$的必要非充分条件, 则$\overline{A}$是$\overline{B}$的\blank{50}条件.
\item {\tiny (002735)}下列各组命题中互为等价命题的是\bracket{20}.
\twoch{$A\subseteq B$与$A\cup B=B$}{$x\in A$且$x\in B$与$x\in A\cup B$}{$a\in A\cap B$与$a\in A$或$a\in B$}{$m\in A\cap B$与$m\in A\cup B$}
\item {\tiny (002737)}``$a>0b>0$''的一个必要非充分条件是\bracket{20}.
\fourch{$a>0$}{$b>0$}{$a>0b>0$}{$a,b\in \mathbf{R}$}
\item {\tiny (002738)}``函数$f(x)\ (x\in \mathbf{R})$存在反函数''是``函数$f(x)$在$\mathbf{R}$上为增函数''的\bracket{20}.
\twoch{充分而不必要条件}{必要而不充分条件}{充分必要条件}{既不充分也不必要条件}
\item {\tiny (002740)}(1) 是否存在实数$m$, 使得$2x+m<0$是${x^2}-2x-3>0$的充分条件? 说明理由.\\
(2) 是否存在实数$m$, 使得$2x+m<0$是$x^2-2x-3>0$的必要条件? 说明理由.
\item {\tiny (002741)}已知关于$x$的实系数二次方程$a x^2 +bx+c=0\ (a>0)$, 分别求下列命题的一个充要条件:\\
(1) 方程有一正根, 一根是零;\\
(2) 两根都比$2$小.
\item {\tiny (002742)}设$a,b\in \mathbf{R}$, 写出命题``若$a+b>0$且$ab>0$, 则$a>0$且$b>0$''的逆否命题.
\item {\tiny (002744)}已知$x,y\in \mathbf{R}$, 有如下四个命题: \textcircled{1} $x^2+y^2<1$; \textcircled{2} $|x|+|y|<1$; \textcircled{3} $|x|<1$且$y|<1$; \textcircled{4} $|x+y|<1$. 则\blank{50}是\blank{50}的充分非必要条件(答案可能不唯一).
\item {\tiny (002745)}使不等式$2x^2-5x-3\ge 0$成立的一个充分不必要条件是\bracket{20}. 
\fourch{$x<0$}{$x\ge 0$}{$x\in \{-1,3,5\}$}{$x\le \dfrac12$或x$\ge 3$}
\item {\tiny (002746)}已知$\alpha$:``$x\ge a$'', $\beta$:``$|x-1|\le 1$'', 若$\alpha$是$\beta$的必要非充分条件, 则实数$a$的取值范围是\blank{50}.
\item {\tiny (002747)}命题甲: 关于$x$的方程$x^2+x+m=0$有两个相异的负根; 命题乙: 关于$x$的方程$4x^2+x+m=0$无实根, 若这两个命题有且只有一个是真命题, 求实数$m$的取值范围.
*
\item {\tiny (002748)}已知$P=\{x|x^2-8x-20 \le 0\}$, $S=\{x||x-a|\le m\}$, 求实数$a,m$的值, 使得``$x\in P$''是``$x\in S$''的充要条件.
*
\item {\tiny (002750)}命题(1) $a>b\Rightarrow ac^2>bc^2$;   (2) $ac^2>bc^2\Rightarrow a>b$;     (3) $a>b\Rightarrow \dfrac 1a<\dfrac 1b$; (4) $a<b<0, \ c<d<0\Rightarrow ac>bd$;   (5) $\sqrt[n]a>\sqrt[n]b\Rightarrow a>b \ (n\in \mathbf{N}^*)$;    (6) $a+c<b+d\Leftrightarrow \begin{cases} a<b, \\ c<d; \end{cases}$ (7) $a<b<0\Rightarrow a^2>ab>b^2$. 其中真命题的序号是\blank{50}.
\item {\tiny (002751)}已知$a,b\in \mathbf{R}$, 则$ab(a-b)<0$成立的一个充要条件是\bracket{20}.
\fourch{$\dfrac 1a>\dfrac 1b>0$}{$\dfrac 1a<\dfrac 1b$}{$0<\dfrac 1a<\dfrac 1b$}{$\dfrac 1a>\dfrac 1b$}
\item {\tiny (003665)}已知$a\in \mathbf{R}$, 则``$a>1$''是``$\dfrac{1}{a}<1$''的\bracket{15}.
\twoch{充分非必要条件}{必要非充分条件}{充要条件}{既非充分又非必要条件}
\item {\tiny (003708)}设$\alpha:2\le x\le 4$, $\beta: m+1\le x\le 2m+4, \ m\in \mathbf{R}$, 如果$\alpha$是$\beta$的充分非必要条件, 则$m$的范围是\blank{50}.
\item {\tiny (003729)}``$(2x+1)x=0$''是``$x=0$''的\blank{30}.
\twoch{充分不必要条件}{必要不充分条件}{充分必要条件}{既不充分也不必要条件}
\item {\tiny (003758)}已知$a\in\mathbf{R}$, 命题$P:$``实系数一元二次方程$x^2+ax+2=0$的两根都是虚数''; 命题$Q:$``存在复数$z$同时满足$|z|=2$且$|z+a|=1$''.
是判断命题$P$和命题$Q$之间是否存在推出关系? 说明你的理由.
\item {\tiny (003800)}下列命题中正确的是\blank{30}.
\twoch{若$ac>bc$, 则$a>b$}{若$a^2>b^2$, 则$a>b$}{若$\dfrac 1a>\dfrac 1b$, 则$a<b$}{若$\sqrt{a}<\sqrt{b}$, 则$a<b$}
\item {\tiny (003818)}已知$P=\{x|x^2-8x-20\le 0\}$, $S=\{x|1-m\le x\le 1+m\}$.\\
(1) 是否存在实数$m$, 使$x\in P$是$x\in S$的充要条件, 若存在, 求出$m$的范围;\\
(2) 是否存在实数$m$, 使$x\in P$是$x\in S$的必要条件, 若存在, 求出$m$的范围.
\item {\tiny (003861)}设$A(-1,0)$, $B(1,0)$, 条件甲: $A,B,C$是以$C$为直角顶点的三角形的三个顶点; 条件乙: $C$的坐标是方程$x^2+y^2=1$的解, 则甲是乙的\blank{30}.
\fourch{充分非必要条件}{必要非充分条件}{充要条件}{既不充分又不必要条件}
\item {\tiny (003905)}已知条件$p: |x+1|>2$, 条件$q: x>a$, 且$\bar{p}$是$\bar{q}$的充分不必要条件, 则$a$的取值范围可以是\blank{30}.
\fourch{$a\ge 1$}{$a\le 1$}{$a\ge -1$}{$a\le -3$}
\item {\tiny (003980)}(理科)在极坐标系中, ``点$P$是极点''是``点$P$的极坐标是$(0,0)$''成立的\blank{30}.
\fourch{充分不必要条件}{必要不充分条件}{充要条件}{既不充分也不必要条件}\\
(文科)$\overrightarrow a,\overrightarrow b$为非零向量, ``函数$f(x)=(x\overrightarrow a+\overrightarrow b)^2$为偶函数''是``$\overrightarrow a\perp \overrightarrow b$''的\blank{30}.
\fourch{充分不必要条件}{必要不充分条件}{充要条件}{既不充分也不必要条件}
\item {\tiny (004092)}已知$\alpha,\beta$是两个不同平面, $m$为$\alpha$内的一条直线, 则``$m\parallel\beta$''是``$\alpha\parallel\beta$''的\bracket{20}.
\twoch{充分不必要条件}{必要不充分条件}{充要条件}{既不充分也不必要条件}
\item {\tiny (004113)}若$m,n\in \mathbf{R}$, $\mathrm{i}$是虚数单位, 则 ``$m^2=n^2$''是``$(m-n)+(m+n)\mathrm{i}$为纯虚数 ''的\bracket{20}.
\twoch{充分不必要条件}{必要不充分条件}{充要条件}{既不充分也不必要条件}
\item {\tiny (004136)}设$a,b,c$表示三条互不重合的直线, $\alpha$、$\beta$表示两个不重合的平面, 则使得$a\parallel b$成立的一个充分条件为\bracket{20}.
\twoch{$a\perp c$, $b\perp c$}{$a\parallel \alpha$, $b\parallel \alpha$}{$a\parallel \alpha$, $a\parallel \beta$, $\alpha\cap \beta = b$}{$b\perp \alpha$, $c\parallel \alpha$, $a\perp c$}
\item {\tiny (004282)}若$|x-a|\le 1$成立的一个充分不必要条件是$1\le x\le 2$, 则实数$a$的取值范围是\bracket{20}.
\fourch{$1\le a\le 2$}{$a\ge 1$}{$a\le 2$}{$a\ge 1$或$a\le 2$}
\item {\tiny (004323)}设$x\in \mathbf{R}$, $y\in \mathbf{R}$.那么``$x>0$''是``$xy>0$''的\bracket{20}.
\twoch{充分非必要条件}{必要非充分条件}{充要条件}{既非充分又非必要条件}
\item {\tiny (004365)}已知$x\in\mathbf{R}$, 则"$x>0$"是"$x>1$"的\bracket{20}.
\fourch{充分非必要条件}{必要非充分条件}{充要条件}{既非充分又非必要条件}
\item {\tiny (004378)}已知常数$a\in \mathbf{R}$, 设$p:1\le x<2$, $q:x<a$. 若$p$是$q$的充分条件, 则$a$的取值范围为\blank{50}.
\item {\tiny (004399)}对于全集$\mathbf{R}$的子集$A$, 定义函数$f_A(x)=\begin{cases}
1, &  x\in A,  \\0, & x\in \complement_{\mathbf{R}}A  \end{cases}$为$A$的特征函数, 设$A,B$为全集$\mathbf{R}$的子集,\\
\textcircled{1} 若$A\subseteq B$, 则$f_A(x)\le f_B(x)$; \textcircled{2} $f_{\complement_{\mathbf{R}}A}(x)=1-f_A(x)$;\\
\textcircled{3} ${f_{A\cap B}}(x)=f_A(x)\cdot f_B(x)$; \textcircled{4} $f_{A\cup B}(x)=f_A(x)+f_B(x)$;\\ \textcircled{5} $f_{A\cap \complement_\mathbf{R}B}(x)=f_A(x)-f_B(x)$; \textcircled{6} 对于任意$x\in \mathbf{R}$, 若$f_A(x)\cdot f_B(x)=0$恒成立, 则$A\cap B=\varnothing$.\\
其中正确的命题为\blank{50}(填所有正确命题的序号).
\item {\tiny (004403)}设集合$A=\{y|y=a^x,\ x>0\}$(其中常数$a>0,  \ a\ne 1$), $B=\{y|y=x^k,\ x\in A\}$(其中常数$k\in \mathbf{Q}$), 则``$k<0$''是``$A\cap B=\varnothing$''的\bracket{20}.
\twoch{充分非必要条件}{必要非充分条件}{充分必要条件}{既非充分又非必要条件}
\item {\tiny (004458)}已知$x\in \mathbf{R}$, 则``$\sin x=1$''是``$\cos x=0$''的\bracket{20}.
\twoch{充分不必要条件}{必要不充分条件}{充要条件}{非充分非必要条件}
\item {\tiny (004502)}已知两条直线$l_1$、$l_2$的方程分别为$l_1:ax+y-1=0$和$l_2:x-y+1=0$, 则``$a=1$''是``直线$l_1\perp l_2$''的\bracket{20}.
\fourch{充分不必要条件}{必要不充分条件}{充要条件}{既不充分也不必要条件}
\item {\tiny (004522)}若$a$、$b$是实数, 则$a>b$是$2^a>2^b$的\bracket{20}.
\fourch{充分非必要条件}{必要非充分条件}{充要条件}{既非充分又非必要条件}
\item {\tiny (004564)}已知$a,b\in\mathbf{R}$, 则``$a^2>b^2$''是``$|a|>|b|$''的\bracket{20}.
\twoch{充分非必要条件}{必要非充分条件}{充要条件}{既非充分又非必要条件}
\item {\tiny (004631)}``函数$y=f(x), \ x\in \mathbf{R}$是增函数''是``函数$y=2-f(x), \ x\in \mathbf{R}$是减函数''的\bracket{20}.
\fourch{充分非必要条件}{必要非充分条件}{充要条件}{既非充分又非必要条件}
\item {\tiny (004673)}下列是``$a>b$''的充分不必要条件的是\bracket{20}.
\fourch{$a>b+1$}{$\dfrac ab>1$}{$a^2>b^2$}{$a^3>b^3$}
\item {\tiny (004697)}已知非空集合$A,B$满足: $A\cup B=R$, $A\cap B=\varnothing$, 函数$f(x)=\begin{cases}
x^2, &  x\in A,  \\ 2x-1, &  x\in B.  \end{cases}$ 对于下列两个命题: \textcircled{1} 存在唯一的非空集合对$(A,B)$, 使得$f(x)$为偶函数; \textcircled{2} 存在无穷多非空集合对$(A,B)$, 使得方程$f(x)=2$无解. 下面判断正确的是\bracket{20}.
\fourch{\textcircled{1} 正确, \textcircled{2} 错误}{\textcircled{1} 错误, \textcircled{2} 正确}{\textcircled{1} 、\textcircled{2} 都正确}{\textcircled{1} 、\textcircled{2} 都错误}
\item {\tiny (004715)}设$x_1,x_2\in \mathbf{R}$, 则``$x_1+x_2>6$且$x_1x_2>9$''是``$x_1>3$且$x_2>3$''的\bracket{20}.
\twoch{充分不必要条件}{必要不充分条件}{充要条件}{既不充分也不必要条件}
\item {\tiny (004736)}``$\alpha \in (0,\dfrac{\pi}2)$''是``$\alpha$ 为第一象限角''的\bracket{20}.
\twoch{充分不必要条件}{必要不充分条件}{充要条件}{既不充分又不必要条件}
\item {\tiny (004843)}下列语句哪些不是命题? 哪些是命题? 如果是命题, 那么它们是真命题还是假命题? 为什么?\\
(1) 你到过北京吗?\\
(2) 当$x=4$时, $2x<0$;\\
(3) 若$x\in \mathbf{R}$, 则方程$x^2-x+1=0$无实数根;\\
(4) $1+2=5$或$3\ge 3$;\\
(5) $x<-2$或$x>2$;\\
\item {\tiny (004845)}写出下列命题的否定式:\\
(1) 不论$k$取何实数, $x^2+x+k=0$必有实数根;\\
(2) 三角形中至多有一个钝角;\\
(3) 正$n(n\ge 3)$边形的$n$个内角全相等;\\
(4) 张三是科大或北大的学生;\\
(5) 如果$x^2-x-2=0$, 那么$x\ne -1$且$x\ne -2$.
\item {\tiny (004847)}下列说法是否正确? 为什么?\\
(1) $x^2=y^2\Rightarrow x=-y$;\\
(2) $x^2\ne y^2\Rightarrow x\ne y$或$x\ne -y$.
\item {\tiny (004848)}已知命题$\alpha$: 方程$x^2+mx+1=0$有两个相异负实数根, 命题$\beta$: $4x^2+4(m-2)x+1=0$无实数根, 命题$\alpha,\beta$有且只有一个为真命题, 求实数$m$的取值范围.
\item {\tiny (004854)}已知命题``非空集合$M$的元素都是集合$P$的元素''是假命题, 给出下列命题: \textcircled{1} $M$中的元素都不是$P$的元素; \textcircled{2} $M$中有不属于$P$的元素; \textcircled{3} $M$中有$P$的元素; \textcircled{4} $M$中的元素不都是$P$的元素. 其中假命题的个数是\bracket{20}.
\fourch{$1$}{$2$}{$3$}{$4$}
\item {\tiny (004861)}已知命题$p:$存在$x\in \mathbf{R}$, 使得$x^2+2ax+a\le 0$, 若命题$p$是假命题, 则实数$a$的取值范围是\blank{50}.
\item {\tiny (004864)}已知命题$A:$如果$a^2+2ab+b^2+a+b-2\ne 0$, 那么$a+b\ne 1$, 求证: 命题$A$是真命题.
\item {\tiny (004865)}已知$\alpha :|a-1|<2$, $\beta:$方程$x^2+(a+2)x+1=0(x\in \mathbf{R})$没有正根, 求实数$a$的取值范围, 使$\alpha,\beta$有且只有一个为真命题.
\item {\tiny (004866)}已知关于$x$的方程$(x^2-1)^2-|x^2-1|+k=0$. 判断下列命题的真假:\\
(1) 存在实数$k$, 使得方程恰有$2$个不同的实数根;\\
(2) 存在实数$k$, 使得方程恰有$4$个不同的实数根;\\
(3) 存在实数$k$, 使得方程恰有$5$个不同的实数根;\\
(4) 存在实数$k$, 使得方程恰有$8$个不同的实数根.
\item {\tiny (004867)}如果$a,b,c$都是实数, 那么``$ac<0$''是``关于$x$的方程$ax^2+bx+c=0$有一个正根和一个负根''的\bracket{20}.
\twoch{必要不充分条件}{充分不必要条件}{充要条件}{既不充分也不必要条件}
\item {\tiny (004870)}已知$p:|x-3|\le 2$, $q:(x-m+1)(x-m-1)\le 0$, 若$\overline p$是$\overline q$的充分不必要条件, 求实数$m$的取值范围.
\item {\tiny (004871)}已知集合$A=\{x|x<-3\text{或}x>5\}$, $B=\{x|a\le x\le 8\}$.\\
(1) 求实数$a$的取值范围, 使它成为$A\cap B=\{x|5<x\le 8\}$的充要条件;\\
(2) 求实数$a$的一个值, 使它成为$A\cap B=\{x|5<x\le 8\}$的一个充分不必要条件;\\
(3) 求实数$a$的一个值, 使它成为$A\cap B=\{x|5<x\le 8\}$的一个必要不充分条件.
\item {\tiny (004872)}已知$a:0\le x<3$, $\beta:-1<x\le 4$, $\gamma:2x^2+mx-1<0$.\\
(1) 若$a$是$\gamma$的充分条件, 求实数$m$的取值范围;\\
(2) 若$\beta$是$\gamma$的充分条件, 求实数$m$的取值范围.
\item {\tiny (004873)}已知$\triangle ABC$的三边为$a,b,c$求证: 关于$x$的方程$x^2+2ax+b^2=0$与$x^2+2cx-b^2=0$有公共根的充要条件是$A=90^\circ$.
\item {\tiny (004874)}``$m=2$''是``函数$f(x)=x^2+mx-3$有两个零点''的\bracket{20}.
\twoch{充分不必要条件}{必要不充分条件}{充要条件}{既不充分也不必要条件}
\item {\tiny (004875)}``$a\ne 1$或$b\ne 2$''是``$a+b\ne 3$''的\bracket{20}.
\twoch{充分不必要条件}{必要不充分条件}{充要条件}{既不充分也不必要条件}
\item {\tiny (004876)}如果$x,y\in \mathbf{R}$, 那么``$x>1$或$y>2$''是``$x+y>3$''的\bracket{20}.
\twoch{充分不必要条件}{必要不充分条件}{充要条件}{既不充分也不必要条件}
\item {\tiny (004877)}``$\dfrac{x^2+x+1}{3x+2}<0$''是``$3x+2<0$''的\bracket{20}.
\twoch{充分不必要条件}{必要不充分条件}{充要条件}{既不充分也不必要条件}
\item {\tiny (004878)}$a,b,c$三个数不全为零的充要条件是\bracket{20}.
\twoch{$a,b,c$三个数都不是零}{$a,b,c$三个数中之多有一个是零}{$a,b,c$三个数中只有一个是零}{$a,b,c$三个数中至少有一个不是零}
\item {\tiny (004879)}已知$p:x^2+x-2>0$, $q:x>a$.若q是p的充分不必要条件, 则实数$a$的取值范围是\bracket{20}
\fourch{$a\ge 1$}{$a\ge 1$}{$a\ge -1$}{$a\le -3$}
\item {\tiny (004880)}方程$ax^2+2x+1=0$至少有一个负实数根的充要条件是\bracket{20}.
\fourch{$0<a\le 1$}{$a>1$}{$a\le 1$}{$0<a\le 1$或$a<0$}
\item {\tiny (004882)}给出下列命题: \textcircled{1} ``$x+y=0$''是``$x^2-y^2+x+y=0$''的充分不必要条件; \textcircled{2} ``$a-b<0$''是``$a^2-b^2<0$''的充分不必要条件; \textcircled{3} ``$a-b<0$''是``$a^2-b^2<0$''的必要不充分条件; \textcircled{4} ``两个三角形全等''是``两边和夹角对应相等''的充要条件.
其中属真命题的是\bracket{20}.
\fourch{\textcircled{1}\textcircled{2}}{\textcircled{1}\textcircled{3}}{\textcircled{2}\textcircled{3}}{\textcircled{1}\textcircled{4}}
\item {\tiny (004883)}有限集合$S$中元素的个数记作$\mathrm{card}(S)$, 设$A$, $B$都是有限集合, 给出下列命题:
\textcircled{1} $A\cap B=\varnothing$的充要条件是$\mathrm{card}(A\cup B)=\mathrm{card}(A)+\mathrm{card}(B)$; \textcircled{2} $A\subseteq B$的必要不充分条件是$\mathrm{card}(A)\le \mathrm{card}(B)$; \textcircled{3} $A\subseteq B$的充分不必要条件是$\mathrm{card}(A)\le \mathrm{card}(B)$; \textcircled{4} $A=B$的充要条件是$\mathrm{card}(A)=\mathrm{card}(B)$. 
其中真命题的个数是\bracket{20}.
\fourch{$0$}{$1$}{$2$}{$3$}
\item {\tiny (004885)}已知$p$是$r$的充分不必要条件, $s$是$r$的必要条件, $q$是$s$的必要条件, 那么$p$是$q$的\blank{50}条件.
\item {\tiny (004886)}指出下列各命题中, $p$是$q$的什么条件:\\
(1) $p:0<x<3$, $q:|x-1|<2$;\\
(2) $p:(x-2)(x-3)=0$, $q:x=2$;\\
(3) $p:c=0$, $p$: 抛物线$y=ax^2+bx+c$过原点;\\
(4) $p:A\subseteq B\subseteq U$, $q:\complement_UB\subseteq A$.
\item {\tiny (004887)}``$xy>0$''的一个充分不必要条件是\blank{50}.
\item {\tiny (004888)}``$\sqrt x>\sqrt y$''的一个必要不充分条件是\blank{50}.
\item {\tiny (004891)}若集合$A=\{x|x^2+x-6=0\}$, $B=\{x|mx+1=0\}$, 则$B$是$A$的真子集的一个充分不必要条件是\blank{50}.
\item {\tiny (004893)}已知$m>0$, $p:-2\le x\le 10$, $q:1-m\le x\le -1+m$, 若$\overline p$是$\overline q$的必要不充分条件, 求实数$m$的取值范围.
\item {\tiny (004894)}求证: ``$x+y=5$''是``$x^2+y^2-3x+7y=10$''的充分不必要条件.
\item {\tiny (004895)}设$x,y\in \mathbf{R}$, 求证: $|x+y|=|x|+|y|$成立的充要条件是$xy\ge 0$.
\item {\tiny (004896)}已知函数$f(x)=ax-bx^2$.\\
(1) 当$b>0$时, 若对任何$x\in \mathbf{R}$都有$f(x)\le 1$, 求证: $a\le 2\sqrt b$;\\
(2) 当$b>1$时, 求证: ``对任意$x\in [0,1]$, $|f(x)|\le 1$''的充要条件是$b-1\le a\le 2\sqrt b$.
\item {\tiny (004900)}下列命题中, 不正确的一个是\bracket{20}.
\twoch{若$\sqrt[3]a>\sqrt[3]b$, 则$a>b$}{若$a>b$, $c>d$, 则$a-d>b-c$}{若$a>b>0$, $c>d>0$, 则$\dfrac ad>\dfrac bc$}{若$a>b>0$, $ac>bd$, 则$c>d$}
\item {\tiny (004925)}若$x$为实数, 则下列命题正确的是\bracket{20}.
\onech{$x^2\ge 2$的解集是$\{x|x\ge \pm \sqrt 2\}$}{$(x-1)^2<2$的解集是$\{x|1-\sqrt 2<x<1+\sqrt 2\}$}{$x^2-9<0$的解集是$\{x|x<3\}$}{设$x_1,x_2$为$ax^2+bx+c=0$的两个实根, 且$x_1>x_2$, 则$ax^2+bx+c>0$的解集是$\{x|x_2<x<x_1\}$}
\item {\tiny (005007)}下列命题中, 正确的一个是\bracket{20}.
\twoch{若$a,b,c\in \mathbf{R}$, 且$a>b$, 则$ac^2>bc^2$}{若$a,b\in \mathbf{R}$, 且$a\cdot b\ne 0$, 则$\dfrac ab+\dfrac ba\ge 2$}{若$a,b\in \mathbf{R}$, 且$a>|b|$, 则$a^n>b^n$($n\in \mathbf{N}$)}{若$a>b$, $c<d$, 则$\dfrac ac>\dfrac bd$}
\item {\tiny (005220)}设$x\in \mathbf{R}$, 则$(1-|x|)(1+x)>0$成立的充要条件是\bracket{20}.
\fourch{$|x|<1$}{$x<1$}{$|x|>1$}{$x<1$且$x\ne 1$}
\item {\tiny (007707)}判断下列语句是否为命题, 并在相应的括号内填入``是''或``否''.\\
(1) 正方形是四边形; \blank{20}\\
(2) $0$是自然数吗; \blank{20}\\
(3) 交集和并集; \blank{20}\\
(4) $3<\pi$. \blank{20}
\item {\tiny (007708)}判断下列命题的真假, 并在相应的括号内填入``真命题''或``假命题''.\\
(1) 如果$a$、$b$都是奇数, 那么$a+b$是偶数; \blank{20}\\
(2) 一组对边平行且两对角线相等的四边形是平行四边形; \blank{20}\\
(3) 如果$|a|<2$, 那么$a<2$; \blank{20}\\
(4) 如果$A\cap B=A$, 那么$A\cup B=B$. \blank{20}
\item {\tiny (007709)}如果$a$、$b$、$c$为实数, 设
$A:a=b=c=0$; $B:a,b,c$至少有一个为$0$; $C:a^2+\sqrt b+|c|=0$, 那么$A$\blank{20}$B$; $A$\blank{20}$C$; $B$\blank{20}$C$.(用符号``$\Rightarrow$''、``$\Leftarrow$''或``$\Leftrightarrow$''填空)
\item {\tiny (007710)}已知命题$A$: 如果$x<3$, 那么$x<5$; 命题$B$: 如果$x\ge 3$, 那么$x\ge 5$; 命题$C$: 如果$x\ge 5$, 那么$x\ge 3$.填写各命题之间的关系:
$A$与$B$互为\blank{20}命题, $B$与$C$互为\blank{20}命题, $C$与$A$互为\blank{20}命题.
\item {\tiny (007711)}写出命题``在$\triangle ABC$中, 如果$\angle C>\angle B$, 那么$AB>AC$''的逆命题、否命题和逆否命题, 并判断其真假.
\item {\tiny (007712)}写出命题``如果$\alpha$, 那么$\beta$''的逆命题、否命题和逆否命题.
\item {\tiny (007713)}写出命题``已知$a$、$b$、$c$是实数, 如果$ac<0$, 那么$ax^2+bx+c=0(a\ne 0)$有实数根''的逆命题.否命题和逆否命题, 并判断其真假.
\item {\tiny (007714)}命题``若$x\ne 3$且$x\ne 4$, 则$x^2-7x+12\ne 0$''的逆否命题是\bracket{20}.
\twoch{若$x^2-7x+12=0$, 则$x=3$或$x=4$}{若$x^2-7x+12=0$, 则$x\ne 3$或$x\ne 4$}{若$x^2-7x+12\ne 0$, 则$x\ne 3$且$x\ne 4$}{若$x^2-7x+12=0$, 则$x=3$且$x=4$}
\item {\tiny (007715)}如果命题$A$的逆命题是$B$, 命题$A$的否命题是$C$, 那么命题$B$是命题$C$的\bracket{20}.
\fourch{逆命题}{否命题}{逆否命题}{以上都不正确}
\item {\tiny (007716)}试判断命题$A$: ``在$\triangle ABC$中, $BC^2=AC^2+AB^2$''与命题$B$: ``$\triangle ABC$是直角三角形''是否为等价命题, 并说明理由.
\item {\tiny (007717)}试判断命题$A$: ``三角形任意两边之和大于第三边''与命题$B$: ``三角形任意两边之差小于第三边''是否为等价命题, 并说明理由.
\item {\tiny (007719)}判断下列命题的真假, 并在相应的横线上填入``真命题''或``假命题''.\\
(1) 若$A\cap B\ne \varnothing$, $B\subsetneqq C$, 则$A\cap C\ne \varnothing$\blank{20};\\
(2) 方程$(a+1)x+b=0$($a$、$b\in \mathbf{R}$)的解为$x=-\dfrac b{a+1}$\blank{20};\\
(3)若命题$\alpha$、$\beta$、$\gamma$满足$\alpha \Rightarrow \beta$, $\beta \Rightarrow \gamma$, $\gamma \Rightarrow \alpha$, 则$\alpha \Leftrightarrow \gamma$\blank{20}.
\item {\tiny (007720)}若$\alpha$: $\{2\}\subsetneqq B\subseteq \{2,3,4\}$, $\beta :B=\{2,4\}$, 则$\alpha$与$\beta$的推出关系是\bracket{20}.
\fourch{$\alpha \Rightarrow \beta$}{$\beta \Rightarrow \alpha$}{$\alpha \Leftrightarrow \beta$}{$\alpha \not\Rightarrow \beta$且$\beta \not\Rightarrow \alpha$}
(2)由命题甲成立, 可推出命题乙不成立, 下列说法一定正确的是\bracket{20}.
\twoch{命题甲不成立, 可推出命题乙成立}{命题甲不成立, 可推出命题乙不成立}{命题乙成立, 可推出命题甲成立}{命题乙成立, 可推出命题甲不成立}
\item {\tiny (007721)}已知一个命题的否命题是``两组对边分别相等的四边形是平行四边形'', 试写出原命题的逆命题, 并判断原命题的真假.
\item {\tiny (007722)}已知一个命题的逆命题是``若实数$a$、$b$满足$a=1$且$b=2$, 则$a+b<4$'', 试写出原命题的否命题, 并判断原命题的真假.
\item {\tiny (007723)}类比$A\subseteq B\Leftrightarrow A\cap B=A$, 试再写出两个等价命题:\\
$A\subseteq B\Leftrightarrow$\blank{50};\\
$A\subseteq B\Leftrightarrow$\blank{50}.
\item {\tiny (007724)}下列各题中命题$P$是命题$Q$的什么条件?\\
(1) $P$: 四边形的四条边相等, $Q$: 四边形是正方形;\\
(2) $P$: $\triangle ABC\cong \triangle DEF$,	$Q$: $\triangle ABC$的面积$=\triangle DEF$的面积;\\
(3) $P$: $x$是$2$的倍数, $Q$: $x$是$6$的倍数;\\
(4) $P$: 两个三角形全等, $Q$: 两个三角形的两角和一边对应相等.
\item {\tiny (007730)}有下列四组命题:
\textcircled{1} $P$: 集合$A\subseteq B$, $B\subseteq C$, $C\subseteq A$, 		$Q$: 集合$A=B=C$;
\textcircled{2} $P$: $A\cap B=A\cap C$, 					$Q$: $B=C$;
\textcircled{3} $P$: $(x-2)(x-3)=0$, 				$Q$: $\dfrac{x-2}{x-3}=0$;
\textcircled{4} $P$: 抛物线$y=ax^2+bx+c(a\ne 0)$过原点, $Q$: $c=0$.
其中$P$是$Q$的充要条件的有\bracket{20}.
\fourch{\textcircled{1} 、\textcircled{2} }{\textcircled{1} 、\textcircled{4} }{\textcircled{2} 、\textcircled{3} }{\textcircled{2} 、\textcircled{4}}
\item {\tiny (007731)}写出使实数$a$、$b$一正一负的充要条件.
\item {\tiny (007732)}求证: 实数$a$、$b$均大于$0$的充要条件是$\begin{cases} a+b>0, \\ ab>0. \end{cases}$
\item {\tiny (007733)}命题``$x\in M$或$x\in P$''是命题``$x\in M\cap P$''的什么条件?
\item {\tiny (007734)}写出命题``$x>3$''的一个充分条件和一个必要条件.
\item {\tiny (007735)}如果$\alpha$是$\beta$的充分非必要条件, 那么$\overline{\alpha }$是$\overline{\beta }$的什么条件?
\item {\tiny (007736)}如果$A$是$B$的必要条件, $C$是$B$的充分条件, $A$是$C$的充分条件, 那么$B$、$C$分别是$A$的什么条件?
\item {\tiny (007738)}试用子集与推出关系来判断命题$A$是命题$B$的什么条件.\\
(1) $A$: 该平面图形是四边形, $B$: 该平面图形是梯形;\\
(2) $A$: $x=2$, $B$: $(x-5)(x-2)=0$;\\
(3) $A$: $x^2=y^2$, $B$: $x=y$;\\
(4) $A$: $a=2$, $B$: $a\le 2$.
\item {\tiny (007739)}如果命题$p$: $m<-3$, 命题$q$: 方程$x^2-x-m=0$无实数根, 那么$p$是$q$的什么条件?
\item {\tiny (007740)}已知命题$\alpha$: $2\le x<4$, 命题$\beta$: $3m-1\le x\le -m$, 且$\alpha$是$\beta$的充分条件, 求实数$m$的取值范围.
\item {\tiny (007741)}如果命题$p$: $A\subseteq B$, 命题$q$: $A\subsetneqq B$, 那么$p$是$q$的什么条件?
\item {\tiny (007742)}已知$a$为实数, 写出关于$x$的方程$ax^2+2x+1=0$至少有一个实数根的一个充要条件、一个充分条件、一个必要条件.
\item {\tiny (007743)}下列命题中正确的是\bracket{20}.
\twoch{自然数集$\mathbf{N}$中最小的数是$1$}{空集是任何集合的真子集}{如果$A\subseteq B$, 且$A\ne B$, 那么$A$是$B$的真子集}{$\{y|y=x+3,\ x\in \mathbf{N}\}$中的最小值是$4$}
\item {\tiny (007746)}若命题$p$: $x^2-5x+6=0$, 命题$q$: $x=2$, 则$p$是$q$的\blank{50}条件。
\item {\tiny (007754)}写出命题: 若$x>1$, 则$x>0$的逆命题、否命题、逆否命题, 并指出哪些是真命题.
\item {\tiny (007758)}若$A$是$B$的必要非充分条件, $B$是$C$的充要条件, $C$是$D$的必要非充分条件, 则$D$是$A$的\blank{50}条件, $C$是$A$的\blank{50}条件.
\item {\tiny (007779)}如果$a>b$, 那么$\dfrac 1a<\dfrac 1b$成立的充要条件是\blank{50}.
\item {\tiny (007986)}``$x\ne 1$且$y\ne 2$''是``$x+y\ne 3$''的\bracket{20}.
\twoch{充分非必要条件}{必要非充分条件}{充要条件}{既非充分又非必要条件}
\item {\tiny (009437)}举几个生活中的命题的例子, 并判断其真假.
\item {\tiny (009438)}判断下列命题的真假, 并说明理由:\\
(1) 所有偶数都不是素数;\\
(2) $\{1\}$是$\{0, 1, 2\}$的真子集;\\
(3) $0$是$\{0, 1, 2\}$的真子集;\\
(4) 如果集合$A$是集合$B$的子集, 那么$B$不是$A$的子集.
\item {\tiny (009439)}用``$\Rightarrow$''表示下列陈述句$\alpha$与$\beta$之间的推出关系:\\
(1) $\alpha: \triangle ABC$是等边三角形, $\beta: \triangle ABC$是轴对称图形;\\
(2) $\alpha: x^2=4$, $\beta: x=2$.
\item {\tiny (009441)}设$\alpha: 1\le x<4$, $\beta: x<m$, $\alpha$是$\beta$的充分条件. 求实数$m$的取值范围.
\item {\tiny (009444)}设$a$、$b$、$c$、$d$是实数, 判断下列命题的真假, 并说明理由:\\
(1) 若$a^2=b^2$, 则$a=b$;\\
(2) 若$a(c^2+1)=b(c^2+1)$, 则$a=b$;\\
(3) 若$ab=0$, 则$a=0$或$b=0$;\\
(4) 若$\dfrac ac=\dfrac bd$, 且$c+d\ne 0$, 则$\dfrac{a+b}{c+d}=\dfrac ac$.
\item {\tiny (009449)}设$a$、$b$、$c$、$d$为实数, 判断下列命题的真假, 并说明理由:\\
(1) 如果$a>b$, $c>d$, 那么$a+d>b+c$;\\
(2) 如果$ab>ac$, 那么$b>c$;\\
(3) 如果$a\ge b$且$a\le b$, 那么$a=b$;\\
(4) 如果$a>b$, $\dfrac 1c>\dfrac 1d$, 那么$ac>bd$;\\
(5) 如果$\dfrac ba>\dfrac dc$, 那么$bc>ad$.
\item {\tiny (009450)}设$ab>0$, 求证: $a>b$是$\dfrac 1a<\dfrac 1b$的充要条件.
\item {\tiny (009451)}设$a$、$b$、$c$是实数, 判断下列命题的真假, 并说明理由.\\
(1) 如果$ac^2>bc^2$, 那么$a>b$;\\
(2) 如果$ab>c$, 那么$a>\dfrac cb$;\\
(3) 如果$a>b\ge 0$, 那么$\sqrt a>\sqrt b$.
\item {\tiny (010030)}判断下列语句是否为命题:\\
(1) 有的正方形是三角形;\\
(2) 任意一个三角形的内角和都为$180^\circ$;\\
(3) $1$是自然数吗?\\
(4) $3>\pi$;\\
(5) $2\in (0, 5)$, 且$2\in \mathbf{Z}$.
\item {\tiny (010031)}判断下列命题的真假, 并说明理由:\\
(1) 如果$a$、$b$都是奇数, 那么$a+b$是偶数;\\
(2) 一组对边平行且两对角线等长的四边形是平行四边形;\\
(3) 如果$A\cap B=A$, 那么$A\cup B=B$.
\item {\tiny (010032)}如果$a$、$b$、$c$为实数, 设$\alpha$: $a=b=c=0$; $\beta$: $a$、$b$、$c$中至少有一个为$0$; $\gamma$: $a^2+\sqrt b+|c|=0$. 那么$\alpha$\blank{20}$\beta$; $\alpha$\blank{20}$\gamma$; $\beta$ \blank{20}$\gamma$. (用符号``$\Leftarrow$''``$\Rightarrow$''或``$\Leftrightarrow$''填空)
\item {\tiny (010035)}证明: ``四边形$ABCD$是平行四边形''是``四边形$ABCD$的对角线互相平分''的充要条件.
\item {\tiny (010036)}判断下列命题的真假, 并说明理由:\\
(1) 若$A\cap B=\varnothing$, $C\subset B$, 则$A\cap C=\varnothing$;\\
(2) 若$a$、$b\in \mathbf{R}$, 则关于$x$的方程$(a+1)x+b=0$的解为$x=- \dfrac b{a+1}$.
\item {\tiny (010037)}已知$a$为实数. 写出关于$x$的方程$ax^2+2x+1=0$至少有一个实根的一个充要条件、一个充分非必要条件和一个必要非充分条件.
\item {\tiny (010038)}若$\alpha$: $\{2\}\subset B\subseteq \{2, 3, 4\}$, $\beta$: $B=\{2, 4\}$, 则$\alpha$是$\beta$的\bracket{20}.
\twoch{充分非必要条件}{必要非充分条件}{充要条件}{既非充分又非必要条件}
\item {\tiny (010039)}已知$\alpha$: $x<3m-1$或$x>-m$, $\beta$: $x<2$或$x\ge 4$.\\
(1) 若$\alpha$是$\beta$的充分条件, 求实数$m$的取值范围;\\
(2) 若$\alpha$是$\beta$的必要条件, 求实数$m$的取值范围.
\item {\tiny (010046)}设$a$、$b$、$c$、$d$为实数, 判断下列命题的真假:\\
(1) 若$a>b\ge 0$, 则$a^2>b^2$;\\
(2) 若$\sqrt a>\sqrt b$, 则$a>b$;\\
(3) 若$a>b>0, c>d>0$, 则$ac>bd$;\\
(4) 若$\dfrac ba>0$, 则$ab>0$;\\
(5) 若$a>b>0$, 则$a^2>ab>b^2$;\\
(6) 若$\sqrt a>b$, 则$a>b^2$.
\item {\tiny (010050)}证明: ``$a>0$且$b>0$''是``$a+b>0$且$ab>0$''的充要条件.
\item {\tiny (010060)}对一元二次方程$ax^2+bx+c=0$($a\ne 0$), 证明: $ac<0$是该方程有两个异号实根的充要条件.
\item {\tiny (010064)}设$s=a+b$, $p=ab$($a$、$b\in\mathbf{R}$), 写出``$a>1$且$b>1$''用$s$、$p$表示的一个充要条件, 并证明.
\item {\tiny (010065)}原有酒精溶液$a$(单位: $\text{g}$), 其中含有酒精$b$(单位: $\text{g}$), 其酒精浓度为$\dfrac ba$. 为增加酒精浓度, 在原溶液中加入酒精$x$(单位: $\text{g}$), 新溶液的浓度变为$\dfrac{b+x}{a+x}$. 根据这一事实, 可提炼出如下关于不等式的命题:若$a>b>0$, $x>0$, 则$\dfrac ba<\dfrac{b+x}{a+x}<1$. 试加以证明.
\item {\tiny (010089)}如果实数$a$、$b$同号, 那么下列命题中正确的是\bracket{20}.
\fourch{$a^2+b^2>2ab$}{$a+b\ge 2\sqrt {ab}$}{$\dfrac 1a+\dfrac 1b> \dfrac 2{\sqrt {ab}}$}{$\dfrac ba+\dfrac ab\ge 2$}
\item {\tiny (020070)}判断下列语句是否为命题, 并在相应的横线上填入``是''或``否''.\\
(1) 正方形和四边形;\blank{50};\\
(2) 正方形是四边形吗?\blank{50};\\
(3) $\pi>3$;\blank{50};\\
(4) 正方形好美!\blank{50};\\
(5) $2x>4$;\blank{50};\\
(6) $968$能被$11$整除;\blank{50}.
\item {\tiny (020071)}判断下列命题的真假, 并在相应的括号内填入``真''或``假''.\\
(1) $2\sqrt 3>3\sqrt 2$或$1\le 1$;\blank{50};\\
(2) $2\sqrt 3>3\sqrt 2$且$1\le1$;\blank{50};\\
(3) 如果$a$、$b$都是奇数, 那么$ab$也是奇数;\blank{50};\\
(4) $\{1\}$是$\{0, 1, 2\}$的真子集;\blank{50};\\
(5) $1$是$\{0, 1, 2\}$的真子集;\blank{50};\\
(6) 若$x<-2$或$x>2$, 则$x^2>1$;\blank{50};\\
(7) 如果$|a|<2$, 那么$a<2$;\blank{50};\\
(8) 对任意实数$a,b$, 方程$(a+1)x+b=0$的解为$x=-\dfrac b{a+1}$;\blank{50};\\
(9) 若命题$\alpha$、$\beta$、$\gamma$满足$\alpha\Rightarrow \beta$, $\beta\Rightarrow \gamma$, $\gamma\Rightarrow \alpha$, 则$\alpha\Leftrightarrow \gamma$;\blank{50};\\
(10) 若关于$x$的方程$ax^2+bx+c=0$($a\ne 0$)的两实数根之积是正数, 则$ac>0$;\blank{50};\\
(11) 若某个整数不是偶数, 则这个数不能被$4$整除;\blank{50};\\
(12) 合数一定是偶数;\blank{50};\\
(13) 所有的偶数都是素数或合数;\blank{50};\\
(14) 所有的偶数都是素数或所有的偶数都是合数;\blank{50};\\
(15) 如果$A\subset B$, $B\supset C$, 那么$A=C$;\blank{50};\\
(16) 空集是任何集合的真子集;\blank{50};\\
(17) 若$x\in \mathbf{R}$, 则方程$x^2-x+1=0$不成立;\blank{50};\\
(18) 若$A\cap B\ne \varnothing$, $B\subset C$, 则$A\cap C\ne \varnothing$;\blank{50};\\
(19) 存在一个三角形, 它的任意两边的平方和小于第三边的平方;\blank{50};\\
(20) 对于任意一个三角形, 存在一组两边的平方和不等于第三边的平方;\blank{50}.
\item {\tiny (020072)}在下列各题中, 用符号``$\Rightarrow$''``$\Leftarrow$''``$\Leftrightarrow$''把$\alpha$和$\beta$联系起来:\\
(1) $\alpha:a=0$, $\beta:ab=0$; $\alpha$\blank{20}$\beta$;\\
(2) $\alpha:x^2=4$, $\beta:x=2$; $\alpha$\blank{20}$\beta$;\\
(3) $\alpha:$实数$x$适合$x^2-5x+6=0$, $\beta:x=2$; $\alpha$\blank{20}$\beta$;\\
(4) $\alpha:\sqrt {x^2}=x$, $\beta:x>0$; $\alpha$\blank{20}$\beta$;\\
(5) $\alpha:$实数$x$适合$\dfrac{x-3}{x+1}=-1$, $\beta:x=1$; $\alpha$\blank{20}$\beta$;\\
(6) $\alpha:k$除以$4$余$1$, $\beta:k$除以$2$余$1$; $\alpha$\blank{20}$\beta$;\\
(7)$\alpha: \{2\}\subset B\subseteq \{2, 3, 5\}$, $\beta:B=\{2, 5\}$; $\alpha$\blank{20}$\beta$.
\item {\tiny (020073)}已知命题``非空集合$M$的元素都是集合$P$的元素$''$是假命题, 给出下列命题: \textcircled{1} $M$中的元素都不是$P$的元素; \textcircled{2} $M$中有不属于$P$的元素; \textcircled{3} $M$中有$P$的元素; \textcircled{4} $M$中的元素不都是$P$的元素. 其中真命题有\blank{50}.
\item {\tiny (020074)}已知$\alpha: 2\le x<4$, $\beta: 3m-1\le x\le-m$, 且$\alpha\Rightarrow\beta$, 求实数$m$的取值范围.
\item {\tiny (020075)}已知$a$是常数, 命题$\alpha :-1<a<3$, $\beta$: 关于$x$的方程$x+a=0$($x\in \mathbf{R}$)没有正根, 若命题$\alpha$、$\beta$有且只有一个是真命题, 求实数$a$的取值范围.
\item {\tiny (020077)}如果$A$是$B$的必要条件, $C$是$B$的充分条件, $A$是$C$的充分条件, 那么$B$、$C$分别是$A$的\blank{50}和\blank{50}条件.
\item {\tiny (020078)}写出使得``$x>3$''成立的一个充分条件: \blank{50}和一个必要条件: \blank{50}.
\item {\tiny (020079)}一次函数$y=kx+b$的图像经过第二、三、四象限的一个充要条件是\blank{50}.
\item {\tiny (020080)}关于$x$的方程$ax^2=0$至少有一个实数根的一个充要条件是\blank{50}.
\item {\tiny (020081)}已知$x,y\in \mathbf{R}$, ``$x^2+y^2>0$''是``$x\ne 0$或$y\ne 0$''的\bracket{20}.
\twoch{充分而不必要条件}{必要而不充分条件}{充要条件}{既不充分又不必要条件}
\item {\tiny (020082)}三个数$a$、$b$、$c$不全为零的充要条件是\bracket{20}.
\twoch{$a,b,c$都不是零}{$a,b,c$中最多一个零}{$a,b,c$中只有一个是零}{$a,b,c$中至少有一个不是零}
\item {\tiny (020083)}证明: $x_1>2$且$x_2>2$是$x_1+x_2>4$且$x_1\cdot x_2>4$的充分非必要条件.
\item {\tiny (020084)}有限集合$S$中元素的个数记作$\mathrm{card}(S)$, 设$A,B$都是有限集合, 给出下列命题:\\
\textcircled{1} $A\cap B=\varnothing$的一个充要条件是$\mathrm{card}(A\cup B)=\mathrm{card}(A)+\mathrm{card}(B)$;\\
\textcircled{2} $A\subseteq B$的一个必要不充分条件是$\mathrm{card}(A)\le \mathrm{card}(B)$; \\
\textcircled{3} $A$不是$B$的子集的一个充分不必要条件是$\mathrm{card}(A)>\mathrm{card}(B)$;\\ 
\textcircled{4} $A=B$的一个充要条件是$\mathrm{card}(A)=\mathrm{card}(B)$.\\ 
其中真命题的个数是\bracket{20}.
\fourch{$0$}{$1$}{$2$}{$3$}
\item {\tiny (020086)}设$x,y\in \mathbf{R}$, 求证: $|x+y|=|x|+|y|$成立的充要条件是$xy\ge 0$.
\item {\tiny (020088)}在横线上写出下列命题的否定形式, 并判断命题真假, 在相应的位置中填入``真''或``假''.\\
(1) $\pi$是无理数; \blank{20}; \blank{150}; \blank{20};\\
(2) $2+1=4$;  \blank{20}; \blank{150}; \blank{20};\\
(3) 任何实数是正数或负数;  \blank{20}; \blank{150}; \blank{20};\\
(4) 任何实数是正数或任何实数是负数;  \blank{20}; \blank{150}; \blank{20};\\
(5) 对一切实数$x, x^3+1=0$;  \blank{20}; \blank{150}; \blank{20};\\
(6) 存在实数$x, x^3+1=0$;  \blank{20}; \blank{150}; \blank{20};\\
(7) 对于任意实数$k$, 关于$x$的方程$x^2+x+k=0$都有实数根;  \blank{20}; \blank{250}; \blank{20};\\
(8) 任何三角形中至多有一个钝角;  \blank{20}; \blank{150}; \blank{20};\\
(9) 若$a>1$, $b>1$, 则$ab>1$;  \blank{20}; \blank{150}; \blank{20};\\
(10) 能被$2$整除的整数是质数;  \blank{20}; \blank{150}; \blank{20}.\\
\item {\tiny (020089)}写出下列命题的否定形式.\\
(1) 在平面上, 过定点$P$有且只有一条直线垂直于给定直线$l$;\\
(2) 任意两个有理数之间存在一个无理数;\\
(3) 存在实数$a$, 使得关于$x$的不等式$x^2+(a-2)x+a-1\ge 0$至少有一个正数解;\\
(4) 存在实数$a$, 使得关于$x$的不等式$x^2+(a-2)x+a-1\ge 0$恒成立;\\
(5) 存在实数$a$, 使得关于$x$的不等式$x^2+(a-2)x+a-1\ge 0$有解.
\item {\tiny (020090)}已知甲$\Rightarrow$乙, 下列说法一定正确的是\bracket{20}.
\twoch{甲不成立, 可推出乙成立}{甲不成立, 可推出乙不成立}{乙不成立, 可推出甲成立}{乙不成立, 可推出甲不成立}
\item {\tiny (020091)}``$a\ne 1$且$b\ne 2$''是``$a+b\ne 3$''的\bracket{20}.
\twoch{充分非必要条件}{必要非充分条件}{充要条件}{既非充分又非必要条件}
\item {\tiny (020094)}``$x\ne 3$或$x\ne 4$'' 是``$x^2-7x+12\ne 0$''的\bracket{20}.
\twoch{充分非必要条件}{必要非充分条件}{充要条件}{既非充分又非必要条件}
\end{enumerate}



\end{document}