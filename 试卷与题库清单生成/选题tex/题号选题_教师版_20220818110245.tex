\documentclass[10pt,a4paper]{article}
\usepackage[UTF8,fontset = windows]{ctex}
\setCJKmainfont[BoldFont=黑体,ItalicFont=楷体]{华文中宋}
\usepackage{amssymb,amsmath,amsfonts,amsthm,mathrsfs,dsfont,graphicx}
\usepackage{ifthen,indentfirst,enumerate,color,titletoc}
\usepackage{tikz}
\usepackage{multicol}
\usepackage{makecell}
\usepackage{longtable}
\usetikzlibrary{arrows,calc,intersections,patterns,decorations.pathreplacing,3d,angles,quotes,positioning}
\usepackage[bf,small,indentafter,pagestyles]{titlesec}
\usepackage[top=1in, bottom=1in,left=0.8in,right=0.8in]{geometry}
\renewcommand{\baselinestretch}{1.65}
\newtheorem{defi}{定义~}
\newtheorem{eg}{例~}
\newtheorem{ex}{~}
\newtheorem{rem}{注~}
\newtheorem{thm}{定理~}
\newtheorem{coro}{推论~}
\newtheorem{axiom}{公理~}
\newtheorem{prop}{性质~}
\newcommand{\blank}[1]{\underline{\hbox to #1pt{}}}
\newcommand{\bracket}[1]{(\hbox to #1pt{})}
\newcommand{\onech}[4]{\par\begin{tabular}{p{.9\textwidth}}
A.~#1\\
B.~#2\\
C.~#3\\
D.~#4
\end{tabular}}
\newcommand{\twoch}[4]{\par\begin{tabular}{p{.46\textwidth}p{.46\textwidth}}
A.~#1& B.~#2\\
C.~#3& D.~#4
\end{tabular}}
\newcommand{\vartwoch}[4]{\par\begin{tabular}{p{.46\textwidth}p{.46\textwidth}}
(1)~#1& (2)~#2\\
(3)~#3& (4)~#4
\end{tabular}}
\newcommand{\fourch}[4]{\par\begin{tabular}{p{.23\textwidth}p{.23\textwidth}p{.23\textwidth}p{.23\textwidth}}
A.~#1 &B.~#2& C.~#3& D.~#4
\end{tabular}}
\newcommand{\varfourch}[4]{\par\begin{tabular}{p{.23\textwidth}p{.23\textwidth}p{.23\textwidth}p{.23\textwidth}}
(1)~#1 &(2)~#2& (3)~#3& (4)~#4
\end{tabular}}
\begin{document}

\begin{enumerate}[1.]

\item { (010071)}若下列关于$x$的方程有实数解, 求实数$k$的取值范围:\\
(1) $x^2+kx-k+3=0$;\\
(2) $x^2+2\sqrt 2x+k(k-1)=0$.


关联目标:

暂未关联目标



标签: 第一单元

答案: 暂无答案

解答或提示: 暂无解答与提示

使用记录:

暂无使用记录


出处: 新教材必修第一册习题
\item { (010072)}解下列不等式:\\
(1) $\dfrac 13x^2\le 2x-3$;\\
(2) $4x^2\ge 12x-9$;\\
(3) $x^2-x+\dfrac 14<0$;\\
(4) $x^2+\dfrac 49>\dfrac 23x$.


关联目标:

暂未关联目标



标签: 第一单元

答案: 暂无答案

解答或提示: 暂无解答与提示

使用记录:

暂无使用记录


出处: 新教材必修第一册习题
\item { (010073)}解下列不等式:\\
(1) $x^2+x+1>0$;\\
(2) $3-2\sqrt 2x\ge -x^2$;\\
(3) $2x^2+3x+4<0$;\\
(4) $x^2\le 3x-4$.


关联目标:

暂未关联目标



标签: 第一单元

答案: 暂无答案

解答或提示: 暂无解答与提示

使用记录:

暂无使用记录


出处: 新教材必修第一册习题
\item { (010074)}已知关于$x$的一元二次方程$2x^2+ax+1=0$无实数解, 求实数$a$的取值范围.


关联目标:

暂未关联目标



标签: 第一单元

答案: 暂无答案

解答或提示: 暂无解答与提示

使用记录:

暂无使用记录


出处: 新教材必修第一册习题
\item { (010075)}已知关于$x$的一元二次不等式$x^2+ax+b<0$的解集为$(-3, -1)$, 求实数$a$及$b$的值.


关联目标:

暂未关联目标



标签: 第一单元

答案: 暂无答案

解答或提示: 暂无解答与提示

使用记录:

暂无使用记录


出处: 新教材必修第一册习题
\item { (010076)}解下列不等式组:\\
(1) $\begin{cases} 6-x-x^2\le 0, \\ x^2+3x-4<0; \end{cases}$\\
(2) $\begin{cases} 4x^2-27x+18>0,\\ x^2-6x+4<0; \end{cases}$\\
(3) $\begin{cases} 3x^2+x-2\ge 0, \\ 4x^2-15x+9>0. \end{cases}$


关联目标:

暂未关联目标



标签: 第一单元

答案: 暂无答案

解答或提示: 暂无解答与提示

使用记录:

暂无使用记录


出处: 新教材必修第一册习题
\item { (010077)}解下列不等式:\\
(1) $\dfrac{x+1}{x-2}>0$;\\
(2) $\dfrac 1x<1$;\\
(3) $\dfrac 2{3-4x}\ge 1$;\\
(4) $\dfrac 5{x+2}\le 2$;\\
(5) $\dfrac{4x+3}{x-1}>5$.


关联目标:

暂未关联目标



标签: 第一单元

答案: 暂无答案

解答或提示: 暂无解答与提示

使用记录:

暂无使用记录


出处: 新教材必修第一册习题
\item { (010078)}当关于$x$的方程$4k-3x=2(k+2)x$的解分别满足以下条件时, 求实数$k$的取值范围.\\
(1) 正数;\\
(2) 负数.


关联目标:

暂未关联目标



标签: 第一单元

答案: 暂无答案

解答或提示: 暂无解答与提示

使用记录:

暂无使用记录


出处: 新教材必修第一册习题
\item { (010079)}解下列不等式:\\
(1) $|1-4x|<5$;\\
(2) $|x-4|<2x$;\\
(3) $|3x-4|\ge x+2$;\\
(4) $|x+2|+|x-3|<7$.


关联目标:

暂未关联目标



标签: 第一单元

答案: 暂无答案

解答或提示: 暂无解答与提示

使用记录:

暂无使用记录


出处: 新教材必修第一册习题
\item { (010080)}某船从甲码头顺流航行$75\text{km}$到达乙码头, 停留$30\text{min}$后再逆流航行$126\text{km}$到达丙码头. 如果水流速度为$4\text{km}/\text{h}$, 该船要在$5\text{h}$内(包含$5\text{h}$)完成整个航行任务, 那么船的速度至少要达到多少?


关联目标:

暂未关联目标



标签: 第一单元

答案: 暂无答案

解答或提示: 暂无解答与提示

使用记录:

暂无使用记录


出处: 新教材必修第一册习题
\item { (010081)}设$a$、$b\in \mathbf{R}$, 解关于$x$的不等式$ax>b$.


关联目标:

暂未关联目标



标签: 第一单元

答案: 暂无答案

解答或提示: 暂无解答与提示

使用记录:

暂无使用记录


出处: 新教材必修第一册习题
\item { (010082)}设$a\in \mathbf{R}$, 解下列关于$x$的不等式:\\
(1) $(x-a)(x+3)\ge 0$;\\
(2) $(x-a)(x-2a)>0$;\\
(3) $x(x-a)\ge (a+1)(x-a)$.


关联目标:

暂未关联目标



标签: 第一单元

答案: 暂无答案

解答或提示: 暂无解答与提示

使用记录:

暂无使用记录


出处: 新教材必修第一册习题
\item { (010083)}已知关于$x$的不等式$x^2+bx+c>0$的解集是$(-\infty, \dfrac 12)\cup(2, +\infty)$, 求实数$b$及$c$的值, 并求$x^2-bx+c\le 0$的解集.


关联目标:

暂未关联目标



标签: 第一单元

答案: 暂无答案

解答或提示: 暂无解答与提示

使用记录:

暂无使用记录


出处: 新教材必修第一册习题
\item { (010084)}解下列不等式:\\
(1) $2< \dfrac 1{3x-1}\le 3$;\\
(2) $\dfrac 1x>x$;\\
(3) $\dfrac 1{x-4}\le 1- \dfrac x{4-x}$.


关联目标:

暂未关联目标



标签: 第一单元

答案: 暂无答案

解答或提示: 暂无解答与提示

使用记录:

暂无使用记录


出处: 新教材必修第一册习题
\item { (010085)}解下列不等式:\\
(1) $ \dfrac{3x^2+2x+1}{x^2+x+2}\le 1$;\\
(2) $\dfrac{x-1}{x^2-4x+4}\ge 0$.


关联目标:

暂未关联目标



标签: 第一单元

答案: 暂无答案

解答或提示: 暂无解答与提示

使用记录:

暂无使用记录


出处: 新教材必修第一册习题
\item { (010086)}解下列不等式:\\
(1) $1<|1-2x| \le 7$;\\
(2) $3<|x-2|<6$;\\
(3) $|x+2|-|3-2x| <1$;\\
(4) $|\dfrac x{x+1}| > \dfrac x{x+1}$.


关联目标:

暂未关联目标



标签: 第一单元

答案: 暂无答案

解答或提示: 暂无解答与提示

使用记录:

暂无使用记录


出处: 新教材必修第一册习题
\item { (010087)}若关于$x$的不等式组$\begin{cases} (2x-3)(3x+2)\le 0, \\  x-a>0 \end{cases}$没有实数解, 求实数$a$的取值范围.


关联目标:

暂未关联目标



标签: 第一单元

答案: 暂无答案

解答或提示: 暂无解答与提示

使用记录:

暂无使用记录


出处: 新教材必修第一册习题
\item { (010088)}若关于$x$的不等式$2kx^2+kx+\dfrac 18>0$对于一切实数$x$都成立, 求实数$k$的取值范围.


关联目标:

暂未关联目标



标签: 第一单元

答案: 暂无答案

解答或提示: 暂无解答与提示

使用记录:

暂无使用记录


出处: 新教材必修第一册习题
\item { (010089)}如果实数$a$、$b$同号, 那么下列命题中正确的是\bracket{20}.
\fourch{$a^2+b^2>2ab$}{$a+b\ge 2\sqrt {ab}$}{$\dfrac 1a+\dfrac 1b> \dfrac 2{\sqrt {ab}}$}{$\dfrac ba+\dfrac ab\ge 2$}


关联目标:

暂未关联目标



标签: 第一单元

答案: 暂无答案

解答或提示: 暂无解答与提示

使用记录:

暂无使用记录


出处: 新教材必修第一册习题
\item { (010090)}设$a>b>0$, 将四个正数$a$、$b$、$\sqrt {ab}$、$\dfrac{a+b}2$按从小到大的顺序排列, 并说明理由.


关联目标:

暂未关联目标



标签: 第一单元

答案: 暂无答案

解答或提示: 暂无解答与提示

使用记录:

暂无使用记录


出处: 新教材必修第一册习题
\item { (010091)}已知$a$、$b$为正数, 求证:$\dfrac 2{\dfrac 1a+\dfrac 1b}
\le \sqrt {ab}$, 并指出等号的成立条件.


关联目标:

暂未关联目标



标签: 第一单元

答案: 暂无答案

解答或提示: 暂无解答与提示

使用记录:

暂无使用记录


出处: 新教材必修第一册习题
\item { (010092)}设$a$、$b\in \mathbf{R}$, 求证: $a^2+2b^2+1\ge 2b(a+1)$.


关联目标:

暂未关联目标



标签: 第一单元

答案: 暂无答案

解答或提示: 暂无解答与提示

使用记录:

暂无使用记录


出处: 新教材必修第一册习题
\item { (010093)}设$x\in \mathbf{R}$, 求二次函数$y=(x-1)(5-x)$的最大值.


关联目标:

暂未关联目标



标签: 第一单元

答案: 暂无答案

解答或提示: 暂无解答与提示

使用记录:

暂无使用记录


出处: 新教材必修第一册习题
\item { (010094)}已知直角三角形斜边长等于$10\text{cm}$, 求直角三角形面积的最大值.


关联目标:

暂未关联目标



标签: 第一单元

答案: 暂无答案

解答或提示: 暂无解答与提示

使用记录:

暂无使用记录


出处: 新教材必修第一册习题
\item { (010095)}已知$a$、$b$、$c$为实数, 求证: $|a-b| \le |a-c| +|c-b|$.


关联目标:

暂未关联目标



标签: 第一单元

答案: 暂无答案

解答或提示: 暂无解答与提示

使用记录:

暂无使用记录


出处: 新教材必修第一册习题
\item { (010096)}设$x\in \mathbf{R}$, 求方程$|x-2|+|2x-3|=|3x-5|$的解集.


关联目标:

暂未关联目标



标签: 第一单元

答案: 暂无答案

解答或提示: 暂无解答与提示

使用记录:

暂无使用记录


出处: 新教材必修第一册习题
\item { (010097)}设$0<a<b$, 且$a+b=1$, 请将$a$、$b$、$\dfrac 12$、$2ab$、$a^2+b^2$从小到大排列, 并说明理由.


关联目标:

暂未关联目标



标签: 第一单元

答案: 暂无答案

解答或提示: 暂无解答与提示

使用记录:

暂无使用记录


出处: 新教材必修第一册习题
\item { (010098)}已知$a$为正数, 比较$\dfrac{a^2+2a+1}a$的值与$4$的大小.


关联目标:

暂未关联目标



标签: 第一单元

答案: 暂无答案

解答或提示: 暂无解答与提示

使用记录:

暂无使用记录


出处: 新教材必修第一册习题
\item { (010099)}已知$a$、$b$为正数, 求证: $(a+b)(\dfrac 1a+\dfrac 1b)\ge 4$.


关联目标:

暂未关联目标



标签: 第一单元

答案: 暂无答案

解答或提示: 暂无解答与提示

使用记录:

暂无使用记录


出处: 新教材必修第一册习题
\item { (010100)}已知$a$、$b$是互不相等的正数, 求证: $(a^2+1)(b^2+1)>4ab$.


关联目标:

暂未关联目标



标签: 第一单元

答案: 暂无答案

解答或提示: 暂无解答与提示

使用记录:

暂无使用记录


出处: 新教材必修第一册习题
\end{enumerate}



\end{document}