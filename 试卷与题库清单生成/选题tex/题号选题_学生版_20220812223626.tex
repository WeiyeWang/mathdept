\documentclass[10pt,a4paper]{article}
\usepackage[UTF8,fontset = windows]{ctex}
\setCJKmainfont[BoldFont=黑体,ItalicFont=楷体]{华文中宋}
\usepackage{amssymb,amsmath,amsfonts,amsthm,mathrsfs,dsfont,graphicx}
\usepackage{ifthen,indentfirst,enumerate,color,titletoc}
\usepackage{tikz}
\usepackage{multicol}
\usepackage{makecell}
\usepackage{longtable}
\usetikzlibrary{arrows,calc,intersections,patterns,decorations.pathreplacing,3d,angles,quotes,positioning}
\usepackage[bf,small,indentafter,pagestyles]{titlesec}
\usepackage[top=1in, bottom=1in,left=0.8in,right=0.8in]{geometry}
\renewcommand{\baselinestretch}{1.65}
\newtheorem{defi}{定义~}
\newtheorem{eg}{例~}
\newtheorem{ex}{~}
\newtheorem{rem}{注~}
\newtheorem{thm}{定理~}
\newtheorem{coro}{推论~}
\newtheorem{axiom}{公理~}
\newtheorem{prop}{性质~}
\newcommand{\blank}[1]{\underline{\hbox to #1pt{}}}
\newcommand{\bracket}[1]{(\hbox to #1pt{})}
\newcommand{\onech}[4]{\par\begin{tabular}{p{.9\textwidth}}
A.~#1\\
B.~#2\\
C.~#3\\
D.~#4
\end{tabular}}
\newcommand{\twoch}[4]{\par\begin{tabular}{p{.46\textwidth}p{.46\textwidth}}
A.~#1& B.~#2\\
C.~#3& D.~#4
\end{tabular}}
\newcommand{\vartwoch}[4]{\par\begin{tabular}{p{.46\textwidth}p{.46\textwidth}}
(1)~#1& (2)~#2\\
(3)~#3& (4)~#4
\end{tabular}}
\newcommand{\fourch}[4]{\par\begin{tabular}{p{.23\textwidth}p{.23\textwidth}p{.23\textwidth}p{.23\textwidth}}
A.~#1 &B.~#2& C.~#3& D.~#4
\end{tabular}}
\newcommand{\varfourch}[4]{\par\begin{tabular}{p{.23\textwidth}p{.23\textwidth}p{.23\textwidth}p{.23\textwidth}}
(1)~#1 &(2)~#2& (3)~#3& (4)~#4
\end{tabular}}
\begin{document}

\begin{enumerate}[1.]

\item {\tiny (000047)}方程$(x-1)(x-2)(x-3)=0$的三个根$1$、$2$、$3$将数轴划分为四个区间, 即$(-\infty, 1)$, $(1, 2)$, $(2, 3)$, $(3, +\infty)$. 试在这四个区间上分别考察$(x-1)(x-2)(x-3)$的
符号, 从而得出不等式$(x-1)(x-2)(x-3)>0$与$(x-1)(x-2)(x-3)<0$的解集.\\
一般地, 对$x_1$、$x_2$、$x_3\in \mathbf{R}$, 且$x_1\le x_2\le x_3$, 试分别求不等式$(x-x_1)(x-x_2)(x-x_3)>0$与$(x-x_1)(x-x_2)(x-x_3)<0$的解集(提示: $x_1$、$x_2$、$x_3$相互之间可能相等, 需要分情况讨论).
\item {\tiny (000048)}填空题:\\
(1) 若$x^3=5$, 则$x=$\blank{50}; 若$3^x=5$, 则$x=$\blank{50}.\\
(2) 将$\sqrt[4]{a\sqrt[3]{a}} \ (a>0)$化成有理数指数幂的形式为\blank{50}.\\
(3) 若$\log_8x=-\dfrac 23$, 则$x=$\blank{50}.\\
(4) 若$\log_a b\cdot \log_5 a=3$($a>0$且$a\ne 1$), 则$b=$\blank{50}.
\item {\tiny (000049)}选择题:\\
(1) 若$\lg a$与$\lg b$互为相反数, 则有\bracket{20}.
\fourch{$a+b=0$}{$ab=1$}{$\dfrac ab=1$}{以上答案均不对}
(2) 设$a>0$, 下列计算中正确的是\bracket{20}.
\twoch{$a^\frac{2}{3}\cdot a^\frac{3}{2}=a$}{$a^\frac{2}{3}\div a^\frac{3}{2}=a$}{$a^{-4}\cdot a^4=0$}{$(a^\frac{2}{3})^\frac{3}{2}=a$}
\item {\tiny (001286)}$\dfrac{\sqrt{3\sqrt{3\sqrt{3\sqrt{\dfrac{1}{3}}}}}}{\sqrt{27\sqrt{\dfrac{1}{3}}}}$用$3$的有理数指数幂表示为\blank{80}.
\item {\tiny (001292)}已知$a,b$是实数, 函数$f(x)=a\cdot b^x$, 且$f(4)=648$, $f(5)=1944$, 求$f(9/2)$.
\item {\tiny (010110)}用有理数指数幂的形式表示下列各式(其中$a>0$, $b>0$):\\
(1) $a^\frac 13a^\frac 14$;\\
(2) $\sqrt[3]{a\sqrt a}$;\\
(3) $(a^\frac 14b^{-\frac 38})^8$;\\
(4) $(\dfrac {a^{-3}b^4}{\sqrt b})^{-\frac 13}$.
\item {\tiny (001287)}已知$m,n$是有理数, 则以下各说法中, 正确的有\blank{50}.
\vartwoch{对一切$m,n$均成立$2^m2^n=2^{m+n}$}{存在$m,n$使得$2^m2^n=2^{mn}$}{存在$m,n$使得$2^m+2^n=2^{m+n}$}{存在$m,n$使得$(2^m)^n=2^{m^n}$}
\item {\tiny (000058)}已知$a>1$, $b>0$. 求证: 对任意给定的实数$k$, $a^{2b+k}-a^{b+k}>a^{b+k}-a^k$.
\item {\tiny (003662)}已知常数$a>0$, 函数$f(x)=\dfrac{2^x}{2^x+ax}$的图像经过点$P\left(p,\dfrac{6}{5}\right)$, $Q\left(q,-\dfrac{1}{5}\right)$. 若$2^{p+q}=36pq$, 则$a=$\blank{50}.
\item {\tiny (005621)}若$x=t^{\frac 1{t-1}}$, $y=t^{\frac t{t-1}}$($t>0$, $t\ne 1$), 则$x,y$之间的关系是\bracket{20}.
\fourch{$y^x=x^{\frac 1y}$}{$y^{\frac 1x}=x^y$}{$y^x=x^y$}{$x^x=y^y$}
\item {\tiny (010114)}设$a>b>0$, 求证: $a^ab^b>(ab)^\frac{a+b}2$.
\item {\tiny (001296)}求值: $\log_2 0.5=$\blank{80}, $\log_9 27=$\blank{80}, $3^{1+\log_3 5}=$\blank{80}.
\item {\tiny (000053)}已知$m=\log_2 10$, 求$2^m-m\lg 2-4$的值.
\item {\tiny (000054)}填空题:\\
(1) 若$4^x=2^{-\frac{1}{2}}$, $4^y=\sqrt[3]{32}$, 则$2x-3y=$\blank{50}.\\
(2) 若$\log_3(\log_4 x)=1$, 则$x=$\blank{50}.\\
(3) 若$3^a=7^b=63$, 则$\dfrac 2a+\dfrac 1b$的值为\blank{50}.\\
\item {\tiny (000074)}$\log_23$是有理数吗? 请证明你的结论.
\item {\tiny (005610)}已知$x=a^{\frac 1{1-\log_ay}}$, $y=a^{\frac 1{1-\log_az}}$求证: $z=a^{\frac 1{1-\log_ax}}$.
\item {\tiny (005650)}已知不相等的两个正数$a,b$满足$a^{\lg ax}=b^{\lg bx}$, 求$(ab)^{\lg abx}$的值.
\item {\tiny (001353)}解方程: $x^{\log_2 x}=32x^4$.
\item {\tiny (001300)}用不含对数的式子表示:\\ 
(1) 若$\log_7 2=a$, 则$\log_7 14=$\blank{80}, $\log_7 \sqrt{3.5}=$\blank{80}.\\ 
(2) 若$\log_3 2=a$, 则$\log_3 4=$\blank{80}, $\log_3 \dfrac{2}{3}=$\blank{80}.\\ 
(3) 若$\lg 2=a$, 则$\lg 25=$\blank{80}.
\item {\tiny (001305)}计算下列各式(要有必要的过程):
\begin{multicols}{2}
(1) $\dfrac{1}{2}\log_{20}45-\log_{20}30$;\\ 
(2) $\dfrac{\lg3+\dfrac{2}{5}\lg9+\dfrac{3}{5}\lg\sqrt{27}-\lg\sqrt{3}}{\lg81-\lg27}$;\\ 
\end{multicols}
\begin{multicols}{2}
(3) $\lg^22+\lg^25+2\lg2\lg5$; \\ 
(4) $\lg^32+\lg^35+3\lg2\lg5$;\hfill\\ 
\end{multicols}
\begin{multicols}{2}
(5) $\lg4+2\sqrt{\lg^26-\lg6^2+1}+\lg9$.\\ 
\end{multicols}
\item {\tiny (001307)}已知$a=\log_3 36$, $b=\log_4 36$. 求$\dfrac{2}{a}+\dfrac{1}{b}$.(提示: 你学过实数指数幂的运算律的)
\item {\tiny (003828)}已知正数$x,y$满足$\ln x+\ln y=\ln (x+y)$, 则$2x+y$的最小值是\blank{50}.
\item {\tiny (009482)}求下列各式的值:\\
(1) $\log_8\dfrac 14$;\\
(2) $\log_ab\cdot\log_bc\cdot\log_ca$($a$、$b$、$c$均为不等于$1$的正数);\\
(3) $3^{2+\log_94}$;\\
(4) $\dfrac{\log_52\times\log_79}{\log_5\dfrac 13\times\log_72}$.
\item {\tiny (010125)}科学家以里氏震级来度量地震的强度, 若设$I$为地震时所散发出来的相对能量程度, 则里氏震级度量$r$可定义为$r=\dfrac 23\lg I+2$. 求$7.8$级地震和$6. 9$级地震的相对能量比值. (结果精确到个位)
\item {\tiny (001308)}[证明对数的{\bf 换底公式}]
若$a,b,N>0$, $a\ne 1, b\ne 1$, 则
$$\log_aN=\dfrac{\log_b N}{\log_b a}.$$
\item {\tiny (000060)}已知$a$、$b$及$c$是不为$1$的正数, 且$\lg a+\lg b+\lg c=0$. 求证: $a^{\frac{1}{\lg b}+\frac{1}{\lg c}}\cdot b^{\frac{1}{\lg c}+\frac{1}{\lg a}}\cdot c^{\frac{1}{\lg a}+\frac{1}{\lg b}}=\dfrac{1}{1000}$.
\item {\tiny (001309)}(1) 若$\lg 3=a$, $\lg 2=b$, 则$\log_6 12=$\blank{80}.\\ 
(2) 若$\log_{\sqrt{3}} 2=a$, 则$\log_{12} 3=$\blank{80}.
\item {\tiny (001312)}计算下列各式(要有必要的过程):\\ 
\begin{multicols}{2}
(1) $\log_3 5\cdot\log_5 7\cdot\log_7 9$; \\ 
(2) $(\log_4 3+\log_8 3)(\log_3 2+\log_9 2)$;\\ 
\end{multicols}
\begin{multicols}{2}
(3) $2\log_{100} 5-\sqrt{1-2\lg2+\lg^2 2}$; \\ 
(4)$\dfrac{\log_5 \sqrt{2}\cdot\log_7 9}{\log_5\dfrac{1}{3}\cdot\log_7\sqrt[3]{4}}$ ;
\end{multicols}
\begin{multicols}{2}
(5)$2^{\log_4(\sqrt{3}-2)^2}+3^{\log_9(\sqrt{3}+2)^2}$;  \\ 
(6)$\dfrac{\log_{36}4}{\log_{18}6}+\log_6^2 3$.\\ 
\end{multicols}
\item {\tiny (001314)}若$2^a=5^b=100$, 求$\dfrac{a+b}{ab}$的值.
\item {\tiny (001316)}若$\log_2 3=a$, $\log_37=b$, 试用$a,b$表示$\log_{42} 56$.
\item {\tiny (001317)}不相等的两个正数$a,b$与另一个正数$x$满足$a^{\lg(ax)}=b^{\lg(bx)}$, 求$abx$的值.
\item {\tiny (002963)}若$\log_23=a$, $3^b=7$, 用$a,b$表示$\log_{3\sqrt 7}2$, 则$\log_{3\sqrt 7}2$=\blank{50}.
\item {\tiny (005013)}若$0<a<1$, $0<b<1$, 则$\log_ab+\log_ba$的最小值为\blank{50}.
\item {\tiny (005014)}若$a>1$, $0<b<1$, 则$\log_ab+\log_ba$的最大值为\blank{50}.
\item {\tiny (005016)}若$a,b,c$均大于1, 且$\log_ac\cdot \log_bc=4$, 则下列各式中, 一定正确的是\bracket{20}.
\fourch{$ac\ge b$}{$ab\ge c$}{$bc\ge a$}{$ab\le c$}
\item {\tiny (005123)}已知$a>1$且$a^{\lg b}=\sqrt[4]2$, 求$\log_2(ab)$的最小值.
\item {\tiny (005678)}已知$a^2+b^2=c^2$, 求证$\log_{(c+b)}a+\log_{(c-b)}a=2\log_{(c+b)}a\cdot \log_{(c-b)}a$.
\end{enumerate}



\end{document}