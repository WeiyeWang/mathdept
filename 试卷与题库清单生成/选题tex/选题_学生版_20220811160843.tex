\documentclass[10pt,a4paper]{article}
\usepackage[UTF8,fontset = windows]{ctex}
\setCJKmainfont[BoldFont=黑体,ItalicFont=楷体]{华文中宋}
\usepackage{amssymb,amsmath,amsfonts,amsthm,mathrsfs,dsfont,graphicx}
\usepackage{ifthen,indentfirst,enumerate,color,titletoc}
\usepackage{tikz}
\usepackage{multicol}
\usepackage{makecell}
\usepackage{longtable}
\usetikzlibrary{arrows,calc,intersections,patterns,decorations.pathreplacing,3d,angles,quotes,positioning}
\usepackage[bf,small,indentafter,pagestyles]{titlesec}
\usepackage[top=1in, bottom=1in,left=0.8in,right=0.8in]{geometry}
\renewcommand{\baselinestretch}{1.65}
\newtheorem{defi}{定义~}
\newtheorem{eg}{例~}
\newtheorem{ex}{~}
\newtheorem{rem}{注~}
\newtheorem{thm}{定理~}
\newtheorem{coro}{推论~}
\newtheorem{axiom}{公理~}
\newtheorem{prop}{性质~}
\newcommand{\blank}[1]{\underline{\hbox to #1pt{}}}
\newcommand{\bracket}[1]{(\hbox to #1pt{})}
\newcommand{\onech}[4]{\par\begin{tabular}{p{.9\textwidth}}
A.~#1\\
B.~#2\\
C.~#3\\
D.~#4
\end{tabular}}
\newcommand{\twoch}[4]{\par\begin{tabular}{p{.46\textwidth}p{.46\textwidth}}
A.~#1& B.~#2\\
C.~#3& D.~#4
\end{tabular}}
\newcommand{\vartwoch}[4]{\par\begin{tabular}{p{.46\textwidth}p{.46\textwidth}}
(1)~#1& (2)~#2\\
(3)~#3& (4)~#4
\end{tabular}}
\newcommand{\fourch}[4]{\par\begin{tabular}{p{.23\textwidth}p{.23\textwidth}p{.23\textwidth}p{.23\textwidth}}
A.~#1 &B.~#2& C.~#3& D.~#4
\end{tabular}}
\newcommand{\varfourch}[4]{\par\begin{tabular}{p{.23\textwidth}p{.23\textwidth}p{.23\textwidth}p{.23\textwidth}}
(1)~#1 &(2)~#2& (3)~#3& (4)~#4
\end{tabular}}
\begin{document}

\begin{enumerate}[1.]

\item {\tiny (000010)}证明: 若梯形的对角线不相等, 则该梯形不是等腰梯形.
\item {\tiny (000017)}证明: $\sqrt[3]{2}$是无理数.
\item {\tiny (000038)}证明: 若$x>-1$, 则$x+\dfrac 1{x+1}\ge 1$, 并指出等号成立的条件.
\item {\tiny (000982)}模仿讲义中的真值表, 列出下列每组逻辑运算的真值表并回答各问题:\\ 
(1) ``非$(\ P$且$Q\ )$''与``$($非$P\ )$或$($非$Q\ )$'' (De Morgan律之一);
\begin{center}
\begin{tabular}{|c|c||c|c||c|c|c||}
\hline
$P$ & $Q$ & $P$且$Q$ & 非$(\ P$且$Q\ )$ & 非$P$ & 非$Q$ & $($非$P\ )$或$($非$Q\ )$\\
\hline
T & T &&&&&\\
\hline
T & F &&&&&\\
\hline
F & T &&&&&\\
\hline
F & F &&&&&\\
\hline\\ 
\end{tabular}
\end{center} 
(2) ``$P$且$(\ Q$且$R\ )$''与``$(\ P$且$Q\ )$且$R$''(模仿(1)完成); 你的结论是什么? 如果把两个运算中的``且''都换成``或'', 结论(毋需证明)又是什么?\\ 
(3) ``$P$且 $(\ Q$或$R\ )$''与``$(\ P$且$Q\ )$或$(\ P$且$R\ )$''(模仿(1)完成); 你的结论是什么? 如果把两个运算中的``且''都换成``或'', 同时把``或''都换成``且'', 结论(毋须证明)又是什么?
\item {\tiny (000983)}用反证法证明如下命题:\\ 
(1) 已知$n$是整数. 如果$3$整除$n^3$, 则$3$整除$n$(提示: 讨论$n=3k,3k+1,3k+2$, 其中$k$是整数);\\ 
(2) 如果实数$x$满足$x^{101}-4x^2+8x-1=0$, 则$x>0$;\\ 
(3) $\sqrt[3]{3}$是无理数(提示: 可借鉴讲义上$\sqrt{6}$是无理数的证明方法);\\ 
(4*) $\sqrt{2}+\sqrt{3}$是无理数.
\item {\tiny (000987)}已知实数$t\ne 0$. 证明: ``$x=t$是方程$a x^3+b x^2+cx+d=0$的根''的充分必要条件是``$x=\dfrac{1}{t}$是方程$d x^3+c x^2+ b x+a=0$的根''.
\item {\tiny (000988)}已知$a,b,c$均为实数. 证明: 这三个数中``任意两数之和大于第三个数''的充分必要条件是``任意两数之差小于第三个数''.
\item {\tiny (001000)}设$A=\{n|\ n=3k+1,k \in \mathbf{Z}^+\}$, $B=\{n|\ n=3k-2,k \in \mathbf{Z}^+\}$.\\ 
(1) 集合$A$与集合$B$是相等的还是有真包含关系还是没有任何包含关系?\\ 
(2) 证明你的结论.
\item {\tiny (001001)}证明或否定: $\{y|y\ge 0\}=\{y|y=x^2, x \in \mathbf{R}\}$.
\item {\tiny (001002)}设$a$是一个实数, 集合$A=\{x|\ x<2\}$, $B=\{x|\ x\leq a\}$, 且$A \subseteq B$.\\ 
(1) 实数$a$的取值范围为\blank{100};\\ 
(2) 试证明(1)的结论.
\item {\tiny (001017)}设$A,B$是两个集合, 求证: ``$A\cap B=A$''当且仅当``$A \subseteq B$''.(用文氏图画一下并不算证明)
\item {\tiny (001087)}证明: 若$a>b$, $c\in\mathbf{R}$, $d<0$, 则$(a-c)d<(b-c)d$.
\item {\tiny (001088)}证明: 若$a_1>b_1>0,a_2>b_2>0,a_3>b_3>0$, 则$a_1a_2a_3>b_1b_2b_3$.
\item {\tiny (001089)}证明: 若$a>b>0$, $c>d>0$, 则$\dfrac{1}{ac}<\dfrac{1}{bd}$.
\item {\tiny (001091)}证明:\\ 
(1) 若$a>b$, 则$a^3>b^3$;\\ 
(2)(选做) 若$a>b$, 则$a^5>b^5$.
\item {\tiny (001094)}(1) 证明或否定: ``$|f(x)|>g(x)$''和``$f(x)>g(x)$且$-f(x)>g(x)$''等价;\\ 
(2) 证明或否定: ``$|f(x)|<g(x)$''和``$f(x)<g(x)$且$-f(x)<g(x)$''等价.
\item {\tiny (001095)}证明或否定: ``$\sqrt{f(x)}>g(x)$''和
``$\left\{\begin{array}{l}f(x)>g^2(x),\\g(x)\ge 0,\end{array}\right.\ \text{或} \ \left\{\begin{array}{l}f(x)\ge 0,\\g(x)<0\end{array}\right.$''
同解.
\item {\tiny (001096)}利用绝对值的三角不等式$|a+b|\le |a|+|b|$, 证明:\\ 
(1) 对任意$x,y\in\mathbf{R}$, $|x-y|\ge |x|-|y|$;\\ 
(2) 对任意$x,y\in\mathbf{R}$, $|x-y|\ge ||x|-|y||$.
\item {\tiny (001099)}已知常数$\varepsilon>0$, 证明存在实常数$N$, 使得当正整数$n>N$时, $\left|\dfrac{n}{2n+3}-\dfrac{1}{2}\right|<\varepsilon$.
\item {\tiny (001123)}试确定实常数$k$使得$a^2+b^2+c^2\geq k(a+b+c)^2\geq ab+bc+ca$对任意的$a,b,c\in \mathbf{R}$成立, 并证明该不等式.
\item {\tiny (001124)}设$a,b,c,d>0$.\\ 
(1) 利用三元的基本不等式``$x,y,z>0$时, $x^3+y^3+z^3\ge 3xyz$'', 证明: $a^3+b^3+c^3+d^3\geq abc+bcd+cda+dab$;\\ 
(2) 该不等式能否加强为$a^3+b^3+c^3+d^3\ge k(abc+bcd+cda+dab)$, 其中$k=1.0001$? 为什么?\\ 
(3) 利用三元的基本不等式``$x,y,z>0$时, $x^3+y^3+z^3\ge 3xyz$'', 证明: $a^3+b^3+c^3+d^3\geq \dfrac{3\sqrt[3]{2}}{2}(abc+bcd)$.
\item {\tiny (001134)}已知$x,y \in \mathbf{R}$, 用比较法证明: $x^2+y^2\ge 4(x+y)-8$.
\item {\tiny (001135)}已知$f(x)=x+\dfrac{1}{x}$, 利用比较法证明:\\ 
(1) 若$a>b\ge 1$, 证明: $f(a)>f(b)$;\\ 
(2) 若$0<a<b\le 1$, 证明: $f(a)>f(b)$.
\item {\tiny (001136)}已知$a<b<0$, 用分析法证明: $\dfrac{a^2+b^2}{a^2-b^2}<\dfrac{a+b}{a-b}$.
\item {\tiny (001137)}已知$a,b,c\in \mathbf{R}^+$, 证明: $a^2(b+c)+b^2(c+a)+c^2(a+b)\ge 6abc$.
\item {\tiny (001138)}已知$a,b,c$是{\bf 不全相等}的正数. 证明: $\dfrac{a+b}{c}+\dfrac{b+c}{a}+\dfrac{c+a}{b}>6$.
\item {\tiny (001139)}已知$x,y\in \mathbf{R}^+$且$x+y>2$, 用反证法证明: $\dfrac{1+y}{x}$与$\dfrac{1+x}{y}$中至少有一个小于$2$.
\item {\tiny (001141)}已知$x,y\in \mathbf{R}$, 证明: $x^2+5y^2+4xy+5\ge 2x+8y$.
\item {\tiny (001142)}已知$g(x)=x^3-3x$.\\ 
(1) 若$a>b\ge 1$, 证明: $g(a)>g(b)$;\\ 
(2) 若$-1\le a<b\le 1$, 证明: $g(a)>g(b)$.
\item {\tiny (001147)}已知$n\in\mathbf{Z}$, $n \ge 3$. 证明: $3^n+4^n+5^n\le 6^n$, 并求出等号成立的条件.
\item {\tiny (001148)}已知$a,b,c\in \mathbf{R}^+$, $a+b+c=3$. 证明:\\ 
(1) $a^2+b^2+c^2\ge 3$;\\ 
(2) $\dfrac1a+\dfrac1b+\dfrac1c\ge 3$;\\ 
(3) $a^4+b^4+c^4\ge 3$;\\ 
(4) (选做) 对一切$n\in \mathbf{N}$, $a^{2^n}+b^{2^n}+c^{2^n}\ge 3$.
\item {\tiny (001150)}(1) 设$a+b+c=6$, 且$a,b,c\in(0,3)$, 证明: $(3-a)(3-b)(3-c)\le 1$;\\ 
(2) 已知三角形的面积可以用Heron公式$S=\sqrt{p(p-a)(p-b)(p-c)}$来计算, 其中$p$是半周长, 即$p=\dfrac{a+b+c}{2}$. 据此求周长为$6$的三角形的面积的最大值.
\item {\tiny (002700)}集合$C=\{x|x=\dfrac k2\pm \dfrac14, \ k\in \mathbf{Z}\},D=\{x|x=\dfrac k4,\ k\in \mathbf{Z}\}$, 试判断$C$与$D$的关系, 并证明.
\vspace*{24ex}
\item {\tiny (002709)}设函数$f(x)=\lg (\dfrac2{x+1}-1)$的定义域为集合$A$, 函数$g(x)=\sqrt{1-|x+a|}$的定义域为集合$B$.\\
(1) 当$a=1$时, 求集合$B$.\\
(2) 问: $a\ge 2$是$A\cap B=\varnothing$的什么条件(在``充分非必要条件、必要非充分条件、充要条件、既非充分也非必要条件''中选一)?并证明你的结论.
\item {\tiny (002758)}(1) 比较$1+a^2$与$\dfrac 1{1-a}$的大小;\\
(2) 设$a>0,\ a\ne 1$, $t>0$, 比较$\dfrac 12\log_at$和$\log_a\dfrac{t+1}2$的大小, 证明你的结论.
\item {\tiny (004997)}用比较法证明以下各题:\\
(1) 已知$a>0$, $b>0$, 求证: $\dfrac 1a+\dfrac 1b\ge \dfrac 2{\sqrt{ab}}$;\\
(2) 已知$a>0$, $b>0$, 求证: $\dfrac b{\sqrt a}+\dfrac a{\sqrt b}\ge \sqrt a+\sqrt b$;\\
(3) 已知$a>0$, $b>0$, 求证: ${a^2}+{b^2}\ge (a+b)\sqrt{ab}$;\\
(4) 已知$0<x<1$, 求证: $\dfrac{a^2}x+\dfrac{b^2}{1-x}\ge (a+b)^2$.
\item {\tiny (005034)}利用$a^2+b^2+c^2\ge ab+bc+ca(a,b,c\in \mathbf{R})$, 证明: 若$a>0$, $b>0$, $c>0$, 则$\dfrac{a^2}{b^2}+{b^2}{c^2}+{c^2}{a^2}{a+b+c}\ge abc$.
\item {\tiny (005035)}利用$a^2+b^2+c^2\ge ab+bc+ca(a,b,c\in \mathbf{R})$, 证明: 若半径为$1$的圆内接$\triangle ABC$的而积为$\dfrac 14$, 二边长分别为$a,b,c$, 则\\(1) $abc=1$;\\
(2) $\sqrt b+\sqrt c<\dfrac 1a+\dfrac 1b+\dfrac 1c$.
\item {\tiny (005036)}利用$a^2+b^2+c^2\ge ab+bc+ca(a,b,c\in \mathbf{R})$, 证明: 若$a,b,c>0$, $n\in \mathbf{N}$, $f(n)=\lg \dfrac{a^n+b^n+c^n}3$, 则$2f(n)\le f(2n)$.
\item {\tiny (005037)}利用放缩法并结合公式$ab\le (\dfrac{a+b}2)^2$, 证明: $\lg 9\cdot \lg 11<1$.
\item {\tiny (005038)}利用放缩法并结合公式$ab\le (\dfrac{a+b}2)^2$, 证明: $\log_a(a-1)\cdot \log_a(a+1)<1$($a>1$).
\item {\tiny (005039)}利用放缩法并结合公式$ab\le (\dfrac{a+b}2)^2$, 证明: 若$a>b>c$, 则$\dfrac 1{a-b}+\dfrac 1{b-c}+\dfrac 4{c-a}\ge 0$.
\item {\tiny (005040)}利用放缩法证明: $\dfrac 1n+\dfrac 1{n+1}+\dfrac 1{n+2}+\dfrac 1{n+3}+\dfrac 1{n+4}+\cdots +\dfrac 1{n^2}>1$($n\in \mathbf{N}$, $n\ge 2$).
\item {\tiny (005041)}利用放缩法证明: $\dfrac 12\le \dfrac 1{n+1}+\dfrac 1{n+2}+\cdots +\dfrac 1{2n}<1$($n\in \mathbf{N}$).
\item {\tiny (005042)}利用放缩法证明: 已知$a>0$, $b>0$, $c>0$, 且$a^2+b^2=c^2$, 求证: $a^n+b^n<c^n$($n\ge 3$, $n\in \mathbf{N}$).
\item {\tiny (005043)}利用拆项法证明: 若$x>y$, $xy=1$, 则$\dfrac{x^2+y^2}{x-y}\ge 2\sqrt 2$.
\item {\tiny (005044)}利用拆项法证明: $\dfrac 12({a^2}+{b^2})+1\ge \sqrt{{a^2}+1}\cdot \sqrt{{b^2}+1}$.
\item {\tiny (005045)}利用拆项法证明: 若$a>0$, $b>0$, $c>0$, 则$2(\dfrac{a+b}2-\sqrt{ab})\le 3(\dfrac{a+b+c}3-\sqrt[3]{abc})$.
\item {\tiny (005046)}利用拆项法证明: $2(\sqrt{n+1}-1)<1+\dfrac 1{\sqrt 2}+\dfrac 1{\sqrt 3}+\cdots +\dfrac 1{\sqrt n}<2\sqrt n$($n\in \mathbf{N}$).
\item {\tiny (005047)}利用逆代法证明: 若正数$x,y$满足$x+2y=1$, 则$\dfrac 1x+\dfrac 1y\ge 3+2\sqrt 2$.
\item {\tiny (005048)}利用逆代法证明: $\dfrac 1{\sin ^2\alpha}+\dfrac 3{\cos^2\alpha}\ge 4+2\sqrt 3$.
\item {\tiny (005049)}利用逆代法证明: 若$x,y>0$, $a,b$为正常数, 且$\dfrac ax+\dfrac ay=1$, 则$x+y\ge (\sqrt a+\sqrt b)^2$.
\item {\tiny (005050)}利用判别式法证明: $\dfrac 13\le \dfrac{x^2-x+1}{x^2+x+1}\le 3$.
\item {\tiny (005051)}利用判别式法证明: 若关于$x$的不等式$(a^2-1)x^2-(a-1)x-1<0(a\in \mathbf{R})$对仟意实数$x$恒成立, 则$-\dfrac 35<a\le 1$.
\item {\tiny (005052)}利用函数的单调性证明: 若$x>0$, $y>0$, $x+y=1$, 则$(x+\dfrac 1x)(y+\dfrac 1y)\ge \dfrac{25}4$.
\item {\tiny (005053)}利用函数的单调性证明: 若$0<a<\dfrac 1k(k\ge 2,k\in \mathbf{N})$, 且$a^2<a-b$, 则$b<\dfrac 1{k+1}$.
\item {\tiny (005054)}利用三角换元法证明: 若$a^2+b^2=1$, 则$a\sin x+b\cos x\le 1$.
\item {\tiny (005055)}利用三角换元法证明: 若$|a|<1$, $|b|<1$, 则$|ab\pm \sqrt{(1-{a^2})(1-{b^2})}|\le 1$.
\item {\tiny (005056)}利用三角换元法证明: 若$x^2+y^2\le 1$, 则$-\sqrt 2\le x^2+2xy-y^2\le \sqrt 2$.
\item {\tiny (005057)}利用三角换元法证明: 若$|x|\le 1$, 则$(1+x)^n+(1-x)^n\le 2^n$.
\item {\tiny (005058)}利用三角换元法证明: 若$a>0$, $b>0$, 且$a-b=1$, 则$0<\dfrac 1a(\sqrt a-\dfrac 1{\sqrt a})(\sqrt b+\dfrac 1{\sqrt b})<1$.
\item {\tiny (005059)}利用三角换元法证明: $0<\sqrt{1+x}-\sqrt x\le 1$.
\item {\tiny (005060)}试构造几何图形证明: 若$f(x)=\sqrt{1+x^2}$, $x>b>0$, 则$|f(a)-f(b)|<|a-b|$.
\item {\tiny (005061)}试构造几何图形证明: 若$x,y,z>0$, 则$\sqrt{x^2+y^2+xy}+\sqrt{y^2+z^2+yz}>\sqrt{z^2+x^2+zx}$.
\item {\tiny (005062)}利用均值换元证明: 若$a>0$, $b>0$, 且$a+b=1$, 则$\dfrac 43\le \dfrac 1{a+1}+\dfrac 1{b+1}<\dfrac 32$.
\item {\tiny (005063)}利用均值换元证明: 若$a+b+c=1$, 则${a^2}+{b^2}+{c^2}\ge \dfrac 13$.
\item {\tiny (005064)}利用设差换元证明: 若$x\ge y\ge 0$, 则$\sqrt{2xy-{y^2}}+\sqrt{x^2-y^2}\ge x$.
\item {\tiny (005098)}利用反证法证明: 若$0<a<1$, $0<b<1$, $0<c<1$, 则$(1-a)b$, $(1-b)c$, $(1-c)a$不能都大于$\dfrac 14$.
\item {\tiny (005099)}利用反证法证明: 若$0<a<2$, $0<b<2$, $0<c<2$, 则$a(2-b)$, $b(2-c)$, $c(2-a)$不可能都大于$1$.
\item {\tiny (005100)}利用反证法证明: 若$x,y>0$, 且$x+y>2$, 则$\dfrac{1+y}x$和$\dfrac{1+x}y$中至少有一个小于$2$.
\item {\tiny (005101)}利用反证法证明: 若$0<a<1$, $b>0$, 且$a^b=b^a$, 则$a=b$.
\item {\tiny (005102)}若$a>0$, $b>0$, 且$a^3+b^3=2$, 试分别利用$x^3+y^3+z^3\ge 3xyz$($x,y,z\ge 0$)构造方程, 并利用判别式以及反证法证明: $a+b\le 2$.
\item {\tiny (007768)}证明: 如果$a>b$, $c<0$, 那么$(a-b)c<0$.
\item {\tiny (007769)}证明: 如果$a<b<0$, 那么$0>\dfrac 1a>\dfrac 1b$.
\item {\tiny (007818)}设$ab\ne 0$, 利用基本不等式有如下证明: $\dfrac ba+\dfrac ab=\dfrac{{b^2}+{a^2}}{ab}\ge \dfrac{2ab}{ab}=2$. 试判断这个证明过程是否正确. 若正确, 请说明每一步的依据; 若不正确, 请说明理由.
\item {\tiny (007837)}证明: 如果$a>b>0$, $c>d>0$, 那么$a^2c>b^2d$.
\item {\tiny (007838)}证明: $a^2+b^2+2\ge 2(a+b)$.
\item {\tiny (007839)}证明: 如果$a$、$b$、$c$都是正数, 那么$(a+b)(b+c)(c+a)\ge 8abc$.
\item {\tiny (009442)}设$n\in \mathbf{Z}$. 证明: 若$n^3$是奇数, 则$n$是奇数.
\item {\tiny (009443)}证明: 对于三个实数$a$、$b$、$c$, 若$a\ne c$, 则$a\ne b$或$b\ne c$.
\item {\tiny (009464)}证明: 若$x<0$, 则$x+\dfrac 1x\le -2$, 并指出等号成立的条件.
\item {\tiny (009468)}已知实数$a$、$b$满足$|a| <\dfrac 12$, $|b| <\dfrac 12$. 证明下列各式:\\
(1) $|a+b| <1$;\\
(2) $|a-b| <1$.
\item {\tiny (010027)}已知集合$A=\{x|x=2n+1,\ n\in \mathbf{Z}\}$, $B=\{x|x=4n-1,\ n\in \mathbf{Z}\}$. 判断集合$A$与$B$的包含关系, 并证明你的结论.
\vspace*{16ex}
\item {\tiny (010035)}证明: ``四边形$ABCD$是平行四边形''是``四边形$ABCD$的对角线互相平分''的充要条件.
\item {\tiny (010050)}证明: ``$a>0$且$b>0$''是``$a+b>0$且$ab>0$''的充要条件.
\item {\tiny (010060)}对一元二次方程$ax^2+bx+c=0$($a\ne 0$), 证明: $ac<0$是该方程有两个异号实根的充要条件.
\item {\tiny (010064)}设$s=a+b$, $p=ab$($a$、$b\in\mathbf{R}$), 写出``$a>1$且$b>1$''用$s$、$p$表示的一个充要条件, 并证明.
\item {\tiny (010065)}原有酒精溶液$a$(单位: $\text{g}$), 其中含有酒精$b$(单位: $\text{g}$), 其酒精浓度为$\dfrac ba$. 为增加酒精浓度, 在原溶液中加入酒精$x$(单位: $\text{g}$), 新溶液的浓度变为$\dfrac{b+x}{a+x}$. 根据这一事实, 可提炼出如下关于不等式的命题:若$a>b>0$, $x>0$, 则$\dfrac ba<\dfrac{b+x}{a+x}<1$. 试加以证明.
\item {\tiny (010101)}证明:对于正数$h$, 如果$|x-a| <\dfrac h2$, $|y-a| <\dfrac h2$, 那么$|x-y| <h$.
\item {\tiny (010104)}证明: $|x+2|-|x-1|\ge -3$, 对所有实数$x$均成立, 并求等号成立时$x$的取值范围.
\item {\tiny (020014)}已知集合$A=\{x|x=a+\sqrt 2b,\ a,b\in \mathbf{Z}\}$, 若$x_1,x_2\in A$, 证明: $x_1x_2\in A$.
\item {\tiny (020024)}证明:集合$A=\{1,2,3\}$是集合$B=\{0,1,2,3,4,5,6\}$的子集.
\item {\tiny (020026)}证明集合$A=\{n|n=2k-1,\ k\in \mathbf{N}\}$不是集合$B=\{n|n=2m+1, \ m\in \mathbf{N}\}$的子集, 且集合$A$真包含集合$B$.
\item {\tiny (020030)}设常数$a\in \mathbf{R}$. 若集合$A=(-\infty ,5)$与$B=(-\infty ,a]$满足$A\subseteq B$, 则$a$的取值范围是\blank{50}.\\
证明: $1^\circ$ 当$a$\blank{50}时, 任取$x\in A$, 则\blank{50}, 所以$x\in B$, 即$A\subseteq B$.\\ 
$2^\circ$ 当$a$\blank{50}时, 取$x_1=$\blank{50}, 则\blank{50}, 所以$x_1\in A$且$x_1\not \in B$.\\
由$1^\circ$、$2^\circ$可得结论.
\item {\tiny (020035)}证明:集合$A=\{x|x=6n-1, \ n\in\mathbf{Z}\}$是$B=\{x|x=3n+2, \ n\in\mathbf{Z}\}$的真子集.
\vspace*{16ex}
\item {\tiny (020041)}已知$A=\{x|x=a+\sqrt 2b,\ a,b\in \mathbf{N}\}$, 若集合$B=\{x|x=\sqrt 2x_1,\  x_1 \in A\}$, 证明$B\subset A$.
\vspace*{24ex}
\item {\tiny (020083)}证明: $x_1>2$且$x_2>2$是$x_1+x_2>4$且$x_1\cdot x_2>4$的充分非必要条件.
\item {\tiny (020085)}设$\alpha,\beta$是方程$x^2-ax+b=0$的两个实数根. 试分析$a>2$且$b>1$是``两个实数根$\alpha,\beta$均大于$1$''的什么条件? 并证明你的结论.
\item {\tiny (020092)}证明: 若$x+2y+z>0$, 则$x,y,z$中至少有一个大于$0$.
\item {\tiny (020093)}证明:对于三个实数$a,b,c$, 若$a\ne c$, 则$a\ne b$或$b\ne c$.
\item {\tiny (020095)}证明: 若$x^2\ne y^2$, 则$x\ne y$或$x\ne -y$.
\item {\tiny (020096)}若$a^3+b^3=2$, 证明: $a+b\le 2$.
\end{enumerate}



\end{document}