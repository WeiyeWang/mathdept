\item (004606)某地区为贯彻习近平总书记关于``绿水青山就是金山银山''的理念, 鼓励农户利用荒坡种植果树. 某农户考察三种不同的果树苗$A,B,C$, 经引种试验后发现, 引种树苗$A$的自然成活率为$0.8$, 引种树苗$B,C$的自然成活率均为$p$($0.7\le p\le 0.9$).\\
(1) 任取树苗$A,B,C$各一棵, 估计自然成活的棵数为$X$, 求$X$的分布列及数学期望$E[X]$;\\
(2) 将(1)中的$E[X]$取得最大值时$p$的值作为$B$种树苗自然成活的概率. 该农户决定引种$n$棵$B$种树苗, 引种后没有自然成活的树苗中有$75\%$的树苗可经过人工栽培技术处理, 处理后成活的概率为$0.8$, 其余的树苗不能成活.\\
\textcircled{1} 求一棵$B$种树苗最终成活的概率;\\
\textcircled{2} 若每棵树苗最终成活后可获利$300$元, 不成活的每棵亏损$50$元, 该农户为了获利不低于$20$万元, 问至少引种$B$种树苗多少棵?
\item (004608)一年之计在于春, 一日之计在于晨, 春天是播种的季节, 是希望的开端. 某种植户对一块地的$n$($n\in \mathbf{N}$, $n>0$)个坑进行播种, 每个坑播$3$粒种子, 每粒种子发芽的概率均为$\dfrac 12$, 且每粒种子是否发芽相互独立. 对每一个坑而言, 如果至少有$2$粒种子发芽, 则不需要进行补播种, 否则要补播种.\\
(1) 设恰有$3$个坑需要补种的概率为$f(n)$($n\in \mathbf{N}$, $n\ge 3$), 当$n$取何值时, $f(n)$取得最大值? $f(n)$的最大值为多少?\\
(2) 当$n=4$时, 用$X$表示要补播种的坑的个数, 求$X$的分布列与数学期望.
\item (004611)河南省三门峡市成功入围``十佳魅力中国城市'', 吸引了大批投资商的目光, 一些投资商积极准备投入到``魅力城市''的建设之中. 某投资公司准备在$2022$年年初将$400$万元投资到三门峡下列两个项目中的一个之中.\\
项目一: 天坑院是黄土高原地域独具特色的民居形式, 是人类``穴居''发展史演变的实物见证. 现准备投资建设$20$个天坑院, 每个天坑院投资$20$万元, 假设每个天坑院是否盈利是相互独立的, 据市场调研, 到$2024$年底每个天坑院盈利的概率为$p$($0<p<1$), 若盈利则盈利投资额的$40\%$, 否则盈利额为$0$.\\
项目二: 天鹅湖国家湿地公园是一处融生态、文化和人文地理于一体的自然山水景区. 据市场调研, 投资到该项目上, 到$2024$年底可能盈利投资额的$50\%$, 也可能亏损投资额的$30\%$, 且这两种情况发生的概率分别为$p$和$1-p$.\\
(1) 若投资项目一, 记$X_1$为盈利的天坑院的个数, 求$E[X_1]$(用$p$表示);\\
(2) 若投资项目二, 记投资项目二的盈利为$X_2$百万元, 求$E[X_2]$(用$p$表示);\\
(3) 在(1)(2)两个条件下, 针对以上两个投资项目, 请你为投资公司选择一个项目, 并说明理由.
\item (004616)在测试中, 客观题难度的计算公式为$P_i=\dfrac{R_i}N$, 其中$P_i$为第$i$题的难度, $R_i$为答对该题的人数, $N$为参加测试的总人数. 现对某校高三年级$240$名学生进行一次测试, 共$5$道客观题, 测试前根据对学生的了解, 预估了每道题的难度, 如下表所示:
\begin{center}
    \begin{tabular}{|c|c|c|c|c|c|}
        \hline
        题号 & $1$ & $2$ & $3$ & $4$ & $5$ \\ \hline
        考前预估难度$P_i$ & $0.9$ & $0.8$ & $0.7$ & $0.6$ & $0.4$ \\ \hline       
    \end{tabular}
\end{center}
测试后, 随机抽取了$20$名学生的答题数据进行统计, 结果如下:
\begin{center}
    \begin{tabular}{|c|c|c|c|c|c|}
        \hline
        题号 & $1$ & $2$ & $3$ & $4$ & $5$ \\ \hline
        实测答对人数 & $16$ & $16$ & $14$ & $14$ & $4$ \\ \hline       
    \end{tabular}
\end{center}
(1) 根据题中数据, 估计这$240$名学生中第$5$题的实测答对人数;\\
(2) 从抽样的$20$名学生中随机抽取$2$名学生, 记这$2$名学生中答对第$5$题的人数为$X$, 求$X$的分布列和数学期望;\\
(3) 试题的预估难度和实测难度之间会有偏差, 设$P_i'$为第$i$题的实测难度, 并定义统计量$S=\dfrac 1n[(P_1'-P_1)^2+(P_2'-P_2)^2+\cdots+(P_n'-P_n)^2]$, 若$S<0.05$, 则本次测试的难度预估合理, 否则不合理, 试检验本次测试对难度的预估是否合理.
\item (004617)在中华人民共和国成立$70$周年时, 《我和我的祖国》《中国机长》《攀登者》三大主旋律电影在国庆期间集体上映. 据统计, 《我和我的祖国》票房收入为$31.46$亿元, 《中国机长》票房收入为$28.84$亿元, 《攀登者》票房收入为$10.88$亿元. 已知国庆过后某城市文化局统计得知大量市民至少观看了一部国庆档电影, 在已观影的市民中随机抽取了$100$人进行调查, 其中观看了《我和我的祖国》的有$49$人, 观看了《中国机长》的有$46$人, 观看了《攀登者》的有$34$人, 统计图如图所示.
\begin{center}
    \begin{tikzpicture}
        \draw (0,0) circle (2) node [above] {《中国机长》} node [below] {$(27)$};
        \draw (3.5,0) circle (2.5) node [above] {《我和我的祖国》} node [below] {$(30)$};
        \draw (1.2,-2.6) circle (1.8) node [above] {《攀登者》} node [below] {$(18)$};
        \draw (1.45,-1) node {$(4)$};
        \draw (1.5,0) node {$(a)$};
        \draw (0.6,-1.3) node {$(b)$};
        \draw (2.2,-1.6) node {$(c)$};
    \end{tikzpicture}
\end{center}
(1) 计算图中$a, b, c$的值;\\
(2) 文化局从只观看了两部电影的观众中采用分层抽样的方法抽取了$7$人进行观影体验的访谈, 了解到他们均表示要观看第三部电影, 现从这$7$人中随机选出$4$人, 用$X$表示这$4$人中将要观看《我和我的祖国》的人数, 求$X$的分布列.
