\item (004572)某地区气象台统计, 该地区下雨的概率是$\dfrac 4{15}$, 刮风的概率是$\dfrac 25$, 既刮风又下雨的概率为$\dfrac 1{10}$, 设事件$A$表示``该地区下雨'', 事件$B$表示``该地区刮风'', 那么$P(B|A)$等于\blank{50}.
\item (004573)已知盒中装有$3$只螺口灯泡与$7$只卡口灯泡, 这些灯泡的外形都相同且灯口向下放着, 现需要安装一只卡口灯泡, 电工师傅每次从盒中任取一只并且不放回, 则在他第$1$次抽到的是螺口灯泡的条件下, 第$2$次抽到的是卡口灯泡的概率为\blank{50}.
\item (004574)近年来, 新能源汽车技术不断推陈出新, 新产品不断涌现, 在汽车市场上影响力不断增大. 动力蓄电池技术作为新能源汽车的核心技术, 它的不断成熟也是推动新能源汽车发展的主要动力. 假定现在市售的某款新能源汽车上, 车载动力蓄电池充放电循环次数达到$2000$次的概率为$85\%$, 充放电循环次数达到$2500$次的概率为$35\%$. 若某用户的自用新能源汽车已经经过了$2000$次充电, 那么他的车能够充电$2500$次的概率为\blank{50}.
\item (004575)将三颗骰子各掷一次, 记事件$A$为``三个点数都不相同'', $B$为``至少出现一个$6$点'', 则条件概率$P(A|B)$=\blank{50}, $P(B|A)$=\blank{50}.
\item (004576)袋中有大小完全相同的$2$个白球和$3$个黄球, 逐个不放回地摸出$2$个球, 设``第一次摸到白球''为事件$A$, ``摸到的$2$个球同色''为事件$B$, 则$P(B|A)$=\blank{50}.
\item (004577)已知$P(A)>0$, $P(B)>0$, $P(B|A)=P(B)$, 证明: $P(A|B)=P(A)$.
\item (004578)*甲、乙、丙三人互相作传球训练, 第$1$次由甲将球传出, 每次传球时, 传球者都等可能地将球传给另外两个人中的任何一个, 求$4$次传球后球在甲手中的概率.
\item (004579)现在有$12$道四选一的单选题, 学生张三对其中$9$道题有思路, $3$道题完全没有思路. 有思路的题做对的概率为$0.9$, 没有思路的题只好任意猜一个答案, 猜对的概率为$0.25$, 张三从这$12$道题中随机选择$1$题, 则他做对该题的概率是\blank{50}.
\item (004580)两批同种规格的产品, 第一批占$40\%$, 次品率为$5\%$;第二批占$60\%$, 次品率为$4\%$, 将这两批产品混合, 从混合的产品中任取一件. 则这件产品时合格品的概率是\blank{50}.
\item (004581)甲和乙两个箱子中各装有$10$个球, 其中甲箱中有$5$个红球、$5$个白球, 乙箱中有$8$个红球、$2$个白球. 掷一枚质地均匀的骰子, 如果点数为$1$或$2$, 从甲箱子随机摸出$1$个球; 如果点数为$3, 4, 5, 6$, 从乙箱子中随机摸出$1$个球, 则摸到红球的概率是\blank{50}.
\item (004582)在$A$、$B$、$C$三个地区暴发了流感, 这三个地区分别有$6\%$, $5\%$, $4\%$的人患了流感, 假设这三个地区的人口数的比为$5: 7: 8$, 现从这三个地区中任意选取一个人. 则这个人患流感的概率是\blank{50}.
\item (004583)甲、乙两人独立地向同一目标各射击一次, 已知甲命中目标的概率为$0.6$, 乙命中目标的概率为$0.5$, 则目标至少被命中一次时, 甲命中目标的概率是\blank{50}.
\item (004584)设$P(A)>0$, 且$B$和$\overline B$是对立事件, 求证: $P(\overline B|A)=1-P(B|A)$.
\item (004585)一批产品共有$100$件, 其中$5$件为不合格品, 收货方从中不放回地随机抽取产品进行检验, 并按以下规则判断是否接受这批产品; 如果抽检的第$1$件产品不合格, 则拒绝整批产品; 如果抽检的第一件产品合格, 则再抽$1$件, 如果抽检的第$2$件产品合格, 则接受整批产品, 否则拒绝整批产品, 求这批产品被拒绝的概率.
\item (004586)在孟德尔豌豆试验中, 子二代(数量充分大)的基因型为DD, Dd, dd, 其中D为显性基因, d为隐性基因, 且这三种基因型的比为$1: 2: 1$. 如果在子二代中任意选取$2$颗豌豆作为父代进行杂交试验, 那么第三代中基因型为dd的概率有多大?
\item (004587)长时间玩手机可能影响视力, 据调查, 某校学生大约$40\%$的人近视, 而该校大约有$20\%$的学生每天玩手机超过$1\text{h}$, 这些人的近视率为$50\%$. 现从每天玩手机不超过$1\text{h}$的学生中任意调查一名学生, 求他的近视概率.
\item (004588)设随机变量$X$的概率分布列如下, 则$P(|X-2|=1)=$\blank{50}.
\begin{center}
    $\begin{pmatrix}
        1 & 2 & 3 & 4\\ 
        \dfrac 16 & \dfrac 14 & m & \dfrac 13      
    \end{pmatrix}$
\end{center}
\item (004589)已知离散型随机变量$X$的分布列为
\begin{center}
    $\begin{pmatrix}
        0 & 1 & 2 \\ 
        0.5 & 1-2q & q^2 
    \end{pmatrix}$
\end{center}
则常数$q=$\blank{50}.
\item (004590)一盒中有$12$个乒乓球, 其中$9$个新的, $3$个旧的, 从盒子中一次性任取$3$个球来用, 用完即为旧的, 用完后装回盒中, 此时盒中旧球个数$X$是一个随机变量, 则$P(X=4)$的值为\blank{50}.
\item (004591)离散型随机变量$X$的概率分布规律为$P(X=n)=\dfrac{a}{n(n+1)}$($n=1, 2, 3, 4$), 其中$a$是常数, 则$P(\dfrac 12<X<\dfrac 52)$的值为\blank{50}.
\item (004592)设离散型随机变量X的分布列如下表, 求$|X-1|$的分布列.
\begin{center}
    $\begin{pmatrix}
        0 & 1 & 2 & 3 & 4 \\ 
        0.2 & 0.1 & 0.1 & 0.3 & m 
    \end{pmatrix}$
\end{center}
\item (004593)某射手有$5$发子弹, 射击一次命中目标的概率为$0.9$, 如果命中就停止射击, 否则一直到子弹用尽, 求耗用子弹数$X$的分布列.
\item (004594)某汽车美容公司为吸引顾客, 推出优惠活动: 对首次消费的顾客, 按$200$元/次收费, 并注册成为会员, 对会员逐次消费给予相应优惠, 标准如下:\\
\begin{center}
    \begin{tabular}{|c|c|c|c|c|c|}
        \hline
        消费次第 & 第$1$次 & 第$2$次 & 第$3$次 & 第$4$次 & $\ge 5$次 \\ \hline
        收费比率 & $1$ & $0.95$ & $0.90$ & $0.85$ & $0.80$\\ \hline
    \end{tabular}
\end{center}
该公司注册的会员中没有消费超过$5$次的, 从注册的会员中, 随机抽取了$100$位进行统计, 得到的统计数据如下:
\begin{center}
    \begin{tabular}{|c|c|c|c|c|c|}
        \hline
        消费次数 & $1$ & $2$ & $3$ & $4$ & $5$ \\ \hline
        人数 & $60$ & $20$ & $10$ & $5$ & $5$\\ \hline
    \end{tabular}
\end{center}
假设汽车美容$1$次, 公司成本为$150$元, 根据所给数据, 解答下列问题:\\
(1) 某会员仅消费$2$次, 求这$2$次消费中, 公司获得的平均利润;\\
(2) 以事件发生的频率作为相应事件发生的概率, 设该公司为$1$位会员服务的平均利润为$X$元, 求$X$的分布列.
\item (004595)习近平总书记在$2020$年新年贺词中勉励大家:``让我们只争朝夕, 不负韶华, 共同迎接$2020$年的到来.'' 其中``只争朝夕, 不负韶华''旋即成了网络热词, 成了大家互相砥砺前行的铮铮誓言, 激励着广大青年朋友奋发有为, 积极进取, 不负青春, 不负时代.\\
``只争朝夕, 不负韶华''用英文可翻译为:``seize the day and live it to the full''\\
(1) 求上述英语译文中, e, i, t, a $4$个字母出现的频率(不计入空格, 小数点后面保留两位有效数字), 并比较$4$个频率的大小(用``>''连接);\\
(2) 在上面的句子中随机取一个单词, 用$X$表示取到的单词所包含的字母个数, 写出$X$的分布列;\\
(3) 从上述单词中任选$2$个单词, 求其字母个数之和为$6$的概率.
\item (004596)已知$X$的分布列为
\begin{center}
    $\begin{pmatrix}
        -1 & 0 & 1 \\ 
        \dfrac 12 & \dfrac 13 & \dfrac 16 
    \end{pmatrix}$
\end{center}
两个随机变量$X$, $Y$满足$X+2Y=4$, 则$E[X]=$\blank{50}, $E[Y]=$\blank{50}.
\item (004597)``过大年, 吃水饺''是我国不少地方过春节的一大习俗. $2021$年春节前夕, $A$市某质量检测部门随机抽取了$100$包某种品牌的速冻水饺, 检测其某项质量指标值, 所得频率分布直方图如图.
\begin{center}
    \begin{tikzpicture}[>=latex]
        \draw [->] (0,0) -- (7,0) node [below] {质量指标值};
        \draw [->] (0,0) -- (0,4) node [left] {$\dfrac{\text{频率}}{\text{组距}}$};
        \draw (0,0) node [below left] {$O$};
        \draw (0,1) node [left] {$0.010$} -- (1,1) -- (1,0) node [below] {$10$};
        \draw (1,1) -- (1,2) -- (2,2) -- (2,0) node [below] {$20$};
        \draw (2,2) -- (2,3) -- (3,3) -- (3,0) node [below] {$30$};
        \draw (3,2.5) -- (4,2.5) -- (4,0) node [below] {$40$};
        \draw (4,1.5) -- (5,1.5) -- (5,0) node [below] {$50$};
        \draw [dashed] (0,1.5) node [left] {$0.015$} -- (4,1.5);
        \draw [dashed] (0,2) node [left] {$0.020$} -- (1,2);
        \draw [dashed] (0,2.5) node [left] {$0.025$} -- (3,2.5);
        \draw [dashed] (0,3) node [left] {$0.030$} -- (2,3);
    \end{tikzpicture}
\end{center}
(1) 求所抽取的$100$包速冻水饺该项质量指标值的样本平均数$\overline x$(同一组中的数据用该组区间的中点值作代表);\\
(2) 将频率视为概率, 若某人从该市某超市购买了$4$包这种品牌的速冻水饺, 记这$4$包速冻水饺中该项质量指标值位于$(10,30]$内的包数为$X$, 求$X$的分布列和期望.
\item (004598)近年来, 祖国各地依托本地自然资源, 打造旅游产业, 旅游业正蓬勃发展. 景区与游客都应树立尊重自然、顺应自然、保护自然的生态文明理念, 合力使旅游市场走上规范有序且可持续的发展轨道. 某景区有一个自愿消费的项目: 在参观某特色景点入口处会为每位游客拍一张与景点的合影, 参观后, 在景点出口处会将刚拍下的照片打印出来, 游客可自由选择是否带走照片, 若带走照片则需支付$20$元, 没有被带走的照片会收集起来统一销毁. 该项目运营一段时间后, 统计出平均只有$30\%$游客会选择带走照片. 为改善运营状况, 该项目组就照片收费与游客消费意愿关系做了市场调研, 发现收费与消费意愿有较强的线性相关性, 并统计出在原有的基础上, 价格每下调$1$元, 游客选择带走照片的可能性平均增加$0.05$. 假设平均每天约有$5000$人参观该特色景点, 每张照片的综合成本为$5$元, 假设每位游客是否购买照片相互独立.\\
(1) 若调整为支付$10$元就可带走照片, 该项目每天的平均利润比调整前多还是少?\\
(2) 要使每天的平均利润达到最大值, 应如何定价?
\item (004599)某种大型医疗检查机器生产商, 对一次性购买$2$台机器的客户, 推出$2$种超过质保期后$2$年内的延保维修优惠方案.\\
方案一: 交纳延保金$7000$元, 在延保的$2$年内可免费维修$2$次, 超过$2$次每次收取维修费$2000$元;\\
方案二: 交纳延保金$10000$元, 在延保的$2$年内可免费维修$4$次, 超过$4$次每次收取维修费$1000$元.\\
某医院准备一次性购买$2$台这种机器. 现需决策在购买机器时应购买哪种延保方案, 为此搜集并整理了$50$台这种机器超过质保期后延保$2$年内维修的次数, 得下表:
\begin{center}
    \begin{tabular}{|c|c|c|c|c|}
        \hline
        维修次数 & $0$ & $1$ & $2$ & $3$\\ \hline
        台数 & $5$ & $10$ & $20$ & $15$\\ \hline
    \end{tabular}
\end{center}
以这$50$台机器维修次数的频率代替$1$台机器维修次数发生的概率. 记$X$表示这$2$台机器超过质保期后延保的$2$年内共需维修的次数.\\
(1) 求$X$的分布列;\\
(2) 以方案一与方案二所需费用(所需延保金及维修费用之和)的期望值为决策依据, 医院选择哪种延保方案更合算?
\item (004600)已知$X$的分布列为
\begin{center}
    $\begin{pmatrix}
        -1 & 0 & 1 \\
        \dfrac 12 & \dfrac 13 & \dfrac 16      
    \end{pmatrix}$
\end{center}
两个随机变量$X$, $Y$满足$X+2Y=4$, 则$D[X]=$\blank{50}, $D[Y]=$\blank{50}.
\item (004601)五个自然数$1, 2, 3, 4, 5$按照一定的顺序排成一排.\\
(1) 求$2$和$4$不相邻的概率;\\
(2) 定义: 若两个数的和为$6$且相邻, 称这两个数为一组``友好数''. 随机变量X表示上述五个自然数组成的一个排列中``友好数''的组数, 求$X$的分布列、数学期望$E[X]$和方差$D[X]$.
