\item (000953)若$\mathrm{i}(b\mathrm{i}+1)$是纯虚数, $\mathrm{i}$是虚数单位, 则实数$b=$\blank{50}.
\item (000954)函数$y=\sqrt{2^x-1}$的定义域是\blank{50}(用区间表示).
\item (000955)已知$\triangle ABC$中, $|\overrightarrow{AB}|=2 $,  $|\overrightarrow{AC}|=3 $, $\overrightarrow{AB}\cdot \overrightarrow{AC}<0$, 且$\triangle ABC$的面积为$\dfrac32$, 则$\angle BAC=$\blank{50}.
\item (000956)双曲线$4 x^2-y^2=1$的一条渐近线与直线$tx+y+1=0$垂直, 则$t=$\blank{50}.
\item (000957)已知抛物线上一点$M(x_0,2 \sqrt3)$, 则点$M$到抛物线焦点的距离为\blank{50}.
\item (000958)无穷等比数列首项为$1$,公比为$q \ (q>0)$, 前$n$项和为$S_n$, 若$\displaystyle\lim_{n\to\infty}S_n=2$, 则$q=$\blank{50}.
\item (000959)在一个水平放置的底面半径为$\sqrt 3$的圆柱形量杯中装有适量的水, 现放入一个半径为$R$的实心铁球, 球完全浸没于水中且无水溢出, 若水面高度恰好上升$R$, 则$R$=\blank{50}.
\item (000960)在平面直角坐标系$xOy$中, 将点$A(2,1)$绕原点$O$逆时针旋转$\dfrac\pi 4$到点$B$, 若直线$OB$的倾斜角为$\alpha$, 则$\cos \alpha$的值为\blank{50}.
\item (000961)已知函数$f(x)=2^x-a\cdot 2^{-x}$的反函数是$f^{-1}(x)$, $f^{-1}(x)$在定义域上是奇函数, 则正实数$a=$\blank{50}.
\item (000962)已知$x\ge 1$, $y\ge 0$, 集合$A=\{(x,y)|x+y\le 4\}$, $B=\{(x,y)|x-y+t=0\}$. 如果$A\cap B\ne \varnothing$,则$t$的取值范围是\blank{50}.
\item (000963)如图, 一个空间几何体的主视图、左视图、俯视图均为全等的等腰直角三角形, 如果直角三角形的直角边长都为$1$, 那么这个几何体的表面积为\blank{50}.
\begin{center}
    \begin{tikzpicture}
        \draw (0,2) -- (2,0) -- (0,0) -- cycle;
        \draw (1,0) node [below] {主视图};
        \draw (3,0) -- (5,0) -- (3,2) -- cycle;
        \draw (4,0) node [below] {左视图};
        \draw (0,-1) -- (2,-1) -- (0,-3) -- cycle;
        \draw (1,-3) node [below] {俯视图};
    \end{tikzpicture}
\end{center}
\item (000964)已知全集$U=\mathbf{R}$, 集合$A=\{x|(x-1)(x-4)\le 0\}$, 则集合$A$的补集$\complement_UA=$\blank{50}.
\item (000965)指数方程$4^x-6 \times 2^x-16=0$的解是\blank{50}.
\item (000966)已知无穷等比数列$\{a_n\}$的首项$a_1=18$, 公比$q=-\dfrac12$, 则无穷等比数列$\{a_n\}$各项的和是\blank{50}.
\item (000967)函数$y=\cos 2x, \ x\in [0,\pi]$的递增区间为\blank{50}.
\item (000968)抛物线$y^2=x$上一点$M$到焦点的距离为$1$, 则点M的横坐标是\blank{50}.
\item (000969)一盒中装有$12$个同样大小的球, 其中$5$个红球, $4$个黑球, $2$个白球, $1$个绿球. 从中随机取出$1$个球, 则取出的$1$个球是红球或黑球或白球的概率为\blank{50}.
\item (000970)关于$\theta$的函数$f(\theta)=\cos^2\theta-2x\cos\theta-1$的最大值记为$M(x)$, 则$M(x)$的解析式为\blank{50}.
\item (000971)如图所示, 是一个由圆柱和球组成的几何体的三视图, 若$a=2$, $b=3$, 则该几何体的体积等于\blank{50}.
\begin{center}
    \begin{tikzpicture}[>=latex,scale = 0.6]
        \draw (0,1) circle (1);
        \draw (-1,-0.1) -- (-1,-0.3) (1,-0.1) -- (1,-0.3);
        \draw [->] (-0.2,-0.2) -- (-1,-0.2);
        \draw [->] (0.2,-0.2) -- (1,-0.2);
        \draw (0,-0.2) node {$a$};
        \draw (1.1,0) -- (1.3,0) (1.1,2) -- (1.3,2);
        \draw [->] (1.2,0.8) -- (1.2,0);
        \draw [->] (1.2,1.2) -- (1.2,2);
        \draw (1.2,1) node {$a$};
        \draw (0,-1) node {俯视图};
        \draw (-1,3.5) rectangle (1,6.5) (0,7.5) circle (1);
        \draw (-1,3.4) -- (-1,3.2) (1,3.4) -- (1,3.2);
        \draw [->] (-0.2,3.3) -- (-1,3.3);
        \draw [->] (0.2,3.3) -- (1,3.3);
        \draw (0,3.3) node {$a$};
        \draw (1.1,3.5) -- (1.3,3.5) (1.1,6.5) -- (1.3,6.5);
        \draw [->] (1.2,4.6) -- (1.2,3.5);
        \draw [->] (1.2,5.4) -- (1.2,6.5);
        \draw (1.2,5) node {$b$};
        \draw (0,2.5) node {主视图};
        \draw (3,3.5) rectangle (5,6.5) (4,7.5) circle (1);
        \draw (3,3.4) -- (3,3.2) (5,3.4) -- (5,3.2);
        \draw [->] (3.8,3.3) -- (3,3.3);
        \draw [->] (4.2,3.3) -- (5,3.3);
        \draw (4,3.3) node {$a$};
        \draw (5.1,3.5) -- (5.3,3.5) (5.1,6.5) -- (5.3,6.5);
        \draw [->] (5.2,4.6) -- (5.2,3.5);
        \draw [->] (5.2,5.4) -- (5.2,6.5);
        \draw (5.2,5) node {$b$};
        \draw (4,2.5) node {左视图};
    \end{tikzpicture}
\end{center}
\item (000972)已知双曲线$x^2-\dfrac{y^2}{m^2}=1 \ (m>0)$的渐近线与圆$x^2+(y+2)^2=1$没有公共点, 则该双曲线的焦距的取值范围为\blank{50}.
\item (000973)已知$\triangle ABC$外接圆的半径为$2$, 圆心为$O$, 且$\overrightarrow{AB}+\overrightarrow{AC}=2 \overrightarrow{AO}$, $|\overrightarrow{AB}|=|\overrightarrow{AO}|$, 则$\overrightarrow{CA}\cdot \overrightarrow{CB}=$\blank{50}.
\item (000974)若不等式组$\begin{cases} x\ge 0, \\ x+3y\ge 4, \\  3x+y\le 4 \end{cases}$所表示的平面区域被直线$y=kx+\dfrac 43$分为面积相等的两部分, 则$k$的值是\blank{50}.
