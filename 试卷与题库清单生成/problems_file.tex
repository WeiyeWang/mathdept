\item (004575)将三颗骰子各掷一次, 记事件$A$为``三个点数都不相同'', $B$为``至少出现一个$6$点'', 则条件概率$P(A|B)$=\blank{50}, $P(B|A)$=\blank{50}.
\item (004586)在孟德尔豌豆试验中, 子二代(数量充分大)的基因型为DD, Dd, dd, 其中D为显性基因, d为隐性基因, 且这三种基因型的比为$1: 2: 1$. 如果在子二代中任意选取$2$颗豌豆作为父代进行杂交试验, 那么第三代中基因型为dd的概率有多大?
\item (004598)近年来, 祖国各地依托本地自然资源, 打造旅游产业, 旅游业正蓬勃发展. 景区与游客都应树立尊重自然、顺应自然、保护自然的生态文明理念, 合力使旅游市场走上规范有序且可持续的发展轨道. 某景区有一个自愿消费的项目: 在参观某特色景点入口处会为每位游客拍一张与景点的合影, 参观后, 在景点出口处会将刚拍下的照片打印出来, 游客可自由选择是否带走照片, 若带走照片则需支付$20$元, 没有被带走的照片会收集起来统一销毁. 该项目运营一段时间后, 统计出平均只有$30\%$游客会选择带走照片. 为改善运营状况, 该项目组就照片收费与游客消费意愿关系做了市场调研, 发现收费与消费意愿有较强的线性相关性, 并统计出在原有的基础上, 价格每下调$1$元, 游客选择带走照片的可能性平均增加$0.05$. 假设平均每天约有$5000$人参观该特色景点, 每张照片的综合成本为$5$元, 假设每位游客是否购买照片相互独立.\\
(1) 若调整为支付$10$元就可带走照片, 该项目每天的平均利润比调整前多还是少?\\
(2) 要使每天的平均利润达到最大值, 应如何定价?
\item (004599)某种大型医疗检查机器生产商, 对一次性购买$2$台机器的客户, 推出$2$种超过质保期后$2$年内的延保维修优惠方案.\\
方案一: 交纳延保金$7000$元, 在延保的$2$年内可免费维修$2$次, 超过$2$次每次收取维修费$2000$元;\\
方案二: 交纳延保金$10000$元, 在延保的$2$年内可免费维修$4$次, 超过$4$次每次收取维修费$1000$元.\\
某医院准备一次性购买$2$台这种机器. 现需决策在购买机器时应购买哪种延保方案, 为此搜集并整理了$50$台这种机器超过质保期后延保$2$年内维修的次数, 得下表:
\begin{center}
    \begin{tabular}{|c|c|c|c|c|}
        \hline
        维修次数 & $0$ & $1$ & $2$ & $3$\\ \hline
        台数 & $5$ & $10$ & $20$ & $15$\\ \hline
    \end{tabular}
\end{center}
以这$50$台机器维修次数的频率代替$1$台机器维修次数发生的概率. 记$X$表示这$2$台机器超过质保期后延保的$2$年内共需维修的次数.\\
(1) 求$X$的分布列;\\
(2) 以方案一与方案二所需费用(所需延保金及维修费用之和)的期望值为决策依据, 医院选择哪种延保方案更合算?
\item (004600)已知$X$的分布列为
\begin{center}
    $\begin{pmatrix}
        -1 & 0 & 1 \\
        \dfrac 12 & \dfrac 13 & \dfrac 16      
    \end{pmatrix}$
\end{center}
两个随机变量$X$, $Y$满足$X+2Y=4$, 则$D[X]=$\blank{50}, $D[Y]=$\blank{50}.
\item (004601)五个自然数$1, 2, 3, 4, 5$按照一定的顺序排成一排.\\
(1) 求$2$和$4$不相邻的概率;\\
(2) 定义: 若两个数的和为$6$且相邻, 称这两个数为一组``友好数''. 随机变量X表示上述五个自然数组成的一个排列中``友好数''的组数, 求$X$的分布列、数学期望$E[X]$和方差$D[X]$.
