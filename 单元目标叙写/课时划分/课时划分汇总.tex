\section*{第1单元, 第1课时}
起始页码: 必修第一册P2; 终止页码: 必修第一册P4.
\begin{itemize}
\item K0101001B|D01001B|通过具体的例子理解集合的含义, 理解元素与集合的``属于''关系, 并能用符号表示.

关联题目数量: 35. 关联题目id: 000756, 000891, 000989, 000994, 000995, 000996, 000997, 002692, 002716, 002720, 002723, 002727, 004311, 004772, 004780, 004781, 004783, 004784, 004786, 004794, 004795, 007682, 007685, 007687, 010026, 020001, 020002, 020003, 020009, 020010, 020014, 020015, 020020, 020039, 030003

\item K0101002B|D01001B|理解有限集、无限集、空集的含义.

关联题目数量: 14. 关联题目id: 000019, 002708, 002714, 002725, 004770, 004784, 004800, 005824, 005837, 007692, 010028, 020001, 020004, 020009

\item K0101003B|D01001B|熟悉常用数集的符号, 能在具体的情境中认识和运用.

关联题目数量: 5. 关联题目id: 001114, 003501, 004774, 007743, 020002

\item K0101004B|D01001B|知道集合相等的定义.

关联题目数量: 25. 关联题目id: 000899, 000993, 001001, 002692, 002693, 002728, 004421, 004461, 004776, 004782, 004820, 007690, 007691, 007757, 010022, 020016, 020020, 020021, 020023, 020025, 020033, 020040, 020049, 020066, 020084

\item 本课时题目数量: 73. 关联题目id: 000019, 000756, 000891, 000899, 000989, 000993, 000994, 000995, 000996, 000997, 001001, 001114, 002692, 002693, 002708, 002714, 002716, 002720, 002723, 002725, 002727, 002728, 003501, 004311, 004421, 004461, 004770, 004772, 004774, 004776, 004780, 004781, 004782, 004783, 004784, 004786, 004794, 004795, 004800, 004820, 005824, 005837, 007682, 007685, 007687, 007690, 007691, 007692, 007743, 007757, 010022, 010026, 010028, 020001, 020002, 020003, 020004, 020009, 020010, 020014, 020015, 020016, 020020, 020021, 020023, 020025, 020033, 020039, 020040, 020049, 020066, 020084, 030003

\end{itemize}

\section*{第1单元, 第2课时}
起始页码: 必修第一册P4; 终止页码: 必修第一册P6.
\begin{itemize}
\item K0102001B|D01001B|能在具体情境中用列举法表示集合.

关联题目数量: 16. 关联题目id: 000001, 000073, 000991, 002722, 004777, 004778, 004781, 005815, 007680, 010017, 020005, 020011, 020012, 020013, 020021, 020028

\item K0102002B|D01001B|能在具体情境中用描述法表示集合.

关联题目数量: 19. 关联题目id: 000002, 000073, 000356, 000397, 002704, 004773, 004783, 007681, 007683, 010018, 010019, 020006, 020013, 020021, 020045, 020046, 020047, 020068, 030004

\item K0102003B|D01001B|会选择合适的表示集合的方式, 会正确地进行表示方式的切换.

关联题目数量: 7. 关联题目id: 000015, 000990, 000999, 002704, 004780, 007684, 020008

\item K0102004B|D01001B|会用区间表示一些实数集合.

关联题目数量: 5. 关联题目id: 000005, 002702, 004822, 020007, 020038

\item 本课时题目数量: 43. 关联题目id: 000001, 000002, 000005, 000015, 000073, 000356, 000397, 000990, 000991, 000999, 002702, 002704, 002722, 004773, 004777, 004778, 004780, 004781, 004783, 004822, 005815, 007680, 007681, 007683, 007684, 010017, 010018, 010019, 020005, 020006, 020007, 020008, 020011, 020012, 020013, 020021, 020028, 020038, 020045, 020046, 020047, 020068, 030004

\end{itemize}

\section*{第1单元, 第3课时}
起始页码: 必修第一册P6; 终止页码: 必修第一册P8.
\begin{itemize}
\item K0103001B|D01001B|理解集合之间包含的概念, 能识别给定集合的子集.

关联题目数量: 77. 关联题目id: 000004, 000011, 000012, 000014, 000015, 000016, 000037, 000910, 000964, 000996, 000997, 000998, 000999, 001004, 001005, 001008, 001052, 002692, 002693, 002701, 002705, 002706, 002707, 002715, 002718, 002724, 002730, 002788, 003695, 004059, 004227, 004414, 004461, 004499, 004662, 004766, 004785, 004786, 004790, 004791, 004792, 004796, 004800, 004819, 004832, 004881, 004884, 004962, 004963, 004964, 005815, 005825, 005837, 007686, 007703, 007750, 007756, 007757, 007760, 007761, 010020, 010027, 010029, 020017, 020018, 020027, 020030, 020032, 020034, 020037, 020038, 020049, 020063, 020064, 020066, 020066, 020084

\item K0103002B|D01001B|能用文氏图表示集合以及集合之间的包含关系.

关联题目数量: 3. 关联题目id: 000992, 002726, 004823

\item K0103003B|D01001B|能在简单的情境中, 证明集合间的子集关系.

关联题目数量: 9. 关联题目id: 001001, 001002, 001017, 002700, 004793, 004801, 010027, 020024, 020025

\item K0103004B|D01001B|理解集合的包含关系具有传递性.

关联题目数量: 2. 关联题目id: 007730, 020022

\item K0103005B|D01001B|理解真子集的概念, 能在具体的例子中证明给定集合间的真子集关系.

关联题目数量: 31. 关联题目id: 000998, 001000, 001003, 002702, 002717, 004787, 004788, 004789, 004790, 004820, 007686, 007687, 007688, 007689, 007757, 010021, 010028, 020016, 020018, 020019, 020020, 020022, 020026, 020029, 020031, 020035, 020036, 020041, 020049, 030003, 030004

\item 本课时题目数量: 113. 关联题目id: 000004, 000011, 000012, 000014, 000015, 000016, 000037, 000910, 000964, 000992, 000996, 000997, 000998, 000999, 001000, 001001, 001002, 001003, 001004, 001005, 001008, 001017, 001052, 002692, 002693, 002700, 002701, 002702, 002705, 002706, 002707, 002715, 002717, 002718, 002724, 002726, 002730, 002788, 003695, 004059, 004227, 004414, 004461, 004499, 004662, 004766, 004785, 004786, 004787, 004788, 004789, 004790, 004791, 004792, 004793, 004796, 004800, 004801, 004819, 004820, 004823, 004832, 004881, 004884, 004962, 004963, 004964, 005815, 005825, 005837, 007686, 007687, 007688, 007689, 007703, 007730, 007750, 007756, 007757, 007760, 007761, 010020, 010021, 010027, 010028, 010029, 020016, 020017, 020018, 020019, 020020, 020022, 020024, 020025, 020026, 020027, 020029, 020030, 020031, 020032, 020034, 020035, 020036, 020037, 020038, 020041, 020049, 020063, 020064, 020066, 020084, 030003, 030004

\end{itemize}

\section*{第1单元, 第4课时}
起始页码: 必修第一册P8; 终止页码: 必修第一册P11.
\begin{itemize}
\item K0104001B|D01001B|理解两个集合的交集的含义, 在具体数学情境中, 能求两个集合的交集.

关联题目数量: 165. 关联题目id: 000004, 000005, 000026, 000028, 000045, 000346, 000356, 000368, 000377, 000386, 000397, 000426, 000446, 000456, 000476, 000496, 000526, 000547, 000556, 000576, 000617, 000636, 000666, 000686, 000706, 000726, 000746, 000768, 000857, 000879, 001006, 001008, 001009, 001014, 001015, 001016, 001114, 002694, 002695, 002696, 002698, 002699, 002701, 002703, 002708, 002711, 002713, 002714, 002715, 002719, 002721, 002724, 002725, 002729, 002730, 002799, 003590, 003610, 003631, 003673, 003695, 003707, 003835, 003860, 003925, 003957, 004080, 004123, 004144, 004271, 004292, 004374, 004399, 004435, 004461, 004468, 004510, 004552, 004619, 004676, 004683, 004708, 004724, 004767, 004769, 004797, 004798, 004799, 004802, 004803, 004804, 004805, 004806, 004807, 004808, 004809, 004810, 004811, 004812, 004813, 004814, 004815, 004817, 004821, 004822, 004825, 004826, 004827, 004828, 004829, 004830, 004831, 004833, 004835, 004836, 004840, 004841, 004865, 004871, 004965, 004966, 004971, 005221, 005824, 007693, 007694, 007695, 007696, 007697, 007701, 007704, 007705, 007733, 007750, 007752, 007755, 009458, 009996, 010023, 010024, 010069, 010076, 010087, 020042, 020043, 020044, 020045, 020046, 020047, 020048, 020049, 020051, 020053, 020054, 020056, 020057, 020059, 020060, 020061, 020062, 020063, 020066, 020067, 020075, 030006

\item K0104002B|D01001B|能用文氏图反映两个集合的交集.

关联题目数量: 25. 关联题目id: 000007, 000013, 000776, 001010, 001011, 001012, 001013, 001017, 002699, 002710, 002712, 002726, 004768, 004823, 004838, 004842, 007702, 007706, 007744, 007745, 007751, 007759, 020065, 020069, 030005

\item K0104003B|D01001B|理解两个集合的并集的含义, 在具体数学情境中, 能求两个集合的并集.

关联题目数量: 74. 关联题目id: 000004, 000005, 000045, 000327, 000466, 000627, 000700, 000836, 000910, 001007, 001011, 001012, 001015, 002699, 002701, 002706, 002711, 002718, 002724, 003695, 003719, 003745, 003760, 004354, 004399, 004421, 004461, 004676, 004767, 004769, 004816, 004818, 004820, 004822, 004824, 004825, 004826, 004827, 004828, 004829, 004830, 004831, 004833, 004836, 004841, 004865, 004948, 004966, 007696, 007697, 007698, 007701, 007703, 007704, 007705, 007750, 007752, 007755, 007756, 010069, 020042, 020043, 020044, 020048, 020050, 020051, 020052, 020054, 020055, 020057, 020059, 020060, 020067, 020068

\item K0104004B|D01001B|能用文氏图反映两个集合的并集.

关联题目数量: 21. 关联题目id: 000007, 000013, 001010, 001011, 002699, 002710, 002712, 002726, 004164, 004768, 004823, 004838, 007702, 007706, 007744, 007751, 007759, 010029, 020065, 020069, 030005

\item K0104005B|D01001B|了解全集的含义.

关联题目数量: 1. 关联题目id: 020058

\item K0104006B|D01001B|理解在给定集合中一个子集的补集的含义, 在具体数学情境中, 能求给定集合中一个子集的补集.

关联题目数量: 62. 关联题目id: 000006, 000028, 000377, 000416, 000476, 000496, 000506, 000536, 000556, 000596, 000627, 000706, 000716, 000736, 000846, 002694, 002695, 002697, 002699, 002724, 002799, 003707, 003775, 004353, 004388, 004399, 004468, 004531, 004832, 004833, 004834, 004835, 004836, 004837, 004839, 004840, 004841, 004860, 004868, 004870, 007699, 007700, 007701, 007705, 007753, 007756, 010025, 010069, 020057, 020058, 020059, 020060, 020061, 020062, 020063, 020064, 020066, 020067, 020068, 020075, 020091, 020094

\item K0104007B|D01001B|能用文氏图反映一个集合的补集.

关联题目数量: 18. 关联题目id: 000776, 001010, 002699, 002703, 002710, 002726, 004768, 004832, 004838, 004842, 007702, 007706, 007744, 007745, 007759, 020065, 020069, 030005

\item 本课时题目数量: 245. 关联题目id: 000004, 000005, 000006, 000007, 000013, 000026, 000028, 000045, 000327, 000346, 000356, 000368, 000377, 000386, 000397, 000416, 000426, 000446, 000456, 000466, 000476, 000496, 000506, 000526, 000536, 000547, 000556, 000576, 000596, 000617, 000627, 000636, 000666, 000686, 000700, 000706, 000716, 000726, 000736, 000746, 000768, 000776, 000836, 000846, 000857, 000879, 000910, 001006, 001007, 001008, 001009, 001010, 001011, 001012, 001013, 001014, 001015, 001016, 001017, 001114, 002694, 002695, 002696, 002697, 002698, 002699, 002701, 002703, 002706, 002708, 002710, 002711, 002712, 002713, 002714, 002715, 002718, 002719, 002721, 002724, 002725, 002726, 002729, 002730, 002799, 003590, 003610, 003631, 003673, 003695, 003707, 003719, 003745, 003760, 003775, 003835, 003860, 003925, 003957, 004080, 004123, 004144, 004164, 004271, 004292, 004353, 004354, 004374, 004388, 004399, 004421, 004435, 004461, 004468, 004510, 004531, 004552, 004619, 004676, 004683, 004708, 004724, 004767, 004768, 004769, 004797, 004798, 004799, 004802, 004803, 004804, 004805, 004806, 004807, 004808, 004809, 004810, 004811, 004812, 004813, 004814, 004815, 004816, 004817, 004818, 004820, 004821, 004822, 004823, 004824, 004825, 004826, 004827, 004828, 004829, 004830, 004831, 004832, 004833, 004834, 004835, 004836, 004837, 004838, 004839, 004840, 004841, 004842, 004860, 004865, 004868, 004870, 004871, 004948, 004965, 004966, 004971, 005221, 005824, 007693, 007694, 007695, 007696, 007697, 007698, 007699, 007700, 007701, 007702, 007703, 007704, 007705, 007706, 007733, 007744, 007745, 007750, 007751, 007752, 007753, 007755, 007756, 007759, 009458, 009996, 010023, 010024, 010025, 010029, 010069, 010076, 010087, 020042, 020043, 020044, 020045, 020046, 020047, 020048, 020049, 020050, 020051, 020052, 020053, 020054, 020055, 020056, 020057, 020058, 020059, 020060, 020061, 020062, 020063, 020064, 020065, 020066, 020067, 020068, 020069, 020075, 020091, 020094, 030005, 030006

\end{itemize}

\section*{第1单元, 第5课时}
起始页码: 必修第一册P13; 终止页码: 必修第一册P15.
\begin{itemize}
\item K0105001B|D01002B|结合集合之间的包含关系, 理解推出关系的含义以及推出关系的传递性.

关联题目数量: 25. 关联题目id: 000009, 000939, 000942, 002746, 002748, 003708, 003758, 004282, 004365, 004378, 004848, 004872, 005238, 007709, 007720, 007723, 007736, 007737, 007738, 007739, 007740, 010039, 020074, 020091, 020094

\item K0105002B|D01002B|理解命题的定义, 能在熟悉的情境中运用推出关系判断条件命题的真假.

关联题目数量: 44. 关联题目id: 000019, 000326, 002732, 002733, 002746, 004323, 004365, 004378, 004843, 004847, 004848, 004865, 007707, 007708, 007711, 007713, 007716, 007717, 007718, 007720, 007721, 007722, 007738, 007739, 007740, 007741, 007746, 007747, 007748, 007749, 007754, 010030, 010031, 010032, 010034, 010036, 010038, 010039, 020070, 020071, 020072, 020075, 020076, 020079

\item 本课时题目数量: 60. 关联题目id: 000009, 000019, 000326, 000939, 000942, 002732, 002733, 002746, 002748, 003708, 003758, 004282, 004323, 004365, 004378, 004843, 004847, 004848, 004865, 004872, 005238, 007707, 007708, 007709, 007711, 007713, 007716, 007717, 007718, 007720, 007721, 007722, 007723, 007736, 007737, 007738, 007739, 007740, 007741, 007746, 007747, 007748, 007749, 007754, 010030, 010031, 010032, 010034, 010036, 010038, 010039, 020070, 020071, 020072, 020074, 020075, 020076, 020079, 020091, 020094

\end{itemize}

\section*{第1单元, 第6课时}
起始页码: 必修第一册P15; 终止页码: 必修第一册P16.
\begin{itemize}
\item K0106001B|D01002B|知道充分条件、必要条件的定义, 充要条件的含义.

关联题目数量: 44. 关联题目id: 000008, 000009, 000012, 000436, 000648, 000988, 002737, 002745, 002746, 002748, 002751, 004261, 004365, 004378, 004871, 004872, 004874, 004894, 005220, 007731, 007734, 007736, 007737, 007738, 007739, 007740, 007741, 007742, 007746, 007747, 007748, 007749, 010037, 010038, 010039, 020076, 020077, 020078, 020079, 020080, 020083, 020084, 020085, 030001

\item K0106002B|D01002B|通过对典型数学命题的梳理与学习, 理解性质定理与必要条件的关系、判定定理与充分条件的关系, 以及数学定义与充要条件的关系.

关联题目数量: 1. 关联题目id: 007733

\item K0106003B|D01002B|能基于推出关系有理有据地判定熟悉的陈述句之间的必要条件关系、充分条件关系和充要条件关系.

关联题目数量: 74. 关联题目id: 000003, 000431, 000932, 000939, 000942, 000984, 000986, 000987, 002709, 002735, 002736, 002739, 002740, 002741, 002743, 002744, 002752, 003665, 003729, 003818, 003861, 003905, 003980, 004092, 004113, 004136, 004403, 004419, 004458, 004502, 004522, 004564, 004631, 004673, 004715, 004736, 004867, 004868, 004869, 004870, 004873, 004877, 004879, 004880, 004882, 004883, 004885, 004886, 004887, 004888, 004889, 004890, 004891, 004892, 004893, 004894, 004895, 004896, 004904, 007719, 007724, 007725, 007726, 007727, 007728, 007729, 007730, 007732, 007735, 007758, 010033, 010035, 010050, 020081

\item 本课时题目数量: 118. 关联题目id: 000003, 000008, 000009, 000012, 000431, 000436, 000648, 000932, 000939, 000942, 000984, 000986, 000987, 000988, 002709, 002735, 002736, 002737, 002739, 002740, 002741, 002743, 002744, 002745, 002746, 002748, 002751, 002752, 003665, 003729, 003818, 003861, 003905, 003980, 004092, 004113, 004136, 004261, 004365, 004378, 004403, 004419, 004458, 004502, 004522, 004564, 004631, 004673, 004715, 004736, 004867, 004868, 004869, 004870, 004871, 004872, 004873, 004874, 004877, 004879, 004880, 004882, 004883, 004885, 004886, 004887, 004888, 004889, 004890, 004891, 004892, 004893, 004894, 004895, 004896, 004904, 005220, 007719, 007724, 007725, 007726, 007727, 007728, 007729, 007730, 007731, 007732, 007733, 007734, 007735, 007736, 007737, 007738, 007739, 007740, 007741, 007742, 007746, 007747, 007748, 007749, 007758, 010033, 010035, 010037, 010038, 010039, 010050, 020076, 020077, 020078, 020079, 020080, 020081, 020083, 020084, 020085, 030001

\end{itemize}

\section*{第1单元, 第7课时}
起始页码: 必修第一册P17; 终止页码: 必修第一册P18.
\begin{itemize}
\item K0107001B|D01002B|知道一些常用的否定形式, 能正确使用存在量词对全称量词命题进行否定, 能正确使用全称量词对存在量词命题进行否定.

关联题目数量: 12. 关联题目id: 000977, 000979, 000980, 000981, 004845, 004854, 004878, 020073, 020082, 020088, 020089, 020092

\item K0107002B|D01002B|能对比较熟悉的陈述句进行否定.

关联题目数量: 15. 关联题目id: 000976, 000978, 002731, 002747, 004860, 004861, 004875, 004876, 020087, 020088, 020091, 020093, 020094, 020095, 020096

\item K0107003B|D01002B|了解反证法的思想以及表达方式, 能正确使用反证法证明一些简单的数学命题.

关联题目数量: 26. 关联题目id: 000010, 000017, 000018, 000019, 000074, 000924, 000983, 001139, 002734, 004864, 004875, 004876, 004980, 004985, 005261, 007720, 007732, 007735, 020081, 020090, 020091, 020092, 020093, 020094, 020095, 020096

\item 本课时题目数量: 44. 关联题目id: 000010, 000017, 000018, 000019, 000074, 000924, 000976, 000977, 000978, 000979, 000980, 000981, 000983, 001139, 002731, 002734, 002747, 004845, 004854, 004860, 004861, 004864, 004875, 004876, 004878, 004980, 004985, 005261, 007720, 007732, 007735, 020073, 020081, 020082, 020087, 020088, 020089, 020090, 020091, 020092, 020093, 020094, 020095, 020096

\end{itemize}

\section*{第1单元, 第8课时}
起始页码: 必修第一册P24; 终止页码: 必修第一册P26.
\begin{itemize}
\item K0108001B|D01004B|知道等式的加法性质、乘法性质和传递性.

关联题目数量: 2. 关联题目id: 007726, 007727

\item K0108002B|D01004B|知道方程、方程的解、方程的解集的定义.

关联题目数量: 15. 关联题目id: 001040, 001041, 001042, 001043, 001044, 001045, 001046, 001047, 001048, 001049, 001050, 001051, 004422, 004852, 004921

\item K0108003B|D01004B|会用集合表示一元一次方程、二元一次方程组的解集.

关联题目数量: 24. 关联题目id: 001049, 001050, 001052, 001053, 001054, 001055, 001056, 001057, 001058, 001059, 001060, 001061, 001070, 003777, 004557, 004771, 004775, 004860, 004959, 007807, 009446, 010040, 010041, 010057

\item 本课时题目数量: 39. 关联题目id: 001040, 001041, 001042, 001043, 001044, 001045, 001046, 001047, 001048, 001049, 001050, 001051, 001052, 001053, 001054, 001055, 001056, 001057, 001058, 001059, 001060, 001061, 001070, 003777, 004422, 004557, 004771, 004775, 004852, 004860, 004921, 004959, 007726, 007727, 007807, 009446, 010040, 010041, 010057

\end{itemize}

\section*{第1单元, 第9课时}
起始页码: 必修第一册P26; 终止页码: 必修第一册P28.
\begin{itemize}
\item K0109001B|D01004B|会用集合表示一元二次方程的解集.

关联题目数量: 14. 关联题目id: 000016, 001069, 001070, 001071, 001072, 002739, 003897, 004779, 004868, 004870, 007750, 010042, 010058, 020085

\item K0109002B|D01004B|知道恒等式成立的意义, 会用赋值法处理恒等式.

关联题目数量: 3. 关联题目id: 010043, 010059, 020004

\item K0109003B|D01004B|知道并会用因式分解法证明一元二次方程根与系数的关系.

关联题目数量: 15. 关联题目id: 001062, 001063, 001064, 001065, 001066, 001067, 001068, 001069, 002741, 004949, 004950, 010060, 010061, 010062, 010064

\item K0109004B|D01004B|在给定二次方程的前提下, 能计算用根表示的简单二元对称多项式的值.

关联题目数量: 18. 关联题目id: 000020, 000033, 000034, 000059, 000082, 001069, 001070, 001071, 001073, 001074, 001075, 001076, 001140, 010044, 010045, 010061, 010062, 020085

\item 本课时题目数量: 43. 关联题目id: 000016, 000020, 000033, 000034, 000059, 000082, 001062, 001063, 001064, 001065, 001066, 001067, 001068, 001069, 001070, 001071, 001072, 001073, 001074, 001075, 001076, 001140, 002739, 002741, 003897, 004779, 004868, 004870, 004949, 004950, 007750, 010042, 010043, 010044, 010045, 010058, 010059, 010060, 010061, 010062, 010064, 020004, 020085

\end{itemize}

\section*{第1单元, 第10课时}
起始页码: 必修第一册P28; 终止页码: 必修第一册P30.
\begin{itemize}
\item K0110001B|D01003B|理解不等式的含义, 通过等式的性质类比并证明不等式的性质(传递性、加法性质、乘法性质).

关联题目数量: 8. 关联题目id: 001093, 002764, 005007, 007749, 007764, 007766, 007768, 007818

\item K0110002B|D01003B|经历不等式的移项法则、不等式的同向可加性的证明过程.

关联题目数量: 2. 关联题目id: 001139, 001144

\item 本课时题目数量: 10. 关联题目id: 001093, 001139, 001144, 002764, 005007, 007749, 007764, 007766, 007768, 007818

\end{itemize}

\section*{第1单元, 第11课时}
起始页码: 必修第一册P30; 终止页码: 必修第一册P31.
\begin{itemize}
\item K0111001B|D01003B|经历不等式的同正同向的可乘性、乘方性质、开方性质(方根在第三章出现, 同一个意思, 不同表达形式)的证明过程.

关联题目数量: 10. 关联题目id: 001085, 001095, 001143, 002750, 004985, 005007, 007766, 007767, 007770, 030002

\item K0111002B|D01003B|掌握常用不等式$a^2+b^2 \ge 2ab$的证明过程及等号成立的条件.

关联题目数量: 16. 关联题目id: 001120, 001121, 001122, 001138, 001144, 002760, 004115, 004451, 004737, 004976, 004977, 005008, 005009, 007818, 009452, 010089

\item K0111003B|D01003B|会用不等式的性质、作差法证明一些简单的不等式.

关联题目数量: 111. 关联题目id: 000021, 000022, 000043, 000326, 000975, 000988, 001085, 001086, 001087, 001088, 001089, 001090, 001091, 001092, 001120, 001134, 001135, 001138, 001139, 001141, 001142, 001148, 001149, 001150, 002739, 002750, 002751, 002752, 002756, 002758, 002761, 002763, 002769, 002789, 002812, 003742, 003800, 003833, 003923, 003965, 004400, 004438, 004564, 004897, 004898, 004899, 004900, 004901, 004903, 004904, 004905, 004906, 004908, 004909, 004910, 004911, 004916, 004927, 004972, 004974, 004978, 004982, 004983, 004984, 004986, 004987, 004988, 004989, 004990, 004997, 004998, 004999, 005007, 005015, 005018, 005019, 005020, 005021, 005022, 005023, 005268, 005270, 007732, 007765, 007769, 007771, 007772, 007773, 007779, 007781, 007828, 007829, 007832, 009451, 010046, 010047, 010048, 010049, 010050, 010051, 010052, 010053, 010054, 010055, 010056, 010063, 010065, 010097, 010098, 020083, 030002

\item 本课时题目数量: 131. 关联题目id: 000021, 000022, 000043, 000326, 000975, 000988, 001085, 001086, 001087, 001088, 001089, 001090, 001091, 001092, 001095, 001120, 001121, 001122, 001134, 001135, 001138, 001139, 001141, 001142, 001143, 001144, 001148, 001149, 001150, 002739, 002750, 002751, 002752, 002756, 002758, 002760, 002761, 002763, 002769, 002789, 002812, 003742, 003800, 003833, 003923, 003965, 004115, 004400, 004438, 004451, 004564, 004737, 004897, 004898, 004899, 004900, 004901, 004903, 004904, 004905, 004906, 004908, 004909, 004910, 004911, 004916, 004927, 004972, 004974, 004976, 004977, 004978, 004982, 004983, 004984, 004985, 004986, 004987, 004988, 004989, 004990, 004997, 004998, 004999, 005007, 005008, 005009, 005015, 005018, 005019, 005020, 005021, 005022, 005023, 005268, 005270, 007732, 007765, 007766, 007767, 007769, 007770, 007771, 007772, 007773, 007779, 007781, 007818, 007828, 007829, 007832, 009451, 009452, 010046, 010047, 010048, 010049, 010050, 010051, 010052, 010053, 010054, 010055, 010056, 010063, 010065, 010089, 010097, 010098, 020083, 030002

\end{itemize}

\section*{第1单元, 第12课时}
起始页码: 必修第一册P34; 终止页码: 必修第一册P35.
\begin{itemize}
\item K0112001B|D01004B|会求解(含有参数的)一元一次不等式(组), 并能用集合表示一元一次不等式(组)的解集.

关联题目数量: 26. 关联题目id: 000023, 000035, 000879, 001100, 001101, 001102, 001103, 002741, 002777, 003777, 004931, 004955, 007762, 007763, 007774, 007775, 007776, 007777, 007778, 007780, 009453, 010066, 010067, 010078, 010081, 010087

\item K0112002B|D01004B|经历从实际情境中抽象出一元二次不等式的过程, 了解一元二次不等式的现实意义.

关联题目数量: 2. 关联题目id: 007802, 010080

\item 本课时题目数量: 28. 关联题目id: 000023, 000035, 000879, 001100, 001101, 001102, 001103, 002741, 002777, 003777, 004931, 004955, 007762, 007763, 007774, 007775, 007776, 007777, 007778, 007780, 007802, 009453, 010066, 010067, 010078, 010080, 010081, 010087

\end{itemize}

\section*{第1单元, 第13课时}
起始页码: 必修第一册P35; 终止页码: 必修第一册P39.
\begin{itemize}
\item K0113001B|D01004B|会用因式分解后两部分符号的讨论求解一元二次不等式.

关联题目数量: 19. 关联题目id: 000047, 000526, 000586, 000636, 000666, 001104, 001106, 002779, 002792, 004080, 004510, 005220, 007782, 007784, 007785, 007786, 009454, 010068, 010082

\item K0113002B|D01004B|建立一元二次不等式与相应的一元二次方程的联系, 通过对方程判别式分类讨论的方式求解一元二次不等式.

关联题目数量: 7. 关联题目id: 000526, 000547, 001104, 003754, 004358, 004957, 010069

\item 本课时题目数量: 24. 关联题目id: 000047, 000526, 000547, 000586, 000636, 000666, 001104, 001106, 002779, 002792, 003754, 004080, 004358, 004510, 004957, 005220, 007782, 007784, 007785, 007786, 009454, 010068, 010069, 010082

\end{itemize}

\section*{第1单元, 第14课时}
起始页码: 必修第一册P130; 终止页码: 必修第一册P131.
\begin{itemize}
\item K0114001B|D01004B|掌握结合一元二次函数的图像求解一元二次不等式的方法.

关联题目数量: 64. 关联题目id: 000024, 000026, 000446, 000526, 000547, 000596, 000736, 000768, 000857, 001075, 001104, 001105, 001107, 002773, 002781, 003963, 004271, 004292, 004435, 004630, 004636, 004912, 004922, 004923, 004924, 004925, 004926, 004928, 004930, 004931, 004932, 004933, 004934, 004935, 004936, 004947, 004948, 004956, 004957, 004963, 004964, 004965, 004966, 005221, 005265, 005267, 007782, 007785, 007786, 007801, 007802, 009455, 009456, 009458, 009460, 009463, 010070, 010071, 010072, 010073, 010074, 010076, 010083, 010087

\item 本课时题目数量: 64. 关联题目id: 000024, 000026, 000446, 000526, 000547, 000596, 000736, 000768, 000857, 001075, 001104, 001105, 001107, 002773, 002781, 003963, 004271, 004292, 004435, 004630, 004636, 004912, 004922, 004923, 004924, 004925, 004926, 004928, 004930, 004931, 004932, 004933, 004934, 004935, 004936, 004947, 004948, 004956, 004957, 004963, 004964, 004965, 004966, 005221, 005265, 005267, 007782, 007785, 007786, 007801, 007802, 009455, 009456, 009458, 009460, 009463, 010070, 010071, 010072, 010073, 010074, 010076, 010083, 010087

\end{itemize}

\section*{第1单元, 第15课时}
起始页码: 必修第一册P39; 终止页码: 必修第一册P40.
\begin{itemize}
\item K0115001B|D01004B|能通过对判别式讨论的方法解决含参一元二次(可能是一元一次, 可能不含未知数)不等式的恒成立问题.

关联题目数量: 24. 关联题目id: 000084, 001108, 002749, 002770, 002776, 002780, 002782, 002785, 003714, 003741, 003774, 004636, 004861, 004915, 004941, 004942, 004943, 004967, 004968, 004969, 004971, 007800, 007991, 010088

\item K0115002B|D01004B|在已知解集的情形下, 会求解含参一元二次不等式系数所满足的关系或者系数值.

关联题目数量: 35. 关联题目id: 000025, 000045, 001115, 002772, 002775, 002778, 002784, 002786, 002788, 004382, 004865, 004866, 004867, 004869, 004872, 004913, 004914, 004937, 004939, 004940, 004944, 004945, 004946, 004951, 004952, 004953, 004954, 004956, 004958, 007801, 009457, 009459, 009460, 010075, 010083

\item 本课时题目数量: 59. 关联题目id: 000025, 000045, 000084, 001108, 001115, 002749, 002770, 002772, 002775, 002776, 002778, 002780, 002782, 002784, 002785, 002786, 002788, 003714, 003741, 003774, 004382, 004636, 004861, 004865, 004866, 004867, 004869, 004872, 004913, 004914, 004915, 004937, 004939, 004940, 004941, 004942, 004943, 004944, 004945, 004946, 004951, 004952, 004953, 004954, 004956, 004958, 004967, 004968, 004969, 004971, 007800, 007801, 007991, 009457, 009459, 009460, 010075, 010083, 010088

\end{itemize}

\section*{第1单元, 第16课时}
起始页码: 必修第一册P40; 终止页码: 必修第一册P42.
\begin{itemize}
\item K0116001B|D01004B|结合分类讨论, 会用不等式(组)解一些简单的分式不等式.

关联题目数量: 25. 关联题目id: 000027, 000047, 000468, 000496, 000507, 001110, 001111, 001112, 002783, 002792, 002797, 002807, 003665, 004125, 004409, 004469, 004969, 005150, 005264, 007803, 007811, 007835, 009461, 010077, 010078

\item K0116002B|D01004B|会用转化为整式不等式(组)解一些简单的分式不等式.

关联题目数量: 49. 关联题目id: 000027, 000036, 000037, 000044, 000368, 000378, 000407, 000459, 000468, 000496, 000507, 000726, 000797, 000846, 001093, 001112, 002790, 002791, 002800, 002801, 003675, 003707, 004125, 004249, 004354, 004708, 004868, 004870, 004926, 004929, 004935, 004938, 004970, 005223, 005226, 005235, 005264, 007804, 007805, 007806, 007807, 007809, 007813, 007815, 009461, 010077, 010080, 010084, 010085

\item 本课时题目数量: 65. 关联题目id: 000027, 000036, 000037, 000044, 000047, 000368, 000378, 000407, 000459, 000468, 000496, 000507, 000726, 000797, 000846, 001093, 001110, 001111, 001112, 002783, 002790, 002791, 002792, 002797, 002800, 002801, 002807, 003665, 003675, 003707, 004125, 004249, 004354, 004409, 004469, 004708, 004868, 004870, 004926, 004929, 004935, 004938, 004969, 004970, 005150, 005223, 005226, 005235, 005264, 007803, 007804, 007805, 007806, 007807, 007809, 007811, 007813, 007815, 007835, 009461, 010077, 010078, 010080, 010084, 010085

\end{itemize}

\section*{第1单元, 第17课时}
起始页码: 必修第一册P42; 终止页码: 必修第一册P43.
\begin{itemize}
\item K0117001B|D01004B|会用绝对值的几何意义求解一些基本的含绝对值的不等式.

关联题目数量: 37. 关联题目id: 000036, 000037, 000044, 000446, 000496, 000540, 000617, 000757, 000768, 000816, 000857, 000879, 001093, 002748, 002793, 002799, 003835, 004144, 004292, 004312, 004354, 004414, 004554, 004865, 004917, 004918, 004919, 004920, 005220, 005221, 005238, 007810, 007811, 007836, 009462, 010079, 010086

\item K0117002B|D01004B|会用分类讨论的思想求解一些基本的含绝对值的不等式.

关联题目数量: 41. 关联题目id: 000029, 000389, 000496, 000540, 000617, 001094, 001099, 001116, 001117, 001118, 001119, 002794, 002797, 002798, 002800, 003714, 003774, 004533, 004929, 005223, 005224, 005226, 005227, 005228, 005229, 005230, 005233, 005234, 005235, 005236, 005237, 005270, 007808, 007809, 007812, 007814, 007815, 007816, 009462, 010079, 010086

\item 本课时题目数量: 72. 关联题目id: 000029, 000036, 000037, 000044, 000389, 000446, 000496, 000540, 000617, 000757, 000768, 000816, 000857, 000879, 001093, 001094, 001099, 001116, 001117, 001118, 001119, 002748, 002793, 002794, 002797, 002798, 002799, 002800, 003714, 003774, 003835, 004144, 004292, 004312, 004354, 004414, 004533, 004554, 004865, 004917, 004918, 004919, 004920, 004929, 005220, 005221, 005223, 005224, 005226, 005227, 005228, 005229, 005230, 005233, 005234, 005235, 005236, 005237, 005238, 005270, 007808, 007809, 007810, 007811, 007812, 007814, 007815, 007816, 007836, 009462, 010079, 010086

\end{itemize}

\section*{第1单元, 第18课时}
起始页码: 必修第一册P46; 终止页码: 必修第一册P48.
\begin{itemize}
\item K0118001B|D01003B|知道算术平均值和几何平均值的概念.

关联题目数量: 1. 关联题目id: 000031

\item K0118002B|D01003B|经历平均值不等式的证明过程, 理解取等号的条件.

关联题目数量: 8. 关联题目id: 000031, 001138, 002759, 002767, 007817, 009464, 009997, 010091

\item K0118003B|D01003B|能运用平均值不等式比较大小、证明一些简单的不等式.

关联题目数量: 67. 关联题目id: 000030, 000038, 000040, 000046, 001120, 001121, 001122, 001123, 001124, 001137, 001138, 001143, 001145, 001146, 002758, 002769, 002812, 003604, 003622, 003716, 004384, 004972, 004987, 004988, 004990, 005000, 005001, 005002, 005003, 005004, 005005, 005008, 005009, 005015, 005017, 005018, 005019, 005020, 005021, 005022, 005023, 005024, 005025, 005026, 005027, 005028, 005029, 005250, 005251, 007817, 007819, 007821, 007825, 007827, 007830, 007831, 007832, 007833, 007834, 009464, 009997, 010090, 010091, 010092, 010097, 010099, 010100

\item 本课时题目数量: 70. 关联题目id: 000030, 000031, 000038, 000040, 000046, 001120, 001121, 001122, 001123, 001124, 001137, 001138, 001143, 001145, 001146, 002758, 002759, 002767, 002769, 002812, 003604, 003622, 003716, 004384, 004972, 004987, 004988, 004990, 005000, 005001, 005002, 005003, 005004, 005005, 005008, 005009, 005015, 005017, 005018, 005019, 005020, 005021, 005022, 005023, 005024, 005025, 005026, 005027, 005028, 005029, 005250, 005251, 007817, 007819, 007821, 007825, 007827, 007830, 007831, 007832, 007833, 007834, 009464, 009997, 010090, 010091, 010092, 010097, 010099, 010100

\end{itemize}

\section*{第1单元, 第19课时}
起始页码: 必修第一册P48; 终止页码: 必修第一册P49.
\begin{itemize}
\item K0119001B|D01003B|会运用平均值不等式求解较简单的最大值和最小值问题.

关联题目数量: 46. 关联题目id: 000039, 000041, 000056, 000371, 000433, 000924, 000939, 001125, 001126, 001127, 001128, 001129, 001130, 001131, 001132, 001133, 001151, 002753, 002754, 002755, 002760, 002765, 002766, 002768, 003637, 003741, 003797, 003810, 003900, 003912, 004360, 004991, 004992, 004993, 004994, 004995, 004996, 005001, 005004, 005010, 005011, 005012, 007820, 007823, 007824, 010093

\item K0119002B|D01003B|会运用平均值不等式解决一些现实情境中的最大值和最小值问题.

关联题目数量: 11. 关联题目id: 001130, 001131, 001132, 002757, 002774, 003863, 007822, 007826, 009465, 009466, 010094

\item 本课时题目数量: 54. 关联题目id: 000039, 000041, 000056, 000371, 000433, 000924, 000939, 001125, 001126, 001127, 001128, 001129, 001130, 001131, 001132, 001133, 001151, 002753, 002754, 002755, 002757, 002760, 002765, 002766, 002768, 002774, 003637, 003741, 003797, 003810, 003863, 003900, 003912, 004360, 004991, 004992, 004993, 004994, 004995, 004996, 005001, 005004, 005010, 005011, 005012, 007820, 007822, 007823, 007824, 007826, 009465, 009466, 010093, 010094

\end{itemize}

\section*{第1单元, 第20课时}
起始页码: 必修第一册P50; 终止页码: 必修第一册P51.
\begin{itemize}
\item K0120001B|D01003B|经历三角不等式的证明过程, 理解取等号的条件.

关联题目数量: 6. 关联题目id: 000032, 005222, 005225, 010096, 010104, 020086

\item K0120002B|D01003B|会运用三角不等式证明一些简单的不等式.

关联题目数量: 14. 关联题目id: 000042, 001096, 001097, 001098, 001113, 004737, 009467, 009468, 010095, 010096, 010101, 010102, 010103, 010104

\item K0120003B|D01003B|会运用三角不等式求解一些简单的最大值或最小值问题.

关联题目数量: 3. 关联题目id: 003774, 003948, 005239

\item 本课时题目数量: 21. 关联题目id: 000032, 000042, 001096, 001097, 001098, 001113, 003774, 003948, 004737, 005222, 005225, 005239, 009467, 009468, 010095, 010096, 010101, 010102, 010103, 010104, 020086

\end{itemize}

\section*{第2单元, 第1课时}
起始页码: 必修第一册P60; 终止页码: 必修第一册P62.
\begin{itemize}
\item K0201001B|D02001B|理解零次幂与负整数幂的定义及运算性质.

关联题目数量: 0. 关联题目id: 

\item K0201002B|D02001B|理解根式及其相关的概念.

关联题目数量: 1. 关联题目id: 000047

\item K0201003B|D02001B|会根据定义求实数的$n$次方根.

关联题目数量: 1. 关联题目id: 000048

\item K0201004B|D02001B|理解底数为正实数$a$的有理数指数幂的定义$a^{m/n}=(a^{m})^{1/n}$,经历等价定义$a^{m/n}= (a^{1/n})^{m}$的推导过程.

关联题目数量: 0. 关联题目id: 

\item 本课时题目数量: 2. 关联题目id: 000047, 000048

\end{itemize}

\section*{第2单元, 第2课时}
起始页码: 必修第一册P62; 终止页码: 必修第一册P63.
\begin{itemize}
\item K0202001B|D02001B|经历在个别情形下验证底数为正实数的有理数指数幂的三条运算性质的过程.

关联题目数量: 0. 关联题目id: 

\item K0202002B|D02001B|会运用底数为正实数的有理数指数幂的定义及运算性质进行幂与根式的互化以及解决相关的化简、计算等问题.

关联题目数量: 6. 关联题目id: 000048, 000049, 000050, 001286, 001292, 010110

\item K0202003B|D02001B|理解底数为负实数的有理数指数幂的定义, 进而理解底数为实数的有理数指数幂的定义.

关联题目数量: 0. 关联题目id: 

\item 本课时题目数量: 6. 关联题目id: 000048, 000049, 000050, 001286, 001292, 010110

\end{itemize}

\section*{第2单元, 第3课时}
起始页码: 必修第一册P63; 终止页码: 必修第一册P64.
\begin{itemize}
\item K0203001B|D02001B|知道底数为正实数的无理数指数幂的定义.

关联题目数量: 0. 关联题目id: 

\item K0203002B|D02001B|熟记底数为正实数的实数指数幂的三条运算性质.

关联题目数量: 1. 关联题目id: 001287

\item K0203003B|D02001B|经历有理数指数幂的基本不等式: ``当实数$a>1$, 有理数$s>0$时, 不等式$a^s>1$成立''的推导过程.

关联题目数量: 0. 关联题目id: 

\item K0203004B|D02001B|知道幂的基本不等式: ``当$a>1$, $s>0$时, $a^s>1$''.

关联题目数量: 2. 关联题目id: 000058, 005006

\item K0203005B|D02001B|会应用底数为正实数的实数指数幂的定义、运算性质以及幂的基本不等式, 解决底数为正实数的实数指数幂的较复杂的表达式的化简、不等式的证明等问题.

关联题目数量: 7. 关联题目id: 000051, 000057, 000058, 001287, 003662, 005621, 010114

\item 本课时题目数量: 8. 关联题目id: 000051, 000057, 000058, 001287, 003662, 005006, 005621, 010114

\end{itemize}

\section*{第2单元, 第4课时}
起始页码: 必修第一册P66; 终止页码: 必修第一册P68.
\begin{itemize}
\item K0204001B|D02001B|理解对数的定义.

关联题目数量: 1. 关联题目id: 001296

\item K0204002B|D02001B|会理解、熟记并应用一些常用的对数等式: ``$a^{\log_aN}=N$, $\log_a1=0$, $\log_aa=1$''.

关联题目数量: 0. 关联题目id: 

\item K0204003B|D02001B|知道常用对数、常数$e$以及自然对数的含义.

关联题目数量: 1. 关联题目id: 000053

\item K0204004B|D02001B|会进行指数式与对数式的互化, 以及对数式的化简.

关联题目数量: 9. 关联题目id: 000048, 000053, 000054, 000055, 000074, 000311, 001353, 005610, 005650

\item 本课时题目数量: 10. 关联题目id: 000048, 000053, 000054, 000055, 000074, 000311, 001296, 001353, 005610, 005650

\end{itemize}

\section*{第2单元, 第5课时}
起始页码: 必修第一册P68; 终止页码: 必修第一册P70.
\begin{itemize}
\item K0205001B|D02001B|经历推导对数运算性质$1$: ``当$M>0$,$N>0$时, $\log_a(MN)=\log_aM+\log_aN$'';性质$2$: ``当$M>0$,$N>0$时, $\log_a(M/N)=\log_aM-\log_aN$'';性质$3$: ``当$N>0$时, 对任何给定的实数$c$, $\log_a(N^{c})=c\log_aN$''的过程, 并熟记这三条运算性质.

关联题目数量: 0. 关联题目id: 

\item K0205002B|D02001B|会运用对数的定义以及运算性质解决简单的求值、化简以及生活实际问题.

关联题目数量: 12. 关联题目id: 000049, 000052, 000054, 000056, 001296, 001300, 001305, 001307, 002758, 003828, 009482, 010125

\item 本课时题目数量: 12. 关联题目id: 000049, 000052, 000054, 000056, 001296, 001300, 001305, 001307, 002758, 003828, 009482, 010125

\end{itemize}

\section*{第2单元, 第6课时}
起始页码: 必修第一册P71; 终止页码: 必修第一册P72.
\begin{itemize}
\item K0206001B|D02001B|经历推导对数换底公式的过程.

关联题目数量: 1. 关联题目id: 001308

\item K0206002B|D02001B|会运用对数的运算性质以及换底公式解决较复杂的求值、化简以及证明等相关问题.

关联题目数量: 17. 关联题目id: 000048, 000055, 000059, 000060, 001307, 001309, 001312, 001314, 001316, 001317, 002963, 005013, 005014, 005016, 005123, 005650, 005678

\item K0206003B|D02001B|会推导并会运用例7的结论: ``当$a>0$, $a\neq1$, 且$N>0$, $m\neq0$时, $\log_a^{m}N^{n}=n/m\log_aN$''.

关联题目数量: 0. 关联题目id: 

\item 本课时题目数量: 18. 关联题目id: 000048, 000055, 000059, 000060, 001307, 001308, 001309, 001312, 001314, 001316, 001317, 002963, 005013, 005014, 005016, 005123, 005650, 005678

\end{itemize}

\section*{第2单元, 第7课时}
起始页码: 必修第一册P80; 终止页码: 必修第一册P82.
\begin{itemize}
\item K0207001B|D02002B|理解幂函数的定义(包含幂函数定义域的概念).

关联题目数量: 3. 关联题目id: 000061, 000065, 001340

\item K0207002B|D02002B|会根据具体的幂指数$a$求解幂函数$y=x^{a}$的定义域.

关联题目数量: 2. 关联题目id: 000063, 001340

\item K0207003B|D02002B|会根据函数定义域, 利用计算器合理采点, 并能通过描点法作出幂函数$y=x^{1/2}$,$y=x^{3}$,$y=x^{-2/3}$的大致图像.

关联题目数量: 3. 关联题目id: 000072, 002909, 002925

\item K0207004B|D02002B|会用图像上任意一点关于原点(或关于$y$轴)的对称点仍落在图像上证明函数的图像关于原点(或$y$轴)对称.

关联题目数量: 6. 关联题目id: 000069, 002911, 002918, 002925, 005463, 010137

\item 本课时题目数量: 12. 关联题目id: 000061, 000063, 000065, 000069, 000072, 001340, 002909, 002911, 002918, 002925, 005463, 010137

\end{itemize}

\section*{第2单元, 第8课时}
起始页码: 必修第一册P83; 终止页码: 必修第一册P85.
\begin{itemize}
\item K0208001B|D02002B|会用不等式的常用性质证明当$x>0$时, 幂函数的函数值总大于$0$.

关联题目数量: 1. 关联题目id: 010137

\item K0208002B|D02002B|会经历作图猜想证明具体的幂函数图像在第一象限的单调性.

关联题目数量: 4. 关联题目id: 002911, 002914, 002918, 010137

\item K0208003B|D02002B|知道幂函数的图像过定点$(1,1)$.

关联题目数量: 2. 关联题目id: 005463, 010137

\item K0208004B|D02002B|会用幂函数的单调性判断两个幂的大小.

关联题目数量: 7. 关联题目id: 000061, 000064, 000066, 000070, 002795, 005464, 005568

\item K0208005B|D02002B|理解函数图像的平移与解析式的关系, 并会以此为依据作出分式线性函数的大致图像.

关联题目数量: 2. 关联题目id: 009488, 009490

\item 本课时题目数量: 14. 关联题目id: 000061, 000064, 000066, 000070, 002795, 002911, 002914, 002918, 005463, 005464, 005568, 009488, 009490, 010137

\end{itemize}

\section*{第2单元, 第9课时}
起始页码: 必修第一册P86; 终止页码: 必修第一册P87.
\begin{itemize}
\item K0209001B|D02002B|理解指数函数的定义(包含指数函数定义域为$\mathbf{R}$).

关联题目数量: 1. 关联题目id: 000061

\item K0209002B|D02002B|会求解有关指数型函数的定义域.

关联题目数量: 1. 关联题目id: 000063

\item K0209003B|D02002B|会根据函数定义域, 利用计算器合理采点, 并能通过描点法作出指数函数$y=2^{x}$, $y=3^{x}$, $y=(1/2)^{x}$的大致图像.

关联题目数量: 2. 关联题目id: 003815, 005569

\item 本课时题目数量: 4. 关联题目id: 000061, 000063, 003815, 005569

\end{itemize}

\section*{第2单元, 第10课时}
起始页码: 必修第一册P87; 终止页码: 必修第一册P90.
\begin{itemize}
\item K0210001B|D02002B|会结合图像, 了解指数函数函数值恒大于$0$.

关联题目数量: 0. 关联题目id: 

\item K0210002B|D02002B|知道指数函数图像过定点$(0,1)$.

关联题目数量: 3. 关联题目id: 000061, 000069, 003815

\item K0210003B|D02002B|会证明指数函数$y=a^{x}$与$y=(1/a)^{x}$($a>0$且$a\neq1$)的图像关于$y$轴对称.

关联题目数量: 0. 关联题目id: 

\item K0210004B|D02002B|会利用幂的基本不等式证明指数函数的单调性.

关联题目数量: 0. 关联题目id: 

\item K0210005B|D02002B|会作出指数函数的大致图像, 能根据其图像特征叙述其函数性质.

关联题目数量: 4. 关联题目id: 000070, 000072, 000073, 005569

\item K0210006B|D02002B|会利用指数函数的单调性判断两个数的大小.

关联题目数量: 6. 关联题目id: 000064, 000066, 000070, 004522, 005568, 005592

\item 本课时题目数量: 12. 关联题目id: 000061, 000064, 000066, 000069, 000070, 000072, 000073, 003815, 004522, 005568, 005569, 005592

\end{itemize}

\section*{第2单元, 第11课时}
起始页码: 必修第一册P90; 终止页码: 必修第一册P91.
\begin{itemize}
\item K0211001B|D02002B|会利用指数函数的单调性解决相关不等式等问题.

关联题目数量: 9. 关联题目id: 000062, 000068, 000071, 000397, 000738, 000954, 001147, 001345, 004975

\item K0211002B|D02002B|会利用指数函数的性质解决其他如最值问题等数学问题和实际生活问题.

关联题目数量: 2. 关联题目id: 000068, 001343

\item 本课时题目数量: 10. 关联题目id: 000062, 000068, 000071, 000397, 000738, 000954, 001147, 001343, 001345, 004975

\end{itemize}

\section*{第2单元, 第12课时}
起始页码: 必修第一册P93; 终止页码: 必修第一册P94.
\begin{itemize}
\item K0212001B|D02002B|理解对数函数的定义(包含对数函数定义域为$(0,+\infty)$).

关联题目数量: 2. 关联题目id: 000061, 005008

\item K0212002B|D02002B|会求解有关对数型函数的定义域.

关联题目数量: 7. 关联题目id: 000063, 000397, 000738, 001324, 001326, 001329, 002871

\item K0212003B|D02002B|会根据函数定义域, 利用计算器合理采点, 并能通过描点法作出对数函数$y=\log_2x,y=\log_3x,y=\log_{1/2}x$的大致图像.

关联题目数量: 2. 关联题目id: 003041, 003815

\item 本课时题目数量: 11. 关联题目id: 000061, 000063, 000397, 000738, 001324, 001326, 001329, 002871, 003041, 003815, 005008

\end{itemize}

\section*{第2单元, 第13课时}
起始页码: 必修第一册P95; 终止页码: 必修第一册P98.
\begin{itemize}
\item K0213001B|D02002B|会利用对数运算性质, 证明函数$y=\log_ax,y=\log_{1/a}x$的图像关于$x$轴对称.

关联题目数量: 5. 关联题目id: 002871, 002874, 002875, 002878, 005720

\item K0213002B|D02002B|知道对数函数的图像过定点$(1,0)$.

关联题目数量: 2. 关联题目id: 002766, 003815

\item K0213003B|D02002B|会联系幂的基本不等式, 利用反证法证明对数的基本不等式.

关联题目数量: 0. 关联题目id: 

\item K0213004B|D02002B|会类比指数函数的单调性的证明, 利用对数的基本不等式证明对数函数的单调性.

关联题目数量: 3. 关联题目id: 001325, 002898, 002905

\item K0213005B|D02002B|会结合图像以及指数与对数互为逆运算的性质, 探究并证明对数函数$y=\log_ax$和指数函数$y=a^{x}$的图像关于直线$y=x$对称.

关联题目数量: 0. 关联题目id: 

\item K0213006B|D02002B|了解逆运算和反函数的概念.

关联题目数量: 0. 关联题目id: 

\item K0213007B|D02002B|会作出对数函数的大致图像, 能根据其图像特征叙述函数性质.

关联题目数量: 6. 关联题目id: 000062, 000063, 000073, 001326, 001330, 002762

\item K0213008B|D02002B|会利用对数函数的单调性判断两个数的大小.

关联题目数量: 8. 关联题目id: 000064, 000066, 000070, 000362, 000386, 002758, 004902, 004907

\item 本课时题目数量: 24. 关联题目id: 000062, 000063, 000064, 000066, 000070, 000073, 000362, 000386, 001325, 001326, 001330, 002758, 002762, 002766, 002871, 002874, 002875, 002878, 002898, 002905, 003815, 004902, 004907, 005720

\end{itemize}

\section*{第2单元, 第14课时}
起始页码: 必修第一册P98; 终止页码: 必修第一册P99.
\begin{itemize}
\item K0214001B|D02002B|会利用对数函数的单调性估算对数型无理数(如$\log_23$).

关联题目数量: 1. 关联题目id: 000075

\item K0214002B|D02002B|会利用对数函数的单调性解决其他相关不等式等数学问题和生活中的实际问题.

关联题目数量: 17. 关联题目id: 000067, 000567, 000795, 001325, 001328, 001351, 001352, 002796, 003747, 003775, 004708, 005023, 005199, 005720, 005723, 005724, 010162

\item 本课时题目数量: 18. 关联题目id: 000067, 000075, 000567, 000795, 001325, 001328, 001351, 001352, 002796, 003747, 003775, 004708, 005023, 005199, 005720, 005723, 005724, 010162

\end{itemize}

\section*{第2单元, 第15课时}
起始页码: 必修第一册P106; 终止页码: 必修第一册P109.
\begin{itemize}
\item K0215001B|D02003B|理解函数的概念, 体会函数即数与数之间的对应关系, 理解函数的定义(包含自变量、函数值、定义域、值域的概念).

关联题目数量: 2. 关联题目id: 004403, 004435

\item K0215002B|D02003B|知道定义域和对应关系为函数的两个要素.

关联题目数量: 0. 关联题目id: 

\item K0215003B|D02003B|会求函数的自然定义域.

关联题目数量: 1. 关联题目id: 000076

\item K0215004B|D02003B|理解两个函数相同的定义, 并会判断两个函数是否是同一函数.

关联题目数量: 0. 关联题目id: 

\item K0215005B|D02003B|会根据已学习过的一些简单函数的值域, 利用复合求解稍为复杂函数的值域.

关联题目数量: 5. 关联题目id: 000415, 002760, 003728, 003753, 004415

\item 本课时题目数量: 8. 关联题目id: 000076, 000415, 002760, 003728, 003753, 004403, 004415, 004435

\end{itemize}

\section*{第2单元, 第16课时}
起始页码: 必修第一册P109; 终止页码: 必修第一册P112.
\begin{itemize}
\item K0216001B|D02003B|知道函数可以用解析式、图像、列表等方式表示.

关联题目数量: 0. 关联题目id: 

\item K0216002B|D02003B|理解函数的图像的概念.

关联题目数量: 0. 关联题目id: 

\item K0216003B|D02003B|会合理利用计算器采点, 通过描点法作出不熟悉函数的大致图像.

关联题目数量: 0. 关联题目id: 

\item K0216004B|D02003B|会利用函数的定义判断坐标系中的图像是否为函数图像.

关联题目数量: 0. 关联题目id: 

\item K0216005B|D02003B|了解并能根据实际情况运用函数的分段表示法.

关联题目数量: 1. 关联题目id: 000083

\item K0216006B|D02003B|知道取整符号$[x]$的含义, 并作出取整函数的大致图像.

关联题目数量: 0. 关联题目id: 

\item 本课时题目数量: 1. 关联题目id: 000083

\end{itemize}

\section*{第2单元, 第17课时}
起始页码: 必修第一册P115; 终止页码: 必修第一册P118.
\begin{itemize}
\item K0217001B|D02003B|知道基于点集的图形关于直线成轴对称的定义.

关联题目数量: 0. 关联题目id: 

\item K0217002B|D02003B|会推导``函数的图像关于$y$轴成轴对称''的等价的代数表达形式, 即偶函数的定义.

关联题目数量: 1. 关联题目id: 003980

\item K0217003B|D02003B|会类比偶函数的定义得到``函数的图像关于原点成中心对称''的等价的代数表达形式, 即奇函数的定义.

关联题目数量: 0. 关联题目id: 

\item K0217004B|D02003B|会运用奇函数、偶函数的定义, 证明一些较为简单的函数是奇函数或是偶函数.

关联题目数量: 4. 关联题目id: 000077, 000085, 000093, 004697

\item 本课时题目数量: 5. 关联题目id: 000077, 000085, 000093, 003980, 004697

\end{itemize}

\section*{第2单元, 第18课时}
起始页码: 必修第一册P118; 终止页码: 必修第一册P119.
\begin{itemize}
\item K0218001B|D02003B|会运用奇函数、偶函数的定义, 通过赋值法或分析定义域, 判断较为复杂(如含参数)的函数的奇偶性问题.

关联题目数量: 1. 关联题目id: 000078

\item 本课时题目数量: 1. 关联题目id: 000078

\end{itemize}

\section*{第2单元, 第19课时}
起始页码: 必修第一册P119; 终止页码: 必修第一册P122.
\begin{itemize}
\item K0219001B|D02003B|理解单调函数、单调区间的定义.

关联题目数量: 2. 关联题目id: 000080, 000092

\item K0219002B|D02003B|会运用函数单调性的定义证明一次函数、二次函数、反比例函数的单调性.

关联题目数量: 2. 关联题目id: 000081, 004562

\item K0219003B|D02003B|会运用函数单调性的定义以及已知的基本初等函数的单调性, 判断较为复杂的函数单调性.

关联题目数量: 2. 关联题目id: 000092, 002795

\item 本课时题目数量: 5. 关联题目id: 000080, 000081, 000092, 002795, 004562

\end{itemize}

\section*{第2单元, 第20课时}
起始页码: 必修第一册P122; 终止页码: 必修第一册P123.
\begin{itemize}
\item K0220001B|D02003B|理解单调函数、单调区间的定义.

关联题目数量: 4. 关联题目id: 002787, 003716, 004007, 004631

\item K0220002B|D02003B|会求函数的单调区间.

关联题目数量: 2. 关联题目id: 000079, 000091

\item K0220003B|D02003B|能直观地感知奇偶性可用于分析单调性并能说理.

关联题目数量: 1. 关联题目id: 000089

\item 本课时题目数量: 7. 关联题目id: 000079, 000089, 000091, 002787, 003716, 004007, 004631

\end{itemize}

\section*{第2单元, 第21课时}
起始页码: 必修第一册P123; 终止页码: 必修第一册P125.
\begin{itemize}
\item K0221001B|D02003B|理解函数最大值、最小值的定义.

关联题目数量: 1. 关联题目id: 000090

\item K0221002B|D02003B|会运用最值的定义, 解决函数的最值问题, 以及含参数的函数最值问题(函数对应关系含参数或者定义域含参数)的数学问题.

关联题目数量: 3. 关联题目id: 000087, 000137, 002780

\item 本课时题目数量: 4. 关联题目id: 000087, 000090, 000137, 002780

\end{itemize}

\section*{第2单元, 第22课时}
起始页码: 必修第一册P127; 终止页码: 必修第一册P129.
\begin{itemize}
\item K0222001B|D02004B|会将现实情境转化为数学模型, 并能分析其中量与量之间的关系.

关联题目数量: 0. 关联题目id: 

\item K0222002B|D02004B|在建立好的数学模型中, 能合理选取变量, 建立变量之间的函数关系, 并能结合实际写出函数的定义域.

关联题目数量: 0. 关联题目id: 

\item 本课时题目数量: 0. 关联题目id: 

\end{itemize}

\section*{第2单元, 第23课时}
起始页码: 必修第一册P129; 终止页码: 必修第一册P131.
\begin{itemize}
\item K0223001B|D02004B|知道函数零点的定义.

关联题目数量: 1. 关联题目id: 004874

\item K0223002B|D02004B|会用函数的观点求解一元二次方程.

关联题目数量: 3. 关联题目id: 004697, 004960, 004961

\item K0223003B|D02004B|会用函数的观点求解一元二次不等式.

关联题目数量: 3. 关联题目id: 000084, 000086, 004962

\item K0223004B|D02004B|会用函数的观点求解较为复杂的方程.

关联题目数量: 2. 关联题目id: 000088, 004422

\item K0223005B|D02004B|会用函数的观点求解较为复杂的不等式.

关联题目数量: 0. 关联题目id: 

\item 本课时题目数量: 9. 关联题目id: 000084, 000086, 000088, 004422, 004697, 004874, 004960, 004961, 004962

\end{itemize}

\section*{第2单元, 第24课时}
起始页码: 必修第一册P132; 终止页码: 必修第一册P134.
\begin{itemize}
\item K0224001B|D02004B|知道零点存在定理, 会用零点存在定理判断连续函数在区间上存在零点.

关联题目数量: 1. 关联题目id: 004014

\item K0224002B|D02004B|理解并会运用二分法寻求连续函数在某个区间上的零点的近似值.

关联题目数量: 0. 关联题目id: 

\item 本课时题目数量: 1. 关联题目id: 004014

\end{itemize}

\section*{第2单元, 第25课时}
起始页码: 必修第一册P136; 终止页码: 必修第一册P138.
\begin{itemize}
\item K0225001B|D02004B|理解反函数的定义.

关联题目数量: 0. 关联题目id: 

\item K0225002B|D02004B|会判断一个函数是否存在反函数.

关联题目数量: 0. 关联题目id: 

\item K0225003B|D02004B|知道反函数与原来函数定义域与值域的关系.

关联题目数量: 0. 关联题目id: 

\item K0225004B|D02004B|会根据反函数与原来函数自变量与函数值的关系, 求解反函数或原来函数的自变量或函数值.

关联题目数量: 0. 关联题目id: 

\item K0225005B|D02004B|会求一个具体函数的反函数.

关联题目数量: 3. 关联题目id: 000132, 000133, 000134

\item 本课时题目数量: 3. 关联题目id: 000132, 000133, 000134

\end{itemize}

\section*{第2单元, 第26课时}
起始页码: 必修第一册P138; 终止页码: 必修第一册P140.
\begin{itemize}
\item K0226001B|D02004B|经历命题``在平面直角坐标系中, 点$P(a,b)$与点$P'(b,a)$关于直线$y=x$成轴对称''的推导过程.

关联题目数量: 0. 关联题目id: 

\item K0226002B|D02004B|会利用命题``在平面直角坐标系中, 点$P(a,b)$与点$P'(b,a)$关于直线$y=x$成轴对称''证明性质: ``互为反函数的两函数的图像关于$y=x$成轴对称''.

关联题目数量: 0. 关联题目id: 

\item K0226003B|D02004B|能根据性质: ``互为反函数的两函数的图像关于$y=x$成轴对称'', 作出具体函数的反函数的大致图像.

关联题目数量: 0. 关联题目id: 

\item K0226004B|D02004B|能根据性质: ``互为反函数的两函数的图像关于$y=x$成轴对称''与命题``在平面直角坐标系中, 点$P(a,b)$与点$P'(b,a)$关于直线$y=x$成轴对称'', 求解函数与其反函数的图像上点的相关问题.

关联题目数量: 1. 关联题目id: 004650

\item K0226005B|D02004B|会探究具体函数与其反函数的基本性质之间的区别与联系.

关联题目数量: 2. 关联题目id: 000094, 002738

\item K0226006B|D02004B|了解符号$f^{-1}(ax+b)$的含义.

关联题目数量: 0. 关联题目id: 

\item 本课时题目数量: 3. 关联题目id: 000094, 002738, 004650

\end{itemize}

\section*{第2单元, 第27课时}
起始页码: 选择性必修第二册P2; 终止页码: 选择性必修第二册P5.
\begin{itemize}
\item K0227001X|D02005X|经历现实情境中变速运动的平均速度与瞬时速度的定义, 感悟极限思想.

关联题目数量: 1. 关联题目id: 004002

\item K0227002X|D02005X|计算已知位移表达式的运动过程中平均速度的极限, 得到某一时刻的瞬时速度, 理解瞬时速度的含义.

关联题目数量: 0. 关联题目id: 

\item K0227003X|D02005X|从瞬时速度的计算过程中抽象出导数的定义, 理解位移在某一时刻的导数就是该时刻的瞬时速度.

关联题目数量: 1. 关联题目id: 004002

\item K0227004X|D02005X|结合导数理解函数的瞬时变化率的概念.

关联题目数量: 1. 关联题目id: 004002

\item K0227005X|D02005X|会对不超过二次的多项式函数通过定义求在自变量取具体值时的导数.

关联题目数量: 0. 关联题目id: 

\item 本课时题目数量: 1. 关联题目id: 004002

\end{itemize}

\section*{第2单元, 第28课时}
起始页码: 选择性必修第二册P5; 终止页码: 选择性必修第二册P9.
\begin{itemize}
\item K0228001X|D02005X|了解一般曲线的切线可定义为割线的极限情形.

关联题目数量: 2. 关联题目id: 004000, 004001

\item K0228002X|D02005X|在具体的情境中, 直观地通过列举斜率数据, 验证圆上一点处用割线极限定义的切线与平面几何中定义的切线是一致的.

关联题目数量: 0. 关联题目id: 

\item K0228003X|D02005X|用代数语言描述函数图像上某点处割线斜率的极限, 进而结合导数的定义, 理解切线的斜率就是函数在该点处的导数.

关联题目数量: 1. 关联题目id: 004009

\item K0228004X|D02005X|会通过求导, 求得不超过二次的多项式函数在图像上一点处的切线方程.

关联题目数量: 2. 关联题目id: 004001, 004008

\item K0228005X|D02005X|理解驻点的概念, 理解函数图像在驻点处的切线是一条水平直线.

关联题目数量: 0. 关联题目id: 

\item 本课时题目数量: 4. 关联题目id: 004000, 004001, 004008, 004009

\end{itemize}

\section*{第2单元, 第29课时}
起始页码: 选择性必修第二册P11; 终止页码: 选择性必修第二册P13.
\begin{itemize}
\item K0229001X|D02005X|将导数的概念一般化, 理解求一点处导数的结果能引出自变量和其导数的对应关系, 理解导函数(也成为导数)的概念.

关联题目数量: 0. 关联题目id: 

\item K0229002X|D02005X|会通过定义求常数函数, 一次函数, 幂函数$f(x)=x^2$, $f(x)=x^{-1}$的导数.

关联题目数量: 0. 关联题目id: 

\item K0229003X|D02005X|会通过定义求$f(x)=x^\frac 12$的导数, 掌握其中的分子有理化的方法.

关联题目数量: 0. 关联题目id: 

\item K0229004X|D02005X|了解幂函数, $f(x)=\mathrm{e}^x$, $f(x)=\ln x$, 正弦函数与余弦函数的导数.

关联题目数量: 1. 关联题目id: 004003

\item K0229005X|D02005X|会利用已知的导数求函数的驻点.

关联题目数量: 0. 关联题目id: 

\item K0229006X|D02005X|了解基本初等函数的概念.

关联题目数量: 0. 关联题目id: 

\item 本课时题目数量: 1. 关联题目id: 004003

\end{itemize}

\section*{第2单元, 第30课时}
起始页码: 选择性必修第二册P13; 终止页码: 选择性必修第二册P15.
\begin{itemize}
\item K0230001X|D02005X|经历和的求导公式的代数推导过程, 掌握和与差的求导公式.

关联题目数量: 0. 关联题目id: 

\item K0230002X|D02005X|掌握并熟记积与商的求导公式.

关联题目数量: 0. 关联题目id: 

\item K0230003X|D02005X|会用积的求导公式推导数乘的求导公式.

关联题目数量: 0. 关联题目id: 

\item K0230004X|D02005X|会将简单的初等函数表达为若干个基本初等函数的四则运算, 并用导数的四则运算法则求导.

关联题目数量: 2. 关联题目id: 004003, 004004

\item K0230005X|D02005X|会用换底公式求得一般对数函数的导数.

关联题目数量: 0. 关联题目id: 

\item 本课时题目数量: 2. 关联题目id: 004003, 004004

\end{itemize}

\section*{第2单元, 第31课时}
起始页码: 选择性必修第二册P15; 终止页码: 选择性必修第二册P16.
\begin{itemize}
\item K0231001X|D02005X|通过实例直观地了解复合函数的概念.

关联题目数量: 0. 关联题目id: 

\item K0231002X|D02005X|经历函数$y=f(ax+b)$的求导公式的推导过程, 会借助中间变量$u$记忆$f(ax+b)$型复合函数的求导法则.

关联题目数量: 0. 关联题目id: 

\item K0231003X|D02005X|会结合使用$f(ax+b)$型复合函数的求导法则及四则运算的求导法则求初等函数的导数.

关联题目数量: 2. 关联题目id: 004003, 004004

\item K0231004X|D02005X|会通过代数变形将$a^x$转换成底为$\mathrm{e}$的函数, 进而求得一般指数函数的导数.

关联题目数量: 0. 关联题目id: 

\item 本课时题目数量: 2. 关联题目id: 004003, 004004

\end{itemize}

\section*{第2单元, 第32课时}
起始页码: 选择性必修第二册P19; 终止页码: 选择性必修第二册P21.
\begin{itemize}
\item K0232001X|D02006X|了解根据区间上导数的正负号可以用于判断函数在该区间上的单调性.

关联题目数量: 1. 关联题目id: 004013

\item K0232002X|D02006X|会根据驻点进行分段, 用导数的正负性研究一些初等函数的单调性.

关联题目数量: 2. 关联题目id: 004005, 004013

\item K0232003X|D02006X|通过具体的实例(如$f(x)=x^3$)了解驻点不一定是单调性的分界点, 以及在函数严格递增处导数也不一定恒为正.

关联题目数量: 0. 关联题目id: 

\item K0232004X|D02006X|了解导数的数值可以判断函数变化速度的快慢, 直观地反映为曲线的倾斜程度.

关联题目数量: 0. 关联题目id: 

\item 本课时题目数量: 2. 关联题目id: 004005, 004013

\end{itemize}

\section*{第2单元, 第33课时}
起始页码: 选择性必修第二册P21; 终止页码: 选择性必修第二册P24.
\begin{itemize}
\item K0233001X|D02006X|结合图像直观, 理解极大值与极大值点, 极小值与极小值点、极值与极值点的定义.

关联题目数量: 1. 关联题目id: 004007

\item K0233002X|D02006X|结合图像直观, 理解根据驻点周围导数的符号确定驻点为极大(小)值点的充分条件.

关联题目数量: 0. 关联题目id: 

\item K0233003X|D02006X|会通过求导求得不超过三次的多项式函数与简单三角函数的极值点与极值.

关联题目数量: 2. 关联题目id: 004005, 004009

\item 本课时题目数量: 3. 关联题目id: 004005, 004007, 004009

\end{itemize}

\section*{第2单元, 第34课时}
起始页码: 选择性必修第二册P24; 终止页码: 选择性必修第二册P27.
\begin{itemize}
\item K0234001X|D02006X|能说出极值与最值的联系与区别.

关联题目数量: 0. 关联题目id: 

\item K0234002X|D02006X|结合图像直观, 知道比区间上的连续函数一定存在最大值和最小值.

关联题目数量: 0. 关联题目id: 

\item K0234003X|D02006X|对于导数存在的函数而言, 会通过分析驻点和定义域端点的函数值求得其最大值与最小值.

关联题目数量: 2. 关联题目id: 004006, 004014

\item K0234004X|D02006X|经历用导数研究一般二次函数单调性的过程.

关联题目数量: 0. 关联题目id: 

\item K0234005X|D02006X|理解能通过导数分析相应函数单调性, 结合相应方程的零点求得不等式的解集.

关联题目数量: 0. 关联题目id: 

\item 本课时题目数量: 2. 关联题目id: 004006, 004014

\end{itemize}

\section*{第2单元, 第35课时}
起始页码: 选择性必修第二册P27; 终止页码: 选择性必修第二册P29.
\begin{itemize}
\item K0235001X|D02006X|在现实情境的问题中, 能通过建模, 求导, 解决现实中的最大值或最小值问题.

关联题目数量: 3. 关联题目id: 004010, 004011, 004012

\item 本课时题目数量: 3. 关联题目id: 004010, 004011, 004012

\end{itemize}

\section*{第3单元, 第1课时}
起始页码: 必修第二册P2; 终止页码: 必修第二册P4.
\begin{itemize}
\item K0301001B|D03001B|理解任意角的概念及相关概念.

关联题目数量: 0. 关联题目id: 

\item K0301002B|D03001B|会判断角在平面直角坐标系中的位置.

关联题目数量: 1. 关联题目id: 004736

\item K0301003B|D03001B|角的加减运算与角终边的旋转之间的关系.

关联题目数量: 0. 关联题目id: 

\item K0301004B|D03001B|终边有特殊位置关系的角之间的等量关系.

关联题目数量: 0. 关联题目id: 

\item 本课时题目数量: 1. 关联题目id: 004736

\end{itemize}

\section*{第3单元, 第2课时}
起始页码: 必修第二册P4; 终止页码: 必修第二册P7.
\begin{itemize}
\item K0302001B|D03001B|了解弧度制, 能进行一般的角度制与弧度制的转化.

关联题目数量: 0. 关联题目id: 

\item K0302002B|D03001B|掌握弧度制下扇形的弧长和面积公式.

关联题目数量: 2. 关联题目id: 000097, 000115

\item K0302003B|D03001B|在弧度制下会用代数方法表示和研究角.

关联题目数量: 0. 关联题目id: 

\item 本课时题目数量: 2. 关联题目id: 000097, 000115

\end{itemize}

\section*{第3单元, 第3课时}
起始页码: 必修第二册P7; 终止页码: 必修第二册P9.
\begin{itemize}
\item K0303001B|D03001B|掌握任意角的用比值给出的正弦、余弦、正切、余切的定义.

关联题目数量: 2. 关联题目id: 000098, 000105

\item K0303002B|D03001B|掌握不同象限的角的正弦、余弦、正切和余切的符号.

关联题目数量: 2. 关联题目id: 000096, 000105

\item 本课时题目数量: 3. 关联题目id: 000096, 000098, 000105

\end{itemize}

\section*{第3单元, 第4课时}
起始页码: 必修第二册P9; 终止页码: 必修第二册P12.
\begin{itemize}
\item K0304001B|D03001B|理解角的终边和单位圆的交点的坐标与角的正弦、余弦、正切和余切的关系.

关联题目数量: 0. 关联题目id: 

\item K0304002B|D03001B|经历$\sin^2\alpha+\cos^2\alpha=1$; $\tan\alpha=\dfrac{\sin\alpha}{\cos\alpha}$; $\cot\alpha=\dfrac{\cos\alpha}{\sin\alpha}$; $\tan\alpha\cdot \cot\alpha=1$的推导.

关联题目数量: 0. 关联题目id: 

\item K0304003B|D03001B|会用$\sin^2\alpha+\cos^2\alpha=1$; $\tan\alpha=\dfrac{\sin\alpha}{\cos\alpha}$; $\cot\alpha=\dfrac{\cos\alpha}{\sin\alpha}$; $\tan\alpha\cdot \cot\alpha=1$解决``已知一个三角比的值, 求其他三角比的值''的问题.

关联题目数量: 6. 关联题目id: 000100, 000107, 000108, 000109, 000110, 000111

\item 本课时题目数量: 6. 关联题目id: 000100, 000107, 000108, 000109, 000110, 000111

\end{itemize}

\section*{第3单元, 第5课时}
起始页码: 必修第二册P12; 终止页码: 必修第二册P13.
\begin{itemize}
\item K0305001B|D03001B|会用同角三角函数的基本关系式($\sin^2\alpha+\cos^2\alpha=1$; $\tan\alpha=\dfrac{\sin\alpha}{\cos\alpha}$; $\cot\alpha=\dfrac{\cos\alpha}{\sin\alpha}$; $\tan\alpha\cdot \cot\alpha=1$), 在熟悉的情境中, 解决一些三角恒等式的化简与证明问题.

关联题目数量: 12. 关联题目id: 000095, 000096, 000099, 000104, 000105, 000106, 000112, 000135, 004115, 004979, 005021, 005269

\item 本课时题目数量: 12. 关联题目id: 000095, 000096, 000099, 000104, 000105, 000106, 000112, 000135, 004115, 004979, 005021, 005269

\end{itemize}

\section*{第3单元, 第6课时}
起始页码: 必修第二册P13; 终止页码: 必修第二册P16.
\begin{itemize}
\item K0306001B|D03002B|借助单位圆的对称性, 经历利用定义推导出第一组诱导公式(有关$k\pi\pm \alpha$)的正弦、余弦、正切和余切的过程.

关联题目数量: 0. 关联题目id: 

\item K0306002B|D03002B|会利用第一组诱导公式(有关$k\pi\pm \alpha$)进行简单的求值、化简与证明.

关联题目数量: 1. 关联题目id: 000099

\item 本课时题目数量: 1. 关联题目id: 000099

\end{itemize}

\section*{第3单元, 第7课时}
起始页码: 必修第二册P16; 终止页码: 必修第二册P18.
\begin{itemize}
\item K0307001B|D03002B|借助单位圆的对称性, 经历利用定义推导出第二组诱导公式(有关$(k+\dfrac 12)\pi\pm \alpha$)的正弦、余弦、正切和余切的过程.

关联题目数量: 0. 关联题目id: 

\item K0307002B|D03002B|会通过``奇变偶不变, 符号看象限''来记忆诱导公式.

关联题目数量: 0. 关联题目id: 

\item K0307003B|D03002B|会利用第二组诱导公式(有关$(k+\dfrac 12)\pi\pm \alpha$)进行简单的求值、化简与证明.

关联题目数量: 2. 关联题目id: 000095, 000099

\item K0307004B|D03002B|理解可以通过终边的旋转、对称等方式, 利用诱导公式研究平面上的坐标变换.

关联题目数量: 0. 关联题目id: 

\item 本课时题目数量: 2. 关联题目id: 000095, 000099

\end{itemize}

\section*{第3单元, 第8课时}
起始页码: 必修第二册P19,必修第二册P45; 终止页码: 必修第二册P24,必修第二册P47.
\begin{itemize}
\item K0308001B|D03002B|能够从已知特殊三角值的角的正弦、余弦、正切值求角的集合, 并能简单应用.

关联题目数量: 1. 关联题目id: 000121

\item K0308002B|D03002B|能借助角的三角比的特殊值解简单的三角方程.

关联题目数量: 2. 关联题目id: 000105, 000130

\item K0308003B|D03002B|掌握锐角的反三角函数表示, 并能用计算器求出近似值.

关联题目数量: 1. 关联题目id: 000103

\item K0308004B|D03002B|借助单位圆, 能用反三角符号表示的锐角表示一般角.

关联题目数量: 0. 关联题目id: 

\item K0308005B|D03002B|会解具体的最简三角方程.

关联题目数量: 4. 关联题目id: 000126, 000128, 000131, 005269

\item 本课时题目数量: 8. 关联题目id: 000103, 000105, 000121, 000126, 000128, 000130, 000131, 005269

\end{itemize}

\section*{第3单元, 第9课时}
起始页码: 必修第二册P25; 终止页码: 必修第二册P27.
\begin{itemize}
\item K0309001B|D03002B|经历两角差的余弦公式的坐标法的推导过程, 知道两角差的余弦公式的意义.

关联题目数量: 0. 关联题目id: 

\item K0309002B|D03002B|两角差的余弦推导两角和的余弦.

关联题目数量: 0. 关联题目id: 

\item K0309003B|D03002B|会灵活选择角, 用两角和差的余弦公式求值及化简.

关联题目数量: 0. 关联题目id: 

\item 本课时题目数量: 0. 关联题目id: 

\end{itemize}

\section*{第3单元, 第10课时}
起始页码: 必修第二册P27; 终止页码: 必修第二册P30.
\begin{itemize}
\item K0310001B|D03002B|了解两角差的余弦公式推导两角和与差的正弦、正切公式的路径, 并经历推导过程.

关联题目数量: 0. 关联题目id: 

\item K0310002B|D03002B|会灵活选择角, 用两角和、差的正弦和正切公式求值及化简.

关联题目数量: 9. 关联题目id: 000099, 000103, 000108, 000109, 000110, 000111, 000113, 000114, 000117

\item 本课时题目数量: 9. 关联题目id: 000099, 000103, 000108, 000109, 000110, 000111, 000113, 000114, 000117

\end{itemize}

\section*{第3单元, 第11课时}
起始页码: 必修第二册P30; 终止页码: 必修第二册P31.
\begin{itemize}
\item K0311001B|D03002B|理解可以通过终边的旋转、对称等方式, 利用两角和、差的正弦、余弦公式研究平面上的坐标变换.

关联题目数量: 0. 关联题目id: 

\item K0311002B|D03002B|会用辅助角公式将$a\sin\alpha+b\cos\alpha$型的表达式整理为$A\sin(\alpha+\varphi)$及$A\cos(\alpha+\varphi)$的形式, 并能明确给出辅助角$\phi$的正弦与余弦值.

关联题目数量: 4. 关联题目id: 000126, 000127, 000129, 000137

\item 本课时题目数量: 4. 关联题目id: 000126, 000127, 000129, 000137

\end{itemize}

\section*{第3单元, 第12课时}
起始页码: 必修第二册P31; 终止页码: 必修第二册P33.
\begin{itemize}
\item K0312001B|D03002B|经历利用两角和公式, 推导出二倍角的正弦、余弦、正切公式的过程, 并了解它们的内在联系.

关联题目数量: 0. 关联题目id: 

\item K0312002B|D03002B|熟悉二倍角余弦公式的三种不同形式.

关联题目数量: 0. 关联题目id: 

\item K0312003B|D03002B|利用二倍角公式, 进行求值、化简和证明.

关联题目数量: 10. 关联题目id: 000095, 000101, 000104, 000111, 000112, 000113, 000114, 000117, 000129, 000490

\item 本课时题目数量: 10. 关联题目id: 000095, 000101, 000104, 000111, 000112, 000113, 000114, 000117, 000129, 000490

\end{itemize}

\section*{第3单元, 第13课时}
起始页码: 必修第二册P33; 终止页码: 必修第二册P39.
\begin{itemize}
\item K0313001B|D03002B|能运用所学公式进行简单的恒等变换, 推导半角公式.

关联题目数量: 0. 关联题目id: 

\item K0313002B|D03002B|能运用所学公式进行简单的恒等变换, 推导积化和差公式.

关联题目数量: 1. 关联题目id: 003965

\item K0313003B|D03002B|能运用所学公式进行简单的恒等变换, 推导和差化积公式.

关联题目数量: 1. 关联题目id: 003965

\item 本课时题目数量: 1. 关联题目id: 003965

\end{itemize}

\section*{第3单元, 第14课时}
起始页码: 必修第二册P40; 终止页码: 必修第二册P42.
\begin{itemize}
\item K0314001B|D03003B|经历用坐标法推导三角形面积公式$S=\dfrac{1}{2}ab\sin C$等的过程.

关联题目数量: 0. 关联题目id: 

\item K0314002B|D03003B|利用三角形面积公式推导得到正弦定理.

关联题目数量: 0. 关联题目id: 

\item K0314003B|D03003B|会用正弦定理解决``ASA''型的解三角形问题.

关联题目数量: 0. 关联题目id: 

\item K0314004B|D03003B|会用正弦定理及面积公式证明三角形中关于边、角和面积的恒等式.

关联题目数量: 1. 关联题目id: 000102

\item K0314005B|D03003B|理解圆周角和圆心角的$2$倍关系.

关联题目数量: 0. 关联题目id: 

\item K0314006B|D03003B|利用圆周角均相等推导含$2R$的正弦定理的过程.

关联题目数量: 0. 关联题目id: 

\item 本课时题目数量: 1. 关联题目id: 000102

\end{itemize}

\section*{第3单元, 第15课时}
起始页码: 必修第二册P42; 终止页码: 必修第二册P45.
\begin{itemize}
\item K0315001B|D03003B|经历用坐标法推导余弦定理的过程.

关联题目数量: 0. 关联题目id: 

\item K0315002B|D03003B|熟悉并记忆余弦定理.

关联题目数量: 0. 关联题目id: 

\item K0315003B|D03003B|会用余弦定理解``SSS''``SAS''型的解三角形问题.

关联题目数量: 1. 关联题目id: 000106

\item K0315004B|D03003B|能够灵活运用正弦定理、余弦定理解决``SSA''型的解三角形问题, 并能正确取舍解得结果.

关联题目数量: 2. 关联题目id: 000101, 000116

\item 本课时题目数量: 3. 关联题目id: 000101, 000106, 000116

\end{itemize}

\section*{第3单元, 第16课时}
起始页码: 必修第二册P45; 终止页码: 必修第二册P45.
\begin{itemize}
\item K0316001B|D03003B|会灵活运用正弦定理和余弦定理证明三角形中的等式.

关联题目数量: 2. 关联题目id: 005249, 005251

\item K0316002B|D03003B|会灵活运用正弦定理和余弦定理判断三角形的形状.

关联题目数量: 1. 关联题目id: 000113

\item 本课时题目数量: 3. 关联题目id: 000113, 005249, 005251

\end{itemize}

\section*{第3单元, 第17课时}
起始页码: 必修第二册P47; 终止页码: 必修第二册P55.
\begin{itemize}
\item K0317001B|D03003B|能用正弦定理、余弦定理解决简单的实际问题.

关联题目数量: 1. 关联题目id: 000118

\item K0317002B|D03003B|能将有关测量的问题转化为解三角形问题, 并灵活运用正弦定理和余弦定理求解.

关联题目数量: 0. 关联题目id: 

\item 本课时题目数量: 1. 关联题目id: 000118

\end{itemize}

\section*{第3单元, 第18课时}
起始页码: 必修第二册P58; 终止页码: 必修第二册P60.
\begin{itemize}
\item K0318001B|D03004B|建立正弦函数的概念.

关联题目数量: 1. 关联题目id: 000135

\item K0318002B|D03004B|经历描点, 平移绘制正弦函数图像的过程.

关联题目数量: 0. 关联题目id: 

\item K0318003B|D03004B|会用五点法绘制正弦函数、与正弦函数相关的函数的大致图像.

关联题目数量: 0. 关联题目id: 

\item 本课时题目数量: 1. 关联题目id: 000135

\end{itemize}

\section*{第3单元, 第19课时}
起始页码: 必修第二册P60; 终止页码: 必修第二册P64.
\begin{itemize}
\item K0319001B|D03004B|结合诱导公式理解正弦函数的周期性.

关联题目数量: 0. 关联题目id: 

\item K0319002B|D03004B|直观地理解周期函数的定义.

关联题目数量: 2. 关联题目id: 000136, 001491

\item K0319003B|D03004B|能用``数学语言''准确地给出周期函数的定义.

关联题目数量: 2. 关联题目id: 000136, 001491

\item K0319004B|D03004B|会证明正弦函数的最小正周期是$2\pi$.

关联题目数量: 0. 关联题目id: 

\item K0319005B|D03004B|了解函数$y=A\sin(\omega x+\varphi)$的周期, 并会粗略地说明道理.

关联题目数量: 1. 关联题目id: 000119

\item 本课时题目数量: 3. 关联题目id: 000119, 000136, 001491

\end{itemize}

\section*{第3单元, 第20课时}
起始页码: 必修第二册P64; 终止页码: 必修第二册P67.
\begin{itemize}
\item K0320001B|D03004B|借助单位圆理解正弦函数的值域与最值.

关联题目数量: 4. 关联题目id: 000490, 003745, 004261, 004970

\item K0320002B|D03004B|能运用正弦函数的值域与最值解决简单的正弦型函数的相应问题.

关联题目数量: 4. 关联题目id: 000125, 000129, 000130, 000137

\item K0320003B|D03004B|会将与正弦有关的现实情境中的问题转化为正弦函数的最值问题, 并加以解决.

关联题目数量: 1. 关联题目id: 000137

\item 本课时题目数量: 8. 关联题目id: 000125, 000129, 000130, 000137, 000490, 003745, 004261, 004970

\end{itemize}

\section*{第3单元, 第21课时}
起始页码: 必修第二册P67; 终止页码: 必修第二册P72.
\begin{itemize}
\item K0321001B|D03004B|会判断并证明与正弦函数相关的函数的奇偶性.

关联题目数量: 1. 关联题目id: 000120

\item K0321002B|D03004B|会借助单位圆及函数图像, 直观地理解正弦函数的单调性.

关联题目数量: 0. 关联题目id: 

\item K0321003B|D03004B|能求$y=A\sin(\omega x+\varphi)$型函数的单调区间, 其中$A>0$, $\omega>0$.

关联题目数量: 3. 关联题目id: 000126, 000128, 000131

\item K0321004B|D03004B|能求$y=A\sin(\omega x+\varphi)$型函数的单调区间, 其中$A$与$\omega$不全大于零.

关联题目数量: 1. 关联题目id: 000122

\item 本课时题目数量: 5. 关联题目id: 000120, 000122, 000126, 000128, 000131

\end{itemize}

\section*{第3单元, 第22课时}
起始页码: 必修第二册P73; 终止页码: 必修第二册P76.
\begin{itemize}
\item K0322001B|D03004B|建立余弦函数的概念.

关联题目数量: 0. 关联题目id: 

\item K0322002B|D03004B|借助正弦函数的相关性质, 掌握余弦函数的奇偶性、周期性、单调性、值域与最值等性质及其图像特征.

关联题目数量: 3. 关联题目id: 000120, 000125, 005251

\item K0322003B|D03004B|会将与余弦函数有关的问题借助第二诱导公式转化为正弦函数有关的问题.

关联题目数量: 1. 关联题目id: 000122

\item 本课时题目数量: 4. 关联题目id: 000120, 000122, 000125, 005251

\end{itemize}

\section*{第3单元, 第23课时}
起始页码: 必修第二册P77; 终止页码: 必修第二册P83.
\begin{itemize}
\item K0323001B|D03004B|结合具体实例, 了解函数$y=A\sin(\omega x+\phi)$以及表达式中参数$A$、$\omega$、$\phi$的实际意义及名称.

关联题目数量: 1. 关联题目id: 000124

\item K0323002B|D03004B|会用三角函数解决简单的与周期变化有关的实际问题, 体会可利用三角函数构建刻画周期变化事物的数学模型.

关联题目数量: 0. 关联题目id: 

\item K0323003B|D03004B|了解函数$y=A\sin(\omega x+\varphi)$参数的变化对函数图像的影响.会用五点作图法作出函数的大致图像.

关联题目数量: 2. 关联题目id: 000123, 000131

\item 本课时题目数量: 3. 关联题目id: 000123, 000124, 000131

\end{itemize}

\section*{第3单元, 第24课时}
起始页码: 必修第二册P84; 终止页码: 必修第二册P91.
\begin{itemize}
\item K0324001B|D03004B|类比正弦函数, 建立正切函数的概念.

关联题目数量: 0. 关联题目id: 

\item K0324002B|D03004B|类比正弦函数, 借助单位圆画出正切函数的图像.

关联题目数量: 0. 关联题目id: 

\item K0324003B|D03004B|直观地掌握正切函数的图像特征.

关联题目数量: 0. 关联题目id: 

\item K0324004B|D03004B|直观地理解正切函数的周期性与值域.

关联题目数量: 0. 关联题目id: 

\item K0324005B|D03004B|会用代数语言表示正切函数的奇偶性及单调性.

关联题目数量: 1. 关联题目id: 000120

\item K0324006B|D03004B|能借助正切函数的单调性求$y=A\tan(\omega x+\varphi)$的单调区间.

关联题目数量: 1. 关联题目id: 000122

\item 本课时题目数量: 2. 关联题目id: 000120, 000122

\end{itemize}

\section*{第4单元, 第1课时}
起始页码: 选择性必修第一册P123; 终止页码: 选择性必修第一册P127.
\begin{itemize}
\item K0401001X|D04001X|了解数列、数列的项、项的序数的概念.

关联题目数量: 0. 关联题目id: 

\item K0401002X|D04001X|经历从具体的问题情境中抽象出等差数列定义的过程, 理解等差数列的概念, 知道公差及等差中项的概念.

关联题目数量: 6. 关联题目id: 000313, 000314, 000317, 000318, 000319, 002769

\item K0401003X|D04001X|经历用累加法由等差数列的定义得到其通项公式的过程, 建立等差数列的通项公式.

关联题目数量: 0. 关联题目id: 

\item K0401004X|D04001X|掌握等差数列的项与序数间的联系, 明白等差数列与一次函数间的关联.

关联题目数量: 1. 关联题目id: 000308

\item K0401005X|D04001X|能根据等差数列的通项公式判断某数是否为该数列的项, 并加以证明.

关联题目数量: 1. 关联题目id: 000306

\item K0401006X|D04001X|能根据数列的通项公式判断某数列是否为等差数列, 并加以证明.

关联题目数量: 0. 关联题目id: 

\item K0401007X|D04001X|能在具体的生活情境中, 发现数列的等差关系, 并能简单运用所学知识解决相应的问题.

关联题目数量: 0. 关联题目id: 

\item 本课时题目数量: 8. 关联题目id: 000306, 000308, 000313, 000314, 000317, 000318, 000319, 002769

\end{itemize}

\section*{第4单元, 第2课时}
起始页码: 选择性必修第一册P127; 终止页码: 选择性必修第一册P131.
\begin{itemize}
\item K0402001X|D04001X|经历从特殊到一般推导等差数列前$n$项和公式的过程, 掌握等差数列的前$n$项和公式的逆序相加的推导方法.

关联题目数量: 0. 关联题目id: 

\item K0402002X|D04001X|会用定义判断并证明一个数列是否为等差数列.

关联题目数量: 2. 关联题目id: 000310, 000321

\item K0402003X|D04001X|明白求和符号$\Sigma$的意义.

关联题目数量: 0. 关联题目id: 

\item K0402004X|D04001X|掌握等差数列前$n$项和公式的两种形式($S_{n}=\dfrac{a_{1}+a_{n}}{2}$, $S_{n}=na_{1}+\dfrac{n(n-1)}{2}d$, 关注公式中所涉及的基本量, 能够根据实际情况合理选择并运用公式解决有关问题.

关联题目数量: 4. 关联题目id: 000310, 000311, 000312, 004630

\item K0402005X|D04001X|能够根据数列的前$n$项和公式推出数列的通项公式.

关联题目数量: 3. 关联题目id: 000309, 000317, 000321

\item K0402006X|D04001X|知道等差数列前$n$项和公式与二次函数间的关联.

关联题目数量: 2. 关联题目id: 000309, 000312

\item 本课时题目数量: 7. 关联题目id: 000309, 000310, 000311, 000312, 000317, 000321, 004630

\end{itemize}

\section*{第4单元, 第3课时}
起始页码: 选择性必修第一册P132; 终止页码: 选择性必修第一册P134.
\begin{itemize}
\item K0403001X|D04002X|从具体问题情境中感受等比关系, 在此基础上类比等差数列的定义得到等比数列的定义, 掌握公比及等比中项的概念.

关联题目数量: 4. 关联题目id: 000313, 000314, 000317, 002769

\item K0403002X|D04002X|类比等差数列的通项公式的得出过程, 经历由等比数列的定义通过累乘得到其通项公式的过程, 建立等比数列的通项公式.

关联题目数量: 0. 关联题目id: 

\item K0403003X|D04002X|会用定义判断及证明一个数列是否为等比数列.

关联题目数量: 1. 关联题目id: 000306

\item K0403004X|D04002X|掌握等比数列的项与序数间的联系, 明白等比数列与指数函数间的关联.

关联题目数量: 1. 关联题目id: 000433

\item K0403005X|D04002X|能在具体的问题情境中, 发现数列的等比关系, 并能简单运用所学知识解决相应的问题.

关联题目数量: 1. 关联题目id: 000325

\item K0403006X|D04002X|体会等差数列和等比数列的特殊联系, 感悟等差数列与正项等比数列之间可以通过指数及对数运算灵活转化.

关联题目数量: 1. 关联题目id: 000319

\item 本课时题目数量: 8. 关联题目id: 000306, 000313, 000314, 000317, 000319, 000325, 000433, 002769

\end{itemize}

\section*{第4单元, 第4课时}
起始页码: 选择性必修第一册P134; 终止页码: 选择性必修第一册P137.
\begin{itemize}
\item K0404001X|D04002X|经历从特殊到一般推导等比数列前$n$项和公式的过程, 掌握等比数列的前$n$项和公式的错位相减的推导方法.

关联题目数量: 1. 关联题目id: 004499

\item K0404002X|D04002X|知道公比为$1$的等比数列前$n$项和$S_{n}=na_{1}$.

关联题目数量: 0. 关联题目id: 

\item K0404003X|D04002X|掌握公比不为$1$的等比数列前$n$项和公式的两种形式($S_{n}=\dfrac{a_{1}(1-q^{n})}{1-q}$、$S_{n}=\dfrac{a_{1}-a_{n+1}}{1-q}$), 关注公式中所涉及的基本量, 能够根据实际情况合理选择并运用公式解决有关问题.

关联题目数量: 5. 关联题目id: 000307, 000318, 000325, 003759, 004630

\item K0404004X|D04002X|知道等比数列前$n$项和公式与$Aq^n+B(q\neq 0$且$q\neq 1)$型函数的关联.

关联题目数量: 0. 关联题目id: 

\item 本课时题目数量: 6. 关联题目id: 000307, 000318, 000325, 003759, 004499, 004630

\end{itemize}

\section*{第4单元, 第5课时}
起始页码: 选择性必修第一册P137; 终止页码: 选择性必修第一册P141.
\begin{itemize}
\item K0405001X|D04002X|借助实例, 理解直观描述下的数列极限的含义.

关联题目数量: 0. 关联题目id: 

\item K0405002X|D04002X|知道符号$\sum\limits_{i=1}^{+\infty }{a_i}$、$\displaystyle\lim_{n\to \infty}S_n$均表示无穷等比数列${a_n}$前$n$项和的极限.

关联题目数量: 0. 关联题目id: 

\item K0405003X|D04002X|知道公比$q$满足$0<|q|<1$的无穷等比数列前$n$项和的极限.

关联题目数量: 0. 关联题目id: 

\item K0405004X|D04002X|知道无限循环小数本质上就是无穷等比数列的前$n$项和的极限的表示方法, 掌握通过等比数列的前$n$项和的极限将无限循环小数化为分数的方法.

关联题目数量: 0. 关联题目id: 

\item K0405005X|D04002X|能在具体问题情境中发现并证明等比关系, 并会利用无穷等比数列的前$n$项和的极限解决有关问题.

关联题目数量: 0. 关联题目id: 

\item 本课时题目数量: 0. 关联题目id: 

\end{itemize}

\section*{第4单元, 第6课时}
起始页码: 选择性必修第一册P142; 终止页码: 选择性必修第一册P145.
\begin{itemize}
\item K0406001X|D04003X|从具体生活与数学情境中抽象概括数列的概念, 理解数列的概念.

关联题目数量: 0. 关联题目id: 

\item K0406002X|D04003X|理解数列的通项公式, 知道数列是一种特殊的函数.

关联题目数量: 0. 关联题目id: 

\item K0406003X|D04003X|会用通项公式、列表等方式表示数列.

关联题目数量: 0. 关联题目id: 

\item K0406004X|D04003X|会用数学语言定义数列的单调性, 能根据定义判断简单数列的单调性.

关联题目数量: 1. 关联题目id: 000320

\item K0406005X|D04003X|能依据数列的单调性求简单数列的最大项、最小项.

关联题目数量: 1. 关联题目id: 000320

\item 本课时题目数量: 1. 关联题目id: 000320

\end{itemize}

\section*{第4单元, 第7课时}
起始页码: 选择性必修第一册P145; 终止页码: 选择性必修第一册P150.
\begin{itemize}
\item K0407001X|D04003X|结合等差数列与等比数列这两类特殊的数列, 理解递推公式是表示数列的一种方法.

关联题目数量: 0. 关联题目id: 

\item K0407002X|D04003X|会用数列的递推公式表示一个数列, 能在一些特殊情形(可累加或可累乘)下根据数列的递推公式求其通项公式.

关联题目数量: 2. 关联题目id: 000307, 000320

\item K0407003X|D04003X|会根据一阶常系数线性递推数列的递推公式求其通项公式.

关联题目数量: 0. 关联题目id: 

\item K0407004X|D04003X|能在具体的问题情境中发现并建立数列的递推关系并解决相应问题, 体会在实际问题中寻找数列的递推关系有时比直接建立通项公式更容易.

关联题目数量: 0. 关联题目id: 

\item 本课时题目数量: 2. 关联题目id: 000307, 000320

\end{itemize}

\section*{第4单元, 第8课时}
起始页码: 选择性必修第一册P151; 终止页码: 选择性必修第一册P154.
\begin{itemize}
\item K0408001X|D04004X|知道通过根据有限的特殊事例(不完全)归纳得到的结论是有待证明的.

关联题目数量: 0. 关联题目id: 

\item K0408002X|D04004X|知道数学归纳法是一种证明与自然数有关的命题的方法, 理解数学归纳法的基本原理.

关联题目数量: 1. 关联题目id: 004981

\item K0408003X|D04004X|初步掌握数学归纳法证明与自然数有关命题的一般步骤, 会用数学归纳法证明一些与自然数有关的一些简单等式.

关联题目数量: 5. 关联题目id: 000315, 000316, 000322, 000323, 004981

\item 本课时题目数量: 5. 关联题目id: 000315, 000316, 000322, 000323, 004981

\end{itemize}

\section*{第4单元, 第9课时}
起始页码: 选择性必修第一册P154; 终止页码: 选择性必修第一册P157.
\begin{itemize}
\item K0409001X|D04004X|在数列求通项、求和等问题中, 经历先猜想后证明的过程, 体会``归纳—猜想—证明''的思想方法.

关联题目数量: 2. 关联题目id: 000316, 000324

\item K0409002X|D04004X|掌握先用待定系数法确定求和公式的必要条件, 后再证明充分性的方法.

关联题目数量: 1. 关联题目id: 000323

\item 本课时题目数量: 3. 关联题目id: 000316, 000323, 000324

\end{itemize}

\section*{第4单元, 第10课时}
起始页码: 选择性必修第一册P158; 终止页码: 选择性必修第一册P161.
\begin{itemize}
\item K0410001X|D04005X|在求$\sqrt 2$的近似值的例子中, 了解基于递推公式的近似计算的迭代算法.

关联题目数量: 0. 关联题目id: 

\item K0410002X|D04005X|通过对巴比伦算法以及另一迭代算法($x_{n+1}=1+\frac{1}{x_{n}+1}$)的迭代的收敛速度的比较, 初步体会算法优劣的评价方式.

关联题目数量: 0. 关联题目id: 

\item 本课时题目数量: 0. 关联题目id: 

\end{itemize}

\section*{第5单元, 第1课时}
起始页码: 必修第二册P94; 终止页码: 必修第二册P96.
\begin{itemize}
\item K0501001B|D05001B|理解向量的描述性定义及相关概念(有向线段、大小、方向、数量等).

关联题目数量: 1. 关联题目id: 000138

\item K0501002B|D05001B|掌握向量的两种表示方法.

关联题目数量: 0. 关联题目id: 

\item K0501003B|D05001B|懂得向量的模的概念.

关联题目数量: 2. 关联题目id: 000138, 000139

\item K0501004B|D05001B|理解平行向量的概念, 并会判断两个向量是否平行.

关联题目数量: 4. 关联题目id: 000140, 000149, 000150, 001856

\item K0501005B|D05001B|理解相等向量的概念, 并会判断两个向量是否相等.

关联题目数量: 0. 关联题目id: 

\item K0501006B|D05001B|理解负向量的概念, 并会用图示法和有向线段表示给定向量的负向量.

关联题目数量: 0. 关联题目id: 

\item 本课时题目数量: 6. 关联题目id: 000138, 000139, 000140, 000149, 000150, 001856

\end{itemize}

\section*{第5单元, 第2课时}
起始页码: 必修第二册P96; 终止页码: 必修第二册P99.
\begin{itemize}
\item K0502001B|D05001B|理解向量加法的平行四边形法则, 能利用它熟练进行向量的加法运算.

关联题目数量: 2. 关联题目id: 000147, 000148

\item K0502002B|D05001B|理解向量加法的三角形法则, 能利用它熟练进行向量的加法运算.

关联题目数量: 3. 关联题目id: 000139, 000147, 001856

\item K0502003B|D05001B|类比实数的加法运算律猜想并作图验证向量加法的运算律.

关联题目数量: 0. 关联题目id: 

\item K0502004B|D05001B|在向量加法结合律的基础上理解首尾规则, 并能用于向量的加法运算.

关联题目数量: 0. 关联题目id: 

\item K0502005B|D05001B|知道向量的减法是用向量的加法的逆运算来定义的.

关联题目数量: 0. 关联题目id: 

\item K0502006B|D05001B|经历向量的减法可以转化为与负向量的加法的推导过程.

关联题目数量: 0. 关联题目id: 

\item K0502007B|D05001B|能熟练进行向量的减法运算.

关联题目数量: 0. 关联题目id: 

\item 本课时题目数量: 4. 关联题目id: 000139, 000147, 000148, 001856

\end{itemize}

\section*{第5单元, 第3课时}
起始页码: 必修第二册P99; 终止页码: 必修第二册P102.
\begin{itemize}
\item K0503001B|D05001B|理解实数与向量乘法的概念与几何意义.

关联题目数量: 0. 关联题目id: 

\item K0503002B|D05001B|知道单位向量的概念, 能用实数与向量的乘法表示单位向量.

关联题目数量: 0. 关联题目id: 

\item K0503003B|D05001B|掌握实数与向量相乘的运算律, 能熟练进行向量的数乘运算并作化简.

关联题目数量: 0. 关联题目id: 

\item K0503004B|D05001B|知道线性运算、线性组合的概念.

关联题目数量: 0. 关联题目id: 

\item K0503005B|D05001B|能熟练运用向量的线性运算, 并会借助基向量对向量进行线性表示.

关联题目数量: 2. 关联题目id: 000149, 000151

\item 本课时题目数量: 2. 关联题目id: 000149, 000151

\end{itemize}

\section*{第5单元, 第4课时}
起始页码: 必修第二册P105; 终止页码: 必修第二册P108.
\begin{itemize}
\item K0504001B|D05001B|知道两个非零向量夹角的概念并会用符号表示.

关联题目数量: 0. 关联题目id: 

\item K0504002B|D05001B|理解数量投影的概念,知道数量投影的符号与向量夹角是锐角、直角、钝角之间的关系.

关联题目数量: 0. 关联题目id: 

\item K0504003B|D05001B|会用向量的夹角、模表示一个向量在另一个非零向量方向上的投影向量.

关联题目数量: 0. 关联题目id: 

\item K0504004B|D05001B|经历向量投影分配律的推导过程.

关联题目数量: 0. 关联题目id: 

\item K0504005B|D05001B|知道投影向量与数量投影两个概念的区别和联系.

关联题目数量: 0. 关联题目id: 

\item K0504006B|D05001B|理解向量数量积的概念.

关联题目数量: 2. 关联题目id: 000154, 003980

\item K0504007B|D05001B|知道数量积与数量投影的联系.

关联题目数量: 0. 关联题目id: 

\item 本课时题目数量: 2. 关联题目id: 000154, 003980

\end{itemize}

\section*{第5单元, 第5课时}
起始页码: 必修第二册P108; 终止页码: 必修第二册P110.
\begin{itemize}
\item K0505001B|D05001B|会用向量的数量积判断两个平面向量的垂直关系, 初步了解向量的数量积在几何上的应用.

关联题目数量: 3. 关联题目id: 000147, 000152, 003980

\item K0505002B|D05001B|会用向量的数量积判断两个平面向量的平行关系, 初步了解向量的数量积在几何上的应用.

关联题目数量: 0. 关联题目id: 

\item K0505003B|D05001B|经历数量积的交换律、结合律、分配律的推导过程.

关联题目数量: 0. 关联题目id: 

\item K0505004B|D05001B|会用向量数量积和线性运算的运算律证明简单的向量恒等式.

关联题目数量: 0. 关联题目id: 

\item K0505005B|D05001B|会用数量积及其运算律解决求向量的夹角、模的问题.

关联题目数量: 2. 关联题目id: 000142, 000154

\item 本课时题目数量: 5. 关联题目id: 000142, 000147, 000152, 000154, 003980

\end{itemize}

\section*{第5单元, 第6课时}
起始页码: 必修第二册P112; 终止页码: 必修第二册P115.
\begin{itemize}
\item K0506001B|D05002B|会正确表述向量基本定理并进行证明.

关联题目数量: 0. 关联题目id: 

\item K0506002B|D05002B|知道平面向量基的概念, 会用基的概念表达向量基本定理.

关联题目数量: 0. 关联题目id: 

\item K0506003B|D05002B|会用基的线性组合表达给定的向量.

关联题目数量: 0. 关联题目id: 

\item 本课时题目数量: 0. 关联题目id: 

\end{itemize}

\section*{第5单元, 第7课时}
起始页码: 必修第二册P115; 终止页码: 必修第二册P117.
\begin{itemize}
\item K0507001B|D05002B|知道向量的分解的概念.

关联题目数量: 0. 关联题目id: 

\item K0507002B|D05002B|知道向量的正交分解的概念.

关联题目数量: 0. 关联题目id: 

\item K0507003B|D05002B|知道向量的坐标分解的概念.

关联题目数量: 0. 关联题目id: 

\item K0507004B|D05002B|知道位置向量的概念.

关联题目数量: 0. 关联题目id: 

\item K0507005B|D05002B|知道向量的坐标就是其位置向量终点的坐标.

关联题目数量: 0. 关联题目id: 

\item K0507006B|D05002B|能根据所给向量的坐标的定义推导坐标表示下向量线性运算的规则.

关联题目数量: 0. 关联题目id: 

\item K0507007B|D05002B|能根据所给向量的坐标进行向量的线性运算.

关联题目数量: 5. 关联题目id: 000143, 000144, 000145, 000146, 000152

\item K0507008B|D05002B|能根据向量的坐标的定义推导模与坐标的关系.

关联题目数量: 0. 关联题目id: 

\item K0507009B|D05002B|能根据所给向量的坐标进行向量的模的运算.

关联题目数量: 3. 关联题目id: 000143, 000144, 000152

\item 本课时题目数量: 5. 关联题目id: 000143, 000144, 000145, 000146, 000152

\end{itemize}

\section*{第5单元, 第8课时}
起始页码: 必修第二册P117; 终止页码: 必修第二册P119.
\begin{itemize}
\item K0508001B|D05002B|会用坐标的定义推导向量数量积的坐标表示.

关联题目数量: 0. 关联题目id: 

\item K0508002B|D05002B|会推借助数量积及模的坐标运算结果导向量夹角的坐标表示.

关联题目数量: 0. 关联题目id: 

\item K0508003B|D05002B|会在坐标形式下利用相应公式计算向量的数量积与夹角.

关联题目数量: 4. 关联题目id: 000141, 000144, 000153, 000156

\item K0508004B|D05002B|会用坐标形式的向量夹角公式推导两个向量垂直的充要条件.

关联题目数量: 0. 关联题目id: 

\item K0508005B|D05002B|会用坐标形式的向量夹角公式推导两个向量平行的充要条件.

关联题目数量: 2. 关联题目id: 000143, 000156

\item K0508006B|D05002B|能将与坐标运算相关的代数问题转化为向量数量积与夹角, 进而解决问题.

关联题目数量: 0. 关联题目id: 

\item 本课时题目数量: 5. 关联题目id: 000141, 000143, 000144, 000153, 000156

\end{itemize}

\section*{第5单元, 第9课时}
起始页码: 必修第二册P121; 终止页码: 必修第二册P123.
\begin{itemize}
\item K0509001B|D05003B|会用向量的线性运算证明平面几何中的平行关系与长度之比等问题.

关联题目数量: 5. 关联题目id: 000155, 000157, 000158, 000160, 000161

\item K0509002B|D05003B|会用向量的坐标与实数和向量乘积的定义证明定比分点公式.

关联题目数量: 0. 关联题目id: 

\item K0509003B|D05003B|会用向量的定比分点公式求三角形重心的坐标.

关联题目数量: 0. 关联题目id: 

\item K0509004B|D05003B|会用向量的运算对三角形进行研究及求解三角形.

关联题目数量: 3. 关联题目id: 000158, 000160, 000161

\item 本课时题目数量: 5. 关联题目id: 000155, 000157, 000158, 000160, 000161

\end{itemize}

\section*{第5单元, 第10课时}
起始页码: 必修第二册P123; 终止页码: 必修第二册P125.
\begin{itemize}
\item K0510001B|D05003B|会用向量的数量积证明两角差的余弦公式.

关联题目数量: 0. 关联题目id: 

\item K0510002B|D05003B|经历用向量语言描述一些现实情境中的问题并用向量方法解决问题的过程.

关联题目数量: 0. 关联题目id: 

\item K0510003B|D05003B|经历用向量语言描述一些物理情境中的问题并用向量方法解决问题的过程.

关联题目数量: 1. 关联题目id: 000159

\item 本课时题目数量: 1. 关联题目id: 000159

\end{itemize}

\section*{第5单元, 第11课时}
起始页码: 必修第二册P134; 终止页码: 必修第二册P137.
\begin{itemize}
\item K0511001B|D05004B|知道引入复数的直接意义是解决负数的开平方问题.

关联题目数量: 0. 关联题目id: 

\item K0511002B|D05004B|知道虚数单位.

关联题目数量: 0. 关联题目id: 

\item K0511003B|D05004B|知道复数的定义、一般形式与复数集的符号表示.

关联题目数量: 1. 关联题目id: 000162

\item K0511004B|D05004B|了解复数相等的含义.

关联题目数量: 2. 关联题目id: 000165, 000169

\item K0511005B|D05004B|类比多项式的运算, 掌握复数的加、减、乘运算的规则, 能正确进行复数的加、减、乘运算.

关联题目数量: 2. 关联题目id: 000166, 000174

\item K0511006B|D05004B|能用分母实数化的方法理解复数的除法运算法则, 并能正确进行复数除法运算.

关联题目数量: 1. 关联题目id: 000166

\item K0511007B|D05004B|理解 加法、乘法的运算律在复数范围内仍成立.

关联题目数量: 0. 关联题目id: 

\item K0511008B|D05004B|了解复数的整数次幂, 类比实数的整数次幂掌握其运算规则.

关联题目数量: 1. 关联题目id: 000166

\item K0511009B|D05004B|掌握虚数单位的整数次幂的运算规律.

关联题目数量: 0. 关联题目id: 

\item 本课时题目数量: 5. 关联题目id: 000162, 000165, 000166, 000169, 000174

\end{itemize}

\section*{第5单元, 第12课时}
起始页码: 必修第二册P137; 终止页码: 必修第二册P140.
\begin{itemize}
\item K0512001B|D05004B|掌握复数的代数形式的概念.

关联题目数量: 0. 关联题目id: 

\item K0512002B|D05004B|掌握复数的实部和虚部的概念.

关联题目数量: 1. 关联题目id: 000163

\item K0512003B|D05004B|会根据复数的代数形式对复数加以分类.

关联题目数量: 1. 关联题目id: 004113

\item K0512004B|D05004B|会运用复数的分类解决相关问题.

关联题目数量: 1. 关联题目id: 000164

\item K0512005B|D05004B|理解共轭复数的概念.

关联题目数量: 4. 关联题目id: 000163, 000165, 000170, 000174

\item K0512006B|D05004B|经历共轭复数的性质的推导过程.

关联题目数量: 0. 关联题目id: 

\item 本课时题目数量: 6. 关联题目id: 000163, 000164, 000165, 000170, 000174, 004113

\end{itemize}

\section*{第5单元, 第13课时}
起始页码: 必修第二册P143; 终止页码: 必修第二册P145.
\begin{itemize}
\item K0513001B|D05004B|知道复平面、实轴、虚轴的概念.

关联题目数量: 0. 关联题目id: 

\item K0513002B|D05004B|掌握利用共轭描述一个复数是实数的方法.

关联题目数量: 0. 关联题目id: 

\item K0513003B|D05004B|理解复数与复平面上点的对应关系.

关联题目数量: 2. 关联题目id: 000163, 000164

\item K0513004B|D05004B|理解复数与复平面上向量间的对应关系并能用向量方法分析和解决与复数有关的垂直和平行问题.

关联题目数量: 1. 关联题目id: 000162

\item K0513005B|D05004B|理解复数的线性运算与向量线性运算之间的同构关系, 并能用复数的运算解决与平行四边形有关的几何问题.

关联题目数量: 0. 关联题目id: 

\item 本课时题目数量: 3. 关联题目id: 000162, 000163, 000164

\end{itemize}

\section*{第5单元, 第14课时}
起始页码: 必修第二册P145; 终止页码: 必修第二册P148.
\begin{itemize}
\item K0514001B|D05004B|了解复数模的概念.

关联题目数量: 2. 关联题目id: 000170, 000490

\item K0514002B|D05004B|懂得复数模的几何意义及复数的模与向量的模的关系.

关联题目数量: 2. 关联题目id: 000169, 003758

\item K0514003B|D05004B|会证明复数的模的共轭性质及与乘除法可交换的性质.

关联题目数量: 0. 关联题目id: 

\item K0514004B|D05005B|会用复数模的性质计算具体复数的模.

关联题目数量: 4. 关联题目id: 000163, 000167, 000171, 000173

\item K0514005B|D05004B|能运用复数的模的共轭性质判断复数是否是实数.

关联题目数量: 0. 关联题目id: 

\item K0514006B|D05004B|知道复数模的三角不等式.

关联题目数量: 0. 关联题目id: 

\item K0514007B|D05004B|能用平面几何的语言表达复数的差的模的意义, 并能解决相关问题.

关联题目数量: 0. 关联题目id: 

\item 本课时题目数量: 8. 关联题目id: 000163, 000167, 000169, 000170, 000171, 000173, 000490, 003758

\end{itemize}

\section*{第5单元, 第15课时}
起始页码: 必修第二册P150; 终止页码: 必修第二册P152.
\begin{itemize}
\item K0515001B|D05004B|了解复数范围内实数的平方根的概念, 知道其与实数范围内相应问题的异同.

关联题目数量: 0. 关联题目id: 

\item K0515002B|D05004B|会求实数在复数范围内的平方根.

关联题目数量: 0. 关联题目id: 

\item K0515003B|D05004B|理解复数范围内实系数一元二次方程根的情况, 并会用配方法和求根公式求其根.

关联题目数量: 3. 关联题目id: 000168, 003758, 004167

\item K0515004B|D05004B|能在复数范围内分解实系数二次三项式.

关联题目数量: 0. 关联题目id: 

\item K0515005B|D05004B|知道实系数一元二次方程的共轭虚根的性质.

关联题目数量: 2. 关联题目id: 000172, 004167

\item K0515006B|D05004B|会验证韦达定理对任意实系数一元二次方程均成立.

关联题目数量: 0. 关联题目id: 

\item K0515007B|D05004B|能运用韦达定理解决一些简单的实系数一元二次方程的问题.

关联题目数量: 1. 关联题目id: 000172

\item 本课时题目数量: 4. 关联题目id: 000168, 000172, 003758, 004167

\end{itemize}

\section*{第5单元, 第16课时}
起始页码: 必修第二册P154; 终止页码: 必修第二册P156.
\begin{itemize}
\item K0516001B|D05005B|回顾任意角的概念, 了解复数的辐角的概念.

关联题目数量: 0. 关联题目id: 

\item K0516002B|D05005B|知道复数的辐角主值的概念.

关联题目数量: 0. 关联题目id: 

\item K0516003B|D05005B|理解复数的三角形式, 懂得其与复数的代数形式的区别与联系.

关联题目数量: 0. 关联题目id: 

\item K0516004B|D05005B|会通过计算复数的模和辐角用三角形式表示复数.

关联题目数量: 0. 关联题目id: 

\item 本课时题目数量: 0. 关联题目id: 

\end{itemize}

\section*{第5单元, 第17课时}
起始页码: 必修第二册P156; 终止页码: 必修第二册P160.
\begin{itemize}
\item K0517001B|D05005B|经历推导三角形式下复数的乘法公式的过程.

关联题目数量: 0. 关联题目id: 

\item K0517002B|D05005B|掌握三角形式下复数的乘法公式, 会用三角形式计算复数的乘法.

关联题目数量: 0. 关联题目id: 

\item K0517003B|D05005B|理解三角形式下复数乘法运算的几何意义.

关联题目数量: 0. 关联题目id: 

\item K0517004B|D05005B|经历三角形式下复数的除法公式的推导过程, 理解三角形式下除法运算的几何意义.

关联题目数量: 0. 关联题目id: 

\item K0517005B|D05005B|掌握用复数三角形式表示的复数的除法运算公式,会用三角形式计算除法.

关联题目数量: 0. 关联题目id: 

\item K0517006B|D05005B|掌握三角形式下复数的乘方运算公式, 并能进行简单的运算.

关联题目数量: 0. 关联题目id: 

\item K0517007B|D05005B|掌握三角形式下复数的开方运算公式, 并能进行正整数次方根的运算.

关联题目数量: 0. 关联题目id: 

\item 本课时题目数量: 0. 关联题目id: 

\end{itemize}

\section*{第6单元, 第1课时}
起始页码: 必修第三册P2; 终止页码: 必修第三册P4.
\begin{itemize}
\item K0601001B|D06001B|经历从现实情境中抽象平面特征的过程, 会用图形和符号表示平面.

关联题目数量: 0. 关联题目id: 

\item K0601002B|D06001B|直观认识和理解空间中点与直线的位置关系, 并能用文字、图形和符号表示.

关联题目数量: 0. 关联题目id: 

\item K0601003B|D06001B|直观认识和理解空间中点与平面的位置关系, 并能用文字、图形和符号表示.

关联题目数量: 0. 关联题目id: 

\item K0601004B|D06001B|借助集合的包含关系, 直观认识和理解空间中直线与平面的位置关系, 并能用文字(如直线在平面上、平面经过直线等)、图形和符号表示.

关联题目数量: 1. 关联题目id: 001610

\item K0601005B|D06001B|以长方体等较为熟悉的几何体作为载体, 理解公理1, 并能用文字、图形及符号语言表示.

关联题目数量: 0. 关联题目id: 

\item K0601006B|D06001B|会在简单情形下利用公理1说明直线在平面上.

关联题目数量: 0. 关联题目id: 

\item K0601007B|D06001B|知道公理与命题的区别.

关联题目数量: 0. 关联题目id: 

\item 本课时题目数量: 1. 关联题目id: 001610

\end{itemize}

\section*{第6单元, 第2课时}
起始页码: 必修第三册P4; 终止页码: 必修第三册P6.
\begin{itemize}
\item K0602001B|D06001B|通过对现实情境的观察和实验操作, 正确理解公理2, 会用文字、图形语言表述公理2.

关联题目数量: 0. 关联题目id: 

\item K0602002B|D06001B|掌握三个确定平面的推论的内容, 能用文字、图形和符号语言表示三个推论, 并能证明三个推论.

关联题目数量: 0. 关联题目id: 

\item K0602003B|D06001B|知道公理2及推论均为确定平面的依据, 会在简单情形下运用它们判断或证明点或直线共面的问题.

关联题目数量: 2. 关联题目id: 000177, 001595

\item 本课时题目数量: 2. 关联题目id: 000177, 001595

\end{itemize}

\section*{第6单元, 第3课时}
起始页码: 必修第三册P6; 终止页码: 必修第三册P8.
\begin{itemize}
\item K0603001B|D06001B|借助实例理解感受空间中相交平面的位置关系, 理解公理3, 并能用文字、图形及符号语言表示.

关联题目数量: 0. 关联题目id: 

\item K0603002B|D06001B|借助实例感受空间中两个不同平面的位置关系, 会用图形和符号语言表示两个不同平面的两种位置关系.

关联题目数量: 1. 关联题目id: 001642

\item K0603003B|D06001B|能在简单情形下, 通过确定交集中的不同两点画出两相交平面的交线, 并会用文字或符号语言加以说明.

关联题目数量: 0. 关联题目id: 

\item K0603004B|D06001B|会在简单情形下运用公理3证明三点共线.

关联题目数量: 0. 关联题目id: 

\item K0603005B|D06001B|能作出给定平面与正方体表面的交线.

关联题目数量: 0. 关联题目id: 

\item 本课时题目数量: 1. 关联题目id: 001642

\end{itemize}

\section*{第6单元, 第4课时}
起始页码: 必修第三册P8; 终止页码: 必修第三册P11.
\begin{itemize}
\item K0604001B|D06001B|回顾并掌握斜二测画法的画图规则及步骤.

关联题目数量: 0. 关联题目id: 

\item K0604002B|D06001B|能用斜二测画法画出简单直线型平面图形的直观图.

关联题目数量: 0. 关联题目id: 

\item K0604003B|D06001B|能用斜二测画法画出简单直线型空间图形的直观图.

关联题目数量: 0. 关联题目id: 

\item 本课时题目数量: 0. 关联题目id: 

\end{itemize}

\section*{第6单元, 第5课时}
起始页码: 必修第三册P12; 终止页码: 必修第三册P15.
\begin{itemize}
\item K0605001B|D06002B|观察实际情境, 类比平面上平行线的传递性, 将两条直线平行关系的传递性从平面推广到空间, 进而理解公理4.

关联题目数量: 0. 关联题目id: 

\item K0605002B|D06002B|会用符号语言表达公理4.

关联题目数量: 0. 关联题目id: 

\item K0605003B|D06002B|能在简单的情形下用公理4证明空间两条直线平行.

关联题目数量: 1. 关联题目id: 000177

\item K0605004B|D06002B|经历等角定理的证明过程, 掌握等角定理及其两个推论.

关联题目数量: 0. 关联题目id: 

\item K0605005B|D06002B|理解并能运用等角定理证明空间两个角相等.

关联题目数量: 0. 关联题目id: 

\item K0605006B|D06002B|知道空间四边形的相关概念.

关联题目数量: 0. 关联题目id: 

\item 本课时题目数量: 1. 关联题目id: 000177

\end{itemize}

\section*{第6单元, 第6课时}
起始页码: 必修第三册P15; 终止页码: 必修第三册P18.
\begin{itemize}
\item K0606001B|D06002B|通过观察生活实景与长方体模型, 抽象形成异面直线的概念.

关联题目数量: 1. 关联题目id: 001598

\item K0606002B|D06002B|会用反证法证明两条直线是异面直线.

关联题目数量: 0. 关联题目id: 

\item K0606003B|D06002B|知道空间直线与直线的位置关系的分类.

关联题目数量: 4. 关联题目id: 001595, 001598, 001605, 001610

\item K0606004B|D06002B|掌握两条异面直线的一般画法.

关联题目数量: 0. 关联题目id: 

\item K0606005B|D06002B|理解并能证明异面直线判定定理.

关联题目数量: 0. 关联题目id: 

\item K0606006B|D06002B|会用异面直线判定定理证明两条直线是异面直线.

关联题目数量: 0. 关联题目id: 

\item K0606007B|D06002B|知道四面体的相关概念.

关联题目数量: 0. 关联题目id: 

\item 本课时题目数量: 4. 关联题目id: 001595, 001598, 001605, 001610

\end{itemize}

\section*{第6单元, 第7课时}
起始页码: 必修第三册P18; 终止页码: 必修第三册P22.
\begin{itemize}
\item K0607001B|D06002B|经历异面直线所成角概念的形成过程, 理解异面直线所成角的定义, 知道等角定理在定义异面直线所成角时所起的作用.

关联题目数量: 0. 关联题目id: 

\item K0607002B|D06002B|知道异面直线所成角的范围.

关联题目数量: 0. 关联题目id: 

\item K0607003B|D06002B|知道异面直线相互垂直的定义及推广的两直线垂直的符号表示.

关联题目数量: 0. 关联题目id: 

\item K0607004B|D06002B|会在简单的情形中通过平移(如在长方体表面平移及借助中位线)求两条异面直线所成角的大小, 初步体会将空间问题转化为平面问题的思想方法.

关联题目数量: 2. 关联题目id: 000175, 000211

\item 本课时题目数量: 2. 关联题目id: 000175, 000211

\end{itemize}

\section*{第6单元, 第8课时}
起始页码: 必修第三册P23; 终止页码: 必修第三册P25.
\begin{itemize}
\item K0608001B|D06003B|通过对现实情境及熟悉的空间几何体的观察, 感知并用反证法证明直线与平面平行的判定定理, 并能用符号语言表示该判定定理.

关联题目数量: 1. 关联题目id: 000179

\item K0608002B|D06003B|能在具体的情境中用直线与平面平行的判定定理证明简单的相关问题.

关联题目数量: 4. 关联题目id: 000176, 000185, 000186, 000187

\item K0608003B|D06003B|理解并证明直线与平面平行的性质定理, 并能用符号及图形语言表示该性质定理.

关联题目数量: 0. 关联题目id: 

\item K0608004B|D06003B|能在具体的情境中用直线与平面平行的性质定理证明简单的相关问题(如借助平面交线作已知直线的平行线).

关联题目数量: 2. 关联题目id: 000184, 000186

\item 本课时题目数量: 6. 关联题目id: 000176, 000179, 000184, 000185, 000186, 000187

\end{itemize}

\section*{第6单元, 第9课时}
起始页码: 必修第三册P26; 终止页码: 必修第三册P29.
\begin{itemize}
\item K0609001B|D06003B|从现实情境中抽象、形成直线与平面垂直的概念, 并能用图形和符号语言表示.

关联题目数量: 2. 关联题目id: 000178, 000184

\item K0609002B|D06003B|通过实验操作与实际经验, 发现并理解直线与平面垂直的判定定理.

关联题目数量: 0. 关联题目id: 

\item K0609003B|D06003B|在简单情境中, 能运用直线与平面垂直的判定定理证明直线与平面的垂直关系.

关联题目数量: 6. 关联题目id: 000182, 000183, 000187, 000191, 000192, 000212

\item K0609004B|D06003B|理解并经历用反证法证明直线与平面垂直的性质定理的过程, 能用符号语言表示该性质定理.

关联题目数量: 0. 关联题目id: 

\item K0609005B|D06003B|直观上感受过空间一点作已知直线的垂面的存在性与唯一性.

关联题目数量: 0. 关联题目id: 

\item K0609006B|D06003B|直观上感受过空间一点作已知平面的垂线的存在性与唯一性.

关联题目数量: 0. 关联题目id: 

\item K0609007B|D06003B|知道点到平面的距离的概念, 并能解决简单的相关问题.

关联题目数量: 2. 关联题目id: 000179, 000193

\item K0609008B|D06003B|知道直线到与它平行的平面的距离的概念, 理解定义中距离与点的选取无关的原因, 并能根据定义将问题转化为点到平面的距离并求解.

关联题目数量: 0. 关联题目id: 

\item 本课时题目数量: 10. 关联题目id: 000178, 000179, 000182, 000183, 000184, 000187, 000191, 000192, 000193, 000212

\end{itemize}

\section*{第6单元, 第10课时}
起始页码: 必修第三册P29; 终止页码: 必修第三册P31.
\begin{itemize}
\item K0610001B|D06003B|知道直线与平面斜交的相关概念, 会用图形语言表示.

关联题目数量: 2. 关联题目id: 000178, 000179

\item K0610002B|D06003B|知道直线、线段在平面上的投影(射影)的概念.

关联题目数量: 1. 关联题目id: 001629

\item K0610003B|D06003B|经历直线与平面所成角的概念的形成过程, 知道直线与平面所成角的概念.

关联题目数量: 1. 关联题目id: 001629

\item K0610004B|D06003B|继续感悟用平面方法解决空间问题的思想, 能在具体的情形下求出直线与平面所成角的大小.

关联题目数量: 2. 关联题目id: 000194, 000212

\item K0610005B|D06003B|会用代数及三角方法分析求解直线与平面所成角的图形中线段成角的数量关系的相关问题.

关联题目数量: 0. 关联题目id: 

\item 本课时题目数量: 5. 关联题目id: 000178, 000179, 000194, 000212, 001629

\end{itemize}

\section*{第6单元, 第11课时}
起始页码: 必修第三册P31; 终止页码: 必修第三册P34.
\begin{itemize}
\item K0611001B|D06003B|理解三垂线定理, 能用符号及图形语言表示该定理并加以证明.

关联题目数量: 0. 关联题目id: 

\item K0611002B|D06003B|会选择合适的平面, 用三垂线定理判断并证明异面直线间是否垂直.

关联题目数量: 0. 关联题目id: 

\item K0611003B|D06003B|继续感悟用平面方法解决空间问题的思想, 能在现实情境中运用三垂线定理将空间问题平面化后加以解决.

关联题目数量: 1. 关联题目id: 001629

\item 本课时题目数量: 1. 关联题目id: 001629

\end{itemize}

\section*{第6单元, 第12课时}
起始页码: 必修第三册P35; 终止页码: 必修第三册P37.
\begin{itemize}
\item K0612001B|D06004B|经历由直线间或线面间的平行关系出发探索两个平面的平行关系的过程, 发现并用反证法证明两个平面平行的判定定理.

关联题目数量: 1. 关联题目id: 004092

\item K0612002B|D06004B|能在长方体中运用两个平面平行的判定定理证明两个平面平行.

关联题目数量: 2. 关联题目id: 000180, 001642

\item K0612003B|D06004B|理解并能证明两个平面平行的性质定理.

关联题目数量: 0. 关联题目id: 

\item K0612004B|D06004B|能在具体的情形中运用两个平面平行的性质定理证明简单的相关问题(如两直线平行).

关联题目数量: 1. 关联题目id: 000181

\item K0612005B|D06004B|经历类比点到平面的距离与直线到平面的距离的定义获得两个平行平面间的距离的定义的过程, 理解定义中的距离与点的选取方式无关.

关联题目数量: 0. 关联题目id: 

\item K0612006B|D06004B|在长方体情境中, 能找寻合适的点计算与表面有关的平面之间的距离.

关联题目数量: 0. 关联题目id: 

\item 本课时题目数量: 4. 关联题目id: 000180, 000181, 001642, 004092

\end{itemize}

\section*{第6单元, 第13课时}
起始页码: 必修第三册P37; 终止页码: 必修第三册P41.
\begin{itemize}
\item K0613001B|D06004B|结合现实情境中的实例, 抽象形成二面角的概念, 能用图形及符号语言表示二面角.

关联题目数量: 0. 关联题目id: 

\item K0613002B|D06004B|知道二面角的平面角的概念, 理解二面角的大小与点的选取无关, 并能根据定义作出二面角的平面角.

关联题目数量: 0. 关联题目id: 

\item K0613003B|D06004B|能在容易作出平面角的情境中求二面角的大小.

关联题目数量: 0. 关联题目id: 

\item K0613004B|D06004B|知道二面角的取值范围.

关联题目数量: 0. 关联题目id: 

\item K0613005B|D06004B|了解平面与平面垂直的概念, 并能用图形及符号语言表示.

关联题目数量: 1. 关联题目id: 000181

\item K0613006B|D06004B|经历面面垂直的判定定理的发现过程, 并经历定理的证明.

关联题目数量: 0. 关联题目id: 

\item K0613007B|D06004B|会用面面垂直的判定定理判断并证明面面垂直.

关联题目数量: 2. 关联题目id: 000188, 000190

\item K0613008B|D06004B|经历面面垂直的性质定理的发现过程, 并经历定理的证明.

关联题目数量: 0. 关联题目id: 

\item K0613009B|D06004B|会用面面垂直的性质定理判断并证明线面垂直.

关联题目数量: 1. 关联题目id: 000189

\item 本课时题目数量: 4. 关联题目id: 000181, 000188, 000189, 000190

\end{itemize}

\section*{第6单元, 第14课时}
起始页码: 必修第三册P42; 终止页码: 必修第三册P45.
\begin{itemize}
\item K0614001B|D06005B|通过猜测与归纳, 形成两条异面直线的公垂线的概念.

关联题目数量: 0. 关联题目id: 

\item K0614002B|D06005B|在定理的论证过程中, 体会两条异面直线的公垂线的存在性与唯一性.

关联题目数量: 0. 关联题目id: 

\item K0614003B|D06005B|知道两条异面直线的距离的概念.

关联题目数量: 0. 关联题目id: 

\item K0614004B|D06005B|能在简单的情形中识别出异面直线的公垂线段并求出两异面直线的距离.

关联题目数量: 0. 关联题目id: 

\item K0614005B|D06005B|能在长方体情境中将求异面直线距离的问题转化为求线面距离、面面距离的问题.

关联题目数量: 0. 关联题目id: 

\item K0614006B|D06005B|能在以等腰三角形为面的四面体情境中, 通过构造二面角的平面角, 将空间问题转化为平面问题, 构造出异面直线的公垂线段并求出异面直线的距离.

关联题目数量: 0. 关联题目id: 

\item 本课时题目数量: 0. 关联题目id: 

\end{itemize}

\section*{第6单元, 第15课时}
起始页码: 必修第三册P54; 终止页码: 必修第三册P56.
\begin{itemize}
\item K0615001B|D06006B|了解多面体的概念(含面, 棱, 顶点).

关联题目数量: 0. 关联题目id: 

\item K0615002B|D06006B|理解棱柱的概念, 能用数学语言刻画棱柱的特征.

关联题目数量: 1. 关联题目id: 001682

\item K0615003B|D06006B|了解和棱柱有关的名称, 含底面, 侧面, 侧棱, 高.

关联题目数量: 0. 关联题目id: 

\item K0615004B|D06006B|理解根据侧棱和底面是否垂直能将棱柱分为斜棱柱与直棱柱, 知道正棱柱的概念.

关联题目数量: 1. 关联题目id: 001682

\item K0615005B|D06006B|知道棱柱的按底面边数的分类方法, 能用符号规范地表示棱柱.

关联题目数量: 0. 关联题目id: 

\item K0615006B|D06006B|在棱柱中能进行简单的线线关系, 线面关系的分析和论证.

关联题目数量: 0. 关联题目id: 

\item K0615007B|D06006B|了解圆柱及相关概念(含轴, 底面, 侧面, 母线, 高), 能用数学语言刻画圆柱的形成过程.

关联题目数量: 1. 关联题目id: 000196

\item K0615008B|D06006B|直观感知圆柱的过轴与垂直于轴的截面的形状, 并经历根据圆柱的形成简要证明这两类截面形状的过程.

关联题目数量: 0. 关联题目id: 

\item 本课时题目数量: 2. 关联题目id: 000196, 001682

\end{itemize}

\section*{第6单元, 第16课时}
起始页码: 必修第三册P56; 终止页码: 必修第三册P58.
\begin{itemize}
\item K0616001B|D06006B|借助实物直观地了解祖暅原理的内容, 能脱离文本独立复述祖暅原理.

关联题目数量: 0. 关联题目id: 

\item K0616002B|D06006B|经历利用祖暅原理和已知的长方体体积公式推导一般柱体体积公式的过程.

关联题目数量: 0. 关联题目id: 

\item K0616003B|D06006B|会用柱体的体积公式计算直棱柱, 圆柱与斜棱柱(已知母线长及母线与底面的夹角)的体积.

关联题目数量: 2. 关联题目id: 000202, 000204

\item K0616004B|D06006B|会将现实情境中的物体合理抽象为柱体(或柱体的组合)计算体积.

关联题目数量: 0. 关联题目id: 

\item 本课时题目数量: 2. 关联题目id: 000202, 000204

\end{itemize}

\section*{第6单元, 第17课时}
起始页码: 必修第三册P58; 终止页码: 必修第三册P60.
\begin{itemize}
\item K0617001B|D06006B|直观上了解空间几何体的表面积的概念.

关联题目数量: 0. 关联题目id: 

\item K0617002B|D06006B|知道直柱体的侧面可以展开为矩形.

关联题目数量: 1. 关联题目id: 000210

\item K0617003B|D06006B|掌握直棱柱的侧面积公式和表面积公式的推导.

关联题目数量: 0. 关联题目id: 

\item K0617004B|D06006B|能用直棱柱的侧面积公式和表面积公式计算数学情境或现实情境中的直棱柱的侧面积与表面积.

关联题目数量: 0. 关联题目id: 

\item K0617005B|D06006B|掌握圆柱的侧面积公式和表面积公式的推导.

关联题目数量: 0. 关联题目id: 

\item K0617006B|D06006B|能用直圆柱的侧面积公式和表面积公式计算数学情境或现实情境中的直棱柱的侧面积与表面积.

关联题目数量: 1. 关联题目id: 000201

\item K0617007B|D06006B|能通过分割与拼接计算简单的由若干个直柱体与圆柱组成的组合体的表面积.

关联题目数量: 0. 关联题目id: 

\item 本课时题目数量: 2. 关联题目id: 000201, 000210

\end{itemize}

\section*{第6单元, 第18课时}
起始页码: 必修第三册P62; 终止页码: 必修第三册P64.
\begin{itemize}
\item K0618001B|D06006B|理解棱锥的概念, 能用数学语言刻画棱锥的特征.

关联题目数量: 1. 关联题目id: 001695

\item K0618002B|D06006B|了解和棱锥有关的名称, 含底面, 侧面, 侧棱, 顶点, 高, 正棱锥.

关联题目数量: 4. 关联题目id: 000207, 000216, 000217, 001695

\item K0618003B|D06006B|知道棱锥的按底面边数的分类方法, 能用符号规范地表示棱锥.

关联题目数量: 0. 关联题目id: 

\item K0618004B|D06006B|在棱锥中能进行简单的线线关系, 线面关系的分析和论证.

关联题目数量: 0. 关联题目id: 

\item K0618005B|D06006B|了解圆锥及相关概念(轴, 顶点, 底面, 侧面, 母线, 高), 能用数学语言刻画圆锥的形成过程.

关联题目数量: 8. 关联题目id: 000196, 000197, 000198, 000203, 000206, 000208, 000209, 000214

\item K0618006B|D06006B|直观感知圆锥垂直于轴的截面的形状, 并经历根据圆锥的形成确定截面形状的过程.

关联题目数量: 1. 关联题目id: 000197

\item K0618007B|D06006B|了解台体的概念及相关名称(圆台, 棱台, 正棱台), 知道台体的问题可以转换为锥体解决.

关联题目数量: 1. 关联题目id: 000196

\item 本课时题目数量: 12. 关联题目id: 000196, 000197, 000198, 000203, 000206, 000207, 000208, 000209, 000214, 000216, 000217, 001695

\end{itemize}

\section*{第6单元, 第19课时}
起始页码: 必修第三册P64; 终止页码: 必修第三册P68.
\begin{itemize}
\item K0619001B|D06006B|经历将三棱锥补完为三棱柱, 利用祖暅原理说明三棱锥的体积是同底同侧棱的三棱柱的体积的$1/3$的过程.

关联题目数量: 0. 关联题目id: 

\item K0619002B|D06006B|利用祖暅原理和三棱锥的体积公式, 推导一般锥体的体积公式.

关联题目数量: 0. 关联题目id: 

\item K0619003B|D06006B|能合理选择底面和高, 在数学情境中计算锥体的体积.

关联题目数量: 6. 关联题目id: 000199, 000211, 000215, 000216, 000217, 000298

\item K0619004B|D06006B|会用体积算两次的方法求一些难以作出垂线的三棱锥的高.

关联题目数量: 0. 关联题目id: 

\item K0619005B|D06006B|经历从锥体的体积公式推导台体的体积公式的过程.

关联题目数量: 0. 关联题目id: 

\item 本课时题目数量: 6. 关联题目id: 000199, 000211, 000215, 000216, 000217, 000298

\end{itemize}

\section*{第6单元, 第20课时}
起始页码: 必修第三册P68; 终止页码: 必修第三册P69.
\begin{itemize}
\item K0620001B|D06006B|了解棱锥的侧面积及表面积的计算方法.

关联题目数量: 0. 关联题目id: 

\item K0620002B|D06006B|了解棱锥的侧面能展开为一个扇形, 其圆心角由底面周长与母线长的比值确定.

关联题目数量: 0. 关联题目id: 

\item K0620003B|D06006B|掌握圆锥的侧面积公式和表面积公式的推导.

关联题目数量: 0. 关联题目id: 

\item K0620004B|D06006B|会在简单的具体情境中计算锥体的表面积.

关联题目数量: 1. 关联题目id: 000217

\item K0620005B|D06006B|能利用圆锥的侧面展开图研究圆锥表面的最短距离问题.

关联题目数量: 0. 关联题目id: 

\item 本课时题目数量: 1. 关联题目id: 000217

\end{itemize}

\section*{第6单元, 第21课时}
起始页码: 必修第三册P72; 终止页码: 必修第三册P76.
\begin{itemize}
\item K0621001B|D06006B|了解多面体的概念及命名方式.

关联题目数量: 0. 关联题目id: 

\item K0621002B|D06006B|会用反证法证明面数最少的多面体是四面体.

关联题目数量: 0. 关联题目id: 

\item K0621003B|D06006B|能复述正多面体的概念.

关联题目数量: 0. 关联题目id: 

\item K0621004B|D06006B|知道正多面体有且仅有五种, 直观了解每种正多面体的空间形象.

关联题目数量: 0. 关联题目id: 

\item K0621005B|D06006B|了解旋转体的概念与相关概念(轴, 旋转面), 理解旋转体和多面体的区别.

关联题目数量: 2. 关联题目id: 000195, 000196

\item 本课时题目数量: 2. 关联题目id: 000195, 000196

\end{itemize}

\section*{第6单元, 第22课时}
起始页码: 必修第三册P78; 终止页码: 必修第三册P80.
\begin{itemize}
\item K0622001B|D06006B|从旋转体的角度理解球的概念与相关的概念(球面, 球心, 半径, 直径).

关联题目数量: 1. 关联题目id: 000203

\item K0622002B|D06006B|知道球有丰富的对称性, 类比圆理解球的用距离刻画的等价定义.

关联题目数量: 0. 关联题目id: 

\item K0622003B|D06006B|只管感知平面截球所得的截面是圆面, 并能论证该结果.

关联题目数量: 0. 关联题目id: 

\item K0622004B|D06006B|知道球面的大圆和小圆的概念.

关联题目数量: 0. 关联题目id: 

\item K0622005B|D06006B|会根据球心到平面的距离确定小圆的半径 .

关联题目数量: 0. 关联题目id: 

\item K0622006B|D06006B|将地球表面抽象为球面, 通过对线线, 线面关系的分析理解经纬度的数学含义分别是二面角与线面角的大小.

关联题目数量: 0. 关联题目id: 

\item 本课时题目数量: 1. 关联题目id: 000203

\end{itemize}

\section*{第6单元, 第23课时}
起始页码: 必修第三册P80; 终止页码: 必修第三册P82.
\begin{itemize}
\item K0623001B|D06006B|经历利用祖暅原理、构造一个与半球体积相等的几何体推导得到球的体积公式的过程.

关联题目数量: 0. 关联题目id: 

\item K0623002B|D06006B|熟记球体的体积公式, 并能用球体的体积公式计算数学情境与现实情境中球体的体积.

关联题目数量: 2. 关联题目id: 000213, 000215

\item K0623003B|D06006B|类比圆的面积与周长的关系, 初步经历用剖分为以球心为顶点的小锥体的方式得到球面的表面积这一极限过程.

关联题目数量: 0. 关联题目id: 

\item K0623004B|D06006B|熟记球的表面积公式, 能用球的表面积公式计算数学情境与现实情境中球的表面积.

关联题目数量: 4. 关联题目id: 000200, 000201, 000205, 000213

\item 本课时题目数量: 5. 关联题目id: 000200, 000201, 000205, 000213, 000215

\end{itemize}

\section*{第6单元, 第24课时}
起始页码: 选择性必修第一册P90; 终止页码: 选择性必修第一册P92.
\begin{itemize}
\item K0624001X|D06007X|理解空间向量共面的概念, 知道两个向量总是共面的, 对更多共面的空间向量的研究可以转化为对同一平面上的向量的研究.

关联题目数量: 1. 关联题目id: 000291

\item K0624002X|D06007X|通过平面向量的复习, 建立起平面向量和空间向量的密切联系, 把平面向量上已经建立起来的向量的有关概念及向量的线性运算和运算律迁移到空间向量.

关联题目数量: 0. 关联题目id: 

\item K0624003X|D06007X|会在典型空间几何体中用向量的线性组合表示其他向量.

关联题目数量: 0. 关联题目id: 

\item 本课时题目数量: 1. 关联题目id: 000291

\end{itemize}

\section*{第6单元, 第25课时}
起始页码: 选择性必修第一册P92; 终止页码: 选择性必修第一册P94.
\begin{itemize}
\item K0625001X|D06007X|理解空间两个向量的数量积的定义可转化为平面上两个向量的数量积.

关联题目数量: 1. 关联题目id: 000292

\item K0625002X|D06007X|知道空间向量的数量积的交换律, 与实数相乘后的结合律以及分配律依然成立.

关联题目数量: 0. 关联题目id: 

\item K0625003X|D06007X|理解空间中两个向量垂直和平行的充要条件, 并能使用该条件解决与垂直或平行有关的问题.

关联题目数量: 3. 关联题目id: 000301, 000302, 000303

\item K0625004X|D06007X|在简单的空间几何体中类比平面向量的应用, 运用空间向量的数量积运算及线性运算解决一些立体几何问题.

关联题目数量: 1. 关联题目id: 000301

\item 本课时题目数量: 4. 关联题目id: 000292, 000301, 000302, 000303

\end{itemize}

\section*{第6单元, 第26课时}
起始页码: 选择性必修第一册P96; 终止页码: 选择性必修第一册P99.
\begin{itemize}
\item K0626001X|D06008X|经历从平面向量基本定理到向量共面的充要条件转化的过程.

关联题目数量: 0. 关联题目id: 

\item K0626002X|D06008X|会用向量共面的充要条件, 用向量方法证明直线与平面垂直的判定定理.

关联题目数量: 0. 关联题目id: 

\item K0626003X|D06008X|经历向量基本定理从平面推广到空间的过程, 掌握空间向量基本定理.

关联题目数量: 0. 关联题目id: 

\item K0626004X|D06008X|在熟悉的空间图形中, 能把向量在不同的基下进行线性表示, 初步经历基变换的过程.

关联题目数量: 0. 关联题目id: 

\item 本课时题目数量: 0. 关联题目id: 

\end{itemize}

\section*{第6单元, 第27课时}
起始页码: 选择性必修第一册P101; 终止页码: 选择性必修第一册P105.
\begin{itemize}
\item K0627001X|D06008X|类比平面直角坐标系, 了解空间直角坐标系, 并知道相关概念(坐标原点, 横轴, 纵轴, 竖轴, 坐标平面, 卦限(不作第几卦限的区分)).

关联题目数量: 0. 关联题目id: 

\item K0627002X|D06008X|会通过构造长方体的方式确定空间一点的坐标.

关联题目数量: 0. 关联题目id: 

\item K0627003X|D06008X|理解坐标有密切联系的两个点在空间直角坐标系中的位置关系(关于坐标平面对称).

关联题目数量: 0. 关联题目id: 

\item K0627004X|D06008X|类比平面向量的坐标表示, 了解位置向量的概念, 掌握空间向量的坐标表示.

关联题目数量: 0. 关联题目id: 

\item K0627005X|D06008X|类比平面向量在坐标表示下的运算规则, 推导并熟记空间向量在坐标表示下的运算规则.

关联题目数量: 0. 关联题目id: 

\item K0627006X|D06009X|掌握空间坐标表示下向量垂直和平行的充要条件.

关联题目数量: 0. 关联题目id: 

\item K0627007X|D06009X|能用空间坐标表示下的向量方法求两直线的夹角及证明直线垂直.

关联题目数量: 2. 关联题目id: 000297, 000304

\item 本课时题目数量: 2. 关联题目id: 000297, 000304

\end{itemize}

\section*{第6单元, 第28课时}
起始页码: 选择性必修第一册P107; 终止页码: 选择性必修第一册P109.
\begin{itemize}
\item K0628001X|D06009X|知道直线的方向向量和平面的法向量的概念, 理解刻画直线和平面方向时的异同.

关联题目数量: 0. 关联题目id: 

\item K0628002X|D06009X|能将直线的夹角问题转化为方向向量的夹角问题.

关联题目数量: 0. 关联题目id: 

\item K0628003X|D06009X|经历用空间向量方法证明三垂线定理的过程.

关联题目数量: 0. 关联题目id: 

\item K0628004X|D06009X|了解用空间向量刻画直线与平面垂直、平行的方式, 并能应用于熟悉的几何体.

关联题目数量: 3. 关联题目id: 000294, 000299, 000305

\item K0628005X|D06009X|了解用空间向量刻画平面与平面垂直、平行的方式, 并能应用于熟悉的几何体.

关联题目数量: 0. 关联题目id: 

\item 本课时题目数量: 3. 关联题目id: 000294, 000299, 000305

\end{itemize}

\section*{第6单元, 第29课时}
起始页码: 选择性必修第一册P109; 终止页码: 选择性必修第一册P111.
\begin{itemize}
\item K0629001X|D06009X|经历利用投影向量的概念推导空间点到平面距离公式的过程, 掌握点到平面的距离公式.

关联题目数量: 3. 关联题目id: 000293, 000296, 000297

\item K0629002X|D06009X|会通过解方程, 在已知平面上三点的坐标的情境中求平面的法向量.

关联题目数量: 0. 关联题目id: 

\item K0629003X|D06009X|能建立空间直角坐标系, 用点到平面的距离公式求空间中点到平面, 直线到平行平面, 平面与平行平面的距离.

关联题目数量: 0. 关联题目id: 

\item 本课时题目数量: 3. 关联题目id: 000293, 000296, 000297

\end{itemize}

\section*{第6单元, 第30课时}
起始页码: 选择性必修第一册P111; 终止页码: 选择性必修第一册P113.
\begin{itemize}
\item K0630001X|D06009X|经历利用直线的方向向量推导空间中直线与直线所成角的公式的过程.

关联题目数量: 0. 关联题目id: 

\item K0630002X|D06009X|会建立空间直角坐标系, 用两直线的夹角公式求熟悉的几何体中两直线所成的角的大小.

关联题目数量: 3. 关联题目id: 000298, 000302, 000304

\item K0630003X|D06009X|经历利用直线的方向向量、平面的法向量推导空间中直线与平面所成角的公式的过程.

关联题目数量: 0. 关联题目id: 

\item K0630004X|D06009X|会建立空间直角坐标系, 用直线与平面的夹角公式求熟悉的几何体中直线与平面所成的角的大小.

关联题目数量: 2. 关联题目id: 000294, 000300

\item 本课时题目数量: 5. 关联题目id: 000294, 000298, 000300, 000302, 000304

\end{itemize}

\section*{第6单元, 第31课时}
起始页码: 选择性必修第一册P113; 终止页码: 选择性必修第一册P115.
\begin{itemize}
\item K0631001X|D06009X|知道两个平面所称的锐二面角的概念, 通过分情况讨论理解锐二面角与法向量夹角的关系.

关联题目数量: 0. 关联题目id: 

\item K0631002X|D06009X|会用向量方法求两个平面所成的二面角的大小(一般有两个不同的大小).

关联题目数量: 5. 关联题目id: 000295, 000296, 000299, 000303, 000305

\item K0631003X|D06009X|会结合直观在两个平面所成的两个二面角中选择恰当的一个作为两个半平面的二面角.

关联题目数量: 2. 关联题目id: 000295, 000299

\item 本课时题目数量: 5. 关联题目id: 000295, 000296, 000299, 000303, 000305

\end{itemize}

\section*{第7单元, 第1课时}
起始页码: 选择性必修第一册P2; 终止页码: 选择性必修第一册P4.
\begin{itemize}
\item K0701001X|D07001X|经历在平面直角坐标系中探索确定直线位置的几何要素, 理解直线的倾斜角和斜率的概念.

关联题目数量: 2. 关联题目id: 000264, 004000

\item K0701002X|D07001X|能对直线的倾斜角与斜率进行互化.

关联题目数量: 3. 关联题目id: 000243, 000247, 000264

\item K0701003X|D07001X|经历用代数方法刻画直线斜率的过程, 掌握过两点的直线斜率的计算公式.

关联题目数量: 1. 关联题目id: 000256

\item K0701004X|D07001X|知道一次函数的一次项系数就是其对应直线的斜率.

关联题目数量: 0. 关联题目id: 

\item 本课时题目数量: 5. 关联题目id: 000243, 000247, 000256, 000264, 004000

\end{itemize}

\section*{第7单元, 第2课时}
起始页码: 选择性必修第一册P6; 终止页码: 选择性必修第一册P8.
\begin{itemize}
\item K0702001X|D07001X|知道截距的概念.

关联题目数量: 3. 关联题目id: 000248, 000252, 000262

\item K0702002X|D07002X|能根据确定一条直线的几何要素, 掌握直线的点斜式方程及其使用范围.

关联题目数量: 2. 关联题目id: 000248, 000252

\item K0702003X|D07002X|能在具体的平面几何问题中求直线的点斜式方程.

关联题目数量: 2. 关联题目id: 000262, 000277

\item K0702004X|D07002X|知道直线的斜截式方程是点斜式方程的特例, 并能在具体的平面几何问题中求直线的斜截式方程.

关联题目数量: 0. 关联题目id: 

\item K0702005X|D07002X|在具体实例中, 能利用直线的点斜式方程、斜截式方程判断给定的点是否在已知直线上.

关联题目数量: 1. 关联题目id: 004115

\item 本课时题目数量: 5. 关联题目id: 000248, 000252, 000262, 000277, 004115

\end{itemize}

\section*{第7单元, 第3课时}
起始页码: 选择性必修第一册P8; 终止页码: 选择性必修第一册P9.
\begin{itemize}
\item K0703001X|D07002X|能根据确定一条直线的几何要素, 掌握直线的两点式方程及其使用范围.

关联题目数量: 0. 关联题目id: 

\item K0703002X|D07002X|能在具体的平面几何问题中求直线的两点式方程.

关联题目数量: 1. 关联题目id: 000258

\item 本课时题目数量: 1. 关联题目id: 000258

\end{itemize}

\section*{第7单元, 第4课时}
起始页码: 选择性必修第一册P10; 终止页码: 选择性必修第一册P11.
\begin{itemize}
\item K0704001X|D07002X|通过具体实例, 知道直线的方程是一个二元一次方程, 并且任意一个二元一次方程都能表示一条直线.

关联题目数量: 0. 关联题目id: 

\item K0704002X|D07002X|掌握直线的一般式方程及其使用范围.

关联题目数量: 0. 关联题目id: 

\item K0704003X|D07002X|能在具体的实例中研究含参数的一般式方程所对应直线的性质.

关联题目数量: 1. 关联题目id: 000263

\item 本课时题目数量: 1. 关联题目id: 000263

\end{itemize}

\section*{第7单元, 第5课时}
起始页码: 选择性必修第一册P11; 终止页码: 选择性必修第一册P13.
\begin{itemize}
\item K0705001X|D07002X|了解平面上直线的法向量的概念.

关联题目数量: 0. 关联题目id: 

\item K0705002X|D07002X|能根据确定一条直线的几何要素, 掌握直线的点法式方程及其使用范围.

关联题目数量: 0. 关联题目id: 

\item K0705003X|D07002X|能在具体的实例中运用直线的点法式方程解决简单的平面几何问题.

关联题目数量: 2. 关联题目id: 000250, 000253

\item K0705004X|D07002X|能通过代数变形理解直线的一般式方程的系数与法向量的联系.

关联题目数量: 0. 关联题目id: 

\item 本课时题目数量: 2. 关联题目id: 000250, 000253

\end{itemize}

\section*{第7单元, 第6课时}
起始页码: 选择性必修第一册P16; 终止页码: 选择性必修第一册P18.
\begin{itemize}
\item K0706001X|D07003X|理解二元一次方程组的解与两条相交直线的交点坐标之间的对应关系, 能用解方程组的方法求两条直线的交点坐标.

关联题目数量: 4. 关联题目id: 000250, 000259, 000639, 000647

\item K0706002X|D07003X|能用代数方法(根据两条直线的方程的系数)讨论两条直线的位置关系(相交、平行或重合).

关联题目数量: 4. 关联题目id: 000250, 000251, 000255, 000265

\item K0706003X|D07003X|能用几何方法(根据两条直线的斜率、法向量)讨论两条直线的位置关系(相交、平行或重合).

关联题目数量: 1. 关联题目id: 000256

\item 本课时题目数量: 8. 关联题目id: 000250, 000251, 000255, 000256, 000259, 000265, 000639, 000647

\end{itemize}

\section*{第7单元, 第7课时}
起始页码: 选择性必修第一册P19; 终止页码: 选择性必修第一册P22.
\begin{itemize}
\item K0707001X|D07003X|掌握用系数表示的两条直线垂直的充要条件.

关联题目数量: 8. 关联题目id: 000244, 000245, 000251, 000253, 000254, 000255, 000261, 004502

\item K0707002X|D07003X|掌握用斜率描述的两条直线垂直的充要条件及其适用范围.

关联题目数量: 0. 关联题目id: 

\item K0707003X|D07003X|经历将两条直线的夹角转化为对应法向量的夹角的过程, 掌握两条直线夹角的余弦公式.

关联题目数量: 0. 关联题目id: 

\item K0707004X|D07003X|会用两条直线夹角的余弦公式求直线的夹角.

关联题目数量: 0. 关联题目id: 

\item K0707005X|D07003X|会用两条直线夹角的余弦公式根据一条直线的方程及夹角反求另一条直线的方程.

关联题目数量: 4. 关联题目id: 000245, 000246, 000257, 000259

\item 本课时题目数量: 11. 关联题目id: 000244, 000245, 000246, 000251, 000253, 000254, 000255, 000257, 000259, 000261, 004502

\end{itemize}

\section*{第7单元, 第8课时}
起始页码: 选择性必修第一册P24; 终止页码: 选择性必修第一册P26.
\begin{itemize}
\item K0708001X|D07003X|通过具体实例, 利用法向量方向上的投影, 探究并求解点到直线的距离, 掌握点到直线的距离公式.

关联题目数量: 0. 关联题目id: 

\item K0708002X|D07003X|根据点到直线距离公式, 推导及掌握两条平行线之间的距离公式.

关联题目数量: 0. 关联题目id: 

\item K0708003X|D07003X|会用点到直线距离公式、两条平行线之间的距离公式求距离.

关联题目数量: 5. 关联题目id: 000249, 000254, 000260, 000261, 000266

\item K0708004X|D07003X|会用平行线间的距离公式及一条直线的方程和距离反求另一条直线的方程.

关联题目数量: 0. 关联题目id: 

\item 本课时题目数量: 5. 关联题目id: 000249, 000254, 000260, 000261, 000266

\end{itemize}

\section*{第7单元, 第9课时}
起始页码: 选择性必修第一册P32; 终止页码: 选择性必修第一册P34.
\begin{itemize}
\item K0709001X|D07004X|回顾直线方程的概念, 结合具体的实例, 理解曲线方程的概念.

关联题目数量: 0. 关联题目id: 

\item K0709002X|D07004X|能在简单的情境中, 判断曲线与方程是否对应.

关联题目数量: 1. 关联题目id: 000267

\item K0709003X|D07004X|在平面直角坐标系中, 根据确定圆的几何要素, 推导并掌握圆的标准方程.

关联题目数量: 2. 关联题目id: 000269, 000283

\item K0709004X|D07004X|会类比求圆的标准方程求其他方式(直接用点的性质)定义的曲线的方程.

关联题目数量: 0. 关联题目id: 

\item K0709005X|D07004X|会用横坐标、纵坐标的范围在圆上截取相应的圆弧.

关联题目数量: 0. 关联题目id: 

\item K0709006X|D07004X|会结合圆的标准方程用垂径定理建立弦长与弦心距的联系.

关联题目数量: 1. 关联题目id: 000286

\item K0709007X|D07004X|会用待定系数法及解方程(组)求圆的标准方程.

关联题目数量: 0. 关联题目id: 

\item 本课时题目数量: 4. 关联题目id: 000267, 000269, 000283, 000286

\end{itemize}

\section*{第7单元, 第10课时}
起始页码: 选择性必修第一册P35; 终止页码: 选择性必修第一册P36.
\begin{itemize}
\item K0710001X|D07004X|通过对圆的标准方程展开整理, 掌握圆的一般方程.

关联题目数量: 0. 关联题目id: 

\item K0710002X|D07004X|知道不含$xy$项且含$x^2, y^2$的项系数相等的二元二次方程表示的曲线类型, 推导并掌握该方程表示圆时系数满足的充要条件.

关联题目数量: 0. 关联题目id: 

\item K0710003X|D07004X|能利用配方法将圆的一般方程化为标准方程.

关联题目数量: 0. 关联题目id: 

\item K0710004X|D07004X|会用待定系数法及解方程(组)求圆的一般方程.

关联题目数量: 0. 关联题目id: 

\item K0710005X|D07004X|能在具体实例中, 选择合适的方法求圆的方程.

关联题目数量: 0. 关联题目id: 

\item K0710006X|D07004X|能用代数方法及韦达定理处理直线与圆相交所得交点的相关问题.

关联题目数量: 1. 关联题目id: 000281

\item 本课时题目数量: 1. 关联题目id: 000281

\end{itemize}

\section*{第7单元, 第11课时}
起始页码: 选择性必修第一册P37; 终止页码: 选择性必修第一册P39.
\begin{itemize}
\item K0711001X|D07004X|在平面直角坐标系中, 能根据给定直线、圆的方程, 通过代数方法(一元二次方程的判别式)判断直线与圆的位置关系.

关联题目数量: 1. 关联题目id: 000277

\item K0711002X|D07004X|在平面直角坐标系中, 能根据给定直线、圆的方程, 通过几何方法(点到直线的距离与半径的大小关系)判断直线与圆的位置关系.

关联题目数量: 4. 关联题目id: 000268, 000269, 000272, 000277

\item K0711003X|D07004X|会用点法式表示圆心在原点的圆上一点处的切线, 并将圆心推广至一般位置.

关联题目数量: 1. 关联题目id: 000276

\item K0711004X|D07004X|会用待定系数法根据距离求过圆外一点圆的切线方程.

关联题目数量: 1. 关联题目id: 000276

\item K0711005X|D07004X|掌握圆的过定点的弦的中点的轨迹的求法, 知道轨迹的概念以及轨迹与轨迹方程的异同.

关联题目数量: 1. 关联题目id: 000274

\item 本课时题目数量: 6. 关联题目id: 000268, 000269, 000272, 000274, 000276, 000277

\end{itemize}

\section*{第7单元, 第12课时}
起始页码: 选择性必修第一册P39; 终止页码: 选择性必修第一册P42.
\begin{itemize}
\item K0712001X|D07004X|知道两圆的位置关系基于直观的分类.

关联题目数量: 0. 关联题目id: 

\item K0712002X|D07004X|在平面直角坐标系中, 会用圆心距与两圆半径的关系判断圆与圆的位置关系.

关联题目数量: 0. 关联题目id: 

\item K0712003X|D07004X|在平面直角坐标系中, 会用解方程组的方法判断圆与圆的位置关系.

关联题目数量: 0. 关联题目id: 

\item K0712004X|D07004X|能推导相交两圆的公共弦及相切两圆过公共切点的公切线所在直线的方程, 体会设而不求的思想.

关联题目数量: 0. 关联题目id: 

\item K0712005X|D07004X|会将切线长转化为点到圆心的距离来解决与切线长有关的问题.

关联题目数量: 0. 关联题目id: 

\item 本课时题目数量: 0. 关联题目id: 

\end{itemize}

\section*{第7单元, 第13课时}
起始页码: 选择性必修第一册P45; 终止页码: 选择性必修第一册P47.
\begin{itemize}
\item K0713001X|D07005X|经历从具体情境(天文学、物理学等方面)抽象出椭圆, 并借助信息技术等工具绘制出椭圆的这一过程, 了解椭圆的直观图像.

关联题目数量: 0. 关联题目id: 

\item K0713002X|D07005X|能用语言描述椭圆的定义, 能根据椭圆的定义推导椭圆的标准方程, 并掌握两种类型(中心在原点, 焦点在坐标轴上)椭圆的标准方程.

关联题目数量: 1. 关联题目id: 000283

\item K0713003X|D07005X|能根据椭圆的定义, 由关键的几何量写出椭圆的标准方程.

关联题目数量: 2. 关联题目id: 000278, 000288

\item K0713004X|D07005X|能利用椭圆的定义, 根据椭圆的焦点、椭圆上点的坐标求出椭圆的标准方程.

关联题目数量: 1. 关联题目id: 000270

\item 本课时题目数量: 4. 关联题目id: 000270, 000278, 000283, 000288

\end{itemize}

\section*{第7单元, 第14课时}
起始页码: 选择性必修第一册P47; 终止页码: 选择性必修第一册P49.
\begin{itemize}
\item K0714001X|D07005X|能根据椭圆的标准方程用代数方法研究椭圆是否关于坐标轴、原点对称.

关联题目数量: 0. 关联题目id: 

\item K0714002X|D07005X|知道用标准方程描述的椭圆关于两条坐标轴对称, 关于原点对称且原点是其唯一的对称中心, 知道椭圆中心的概念.

关联题目数量: 0. 关联题目id: 

\item K0714003X|D07005X|知道椭圆的顶点、长轴、短轴的概念, 知道椭圆的长轴、短轴分别所在的直线是椭圆的两条对称轴.

关联题目数量: 0. 关联题目id: 

\item K0714004X|D07005X|能根据椭圆的标准方程得到椭圆上点的横、纵坐标的范围.

关联题目数量: 0. 关联题目id: 

\item K0714005X|D07005X|通过焦距与长轴长之比了解椭圆的离心率, 知道离心率的大小对椭圆扁平程度的影响.

关联题目数量: 1. 关联题目id: 000271

\item 本课时题目数量: 1. 关联题目id: 000271

\end{itemize}

\section*{第7单元, 第15课时}
起始页码: 选择性必修第一册P49; 终止页码: 选择性必修第一册P51.
\begin{itemize}
\item K0715001X|D07005X|了解椭圆在现实情境中的应用.

关联题目数量: 0. 关联题目id: 

\item K0715002X|D07005X|能根据椭圆的标准方程, 用代数方法研究椭圆上的动点到焦点的距离.

关联题目数量: 1. 关联题目id: 000285

\item K0715003X|D07005X|会通过联立方程组研究直线与椭圆的公共点个数, 从代数角度类比直线与圆的位置关系, 并从形的角度掌握直线与椭圆的位置关系.

关联题目数量: 2. 关联题目id: 000273, 000284

\item K0715004X|D07005X|从图形上理解直线与椭圆相切的含义, 知道直线与椭圆相切的直观定义.

关联题目数量: 0. 关联题目id: 

\item K0715005X|D07005X|了解椭圆的光学性质.

关联题目数量: 0. 关联题目id: 

\item 本课时题目数量: 3. 关联题目id: 000273, 000284, 000285

\end{itemize}

\section*{第7单元, 第16课时}
起始页码: 选择性必修第一册P52; 终止页码: 选择性必修第一册P55.
\begin{itemize}
\item K0716001X|D07006X|经历从具体情境(天文学、物理学等方面)抽象出双曲线, 并借助信息技术等工具绘制出双曲线的局部的这一过程, 了解双曲线的直观图像.

关联题目数量: 0. 关联题目id: 

\item K0716002X|D07006X|能用语言描述双曲线的定义, 能根据双曲线的定义推导双曲线的标准方程, 掌握两种类型(中心在原点, 焦点在坐标轴上)双曲线的标准方程.

关联题目数量: 3. 关联题目id: 000280, 000283, 000289

\item K0716003X|D07006X|能根据双曲线的定义, 由关键的几何量写出双曲线的标准方程.

关联题目数量: 1. 关联题目id: 000288

\item K0716004X|D07006X|会用坐标的范围表示双曲线的一支.

关联题目数量: 0. 关联题目id: 

\item 本课时题目数量: 4. 关联题目id: 000280, 000283, 000288, 000289

\end{itemize}

\section*{第7单元, 第17课时}
起始页码: 选择性必修第一册P55; 终止页码: 选择性必修第一册P58.
\begin{itemize}
\item K0717001X|D07006X|知道用标准方程描述的双曲线关于两条坐标轴对称, 关于原点对称且原点是其唯一的对称中心, 知道双曲线中心的概念.

关联题目数量: 0. 关联题目id: 

\item K0717002X|D07006X|知道双曲线的顶点、实轴、虚轴的概念, 知道双曲线实轴、虚轴分别所在的直线是双曲线的两条对称轴, 了解等轴双曲线的概念.

关联题目数量: 1. 关联题目id: 000289

\item K0717003X|D07006X|能根据双曲线的标准方程得到双曲线上点的横、纵坐标的范围.

关联题目数量: 0. 关联题目id: 

\item K0717004X|D07006X|了解双曲线的渐近线的概念, 能由双曲线的标准方程求出渐近线的方程.

关联题目数量: 1. 关联题目id: 000272

\item K0717005X|D07006X|直观上了解双曲线与其渐近线的位置关系, 并会用代数语言描述与论证.

关联题目数量: 0. 关联题目id: 

\item K0717006X|D07006X|会根据渐近线的方程及双曲线上的一点求出双曲线的标准方程.

关联题目数量: 0. 关联题目id: 

\item K0717007X|D07006X|通过焦距与实轴长之比了解双曲线的离心率, 知道离心率的大小对双曲线开口大小的影响.

关联题目数量: 0. 关联题目id: 

\item 本课时题目数量: 2. 关联题目id: 000272, 000289

\end{itemize}

\section*{第7单元, 第18课时}
起始页码: 选择性必修第一册P58; 终止页码: 选择性必修第一册P60.
\begin{itemize}
\item K0718001X|D07006X|会通过联立方程组研究直线与双曲线的公共点个数, 并从形的角度掌握直线与双曲线的位置关系.

关联题目数量: 0. 关联题目id: 

\item K0718002X|D07006X|在现实情境中能把与双曲线有关的问题抽象为数学模型, 建立坐标系, 应用双曲线的标准方程求解.

关联题目数量: 0. 关联题目id: 

\item 本课时题目数量: 0. 关联题目id: 

\end{itemize}

\section*{第7单元, 第19课时}
起始页码: 选择性必修第一册P61; 终止页码: 选择性必修第一册P64.
\begin{itemize}
\item K0719001X|D07007X|经历从具体情境(天文学、物理学等方面)抽象出抛物线, 并借助信息技术等工具绘制出抛物线的这一过程, 了解抛物线的直观图像.

关联题目数量: 0. 关联题目id: 

\item K0719002X|D07007X|能用语言描述抛物线的定义, 推导抛物线的标准方程, 包括证明以所求方程的任意一组解为坐标的点都在该抛物线上.

关联题目数量: 1. 关联题目id: 000290

\item K0719003X|D07007X|知道抛物线的焦点、准线的概念, 掌握四种类型(顶点在原点, 焦点在坐标轴上)抛物线的标准方程.

关联题目数量: 0. 关联题目id: 

\item K0719004X|D07007X|在简单情境中, 会根据焦点、准线或其他条件求出抛物线的标准方程.

关联题目数量: 0. 关联题目id: 

\item K0719005X|D07007X|通过回顾初中熟知的"二次函数的图像是抛物线"这一结论, 了解二次函数的图像经平移后符合四种标准类型抛物线之一的定义.

关联题目数量: 1. 关联题目id: 000290

\item K0719006X|D07007X|会根据抛物线的定义将线段长度作转化证明一些平面几何的命题.

关联题目数量: 1. 关联题目id: 000275

\item 本课时题目数量: 2. 关联题目id: 000275, 000290

\end{itemize}

\section*{第7单元, 第20课时}
起始页码: 选择性必修第一册P64; 终止页码: 选择性必修第一册P66.
\begin{itemize}
\item K0720001X|D07007X|知道用标准方程描述的抛物线关于其中一条坐标轴对称, 知道抛物线顶点的概念, 了解抛物线有且只有一条对称轴.

关联题目数量: 0. 关联题目id: 

\item K0720002X|D07007X|能根据抛物线的标准方程得到抛物线上点的横、纵坐标的范围.

关联题目数量: 0. 关联题目id: 

\item K0720003X|D07007X|会通过联立方程组研究直线与抛物线的公共点个数,并从形的角度掌握直线与抛物线的位置关系.

关联题目数量: 2. 关联题目id: 000279, 000282

\item K0720004X|D07007X|在现实情境中能把与抛物线有关的问题抽象为数学模型, 建立坐标系, 应用抛物线的标准方程求解.

关联题目数量: 0. 关联题目id: 

\item K0720005X|D07007X|了解抛物线的光学性质.

关联题目数量: 0. 关联题目id: 

\item 本课时题目数量: 2. 关联题目id: 000279, 000282

\end{itemize}

\section*{第7单元, 第21课时}
起始页码: 选择性必修第一册P69; 终止页码: 选择性必修第一册P71.
\begin{itemize}
\item K0721001X|D07008X|通过具体的反例, 进一步理解说明一个方程与一条曲线对应时, 需要进行正反两个方面的验证.

关联题目数量: 0. 关联题目id: 

\item K0721002X|D07008X|掌握求简单的轨迹方程的三个基本步骤(建立合适的坐标系, 根据曲线的特征推导方程, 验证以方程的解为坐标的点都在所求曲线上).

关联题目数量: 3. 关联题目id: 000267, 000279, 000287

\item K0721003X|D07008X|在具体的问题中, 了解如何根据图形的几何特征选取合适的坐标系.

关联题目数量: 0. 关联题目id: 

\item K0721004X|D07008X|对于线段的定比分点的轨迹等问题, 能用``等价''符号简化曲线与方程对应的证明过程.

关联题目数量: 1. 关联题目id: 000274

\item 本课时题目数量: 4. 关联题目id: 000267, 000274, 000279, 000287

\end{itemize}

\section*{第7单元, 第22课时}
起始页码: 选择性必修第一册P71; 终止页码: 选择性必修第一册P74.
\begin{itemize}
\item K0722001X|D07008X|借助具体的科学情境, 理解用参数方程来描述曲线的优势.

关联题目数量: 0. 关联题目id: 

\item K0722002X|D07008X|理解参数方程的概念和相关名词(参数, 参变量, 普通方程).

关联题目数量: 0. 关联题目id: 

\item K0722003X|D07008X|能通过问题中自然给出的参数将轨迹用参数方程表示.

关联题目数量: 0. 关联题目id: 

\item K0722004X|D07008X|能通过消参($x,y$中至少有一个变量是参数的一次函数)将参数方程化为普通方程, 并解决曲线上的范围的界定问题.

关联题目数量: 0. 关联题目id: 

\item K0722005X|D07008X|能借助圆或椭圆的参数方程, 将含有一个约束条件的双变量的问题化为仅含一个自由变量的问题.

关联题目数量: 0. 关联题目id: 

\item 本课时题目数量: 0. 关联题目id: 

\end{itemize}

\section*{第7单元, 第23课时}
起始页码: 选择性必修第一册P74; 终止页码: 选择性必修第一册P76.
\begin{itemize}
\item K0723001X|D07008X|联系复数的三角形式, 了解用长度和方向表示点的极坐标的基本思想.

关联题目数量: 0. 关联题目id: 

\item K0723002X|D07008X|理解极坐标系的概念及相关名词(极点, 极轴, 极坐标, 极径, 极角).

关联题目数量: 1. 关联题目id: 003980

\item K0723003X|D07008X|了解直角坐标系下点和数对是一一对应的, 而极坐标系下一个点有多个极坐标.

关联题目数量: 0. 关联题目id: 

\item K0723004X|D07008X|了解相关的极坐标所表示的点的位置关系.

关联题目数量: 0. 关联题目id: 

\item K0723005X|D07008X|了解如何通过限制极角和极径的取值范围使除极点外的点和极坐标形成一一对应.

关联题目数量: 0. 关联题目id: 

\item K0723006X|D07008X|能根据点的位置写出其一个极坐标.

关联题目数量: 0. 关联题目id: 

\item 本课时题目数量: 1. 关联题目id: 003980

\end{itemize}

\section*{第7单元, 第24课时}
起始页码: 选择性必修第一册P76; 终止页码: 选择性必修第一册P78.
\begin{itemize}
\item K0724001X|D07008X|了解极坐标系下曲线与方程$F(\rho,\theta)=0$的对应关系.

关联题目数量: 0. 关联题目id: 

\item K0724002X|D07008X|理解直角坐标系与极坐标系下方程与曲线对应方式的异同.

关联题目数量: 0. 关联题目id: 

\item K0724003X|D07008X|会利用余弦的定义求圆心在极径上, 经过原点的圆的极坐标方程.

关联题目数量: 0. 关联题目id: 

\item K0724004X|D07008X|会分情况讨论, 根据正弦定理推导不过原点的直线的极坐标方程.

关联题目数量: 0. 关联题目id: 

\item K0724005X|D07008X|会结合物理意义推导等速螺线(阿基米德螺线)的极坐标方程.

关联题目数量: 0. 关联题目id: 

\item 本课时题目数量: 0. 关联题目id: 

\end{itemize}

\section*{第7单元, 第25课时}
起始页码: 选择性必修第一册P78; 终止页码: 选择性必修第一册P79.
\begin{itemize}
\item K0725001X|D07008X|能根据正弦和余弦的定义, 推导极坐标转换为直角坐标的公式.

关联题目数量: 0. 关联题目id: 

\item K0725002X|D07008X|能用公式将具体点的极坐标转换为直角坐标.

关联题目数量: 0. 关联题目id: 

\item K0725003X|D07008X|通过求解方程, 掌握将点的直角坐标转换为极坐标的公式, 理解其中极角的选取与正切值及点的具体位置均有关.

关联题目数量: 0. 关联题目id: 

\item K0725004X|D07008X|能用公式将具体点的直角坐标转换为极坐标.

关联题目数量: 0. 关联题目id: 

\item K0725005X|D07008X|会用$x=\rho \cos\theta$, $y=\rho \sin \theta$代入的方法, 在熟悉的情境下将曲线的简单的直角坐标方程转化为极坐标方程.

关联题目数量: 0. 关联题目id: 

\item K0725006X|D07008X|通过将$\rho^2$化为$x^2+y^2$, $\rho\cos\theta$化为$x$, $\rho\sin\theta$化为$y$, 在熟悉的情境下将曲线的极坐标方程转换为直角坐标方程.

关联题目数量: 0. 关联题目id: 

\item K0725007X|D07008X|了解椭圆, 抛物线, 双曲线的方程在以其(一个)焦点为极点的极坐标系中能统一为相同的形式.

关联题目数量: 0. 关联题目id: 

\item 本课时题目数量: 0. 关联题目id: 

\end{itemize}

\section*{第8单元, 第1课时}
起始页码: 必修第三册P90; 终止页码: 必修第三册P92.
\begin{itemize}
\item K0801001B|D08001B|通过具体事例, 认识随机现象在自然界、社会中普遍存在, 理解随机现象的概念.

关联题目数量: 0. 关联题目id: 

\item K0801002B|D08001B|通过具体事例, 理解随机试验的概念.

关联题目数量: 0. 关联题目id: 

\item K0801003B|D08001B|初步了解概率论的起源与发展历史, 了解描述性的概率的概念.

关联题目数量: 0. 关联题目id: 

\item K0801004B|D08001B|能够区分一个现象是随机现象还是确定性现象.

关联题目数量: 0. 关联题目id: 

\item K0801005B|D08001B|通过具体情境, 了解随机试验中含有的随机性.

关联题目数量: 0. 关联题目id: 

\item 本课时题目数量: 0. 关联题目id: 

\end{itemize}

\section*{第8单元, 第2课时}
起始页码: 必修第三册P92; 终止页码: 必修第三册P95.
\begin{itemize}
\item K0802001B|D08001B|了解样本空间, 基本事件(或样本点)的定义,  知道基本事件不能同时发生。.

关联题目数量: 0. 关联题目id: 

\item K0802002B|D08001B|了解在随机现象中, 样本空间的选取可以不唯一.

关联题目数量: 0. 关联题目id: 

\item K0802003B|D08001B|能够写出随机现象的样本空间, 理解随机事件的含义.

关联题目数量: 1. 关联题目id: 000223

\item K0802004B|D08001B|会用集合语言表达随机事件.

关联题目数量: 0. 关联题目id: 

\item K0802005B|D08001B|理解必然事件和不可能事件的概念, 了解它们对应的子集与样本空间的关系. 会判断一个事件是确定事件还是不确定事件.

关联题目数量: 0. 关联题目id: 

\item K0802006B|D08001B|能在熟悉的情境中写出随机试验的样本空间.

关联题目数量: 0. 关联题目id: 

\item 本课时题目数量: 1. 关联题目id: 000223

\end{itemize}

\section*{第8单元, 第3课时}
起始页码: 必修第三册P96; 终止页码: 必修第三册P98.
\begin{itemize}
\item K0803001B|D08001B|基于生活经验, 直观地理解随机试验结果的等可能性.

关联题目数量: 0. 关联题目id: 

\item K0803002B|D08001B|通过具体实例, 理解古典概率模型的两个基本假设: 有限, 等可能. 会基于枚举计数计算古典概率模型中简单随机事件的概率.

关联题目数量: 10. 关联题目id: 000218, 000219, 000220, 000221, 000222, 000223, 000226, 000227, 000228, 004033

\item K0803003B|D08001B|根据定义, 理解概率性质1($P(\Omega)=1$, $P(\varnothing)=0$)和概率性质2($0\le P(A)\le 1$).

关联题目数量: 0. 关联题目id: 

\item 本课时题目数量: 10. 关联题目id: 000218, 000219, 000220, 000221, 000222, 000223, 000226, 000227, 000228, 004033

\end{itemize}

\section*{第8单元, 第4课时}
起始页码: 必修第三册P98; 终止页码: 必修第三册P100.
\begin{itemize}
\item K0804001B|D08001B|通过古典概型实例, 理解随着观察角度的不同, 并非所有的样本空间都有等可能性. 了解只有选取等可能的样本空间, 才能使得事件的概率如定义所示.

关联题目数量: 0. 关联题目id: 

\item K0804002B|D08001B|会对多步独立的等可能随机试验构造等可能的样本空间.

关联题目数量: 0. 关联题目id: 

\item 本课时题目数量: 0. 关联题目id: 

\end{itemize}

\section*{第8单元, 第5课时}
起始页码: 必修第三册P100; 终止页码: 必修第三册P102.
\begin{itemize}
\item K0805001B|D08001B|理解事件之间的子集关系, 会用集合语言表达.

关联题目数量: 0. 关联题目id: 

\item K0805002B|D08001B|通过具体实例, 掌握事件的交、并运算, 懂得事件的运算的含义, 并能够用集合语言表达.

关联题目数量: 0. 关联题目id: 

\item K0805003B|D08001B|通过具体实例, 理解互斥事件的概念.

关联题目数量: 0. 关联题目id: 

\item K0805004B|D08001B|理解两个事件$A$与$B$相互对立当且仅当: $A\cap B=\varnothing$, $A\cup B=\Omega$. 了解$A$的对立事件的符号表示$\overline A$.

关联题目数量: 0. 关联题目id: 

\item K0805005B|D08001B|在具体实例中, 能用语言描述简单的随机事件的对立事件.

关联题目数量: 0. 关联题目id: 

\item K0805006B|D08001B|掌握公式$\overline{A\cap B}=\overline A\cup\overline B$及$\overline{A\cup B}=\overline A\cap\overline B$, 并理解这两个公式对任意多个事件同样成立.

关联题目数量: 0. 关联题目id: 

\item 本课时题目数量: 0. 关联题目id: 

\end{itemize}

\section*{第8单元, 第6课时}
起始页码: 必修第三册P102; 终止页码: 必修第三册P106.
\begin{itemize}
\item K0806001B|D08001B|在古典概率模型中, 能够推导两个不同时发生的事件至少有一个发生的概率是这两个事件的概率之和, 理解概率性质3(可加性).

关联题目数量: 0. 关联题目id: 

\item K0806002B|D08001B|基于概率性质3(可加性), 理解$B=\overline A$时的特殊情况, 掌握概率性质4($P(A)=1-P(\overline A)$).

关联题目数量: 2. 关联题目id: 000225, 000231

\item K0806003B|D08001B|了解在一般概率模型中概率的三个基本性质: $0\le P(A)\le 1$; $P(\Omega)=1$; 若$A\cap B=\varnothing$, 则$P(A\cup B)=P(A)+P(B)$.

关联题目数量: 2. 关联题目id: 000225, 000229

\item K0806004B|D08001B|能利用概率性质3与概率性质4解决简单的相关问题.

关联题目数量: 2. 关联题目id: 000225, 000229

\item K0806005B|D08001B|经历两个事件的可加性推出任意有限个事件的可加性($P(A_1\cup A_2\cup\cdots\cap A_n)=P(A_1)+P(A_2)+\cdots+P(A_n)$)的过程.

关联题目数量: 0. 关联题目id: 

\item 本课时题目数量: 3. 关联题目id: 000225, 000229, 000231

\end{itemize}

\section*{第8单元, 第7课时}
起始页码: 必修第三册P107; 终止页码: 必修第三册P111.
\begin{itemize}
\item K0807001B|D08001B|了解伯努利试验的概念, 通过对实例的观察与分析初步理解伯努利试验中``独立地重复''的含义以及频率的意义.

关联题目数量: 0. 关联题目id: 

\item K0807002B|D08001B|结合试验实例, 归纳并抽象出伯努利大数定律, 了解其意义.

关联题目数量: 0. 关联题目id: 

\item K0807003B|D08001B|掌握事件频率的计算法则, 了解频率也称经验概率. 会用频率估计概率, 解决一些简单的实际问题.

关联题目数量: 2. 关联题目id: 000224, 000230

\item 本课时题目数量: 2. 关联题目id: 000224, 000230

\end{itemize}

\section*{第8单元, 第8课时}
起始页码: 必修第三册P112; 终止页码: 必修第三册P114.
\begin{itemize}
\item K0808001B|D08002B|结合有限样本空间, 通过具体事例, 经历由对事件独立的直观判断到两个事件独立的严格定义($P(A\cap B)=P(A)P(B)$)的形成过程. 在现实情境中理解多个随机试验独立的含义.

关联题目数量: 0. 关联题目id: 

\item K0808002B|D08002B|结合古典概型, 掌握两个独立事件积及多个独立试验中事件的积的概率计算方法.

关联题目数量: 1. 关联题目id: 000231

\item K0808003B|D08002B|掌握事件独立性的性质: 如果$A$与$B$两个事件独立, 那么$A$与$\overline B$也独立, 并了解其现实意义.

关联题目数量: 1. 关联题目id: 000231

\item K0808004B|D08002B|会综合使用可加性与独立性求解相关的概率问题.

关联题目数量: 0. 关联题目id: 

\item 本课时题目数量: 1. 关联题目id: 000231

\end{itemize}

\section*{第8单元, 第9课时}
起始页码: 必修第三册P114; 终止页码: 必修第三册P121.
\begin{itemize}
\item K0809001B|D08002B|会用两个事件相互独立的充要条件判断两个现实情境中的事件是否独立.

关联题目数量: 0. 关联题目id: 

\item K0809002B|D08002B|会用事件$A,B,A\cap B$表示事件$A\cup B$, 并利用可加性求$A\cup B$的概率.

关联题目数量: 0. 关联题目id: 

\item K0809003B|D08002B|会用概率的思想分析现实情境中的问题, 并通过建模, 计算, 给出解答.

关联题目数量: 0. 关联题目id: 

\item 本课时题目数量: 0. 关联题目id: 

\end{itemize}

\section*{第8单元, 第10课时}
起始页码: 选择性必修第二册P38; 终止页码: 选择性必修第二册P40.
\begin{itemize}
\item K0810001X|D08003X|结合具体实例, 掌握分步计数原理(乘法原理).

关联题目数量: 1. 关联题目id: 000491

\item K0810002X|D08003X|理解乘法原理的应用条件.

关联题目数量: 0. 关联题目id: 

\item K0810003X|D08003X|会利用乘法原理解决简单的相关计数问题.

关联题目数量: 1. 关联题目id: 004015

\item 本课时题目数量: 2. 关联题目id: 000491, 004015

\end{itemize}

\section*{第8单元, 第11课时}
起始页码: 选择性必修第二册P40; 终止页码: 选择性必修第二册P43.
\begin{itemize}
\item K0811001X|D08003X|结合具体实例, 掌握分类计数原理(加法原理).

关联题目数量: 1. 关联题目id: 000491

\item K0811002X|D08003X|理解加法原理应用的条件, 体会分类讨论的思想方法.

关联题目数量: 0. 关联题目id: 

\item K0811003X|D08003X|能利用加法原理解决相关简单的计数问题.

关联题目数量: 0. 关联题目id: 

\item K0811004X|D08003X|能够区分相关计数问题是分步计数还是分类计数问题.

关联题目数量: 0. 关联题目id: 

\item K0811005X|D08003X|能利用加法原理与乘法原理解决较为复杂的计数问题.

关联题目数量: 2. 关联题目id: 004026, 004029

\item 本课时题目数量: 3. 关联题目id: 000491, 004026, 004029

\end{itemize}

\section*{第8单元, 第12课时}
起始页码: 选择性必修第二册P44; 终止页码: 选择性必修第二册P47.
\begin{itemize}
\item K0812001X|D08003X|基于乘法原理, 结合具体实例, 引出排列的定义.

关联题目数量: 0. 关联题目id: 

\item K0812002X|D08003X|理解排列的含义.

关联题目数量: 0. 关联题目id: 

\item K0812003X|D08003X|会利用乘法原理求解具体的排列问题.

关联题目数量: 0. 关联题目id: 

\item 本课时题目数量: 0. 关联题目id: 

\end{itemize}

\section*{第8单元, 第13课时}
起始页码: 选择性必修第二册P47; 终止页码: 选择性必修第二册P51.
\begin{itemize}
\item K0813001X|D08003X|结合具体实例, 理解排列数定义, 会规范地表示排列数.

关联题目数量: 0. 关联题目id: 

\item K0813002X|D08003X|会利用乘法原理推导排列数公式, 体会乘法原理在推导排列数公式上的作用.

关联题目数量: 0. 关联题目id: 

\item K0813003X|D08003X|掌握排列数公式, 并能利用排列数公式求解相关的排列问题.

关联题目数量: 5. 关联题目id: 004016, 004017, 004022, 004023, 004025

\item K0813004X|D08003X|能合理分类, 利用排列数公式以及乘法原理和加法原理求解较综合的计数问题.

关联题目数量: 5. 关联题目id: 004016, 004017, 004022, 004023, 004025

\item K0813005X|D08003X|掌握借助计算器求排列数的方法.

关联题目数量: 0. 关联题目id: 

\item 本课时题目数量: 5. 关联题目id: 004016, 004017, 004022, 004023, 004025

\end{itemize}

\section*{第8单元, 第14课时}
起始页码: 选择性必修第二册P51; 终止页码: 选择性必修第二册P54.
\begin{itemize}
\item K0814001X|D08003X|了解全排列的概念, 及全排列数的符号表示, 掌握全排列数的计算公式.

关联题目数量: 0. 关联题目id: 

\item K0814002X|D08003X|了解阶乘的概念, 并能够用阶乘表示排列数公式. 了解$0!=1$的规定及其作用, 领会全排列数$\mathrm{P}_n^n=n!$是排列数公式中$m=n$的特殊情况.

关联题目数量: 0. 关联题目id: 

\item K0814003X|D08003X|能用排列数表示连续的几个正整数相乘.

关联题目数量: 0. 关联题目id: 

\item K0814004X|D08003X|能用排列数公式证明相关的简单恒等式, 如: $\mathrm{P}_n^m=n\mathrm{P}_{n-1}^{m-1}$; $\mathrm{P}_n^m+m\mathrm{P}_n^{m-1}=\mathrm{P}_{n+1}^m$.

关联题目数量: 0. 关联题目id: 

\item K0814005X|D08003X|会将含有排列数的方程化为整式方程.

关联题目数量: 0. 关联题目id: 

\item 本课时题目数量: 0. 关联题目id: 

\end{itemize}

\section*{第8单元, 第15课时}
起始页码: 选择性必修第二册P55; 终止页码: 选择性必修第二册P56.
\begin{itemize}
\item K0815001X|D08003X|基于排列的定义, 理解组合的定义.

关联题目数量: 0. 关联题目id: 

\item K0815002X|D08003X|理解排列与组合的区别, 能够判断问题是排列问题还是组合问题.

关联题目数量: 0. 关联题目id: 

\item K0815003X|D08003X|能够基于枚举求解简单的组合问题.

关联题目数量: 2. 关联题目id: 004024, 004026

\item 本课时题目数量: 2. 关联题目id: 004024, 004026

\end{itemize}

\section*{第8单元, 第16课时}
起始页码: 选择性必修第二册P56; 终止页码: 选择性必修第二册P59.
\begin{itemize}
\item K0816001X|D08003X|结合排列数定义, 理解组合数定义, 并掌握组合数的符号表示.

关联题目数量: 0. 关联题目id: 

\item K0816002X|D08003X|理解排列与相应的组合之间的对应关系, 进而能利用排列数公式和乘法原理推导组合数公式.

关联题目数量: 0. 关联题目id: 

\item K0816003X|D08003X|会利用由排列数构成的组合数公式, 计算组合数.

关联题目数量: 0. 关联题目id: 

\item K0816004X|D08003X|在熟悉的情境中, 能够利用组合数公式求解相关的组合问题.

关联题目数量: 1. 关联题目id: 000491

\item K0816005X|D08003X|掌握借助计算器计算组合数的方法.

关联题目数量: 0. 关联题目id: 

\item 本课时题目数量: 1. 关联题目id: 000491

\end{itemize}

\section*{第8单元, 第17课时}
起始页码: 选择性必修第二册P59; 终止页码: 选择性必修第二册P62.
\begin{itemize}
\item K0817001X|D08003X|会利用公式$\mathrm{P}_n^m=\dfrac{n!}{(n-m)!}$推导出组合数公式: $\mathrm{C}_n^m=\dfrac{n!}{m!(n-m)!}$.

关联题目数量: 0. 关联题目id: 

\item K0817002X|D08003X|了解$\mathrm{C}_n^0=1$的含义.

关联题目数量: 0. 关联题目id: 

\item K0817003X|D08003X|会利用组合数公式证明: $\mathrm{C}_n^m=\dfrac{m+1}{n-m}\mathrm{C}_n^{m+1}$.

关联题目数量: 0. 关联题目id: 

\item K0817004X|D08003X|会利用组合数公式证明组合数的两个基本性质: $\mathrm{C}_n^m=\mathrm{C}_n^{n-m}$; $\mathrm{C}_{n+1}^m=\mathrm{C}_n^m+\mathrm{C}_n^{m-1}$.

关联题目数量: 0. 关联题目id: 

\item K0817005X|D08003X|经历构造组合模型, 证明组合数的两个基本运算性质的过程.

关联题目数量: 0. 关联题目id: 

\item K0817006X|D08003X|会利用组合数公式及两个基本性质计算和转化含有组合数的问题.

关联题目数量: 1. 关联题目id: 004018

\item 本课时题目数量: 1. 关联题目id: 004018

\end{itemize}

\section*{第8单元, 第18课时}
起始页码: 选择性必修第二册P63; 终止页码: 选择性必修第二册P65.
\begin{itemize}
\item K0818001X|D08003X|在古典概率模型中, 能利用排列和组合求随机事件$A$包含的基本事件的个数$k$, 并能结合公式$P(A)=\dfrac kn$求概率.

关联题目数量: 3. 关联题目id: 000221, 004042, 004046

\item K0818002X|D08003X|在具体实例中, 能够合理分析计算难度, 选择从事件本身或对立事件入手计算概率.

关联题目数量: 2. 关联题目id: 000225, 000231

\item 本课时题目数量: 5. 关联题目id: 000221, 000225, 000231, 004042, 004046

\end{itemize}

\section*{第8单元, 第19课时}
起始页码: 选择性必修第二册P66; 终止页码: 选择性必修第二册P68.
\begin{itemize}
\item K0819001X|D08003X|通过具体的实例了解二项展开式的概念.

关联题目数量: 0. 关联题目id: 

\item K0819002X|D08003X|通过具体实例的展开, 归纳出对于任意正整数$n$, $(a+b)^n$的二项展开式的规律.

关联题目数量: 0. 关联题目id: 

\item K0819003X|D08003X|结合杨辉三角掌握二项展开式中各项系数的$3$个特点: 每一行的第一个数和最后一个数都是$1$; 每一行中, 除了第一个数和最后一个数外, 每个数等于它``肩上''的两数之和; 当$n$为偶数时, 最大的系数时中间一项, 当$n$为奇数时, 最大的系数是中间两项. 并能用组合数的形式表示前两个特点.

关联题目数量: 0. 关联题目id: 

\item K0819004X|D08003X|掌握二项式定理, 并能够利用数学归纳法证明二项式定理.

关联题目数量: 0. 关联题目id: 

\item K0819005X|D08003X|能利用二项式定理展开具体的二项式, 并能求其中具体的一项的系数.

关联题目数量: 6. 关联题目id: 003759, 004019, 004020, 004021, 004027, 004030

\item K0819006X|D08003X|能利用二项式定理证明相关的数的整除问题.

关联题目数量: 1. 关联题目id: 004028

\item 本课时题目数量: 7. 关联题目id: 003759, 004019, 004020, 004021, 004027, 004028, 004030

\end{itemize}

\section*{第8单元, 第20课时}
起始页码: 选择性必修第二册P69; 终止页码: 选择性必修第二册P75.
\begin{itemize}
\item K0820001X|D08003X|在二项式定理中, 能利用赋值法证明一些有关系数的恒等式.

关联题目数量: 2. 关联题目id: 004020, 004027

\item K0820002X|D08003X|了解$\mathrm{C}_n^0+\mathrm{C}_n^1+\mathrm{C}_n^2+\cdots+\mathrm{C}_n^n=2^n$及$\mathrm{C}_n^0-\mathrm{C}_n^1+\mathrm{C}_n^2-\cdots+(-1)^n\mathrm{C}_n^n=0$.

关联题目数量: 0. 关联题目id: 

\item K0820003X|D08003X|能利用分析数列单调性的方法研究二项展开式的系数的单调性, 并再次基础上能求系数的最值.

关联题目数量: 0. 关联题目id: 

\item 本课时题目数量: 2. 关联题目id: 004020, 004027

\end{itemize}

\section*{第8单元, 第21课时}
起始页码: 选择性必修第二册P78; 终止页码: 选择性必修第二册P80.
\begin{itemize}
\item K0821001X|D08004X|理解条件概率的概念.

关联题目数量: 3. 关联题目id: 004031, 004032, 004033

\item K0821002X|D08004X|能分辨条件概率与概率的异同.

关联题目数量: 1. 关联题目id: 004033

\item K0821003X|D08004X|在熟悉的情境中能根据条件概率公式用除法计算条件概率.

关联题目数量: 4. 关联题目id: 004031, 004033, 004040, 004041

\item K0821004X|D08004X|知道概率的乘法公式.

关联题目数量: 0. 关联题目id: 

\item K0821005X|D08004X|能用概率的乘法公式求两事件积的概率.

关联题目数量: 0. 关联题目id: 

\item K0821006X|D08004X|了解条件概率与独立事件之间的联系.

关联题目数量: 0. 关联题目id: 

\item 本课时题目数量: 5. 关联题目id: 004031, 004032, 004033, 004040, 004041

\end{itemize}

\section*{第8单元, 第22课时}
起始页码: 选择性必修第二册P80; 终止页码: 选择性必修第二册P83.
\begin{itemize}
\item K0822001X|D08004X|了解加权平均的概念.

关联题目数量: 0. 关联题目id: 

\item K0822002X|D08004X|理解全概率公式, 会用概率乘法公式和可加性推导全概率公式.

关联题目数量: 0. 关联题目id: 

\item K0822003X|D08004X|在熟悉的情境中, 能合理地分拆事件, 用全概率公式计算概率.

关联题目数量: 5. 关联题目id: 004039, 004043, 004044, 004045, 004046

\item 本课时题目数量: 5. 关联题目id: 004039, 004043, 004044, 004045, 004046

\end{itemize}

\section*{第8单元, 第23课时}
起始页码: 选择性必修第二册P83; 终止页码: 选择性必修第二册P85.
\begin{itemize}
\item K0823001X|D08004X|会用概率乘法公式和条件概率公式推导贝叶斯公式.

关联题目数量: 0. 关联题目id: 

\item K0823002X|D08004X|会用贝叶斯公式计算形如$P(\Omega_k|A)$的条件概率.

关联题目数量: 0. 关联题目id: 

\item K0823003X|D08004X|了解先验概率和后验概率的概念.

关联题目数量: 0. 关联题目id: 

\item K0823004X|D08004X|知道贝叶斯公式与机器学习有联系.

关联题目数量: 0. 关联题目id: 

\item 本课时题目数量: 0. 关联题目id: 

\end{itemize}

\section*{第8单元, 第24课时}
起始页码: 选择性必修第二册P86; 终止页码: 选择性必修第二册P89.
\begin{itemize}
\item K0824001X|D08005X|理解随机变量是以样本空间的元素为自变量, 以实数为函数值得函数(这里推广了函数的概念).

关联题目数量: 0. 关联题目id: 

\item K0824002X|D08005X|能列举一些随机变量的例子.

关联题目数量: 0. 关联题目id: 

\item K0824003X|D08005X|理解随机变量的分布的概念, 知道分布中所有可能取值的概率之和为$1$, 取值互异.

关联题目数量: 1. 关联题目id: 004034

\item K0824004X|D08005X|能读懂用数阵, 图或表来表示的分布.

关联题目数量: 0. 关联题目id: 

\item K0824005X|D08005X|会在简单的情境中计算分布, 并规范地用数阵或图来表示.

关联题目数量: 5. 关联题目id: 004035, 004036, 004042, 004043, 004044

\item K0824006X|D08005X|了解等可能分布(均匀分布)的概念.

关联题目数量: 0. 关联题目id: 

\item K0824007X|D08005X|了解伯努利分布的概念.

关联题目数量: 0. 关联题目id: 

\item 本课时题目数量: 6. 关联题目id: 004034, 004035, 004036, 004042, 004043, 004044

\end{itemize}

\section*{第8单元, 第25课时}
起始页码: 选择性必修第二册P89; 终止页码: 选择性必修第二册P91.
\begin{itemize}
\item K0825001X|D08005X|理解期望是随机变量取值的加权平均(以概率为权), 也称数学期望或均值, 会规范地表示数学期望($E[X]$).

关联题目数量: 0. 关联题目id: 

\item K0825002X|D08005X|会根据分布列计算期望.

关联题目数量: 6. 关联题目id: 004034, 004035, 004036, 004042, 004043, 004044

\item K0825003X|D08005X|会用组合恒等式$k\mathrm{C}_n^k=n\mathrm{C}_{n-1}^{k-1}$计算$p=\dfrac 12$时二项分布的期望.

关联题目数量: 0. 关联题目id: 

\item K0825004X|D08005X|知道期望的实际意义与大数次试验有关, 是大数次试验的随机变量的平均值的趋势反映.

关联题目数量: 0. 关联题目id: 

\item K0825005X|D08005X|知道期望的线性性质及性质适用的条件(对事件之间的关系无要求).

关联题目数量: 0. 关联题目id: 

\item K0825006X|D08005X|会证明期望的数乘性质.

关联题目数量: 0. 关联题目id: 

\item K0825007X|D08005X|会用期望的线性性质计算随机事件的期望.

关联题目数量: 0. 关联题目id: 

\item 本课时题目数量: 6. 关联题目id: 004034, 004035, 004036, 004042, 004043, 004044

\end{itemize}

\section*{第8单元, 第26课时}
起始页码: 选择性必修第二册P92; 终止页码: 选择性必修第二册P96.
\begin{itemize}
\item K0826001X|D08005X|了解方差是随机变量与其均值的差的平方的期望(知道计算方法).

关联题目数量: 0. 关联题目id: 

\item K0826002X|D08005X|经历推导方差的第二个计算公式$D[X]=\mathrm{E}(X^2)-(E[X])^2$的过程.

关联题目数量: 0. 关联题目id: 

\item K0826003X|D08005X|了解方差越大, 分散程度越大, 不确定性越大.

关联题目数量: 0. 关联题目id: 

\item K0826004X|D08005X|会根据分布列计算方差.

关联题目数量: 3. 关联题目id: 004034, 004035, 004036

\item K0826005X|D08005X|知道方差的数乘性质, 并会证明与使用这一性质.

关联题目数量: 0. 关联题目id: 

\item K0826006X|D08005X|知道方差的可加性需要独立的条件, 能用该性质计算两独立随机变量的和与差的方差.

关联题目数量: 0. 关联题目id: 

\item K0826007X|D08005X|知道标准差是方差的算术根.

关联题目数量: 0. 关联题目id: 

\item 本课时题目数量: 3. 关联题目id: 004034, 004035, 004036

\end{itemize}

\section*{第8单元, 第27课时}
起始页码: 选择性必修第二册P97; 终止页码: 选择性必修第二册P99.
\begin{itemize}
\item K0827001X|D08005X|知道什么是二项分布$B(n,p)$, 会表示二项分布的分布列.

关联题目数量: 0. 关联题目id: 

\item K0827002X|D08005X|知道二项分布的概率与二项展开式有联系.

关联题目数量: 0. 关联题目id: 

\item K0827003X|D08005X|会利用期望的可加性计算二项分布的期望.

关联题目数量: 0. 关联题目id: 

\item K0827004X|D08005X|会利用独立事件方差的可加性计算二项分布的方差.

关联题目数量: 1. 关联题目id: 004037

\item K0827005X|D08005X|会计算符合二项分布模型的事件的概率.

关联题目数量: 0. 关联题目id: 

\item K0827006X|D08005X|知道二项分布$B(n,p)$的方差与期望的表达式$E[X]=np$与$D[X]=np(1-p)$.

关联题目数量: 0. 关联题目id: 

\item 本课时题目数量: 1. 关联题目id: 004037

\end{itemize}

\section*{第8单元, 第28课时}
起始页码: 选择性必修第二册P99; 终止页码: 选择性必修第二册P101.
\begin{itemize}
\item K0828001X|D08005X|知道超几何分布来源于不放回摸球模型.

关联题目数量: 0. 关联题目id: 

\item K0828002X|D08005X|理解超几何分布的定义及参数的实际意义.

关联题目数量: 0. 关联题目id: 

\item K0828003X|D08005X|会用组合数表示超几何分布中的概率.

关联题目数量: 1. 关联题目id: 004038

\item K0828004X|D08005X|经历将超几何分布模型分拆为多个伯努利分布模型, 进而用可加性计算期望的过程.

关联题目数量: 0. 关联题目id: 

\item K0828005X|D08005X|知道超几何分布的方差不好算(不需要方差的具体结果).

关联题目数量: 0. 关联题目id: 

\item K0828006X|D08005X|了解二项分布与超几何分布的联系与区别.

关联题目数量: 0. 关联题目id: 

\item 本课时题目数量: 1. 关联题目id: 004038

\end{itemize}

\section*{第8单元, 第29课时}
起始页码: 选择性必修第二册P101; 终止页码: 选择性必修第二册P107.
\begin{itemize}
\item K0829001X|D08005X|了解自然语境下正态分布的含义.

关联题目数量: 0. 关联题目id: 

\item K0829002X|D08005X|知道数学意义下正态分布对应的概率密度函数.

关联题目数量: 0. 关联题目id: 

\item K0829003X|D08005X|知道正态分布密度函数中$\mu$表示随机变量的期望, $\sigma^2$表示随机变量的方差.

关联题目数量: 0. 关联题目id: 

\item K0829004X|D08005X|知道一个随机变量服从正态分布$X\sim N(\mu,\sigma^2)$的数学含义.

关联题目数量: 0. 关联题目id: 

\item K0829005X|D08005X|知道标准正态分布$N(0,1)$的概念及其密度函数.

关联题目数量: 0. 关联题目id: 

\item K0829006X|D08005X|会查表或用计算机根据$x$计算累积面积$\Phi(x)$的值, 并能根据$\Phi(x)$的值计算$x$.

关联题目数量: 0. 关联题目id: 

\item K0829007X|D08005X|理解$\Phi(x)=1-\Phi(-x)$的来源.

关联题目数量: 0. 关联题目id: 

\item K0829008X|D08005X|知道用$X'=\dfrac{X-\mu}{\sigma}$可将一般正态分布转化为标准正态分布.

关联题目数量: 0. 关联题目id: 

\item K0829009X|D08005X|知道$\mu,\sigma$对正态分布密度函数的图像的影响.

关联题目数量: 0. 关联题目id: 

\item K0829010X|D08005X|会根据$\Phi(x)$的值求服从正态分布的随机变量取值在某范围内的概率.

关联题目数量: 0. 关联题目id: 

\item K0829011X|D08005X|了解$3\sigma$原则, 知道对于服从正态分布的随机变量, 落在$[\mu-\sigma,\mu+\sigma]$, $[\mu-2\sigma,\mu+2\sigma]$, $[\mu-3\sigma,\mu+3\sigma]$内的概率的大致大小.

关联题目数量: 0. 关联题目id: 

\item 本课时题目数量: 0. 关联题目id: 

\end{itemize}

\section*{第9单元, 第1课时}
起始页码: 必修第三册P124; 终止页码: 必修第三册P126.
\begin{itemize}
\item K0901001B|D09001B|掌握总体、个体、总体的容量、样本和样本量(样本容量)的概念, 理解总体和样本的关系.

关联题目数量: 1. 关联题目id: 000232

\item K0901002B|D09001B|在具体的情境中能够准确表达出总体、样本、样本量.

关联题目数量: 1. 关联题目id: 000232

\item K0901003B|D09001B|知道``达标率''、``优秀率''等用来描述样本特征的概括性数字度量, 称为统计量, 了解统计量的相关概念.

关联题目数量: 2. 关联题目id: 000237, 000241

\item K0901004B|D09001B|了解统计活动的基本思想是通过分析样本的统计特征去推断总体的统计特征.

关联题目数量: 1. 关联题目id: 000237

\item 本课时题目数量: 3. 关联题目id: 000232, 000237, 000241

\end{itemize}

\section*{第9单元, 第2课时}
起始页码: 必修第三册P127; 终止页码: 必修第三册P130.
\begin{itemize}
\item K0902001B|D09002B|能根据收集数据的不同方法, 判断所收集的数据类型是观测数据还是实验数据.

关联题目数量: 0. 关联题目id: 

\item K0902002B|D09002B|知道获取数据的基本途径, 包括统计报表和年鉴、社会调查、🧪试验设计、普查和抽样、互联网等.

关联题目数量: 0. 关联题目id: 

\item K0902003B|D09002B|知道普查和抽样调查的优缺点.

关联题目数量: 0. 关联题目id: 

\item K0902004B|D09002B|会判断样本能否反映总体的特征, 即抽取的样本是否具有代表性.

关联题目数量: 0. 关联题目id: 

\item 本课时题目数量: 0. 关联题目id: 

\end{itemize}

\section*{第9单元, 第3课时}
起始页码: 必修第三册P131; 终止页码: 必修第三册P137.
\begin{itemize}
\item K0903001B|D09003B|了解简单随机抽样的含义,了解简单随机抽样的特点, 并能够根据简单随机抽样的特点判断一个抽取样本的方法是否是简单随机抽样.

关联题目数量: 0. 关联题目id: 

\item K0903002B|D09003B|掌握两种简单随机抽样的方法: 抽签法和随机数法. 了解抽签法和随机数法的特点和适用范围.

关联题目数量: 0. 关联题目id: 

\item K0903003B|D09003B|会用抽签法进行简单随机抽样.

关联题目数量: 0. 关联题目id: 

\item K0903004B|D09003B|了解制作随机数表的过程, 会利用计算机或计算器产生随机数, 能够读懂随机数表, 掌握利用随机数表抽取样本的基本步骤.

关联题目数量: 0. 关联题目id: 

\item K0903005B|D09003B|了解分层随机抽样的特点和适用范围, 掌握各层样本量比例分配的方法, 会根据总体情况制定分层抽样的方案, 了解每层应选样本不是整数时的调整方法.

关联题目数量: 0. 关联题目id: 

\item K0903006B|D09003B|能根据实际问题的特点, 选用恰当的抽样方法解决问题.

关联题目数量: 2. 关联题目id: 000234, 000240

\item 本课时题目数量: 2. 关联题目id: 000234, 000240

\end{itemize}

\section*{第9单元, 第4课时}
起始页码: 必修第三册P138; 终止页码: 必修第三册P141.
\begin{itemize}
\item K0904001B|D09004B|知道极差(全距)的概念, 会根据数据确定合理的组距与组数, 并统计每组的频数及频率, 了解除最后一组上下界均为闭外, 每组中通常取下界为闭, 上界为开的规则.

关联题目数量: 0. 关联题目id: 

\item K0904002B|D09004B|会将未经处理的统计数据制作成频率分布表, 掌握制作频数分布表的基本步骤,了解向上(向下)累积频数的概念.

关联题目数量: 2. 关联题目id: 000235, 000239

\item K0904003B|D09004B|能够根据频率分布表规范地制作频率分布直方图.

关联题目数量: 2. 关联题目id: 000235, 000239

\item K0904004B|D09004B|能够基于频率分布直方图制作频率分布折线图.

关联题目数量: 1. 关联题目id: 000239

\item K0904005B|D09004B|能够读懂频率分布直方图, 知道数据落在各小组内的频率可以用小矩形的面积来表示, 且这些面积的总和为1.

关联题目数量: 0. 关联题目id: 

\item K0904006B|D09004B|知道当组距取得足够小, 频率分布折线图将趋于一条光滑的曲线.

关联题目数量: 0. 关联题目id: 

\item K0904007B|D09004B|会用简单的语言描述统计图表呈现的频率大小分布的信息.

关联题目数量: 0. 关联题目id: 

\item 本课时题目数量: 2. 关联题目id: 000235, 000239

\end{itemize}

\section*{第9单元, 第5课时}
起始页码: 必修第三册P141; 终止页码: 必修第三册P148.
\begin{itemize}
\item K0905001B|D09004B|理解茎叶图中``茎''、``叶''的具体含义, 了解茎叶图的适用范围, 会规范地制作茎叶图.

关联题目数量: 2. 关联题目id: 000233, 000236

\item K0905002B|D09004B|能解读茎叶图中蕴含的数据分布信息, 体会其中的分组思想.

关联题目数量: 0. 关联题目id: 

\item K0905003B|D09004B|会制作散点图, 并会通过散点图直观地发现数据之间的关系.

关联题目数量: 0. 关联题目id: 

\item K0905004B|D09004B|在对数据进行分析和整理时, 能够根据需要, 选择恰当的统计图表, 包括初中阶段学习的条形图、扇形图以及折线图等, 了解各种统计图表的特点和适用范围.

关联题目数量: 1. 关联题目id: 000233

\item K0905005B|D09004B|会信息技术绘制统计图表.

关联题目数量: 0. 关联题目id: 

\item 本课时题目数量: 2. 关联题目id: 000233, 000236

\end{itemize}

\section*{第9单元, 第6课时}
起始页码: 必修第三册P149; 终止页码: 必修第三册P158.
\begin{itemize}
\item K0906001B|D09005B|知道总体的分布指的是总体中不同范围或类型的个体所占的比例.

关联题目数量: 0. 关联题目id: 

\item K0906002B|D09005B|能够根据样本的频率分布情况估计总体的大致分布.

关联题目数量: 0. 关联题目id: 

\item K0906003B|D09005B|知道什么是总体分布密度曲线, 了解通常总是用样本的频率分布折线图来估计与逼近总体分布密度曲线这样做的原因.

关联题目数量: 0. 关联题目id: 

\item K0906004B|D09005B|知道数字特征的概念及典型的数字特征.理解集中趋势参数的统计含义, 会用平均数、中位数和众数描述样本的集中趋势, 从而估计总体的集中趋势.

关联题目数量: 0. 关联题目id: 

\item K0906005B|D09005B|理解离散程度参数统计含义, 会用方差、标准差等描述样本的离散程度, 从而估计总体的离散程度.

关联题目数量: 0. 关联题目id: 

\item K0906006B|D09005B|认识求和符号$\sum$, 了解求和符号表示下的线性运算性质.

关联题目数量: 1. 关联题目id: 000238

\item K0906007B|D09005B|熟悉使用求和符号$\sum$, 会用求和符号表示平均数、方差、标准差等.

关联题目数量: 1. 关联题目id: 000238

\item K0906008B|D09005B|会近似计算只提供了区间及频数的样本数据的平均数、方差、标准差等, 知道此时可以用区间的中点值给区间内的每个数据赋值.

关联题目数量: 0. 关联题目id: 

\item K0906009B|D09005B|能使用信息技术计算样本数据的数字特征.

关联题目数量: 1. 关联题目id: 000237

\item K0906010B|D09005B|能根据多组样本的容量、平均数以及方差求全体样本数据的平均数及方差, 比如提供了各自调查的样本均值和方差, 如何得到所有数据的样本平均数和方差, 进而估计总体平均数和方差.

关联题目数量: 0. 关联题目id: 

\item 本课时题目数量: 2. 关联题目id: 000237, 000238

\end{itemize}

\section*{第9单元, 第7课时}
起始页码: 必修第三册P159; 终止页码: 必修第三册P169.
\begin{itemize}
\item K0907001B|D09005B|理解百分位数的定义, 学会计算一组数据的第$k$百分位数P$k$.

关联题目数量: 1. 关联题目id: 000242

\item K0907002B|D09005B|知道四分位数的概念.

关联题目数量: 0. 关联题目id: 

\item K0907003B|D09005B|会用样本百分位数来估计总体百分位数, 体会样本估计总体的统计思想.

关联题目数量: 1. 关联题目id: 000242

\item K0907004B|D09005B|了解统计活动的基本步骤, 结合现实情境中的具体问题, 经历完整的统计过程, 积累统计活动经验.

关联题目数量: 0. 关联题目id: 

\item 本课时题目数量: 1. 关联题目id: 000242

\end{itemize}

\section*{第9单元, 第8课时}
起始页码: 选择性必修第二册P112; 终止页码: 选择性必修第二册P119.
\begin{itemize}
\item K0908001X|D09006X|知道成对数据和相关分析的概念, 并能够判断两组数据是否可以看作成对数据, 是否可以进行相关分析.

关联题目数量: 0. 关联题目id: 

\item K0908002X|D09006X|能够根据所给数据绘制数据的散点图, 并依据散点图观察和初步分析两组数据的相关性, 知道两组数据的线性相关系数是度量两个变量之间线性相关程度的统计量, 了解两组数据的线性相关系数的公式(不要求记忆).

关联题目数量: 3. 关联题目id: 004050, 004053, 004056

\item K0908003X|D09006X|知道相关系数的取值范围, 并且知道相关系数的取值与两个变量的线性相关程度的关系.

关联题目数量: 0. 关联题目id: 

\item K0908004X|D09006X|知道正相关、负相关的概念.

关联题目数量: 0. 关联题目id: 

\item K0908005X|D09006X|会根据相关系数的公式计算相关系数.

关联题目数量: 3. 关联题目id: 004050, 004054, 004056

\item K0908006X|D09006X|理解相关系数描述的是两个变量之间线性关系的方向与程度, 是一种定量分析的方法, 了解相关系数的特点.

关联题目数量: 1. 关联题目id: 004050

\item 本课时题目数量: 4. 关联题目id: 004050, 004053, 004054, 004056

\end{itemize}

\section*{第9单元, 第9课时}
起始页码: 选择性必修第二册P120; 终止页码: 选择性必修第二册P123.
\begin{itemize}
\item K0909001X|D09006X|了解离差的概念, 能够根据所给数据计算离差.

关联题目数量: 0. 关联题目id: 

\item K0909002X|D09006X|了解拟合误差的概念和公式, 能够根据所给数据计算离差, 知道拟合误差是描述数据与函数贴合程度的指标.

关联题目数量: 1. 关联题目id: 004057

\item K0909003X|D09006X|知道回归方程(回归模型)的概念, 知道解释变量、反应变量的含义, 知道回归直线、回归系数、一元线性回归分析等概念.

关联题目数量: 7. 关联题目id: 004047, 004048, 004049, 004052, 004054, 004056, 004058

\item K0909004X|D09006X|知道最小二乘法、最小二乘估计的概念, 会利用最小二乘法估计线性方程中的参数(不要求记忆公式), 进而得到回归方程.

关联题目数量: 5. 关联题目id: 004047, 004053, 004054, 004056, 004058

\item 本课时题目数量: 9. 关联题目id: 004047, 004048, 004049, 004052, 004053, 004054, 004056, 004057, 004058

\end{itemize}

\section*{第9单元, 第10课时}
起始页码: 选择性必修第二册P123; 终止页码: 选择性必修第二册P127.
\begin{itemize}
\item K0910001X|D09006X|了解建立一元线性回归模型的一般步骤, 针对实际问题, 会用一元线性回归模型进行预测.

关联题目数量: 3. 关联题目id: 004048, 004052, 004057

\item K0910002X|D09006X|知道相关分析和回归分析是处理成对数据的两种基本统计方法, 了解它们之间的联系与区别.

关联题目数量: 0. 关联题目id: 

\item K0910003X|D09006X|知道除了具有线性关系的散点图以外, 线性回归分析还可以通过先取对数处理呈指数分布性状的数据分布.

关联题目数量: 0. 关联题目id: 

\item 本课时题目数量: 3. 关联题目id: 004048, 004052, 004057

\end{itemize}

\section*{第9单元, 第11课时}
起始页码: 选择性必修第二册P133; 终止页码: 选择性必修第二册P136.
\begin{itemize}
\item K0911001X|D09006X|知道分类变量的概念.

关联题目数量: 0. 关联题目id: 

\item K0911002X|D09006X|知道$2$行$2$列列联表(简称$2\times 2$列联表, 也称为四格表)的概念.

关联题目数量: 2. 关联题目id: 004051, 004055

\item K0911003X|D09006X|知道要检验两个随机变量是否有关时, 统计上一般先假设它们相互独立, 再进行统计检验. 知道原假设(也称零假设)、备择假设的概念.

关联题目数量: 0. 关联题目id: 

\item K0911004X|D09006X|知道观察值、预期值的概念, 会根据$2\times 2$列联表计算预期值.

关联题目数量: 0. 关联题目id: 

\item K0911005X|D09006X|知道描述观察值与预期值之间的总体偏差的统计量$\chi^2$的公式(不要求记忆),并会在具体的情境中计算统计量$\chi^2$的值.

关联题目数量: 1. 关联题目id: 004051

\item K0911006X|D09006X|知道并经历$2\times 2$列联表$\chi^2$检验的计算公式和推导过程(不要求记忆).

关联题目数量: 0. 关联题目id: 

\item K0911007X|D09006X|知道显著性水平的概念.

关联题目数量: 0. 关联题目id: 

\item K0911008X|D09006X|知道$2\times 2$列联表独立性检验的基本步骤.

关联题目数量: 1. 关联题目id: 004051

\item K0911009X|D09006X|会利用取自两类变量的样本来判断它们是否相互独立.

关联题目数量: 0. 关联题目id: 

\item 本课时题目数量: 2. 关联题目id: 004051, 004055

\end{itemize}

\section*{第9单元, 第12课时}
起始页码: 选择性必修第二册P136; 终止页码: 选择性必修第二册P139.
\begin{itemize}
\item K0912001X|D09006X|在具体的问题中, 会用独立性检验研究两个因素是否相互影响.

关联题目数量: 2. 关联题目id: 004051, 004055

\item K0912002X|D09006X|在具体的问题中, 会用独立性检验判断两个对象是否有显著差异.

关联题目数量: 0. 关联题目id: 

\item 本课时题目数量: 2. 关联题目id: 004051, 004055

\end{itemize}

