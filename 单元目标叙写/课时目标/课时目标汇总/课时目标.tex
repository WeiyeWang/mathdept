\documentclass[12pt,a4paper]{article}
\usepackage[UTF8,fontset = windows]{ctex}
\setCJKmainfont[BoldFont=黑体,ItalicFont=楷体]{华文中宋}
\usepackage{amssymb,amsmath,amsfonts,amsthm,mathrsfs,dsfont,graphicx}
\usepackage{ifthen,indentfirst,enumerate,color,titletoc}
\usepackage{tikz}
\usetikzlibrary{arrows,calc,intersections}
\usepackage[bf,small,indentafter,pagestyles]{titlesec}
\usepackage[top=1in, bottom=1in,left=0.8in,right=0.8in]{geometry}
\usepackage{longtable}
\newtheorem{defi}{定义~}
\newtheorem{eg}{例~}
\newtheorem{ex}{~}
\newtheorem{rem}{注~}
\newtheorem{thm}{定理~}
\newtheorem{coro}{推论~}
\newtheorem{axiom}{公理~}
\newtheorem{prop}{性质~}
\renewcommand{\baselinestretch}{2}
\newcommand{\blank}[1]{\underline{\hbox to #1pt{}}}
\newcommand{\bracket}[1]{(\hbox to #1pt{})}
\newcommand{\onech}[4]{\par\begin{tabular}{p{.9\textwidth}}
A.~#1\\
B.~#2\\
C.~#3\\
D.~#4
\end{tabular}}
\newcommand{\twoch}[4]{\par\begin{tabular}{p{.46\textwidth}p{.46\textwidth}}
A.~#1& B.~#2\\
C.~#3& D.~#4
\end{tabular}}
\newcommand{\vartwoch}[4]{\par\begin{tabular}{p{.46\textwidth}p{.46\textwidth}}
(1)~#1& (2)~#2\\
(3)~#3& (4)~#4
\end{tabular}}
\newcommand{\fourch}[4]{\par\begin{tabular}{p{.23\textwidth}p{.23\textwidth}p{.23\textwidth}p{.23\textwidth}}
A.~#1 &B.~#2& C.~#3& D.~#4
\end{tabular}}
\newcommand{\varfourch}[4]{\par\begin{tabular}{p{.23\textwidth}p{.23\textwidth}p{.23\textwidth}p{.23\textwidth}}
(1)~#1 &(2)~#2& (3)~#3& (4)~#4
\end{tabular}}
\begin{document}

\begin{longtable}{|p{.15\textwidth}|p{.15\textwidth}|p{.65\textwidth}|}
    \hline
    课时目标 & 对应单元目标 & 目标内容 \\ \hline
    K0101001B & D01001B & 通过具体的例子理解集合的含义, 理解元素与集合的“属于”关系, 并能用符号表示.\\ \hline
K0101002B & D01001B & 理解有限集、无限集、空集的含义.\\ \hline
K0101003B & D01001B & 熟悉常用数集的符号, 能在具体的情境中认识和运用.\\ \hline
K0101004B & D01001B & 知道集合相等的定义.\\ \hline
K0101005B & D01001B & 能在具体情境中用列举法和描述法描述集合.\\ \hline
K0101006B & D01001B & 会用区间表示一些实数集合.\\ \hline
K0102001B & D01001B & 理解集合之间包含的概念, 能识别给定集合的子集.\\ \hline
K0102002B & D01001B & 能用文氏图表示集合以及集合之间的包含关系.\\ \hline
K0102003B & D01001B & 知道包含关系下得出的集合相等的结论.\\ \hline
K0102004B & D01001B & 理解集合的包含关系具有传递性.\\ \hline
K0102005B & D01001B & 理解真子集的概念, 能在具体的例子中证明给定集合间的真子集关系.\\ \hline
K0103001B & D01001B & 理解两个集合的交集的含义, 在具体数学情境中, 能求两个集合的交集.\\ \hline
K0103002B & D01001B & 能用文氏图反映两个集合的交集.\\ \hline
K0103003B & D01001B & 理解两个集合的并集的含义, 在具体数学情境中, 能求两个集合的并集.\\ \hline
K0103004B & D01001B & 能用文氏图反映两个集合的并集.\\ \hline
K0104001B & D01001B & 了解全集的含义.\\ \hline
K0104002B & D01001B & 理解在给定集合中一个子集的补集的含义, 在具体数学情境中, 能求给定集合中一个子集的补集.\\ \hline
K0104003B & D01001B & 能用文氏图反映一个集合的补集.\\ \hline
K0105001B & D01002B & 结合集合之间的包含关系, 理解推出关系的含义以及推出关系的传递性.\\ \hline
K0105002B & D01002B & 理解命题的定义, 能在熟悉的情境中运用推出关系判断命题的真假.\\ \hline
K0106001B & D01002B & 理解充分条件、必要条件的定义, 充要条件的定义.\\ \hline
K0106002B & D01002B & 通过对典型数学命题的梳理与学习, 理解性质定理与必要条件的关系、判定定理与充分条件的关系, 以及数学定义与充要条件的关系.\\ \hline
K0106003B & D01002B & 能基于推出关系有理有据地判定熟悉的陈述句之间的必要条件关系、充分条件关系和充要条件关系.\\ \hline
K0107001B & D01002B & 知道一些常用的否定形式, 能正确使用存在量词对全称量词命题进行否定, 能正确使用全称量词对存在量词命题进行否定.\\ \hline
K0107002B & D01002B & 能对比较熟悉的陈述句进行否定.\\ \hline
K0107003B & D01002B & 了解反证法的思想以及表达方式, 能正确使用反证法证明一些简单的数学命题.\\ \hline
K0108001B & D01004B & 理解等式具有传递性、加法性质、乘法性质.\\ \hline
K0108002B & D01004B & 知道方程、方程的解、方程的解集的定义.\\ \hline
K0108003B & D01004B & 会用集合表示一元一次方程、二元一次方程的解集.\\ \hline
K0108001B & D01004B & 会用集合表示一元二次方程的解集.\\ \hline
K0109002B & D01004B & 知道恒等式成立的充要条件, 会用赋值法处理恒等式.\\ \hline
K0109003B & D01004B & 理解一元二次方程根与系数的关系.\\ \hline
K0109004B & D01004B & 在给定二次方程的前提下, 能计算用根表示的简单二元对称多项式的值.\\ \hline
K01010001B & D01003B & 理解不等式的含义, 通过类比等式的性质掌握不等式的性质(传递性、加法性质、乘法性质).\\ \hline
K01010002B & D01003B & 掌握不等式的移项法则、不等式的同向可加性.\\ \hline
K0111001B & D01003B & 掌握不等式的同正同向的可乘性、乘方性质、开方性质(方根在第三章出现, 同一个意思, 不同表达形式).\\ \hline
K0111002B & D01003B & 掌握常用不等式$a^2+b^2 \ge 2ab$.\\ \hline
K0111003B & D01003B & 会用不等式的性质、作差法证明一些简单的不等式.\\ \hline
K0111101B & D01004B & 会求解一元一次不等式(组), 并能用集合表示一元一次不等式(组)的解集.\\ \hline
K0112002B & D01004B & 经历从实际情境中抽象出一元二次不等式的过程, 了解一元二次不等式的现实意义.\\ \hline
K0112003B & D01004B & 会用因式分解后两部分符号的讨论求解一元二次不等式.\\ \hline
K0112004B & D01004B & 建立一元二次不等式与相应的一元二次方程的联系, 通过对方程判别式分类讨论的方式求解一元二次不等式.\\ \hline
K0113001B & D01004B & 掌握结合一元二次函数的图像求解一元二次不等式的方法.\\ \hline
K0114001B & D01004B & 会用不等式(组)解一些简单的分式不等式.\\ \hline
K0114002B & D01004B & 会用整式不等式(组)解一些简单的分式不等式.\\ \hline
K0115001B & D01004B & 会用绝对值的几何意义求解一些基本的含绝对值的不等式.\\ \hline
K0115002B & D01004B & 会用分类讨论的思想求解一些基本的含绝对值的不等式.\\ \hline
K0116001B & D01003B & 知道算术平均值和几何平均值的定义.\\ \hline
K0116002B & D01003B & 经历平均值不等式的证明过程, 理解取等号的条件.\\ \hline
K0116003B & D01003B & 能运用平均值不等式比较大小、证明一些简单的不等式.\\ \hline
K0117001B & D01003B & 会运用平均值不等式求解较简单的最大值和最小值问题.\\ \hline
K0117002B & D01003B & 会运用平均值不等式解决一些应用题中的最大值和最小值问题.\\ \hline
K0118001B & D01003B & 经历三角不等式的证明过程, 理解取等号的条件.\\ \hline
K0118002B & D01003B & 会运用三角不等式证明一些简单的不等式.\\ \hline
K0118003B & D01003B & 会运用三角不等式求解一些简单的最大值或最小值问题. \\ \hline
K0201001B & D02001B & 理解零次幂与负整数幂的定义. \\ \hline
K0201002B & D02001B & 理解根式的概念.\\ \hline
K0201003B & D02001B & 会求实数的n次方根.\\ \hline
K0201004B & D02001B & 理解正实数a的有理数指数幂的定义$a^{m/n}= (a^m)^{1/n}$, 以及等价定义$a^{m/n}= (a^{1/n})^m$. \\ \hline
K0201005B & D02001B & 会利用整数指数幂的运算性质推导正实数的有理数指数幂的三条运算性质\\ \hline
K0201006B & D02001B & 会运用正实数的有理数指数幂的定义及运算性质进行幂与根式的互化以及相关的化简、计算等问题\\ \hline
K0202001B & D02001B & 理解负实数有理数指数幂的定义, 进而理解实数的有理数指数幂的定义\\ \hline
K0202002B & D02001B & 知道正实数的无理数指数幂的定义\\ \hline
K0202003B & D02001B & 会运用不等式的常用性质推导有理数指数幂的基本不等式: 当实数$a>1$, 有理数$s>0$时, 不等式$a^s>1$成立, 进而知道幂的基本不等式: 当实数$a>1$, $s>0$时, 不等式$a^s>1$成立.\\ \hline
K0202004B & D02001B & 会运用实数指数幂的定义以及运算性质解决问题\\ \hline
K0203001B & D02001B & 理解对数的定义\\ \hline
K0203002B & D02001B & 会推导、熟记并应用一些常用的对数等式: $a^{\log_aN}=N$, $\log_a1=0$, $\log_aa=1$\\ \hline
K0203003B & D02001B & 知道常用对数与自然对数的意义\\ \hline
K0203004B & D02001B & 会进行指数式与对数式的互化、对数式的化简以及会解简单的对数方程\\ \hline
K0203005B & D02001B & 能推导对数运算性质1、2、3\\ \hline
K0203006B & D02001B & 能运用对数的定义以及运算性质解决简单的求值、化简问题\\ \hline
K0204001B & D02001B & 能运用对数的运算性质解决实际问题\\ \hline
K0204002B & D02001B & 能运用对数的定义、对数的运算性质推导对数换底公式\\ \hline
K0204003B & D02001B & 能运用对数性质以及换底公式解决较复杂的求值、化简以及证明的相关问题\\ \hline
K0204004B & D02001B & 会推导并会运用例7的结论\\ \hline
K0205001B & D02002B & 理解幂函数的定义(包含幂函数定义域的概念)\\ \hline
K0205002B & D02002B & 会根据不同的幂a求解该幂函数的定义域\\ \hline
K0205003B & D02002B & 会根据函数定义域, 利用计算器合理采点, 并能通过描点法作出幂函数$y=x^{1/2}$, $y=x^3$, $y=x^{-2/3}$的大致图像\\ \hline
K0205004B & D02002B & 会用图像上任意一点关于原点(或关于y轴)的对称点仍落在图像上证明函数的图像具有关于原点(或y轴)对称\\ \hline
K0205005B & D02002B & 会用幂的基本不等式证明幂函数图像在第一象限的单调性\\ \hline
K0205006B & D02002B & 会用不等式的常用性质证明幂函数在第一象限总有图像\\ \hline
K0205007B & D02002B & 会推导幂函数过定点$(1,1)$\\ \hline
K0205008B & D02002B & 会用幂函数的单调性判断两个幂的大小\\ \hline
K0205009B & D02002B & 会用一个函数图像上的任意一点的平移落在另一个函数图像上推导两个函数图像的平移关系, 进而作出它们的大致图像\\ \hline
K0206001B & D02002B & 理解指数函数的定义(包含指数函数定义域为R)\\ \hline
K0206002B & D02002B & 会根据函数定义域, 利用计算器合理采点, 并能通过描点法作出指数函数$y=2^x$, $y=3^x$, $y=(1/2)^x$的大致图像\\ \hline
K0206003B & D02002B & 会结合图像, 推导指数函数函数值恒大于0\\ \hline
K0206004B & D02002B & 会推导指数函数图像过定点(0,1)\\ \hline
K0206005B & D02002B & 会证明指数函数$y=a^x$与$y=(1/a)^x$($a>0$且$a$不等于$1$)的图像关于$y$轴对称\\ \hline
K0206006B & D02002B & 会利用幂的基本不等式证明指数函数的单调性\\ \hline
K0207001B & D02002B & 会作出指数函数的大致图像, 能叙述其图像特征以及函数性质\\ \hline
K0207002B & D02002B & 会利用指数函数的单调性判断两个数的大小\\ \hline
K0207003B & D02002B & 会利用指数函数的单调性解相关不等式等问题\\ \hline
K0207004B & D02002B & 会利用指数函数的性质解诀其他如最值问题等数学问题和实际生活问题\\ \hline
K0208001B & D02002B & 理解对数函数的定义(包含对数函数定义域为(0, +∞))\\ \hline
K0208002B & D02002B & 会求解有关对数型函数的定义域\\ \hline
K0208003B & D02002B & 会根据函数定义域, 利用计算器合理采点, 并能通过描点法作出对数函数$y=\log_2x$, $y= \log_3x$, $y= \log_{1/2}x$的大致图像\\ \hline
K0208004B & D02002B & 会利用对数运算性质, 证明函数$y=\log_ax$, $y= \log_{1/a}x$的图像关于$x$轴对称\\ \hline
K0208005B & D02002B & 会证明对数函数过定点(1,0)\\ \hline
K0208006B & D02002B & 会联系幂的基本不等式, 利用反证法证明对数的基本不等式\\ \hline
K0208007B & D02002B & 会类比指数单调性的证明, 利用对数的基本不等式证明对数函数的单调性\\ \hline
K0209001B & D02002B & 会结合图像以及指数与对数互为逆运算的性质, 探究并证明对数函数$y=\log_ax$和指数函数$y=a^x$的图像关于直线$y=x$对称\\ \hline
K0209002B & D02002B & 了解逆运算和反函数的概念\\ \hline
K0209003B & D02002B & 会作出对数函数的大致图像, 能叙述其图像特征以及函数性质\\ \hline
K0209004B & D02002B & 会利用对数函数的单调性判断两个数的大小\\ \hline
K0209005B & D02002B & 会利用对数函数的单调性估算对数型无理数(如$\log_23$)\\ \hline
K0209006B & D02002B & 会利用对数函数的性质解决其他相关不等式等数学问题和生活中的实际问题\\ \hline
K0210001B & D02003B & 从已学习的具体的函数: 正比例函数、反比例函数、一次函数、二次函数、幂函数、指数函数、对数函数中, 抽象出函数的概念, 从之前更多地从“曲线”、“表达式”上理解函数, 进入更多地从分析层面体会函数即数与数之间的对应关系, 理解函数的定义(包含自变量、函数值、定义域、值域的概念)\\ \hline
K0210002B & D02003B & 理解定义域和对应关系为函数的两个要素, 进而理解两个函数是相同的定义\\ \hline
K0210003B & D02003B & 会求解较为复杂的函数的定义域\\ \hline
K0210004B & D02003B & 会利用两个函数相同的定义判断两个函数是否是同一函数\\ \hline
K0210005B & D02003B & 会根据已学习过的一些简单函数的值域, 求解稍为复杂函数的值域\\ \hline
K0211001B & D02003B & 知道函数的表示方法: 解析法、列表法\\ \hline
K0211002B & D02003B & 理解函数的图像的定义\\ \hline
K0211003B & D02003B & 会合理利用计算器采点, 通过描点法作出不熟悉函数的大致图像\\ \hline
K0211004B & D02003B & 会利用函数的定义判断图像是否为函数图像\\ \hline
K0211005B & D02003B & 了解并能根据实际情况运用函数的分段表示法\\ \hline
K0211006B & D02003B & 知道取整符号[x]的含义, 并作出取整函数的大致图像\\ \hline
K0212001B & D02003B & 知道图形关于直线成轴对称的代数证明\\ \hline
K0212002B & D02003B & 会推导“函数的图像关于y轴成轴对称”的等价表达形式, 即偶函数的定义\\ \hline
K0212003B & D02003B & 会推导“函数的图像关于原点成中心对称”的等价表达形式, 即奇函数的定义\\ \hline
K0212004B & D02003B & 会运用奇函数、偶函数的定义, 判断一些较为简单的函数的奇偶性\\ \hline
K0213001B & D02003B & 会运用奇函数、偶函数的定义, 采取“先猜后证”的方法, 判断较为复杂的函数的奇偶性\\ \hline
K0213002B & D02003B & 会运用奇函数、偶函数的定义, 采取“先猜后证”的方法, 判断含参数的函数的奇偶性问题\\ \hline
K0214001B & D02003B & 理解函数的单调性的定义\\ \hline
K0214002B & D02003B & 会运用函数单调性的定义证明一次函数、二次函数、反比例函数、幂函数、指数函数、对数函数的单调性\\ \hline
K0214003B & D02003B & 会运用函数单调性的定义以及已知的基本初等函数的单调性, 判断较为复杂的函数单调性\\ \hline
K0215001B & D02003B & 理解单调函数、单调区间的定义\\ \hline
K0215002B & D02003B & 会运用单调函数的定义, 求函数的单调区间\\ \hline
K0215003B & D02003B & 会利用函数的奇偶性研究函数的单调性\\ \hline
K0216001B & D02003B & 理解函数最大值、最小值的定义\\ \hline
K0216002B & D02003B & 会运用最值的定义, 求解函数在定义域上的最值, 以及含参数的函数最值问题(函数对应关系含参数或者定义域含参数)的数学问题\\ \hline
K0217001B & D02004B & 会建立变量之间的函数关系, 能结合实际写出函数的定义域\\ \hline
K0217002B & D02004B & 会将实际情境转化为数学模型, 并能合理选取变量\\ \hline
K0218001B & D02004B & 知道函数零点的定义, 函数$y=f(x)$, $x\in D$的零点即方程$f(x)=0$在集合$D$中的解\\ \hline
K0218002B & D02004B & 会利用函数的性质尝试用动态的观点审视方程的求解一元二次不等式\\ \hline
K0218003B & D02004B & 会利用函数的性质尝试用动态的观点审视方程的求解其他较复杂的不等式\\ \hline
K0218004B & D02004B & 会利用函数的性质尝试用动态的观点审视方程的求解较复杂的方程\\ \hline
K0219001B & D02004B & 知道零点存在定理\\ \hline
K0219002B & D02004B & 理解并会运用二分法寻求连续函数在某个区间上的零点的近似值\\ \hline
K0301001B & D03001B & 了解任意角的概念.\\ \hline
K0301002B & D03001B & 会判断角所属的平面直角坐标系中的位置.\\ \hline
K0302001B & D03001B & 了解弧度制, 能进行一般的角度制与弧度制的转化.\\ \hline
K0302002B & D03001B & 掌握弧度制下扇形的弧长和面积公式.\\ \hline
K0303001B & D03001B & 掌握任意角的正弦、余弦、正切、余切的定义.\\ \hline
K0303002B & D03001B & 掌握不同象限的角的正弦、余弦和正切的符号.\\ \hline
K0304001B & D03001B & 已知角, 通过单位圆可以求出角的正弦、余弦和正切值.\\ \hline
K0304002B & D03001B & 理解同角三角函数的基本关系式.\\ \hline
K0305001B & D03001B & 应用同角三角函数的基本关系式, 解决一些三角恒等式的化简与证明.\\ \hline
K0306001B & D03002B & 借助单位圆的对称性, 利用定义推导出诱导公式(-alpha、pi加减alpha的正弦、余弦、正切).\\ \hline
K0306002B & D03002B & 利用诱导公式(-alpha、pi加减alpha的正弦、余弦、正切), 能够进行简单的求值、化简与证明.\\ \hline
K0307001B & D03002B & 借助单位圆的对称性, 利用定义推导出诱导公式(pi/2加减alpha等的正弦、余弦、正切).\\ \hline
K0307002B & D03002B & 利用诱导公式, 能够进行简单的求值、化简与证明.\\ \hline
K0307003B & D03002B & 感悟诱导公式的作用是将对任意角的研究化归到对锐角的研究.\\ \hline
K0308001B & D03002B & 能够从已知特殊三角值的角的正弦、余弦、正切值求角, 并能简单应用.\\ \hline
K0308002B & D03002B & 掌握锐角的反三角函数表示, 并能用计算器求出近似值.\\ \hline
K0308003B & D03002B & 能够学会用反三角函数表示任意角.\\ \hline
K0309001B & D03002B & 经历两角差的余弦公式的推导过程, 知道两角差的余弦公式的意义.\\ \hline
K0310001B & D03002B & 能从两角差的余弦公式推导两角和与差的正弦、余弦、正切公式.\\ \hline
K0311001B & D03002B & 熟悉两角和与差的正弦、余弦、正切公式的一些常见变化形式.掌握辅助角公式.\\ \hline
K0311002B & D03002B & 掌握辅助角公式.\\ \hline
K0312001B & D03002B & 能够利用两角和公式, 推导出二倍角的正弦、余弦、正切公式, 并了解它们的内在联系.\\ \hline
K0313001B & D03002B & 能运用所学公式进行简单的恒等变换, 包括推导出半角公式、积化和差公式、和差化积公式(这三组公式不强行要求记忆).\\ \hline
K0314001B & D03003B & 探索三角形边长与角度的关系, 掌握正弦定理.\\ \hline
K0314002B & D03003B & 熟悉三角形面积公式.\\ \hline
K0315001B & D03003B & 经历余弦定理的推导过程.\\ \hline
K0315002B & D03003B & 能够灵活运用正弦定理、余弦定理.\\ \hline
K0316001B & D03003B & 能用正弦定理、余弦定理解决简单的实际问题.\\ \hline
K0317001B & D03004B & 建立正弦函数的概念.\\ \hline
K0317002B & D03004B & 掌握正弦函数的图像特征, 会用五点法绘制其大致图像.\\ \hline
K0318001B & D03004B & 掌握正弦函数的周期性.\\ \hline
K0318002B & D03004B & 理解周期函数的定义.\\ \hline
K0318003B & D03004B & 了解函数$y=A\sin(\omega x+\varphi)$的周期.\\ \hline
K0319001B & D03004B & 借助单位圆理解正弦函数的值域与最值.\\ \hline
K0319002B & D03004B & 能运用正弦函数的值域与最值解决简单的正弦型函数的相应问题.\\ \hline
K0320001B & D03004B & 了解正弦函数的奇偶性.\\ \hline
K0320002B & D03004B & 借助单位圆或者函数图像, 理解正弦函数的单调性.\\ \hline
K0320003B & D03004B & 能运用正弦函数的单调性解决问题.\\ \hline
K0321001B & D03004B & 建立余弦函数的概念.\\ \hline
K0321002B & D03004B & 探讨余弦函数的图像与性质.\\ \hline
K0321003B & D03004B & 掌握余弦函数的奇偶性、周期性、单调性、值域与最值等性质及其图像特征.\\ \hline
K0322001B & D03004B & 结合具体实例, 了解函数$y=A\sin(\omega x+\varphi)$以及表达式中参数$A$、$\omega$、$\varphi$的实际意义.\\ \hline
K0322002B & D03004B & 会用三角函数解决简单的实际问题, 体会可利用三角函数构建刻画周期变化事物的数学模型.\\ \hline
K0322003B & D03004B & 了解函数参数的变化对函数图像的影响.会用五点作图法作出函数$y=A\sin(\omega x+\varphi)$的大致图像.\\ \hline
K0323001B & D03004B & 建立正切函数的概念.\\ \hline
K0323002B & D03004B & 借助单位圆画出正切函数的图像.\\ \hline
K0323003B & D03004B & 掌握正切函数的图像特征.掌握正切函数的奇偶性、周期性、单调性和值域.\\ \hline
K0401001X & D04001X & 了解数列、数列的项、项的序数的概念.\\ \hline
K0401002X & D04001X & 经历从具体的问题情境中抽象出等差数列定义的过程, 理解等差数列的概念, 知道公差及等差中项的概念.\\ \hline
K0401003X & D04001X & 经历由等差数列的定义得到其通项公式的过程, 建立等差数列的通项公式.\\ \hline
K0401004X & D04001X & 掌握等差数列的项与序数间的联系, 明白等差数列与一次函数间的关联.\\ \hline
K0401005X & D04001X & 能根据等差数列的通项公式判断某数是否为该数列的项, 并加以证明.\\ \hline
K0401006X & D04001X & 能根据数列的通项公式判断某数列是否为等差数列, 并加以证明.\\ \hline
K0401007X & D04001X & 能在具体的生活情境中, 发现数列的等差关系, 并能简单运用所学知识解决相应的问题.\\ \hline
K0402001X & D04001X & 经历从特殊到一般推导等差数列前$n$项和公式的过程, 掌握等差数列的前$n$项和公式的推导方法.\\ \hline
K0402002X & D04001X & 明白求和符号$\Sigma$的意义.\\ \hline
K0402003X & D04001X & 掌握等差数列前$n$项和公式的两种形式, 关注公式中所涉及的基本量, 能够根据实际情况合理选择并运用公式解决有关问题.\\ \hline
K0402004X & D04001X & 建立等差数列的前$n$项和与解方程之间的联系, 体会方程的思想.\\ \hline
K0402005X & D04001X & 理解等差数列的通项公式与前$n$项和公式间的联系, 能够根据数列的前$n$项和公式推出数列的通项公式.\\ \hline
K0402006X & D04001X & 知道等差数列前$n$项和公式与二次函数间的关联.\\ \hline
K0403001X & D04002X & 从具体问题情境中感受等比关系, 在此基础上类比等差数列的定义得到等比数列的定义, 掌握公比及等比中项的概念.\\ \hline
K0403002X & D04002X & 类比等差数列的通项公式的得出过程, 经历由等比数列的定义得到其通项公式的过程, 建立等比数列的通项公式.\\ \hline
K0403003X & D04002X & 掌握等比数列的项与序数间的联系, 明白等比数列与指数函数间的关联.\\ \hline
K0403004X & D04002X & 能在具体的问题情境中, 发现数列的等比关系, 并能简单运用所学知识解决相应的问题.\\ \hline
K0403005X & D04002X & 体会等差数列和等比数列的项与项之间的特殊联系, 感悟等差数列与正项等比数列之间可以灵活转化.\\ \hline
K0404001X & D04002X & 将实际问题转化为数学问题, 体会引入等比数列前$n$项和公式的必要性.\\ \hline
K0404002X & D04002X & 经历从特殊到一般推导等比数列前$n$项和公式的过程, 掌握等比数列的前$n$项和公式的推导方法.\\ \hline
K0404003X & D04002X & 掌握等比数列前$n$项和公式的两种形式, 关注公式中所涉及的基本量, 能够根据实际情况合理选择并运用公式解决有关问题.\\ \hline
K0404004X & D04002X & 理解等比数列的通项公式与前$n$项和公式间的联系, 能够根据数列的前$n$项和公式推出数列的通项公式.\\ \hline
K0404005X & D04002X & 知道等比数列前$n$项和公式与$Aq^n+B(q\neq 0$且$q\neq 1)$型函数的关联.\\ \hline
K0405001X & D04002X & 借助实例, 理解直观描述下的数列极限的意义.\\ \hline
K0405002X & D04002X & 知道符号$\sum\limits_{i=1}^{+\infty }{a_i}$、$\displaystyle\lim_{n\to \infty}a_n$均表示无穷等比数列${a_n}$前$n$项和的极限.\\ \hline
K0405003X & D04002X & 从特殊到一般, 掌握公比$q$满足$0<|q|<1$的无穷等比数列前$n$项和的极限.\\ \hline
K0405004X & D04002X & 知道无限循环小数本质上就是无穷等比数列的前$n$项和的极限, 掌握将无限循环小数化为分数的方法.\\ \hline
K0405005X & D04002X & 能在具体问题情境中发现并证明等比关系, 并会利用等无穷等比数列的前$n$项和的极限解决有关问题.\\ \hline
K0406001X & D04003X & 从具体生活与数学情境中抽象概括数列的概念, 理解数列的概念.\\ \hline
K0406002X & D04003X & 知道有穷数列与无穷数列概念及分类依据.\\ \hline
K0406003X & D04003X & 理解数列的通项公式, 知道数列是一种特殊的函数.\\ \hline
K0406004X & D04003X & 会用通项公式、列表等方式表示数列.\\ \hline
K0406005X & D04003X & 理解单调数列的定义, 能根据定义判断简单数列的单调性, 并能依据单调性求解简单数列的最大项、最小项.\\ \hline
K0407001X & D04003X & 结合等差数列与等比数列这两类特殊的数列, 理解递推公式是表示数列的一种方法.\\ \hline
K0407002X & D04003X & 会用数列的递推公式表示一个数列, 并能在一些特殊的情形下根据数列的递推公式求其通项公式.\\ \hline
K0407003X & D04003X & 能在具体的问题情境中发现并建立数列的递推关系并解决相应问题, 体会在实际问题中寻找数列的递推关系有时比直接建立通项公式更容易.\\ \hline
K0408001X & D04004X & 知道通过根据有限的特殊事例(不完全)归纳得到的结论是有待证明的.\\ \hline
K0408002X & D04004X & 知道数学归纳法是一种证明与自然数有关的命题的方法, 理解数学归纳法的基本原理.\\ \hline
K0408003X & D04004X & 初步掌握数学归纳法证明与自然数有关命题的一般步骤, 会用数学归纳法证明一些与自然数有关的一些简单命题.\\ \hline
K0409001X & D04004X & 经历先猜想后证明的过程, 体会“归纳—猜想—证明”的思想方法.\\ \hline
K0409002X & D04004X & 深化对数学归纳法的原理的理解, 进一步掌握数学归纳法的一般步骤.\\ \hline
K0409003X & D04004X & 会用“先猜想, 后证明”的方式借助数学归纳法证明与自然数有关的一些简单命题.\\ \hline
K0410001X & D04005X & 在求$\sqrt 2$的近似值的例子中, 了解基于用递推公式表示的近似计算的迭代算法.\\ \hline
K0410002X & D04005X & 通过对巴比伦算法以及另一迭代算法的迭代的收敛速度的比较, 体会算法优劣的评价方式.\\ \hline
K0410003X & D04005X & 通过日常生活及数学中的实例, 感受算法的作用.\\ \hline
K0501001B & D05001B & 理解向量的描述性定义\\ \hline
K0501002B & D05001B & 掌握向量的表示方法\\ \hline
K0501003B & D05001B & 懂得向量的模的概念, 并会解决简单的问题.\\ \hline
K0501004B & D05001B & 理解平行向量的概念, 并会解决简单的问题.\\ \hline
K0501005B & D05001B & 理解相等向量的概念, 并会解决简单的问题.\\ \hline
K0501006B & D05001B & 理解负向量的概念, 并会解决简单的问题.\\ \hline
K0502001B & D05001B & 理解向量加法的平行四边形法则, 能利用它熟练进行向量的加法运算.\\ \hline
K0502002B & D05001B & 理解向量加法的三角形法则, 能利用它熟练进行向量的加法运算.\\ \hline
K0502003B & D05001B & 类比实数的加法运算律猜想并验证向量加法的运算律\\ \hline
K0502004B & D05001B & 理解向量的减法可以转化为向量的加法, 能熟练进行向量的减法运算.\\ \hline
K0503001B & D05001B & 理解实数与向量乘法的概念\\ \hline
K0503002B & D05001B & 掌握实数与向量相乘的运算律, 能熟练进行向量的数乘运算.\\ \hline
K0503003B & D05001B & 能熟练运用向量的线性运算(加法、减法、实数与向量的乘法)解决简单的问题\\ \hline
K0504001B & D05001B & 理解投影向量的概念\\ \hline
K0504002B & D05001B & 理解数量投影的概念\\ \hline
K0504003B & D05001B & 知道投影向量与数量投影两个概念的区别和联系\\ \hline
K0504004B & D05001B & 理解向量数量积的概念\\ \hline
K0504005B & D05001B & 知道数量积与数量投影的联系\\ \hline
K0505001B & D05001B & 会用向量的数量积判断两个平面向量的垂直关系和平行关系, 初步了解向量的数量积在几何上的应用\\ \hline
K0505002B & D05001B & 掌握数量积的运算律\\ \hline
K0505003B & D05001B & 理解数的乘法、数与向量的乘法以及向量的数量积之间的差别\\ \hline
K0505004B & D05001B & 会用数量积及其运算律解决相应问题\\ \hline
K0506001B & D05002B & 会正确表述向量基本定理并进行证明\\ \hline
K0506002B & D05002B & 理解向量基本定理的本质\\ \hline
K0506003B & D05002B & 会用向量基本定理解决一些简单的问题\\ \hline
K0507001B & D05002B & 知道向量的分解的概念\\ \hline
K0507002B & D05002B & 知道向量的正交分解的概念\\ \hline
K0507003B & D05002B & 知道向量的坐标分解的概念\\ \hline
K0507004B & D05002B & 知道位置向量的概念\\ \hline
K0507005B & D05002B & 理解向量的坐标表示\\ \hline
K0507006B & D05002B & 能根据所给向量的坐标进行向量的加法运算\\ \hline
K0507007B & D05002B & 能根据所给向量的坐标进行向量的减法运算\\ \hline
K0507008B & D05002B & 能根据所给向量的坐标进行向量的数乘运算\\ \hline
K0507008B & D05002B & 能根据所给向量的坐标进行向量的模的运算\\ \hline
K0508001B & D05002B & 会推导向量数量积的坐标表示\\ \hline
K0508002B & D05002B & 会推导向量夹角的坐标表示\\ \hline
K0508003B & D05002B & 会用坐标形式的向量夹角公式推导两个向量垂直的充要条件\\ \hline
K0508004B & D05002B & 会用坐标形式的向量夹角公式推导两个向量平行的充要条件\\ \hline
K0508005B & D05002B & 会用向量数量积与夹角的坐标表示解决相关问题\\ \hline
K0509001B & D05003B & 会用向量的线性运算证明平面几何中的相关问题\\ \hline
K0509002B & D05003B & 会用向量的坐标证明定比分点公式\\ \hline
K0509003B & D05003B & 会用向量的定比分点公式求解三角形重心的坐标\\ \hline
K0509004B & D05003B & 会用向量的数量积和坐标证明三角形的一个面积公式\\ \hline
K0510001B & D05003B & 会用向量的数量积证明两角差的余弦公式\\ \hline
K0510002B & D05003B & 理解向量是解决三角、几何等问题的重要工具\\ \hline
K0510003B & D05003B & 会用向量解决一些实际问题\\ \hline
K0510004B & D05003B & 会用向量解决一些物理问题\\ \hline
K0511001B & D05004B & 知道引入复数的必要性\\ \hline
K0511002B & D05004B & 知道虚数单位的定义\\ \hline
K0511003B & D05004B & 知道复数的定义\\ \hline
K0511004B & D05004B & 理解复数相等的含义\\ \hline
K0511005B & D05004B & 掌握复数的四则运算的公式, 能正确运用公式进行复数的四则运算\\ \hline
K0511006B & D05004B & 会推导复数加法、乘法的运算律\\ \hline
K0511007B & D05004B & 了解复数的整数次幂及运算规则\\ \hline
K0511008B & D05004B & 掌握虚数单位的整数次幂的运算规律\\ \hline
K0512001B & D05004B & 掌握复数的代数形式的概念\\ \hline
K0512002B & D05004B & 掌握复数的实部和虚部的概念\\ \hline
K0512003B & D05004B & 会根据复数的代数形式对复数加以分类\\ \hline
K0512004B & D05004B & 会运用复数的分类解决相关问题\\ \hline
K0512005B & D05004B & 理解共轭复数的概念\\ \hline
K0512006B & D05004B & 掌握共轭复数的性质\\ \hline
K0513001B & D05004B & 知道复平面的概念\\ \hline
K0513002B & D05004B & 知道实轴、虚轴的概念\\ \hline
K0513003B & D05004B & 理解复数与复平面上点的对应关系\\ \hline
K0513004B & D05004B & 理解复数与复平面上向量间的对应关系\\ \hline
K0513005B & D05004B & 掌握复数加法的平行四边形法则\\ \hline
K0513006B & D05004B & 掌握复数减法的平行四边形法则\\ \hline
K0514001B & D05004B & 掌握复数模的概念\\ \hline
K0514002B & D05004B & 懂得复数模的几何意义\\ \hline
K0514003B & D05004B & 会证明复数的模的性质\\ \hline
K0514004B & D05004B & 能运用复数的模的性质解决简单的问题\\ \hline
K0514005B & D05004B & 知道复数模的三角不等式\\ \hline
K0514006B & D05004B & 理解复数的差的模的几何意义, 并能应用它解决相关问题\\ \hline
K0515001B & D05004B & 了解复数范围内实数的平方根的概念, 知道其与实数范围内相应问题的异同\\ \hline
K0515002B & D05004B & 会求实数在复数范围内的平方根\\ \hline
K0515003B & D05004B & 理解复数范围内实系数一元二次方程根的情况, 并会求其根\\ \hline
K0515004B & D05004B & 理解韦达定理对任意实系数一元二次方程均成立\\ \hline
K0515005B & D05004B & 能运用韦达定理解决一些简单的实系数一元二次方程的问题\\ \hline
K0516001B & D05005B & 知道复数的辐角的概念\\ \hline
K0516002B & D05005B & 知道复数的辐角主值的概念\\ \hline
K0516003B & D05005B & 理解复数的三角形式, 懂得其与复数的代数形式的区别与联系\\ \hline
K0516004B & D05005B & 会用复数的模和辐角表示复数\\ \hline
K0517001B & D05005B & 会推导三角形式下复数的乘法公式\\ \hline
K0517002B & D05005B & 掌握三角形式下复数的乘法公式\\ \hline
K0517003B & D05005B & 了解三角形式下复数乘法运算的几何意义\\ \hline
K0517004B & D05005B & 懂得三角形式下复数的除法公式的推导过程\\ \hline
K0517005B & D05005B & 掌握用复数三角形式表示的复数的除法运算公式\\ \hline
K0517006B & D05005B & 掌握三角形式下复数的乘方运算公式, 并能进行简单的运算\\ \hline
K0517007B & D05005B & 掌握三角形式下复数的开方运算公式, 并能进行简单的运算\\ \hline
K0601001B & D06001B & 经历从现实情境中抽象平面特征的过程, 会用图形和符号表示平面.\\ \hline
K0601002B & D06001B & 直观认识和理解空间中点与直线的位置关系, 并能用文字语言、图形和符号表示.\\ \hline
K0601003B & D06001B & 直观认识和理解空间中点与平面的位置关系, 并能用文字语言、图形和符号表示.\\ \hline
K0601004B & D06001B & 直观认识和理解空间中直线与平面的位置关系, 并能用文字语言、图形和符号表示.\\ \hline
K0601005B & D06001B & 以长方体等较为熟悉的几何体作为载体, 理解公理1, 并能用图形及符号语言表示.\\ \hline
K0601006B & D06001B & 会在简单情形下利用公理1说明点或直线在平面上.\\ \hline
K0601007B & D06001B & 知道公理与命题的区别, 初步形成从公理出发进行推理的意识, 体会公理化思想.\\ \hline
K0602001B & D06001B & 通过对现实情境的观察和实验操作, 正确理解公理2.\\ \hline
K0602002B & D06001B & 知道公理与命题的区别, 形成从公理出发进行推理的意识, 进一步体会公理化思想.\\ \hline
K0602003B & D06001B & 掌握公理2的三个推论的内容, 能用图形和符号语言表示三个推论, 并能在此基础上证明三个推论.\\ \hline
K0602004B & D06001B & 知道公理2及其推论均为确定平面的依据, 会在简单情形下运用它们判断或证明点或直线共面的问题.\\ \hline
K0603001B & D06001B & 借助实例理解感受空间中相交平面的位置关系, 理解公理3, 并能用图形及符号语言表示.\\ \hline
K0603002B & D06001B & 借助实例感受空间中两个不同平面的位置关系, 会用图形和符号语言表示两个不同平面的位置关系.\\ \hline
K0603003B & D06001B & 能在简单情形下确定并画出两相交平面的交线.\\ \hline
K0603004B & D06001B & 能证明某直线为两相交平面的交线.\\ \hline
K0603005B & D06001B & 会在简单情形下运用公理3证明三点共线.\\ \hline
K0603006B & D06001B & 能作出给定平面与正方体表面的交线.\\ \hline
K0604001B & D06001B & 回顾并掌握斜二测画法的画图规则及步骤.\\ \hline
K0604002B & D06001B & 能用斜二测画法画出简单平面图形的直观图.\\ \hline
K0604003B & D06001B & 能用斜二测画法画出简单空间图形的直观图, 形成空间的概念.\\ \hline
K0605001B & D06002B & 观察实际情境, 类比平面上平行线的传递性, 将两条直线平行关系的传递性从平面推广到空间, 进而理解公理4.\\ \hline
K0605002B & D06002B & 会用符号语言表达公理4, 并能在简单的情形下证明空间两条直线平行.\\ \hline
K0605003B & D06002B & 经历等角定理的证明过程, 理解并能运用等角定理及其两个推论证明简单情形下两直线平行问题.\\ \hline
K0605004B & D06002B & 知道空间四边形的相关概念.\\ \hline
K0606001B & D06002B & 通过观察生活实景与长方体模型, 抽象形成异面直线的概念.\\ \hline
K0606002B & D06002B & 掌握用反证法证明两条直线是异面直线.\\ \hline
K0606003B & D06002B & 知道空间直线与直线的位置关系的分类.\\ \hline
K0606004B & D06002B & 掌握两条异面直线的一般画法.\\ \hline
K0606005B & D06002B & 理解并能证明异面直线判定定理.\\ \hline
K0606006B & D06002B & 会用异面直线判定定理证明两条直线是异面直线.\\ \hline
K0606007B & D06002B & 知道四面体的相关概念.\\ \hline
K0607001B & D06002B & 经历异面直线所成角概念的形成过程, 理解异面直线所成角的定义.\\ \hline
K0607002B & D06002B & 知道异面直线所成角的范围.\\ \hline
K0607003B & D06002B & 知道异面直线相互垂直的定义及符号表示.\\ \hline
K0607004B & D06002B & 会在简单的情形中通过平移求两条异面直线所成角的大小, 初步体会将空间问题转化为平面问题的思想方法.\\ \hline
K0608001B & D06003B & 通过对现实情境及熟悉的空间几何体的观察, 感知并证明直线与平面平行的判定定理, 并能用符号语言表示该判定定理.\\ \hline
K0608002B & D06003B & 能在具体的情形中用直线与平面平行的判定定理证明简单的相关问题.\\ \hline
K0608003B & D06003B & 理解并证明直线与平面平行的性质定理, 并能用符号语言表示该性质定理.\\ \hline
K0608004B & D06003B & 能在具体的情形中用直线与平面平行的性质定理证明简单的相关问题.\\ \hline
K0609001B & D06003B & 从现实情境中抽象、形成直线与平面垂直的概念, 并能用图形和符号语言表示.\\ \hline
K0609002B & D06003B & 通过实验操作与实际经验, 发现并理解直线与平面垂直的判定定理.\\ \hline
K0609003B & D06003B & 能运用直线与平面垂直的判定定理解决简单的相关问题.\\ \hline
K0609004B & D06003B & 理解并证明直线与平面垂直的性质定理, 并能用符号语言表示该性质定理.\\ \hline
K0609005B & D06003B & 能在具体的情形中用直线与平面垂直的性质定理解决简单的相关问题.\\ \hline
K0609006B & D06003B & 了解并能证明线面垂直性质定理的两个推论.\\ \hline
K0609007B & D06003B & 知道点到平面的距离, 并在能解决简单的相关问题.\\ \hline
K0609008B & D06003B & 知道直线到与它平行的平面的距离, 并能解决简单的相关问题.\\ \hline
K0610001B & D06003B & 知道直线与平面斜交的相关概念, 会用图形符号表示.\\ \hline
K0610002B & D06003B & 知道直线、线段在平面上的投影(射影)的概念.\\ \hline
K0610003B & D06003B & 经历直线与平面所成角的概念的形成过程, 知道直线与平面所成角的概念.\\ \hline
K0610004B & D06003B & 继续感悟用平面方法解决空间问题的思想, 能在具体的情形下求出直线与平面所成角的大小.\\ \hline
K0610005B & D06003B & 会用数学语言求解论证直线与平面所成角的图形中线段与角的相关问题.\\ \hline
K0611001B & D06003B & 理解三垂线定理, 能用符号及图形语言表示该定理并加以证明.\\ \hline
K0611002B & D06003B & 会用三垂线定理论证异面直线间的垂直关系.\\ \hline
K0611003B & D06003B & 继续感悟用平面方法解决空间问题的思想, 能在实际情境中运用三垂线定理解决一些简单的问题.\\ \hline
K0612001B & D06004B & 经历由直线间或线面间的平行关系出发探索两个平面的平行关系的过程, 发现并证明两个平面平行的判定定理.\\ \hline
K0612002B & D06004B & 能在具体的情形中运用两个平面平行的判定定理证明简单的相关问题.\\ \hline
K0612003B & D06004B & 理解并能证明两个平面平行的性质定理.\\ \hline
K0612004B & D06004B & 能在具体的情形中运用两个平面平行的性质定理证明简单的相关问题.\\ \hline
K0612005B & D06004B & 经历类比点到平面的距离与直线到平面的距离的定义获得两个平行平面间的距离的定义的过程, 掌握并能运用两个平行平面间的距离的定义解决简单的相关问题.\\ \hline
K0613001B & D06004B & 结合现实情境中的实例, 抽象形成二面角的概念, 能用图形及符号语言表示二面角.\\ \hline
K0613002B & D06004B & 知道二面角的平面角的概念, 并能作出二面角的平面角.\\ \hline
K0613003B & D06004B & 知道二面角的取值范围.\\ \hline
K0613004B & D06004B & 了解平面与平面垂直的概念, 并能用图形及符号语言表示.\\ \hline
K0613005B & D06004B & 经历面面垂直的判定定理与性质定理的发现与证明的过程, 并能用定理证明简单的相关命题.\\ \hline
K0614001B & D06005B & 通过猜测、归纳、论证的探究过程认识和理解两条异面直线的公垂线及公垂线的存在性与唯一性.\\ \hline
K0614002B & D06005B & 知道两条异面直线的距离的概念.\\ \hline
K0614003B & D06005B & 能在简单的情形中识别出异面直线的公垂线段并求出两异面直线的距离.\\ \hline
K0614004B & D06005B & 能在简单的情形中将求异面直线距离的问题转化为求线面距离、面面距离的问题.\\ \hline
K0614005B & D06005B & 能在简单的情形中将空间问题转化为平面问题, 构造出异面直线的公垂线段并求出异面直线的距离.\\ \hline
K0701001X & D07001X & 经历在平面直角坐标系中探索确定直线位置的几何要素, 理解直线的倾斜角和斜率的概念.\\ \hline
K0701002X & D07001X & 能对直线的倾斜角与斜率进行互化.\\ \hline
K0701003X & D07001X & 经历用代数方法刻画直线斜率的过程, 掌握过两点的直线斜率的计算公式.\\ \hline
K0701004X & D07001X & 知道一次函数的一次项系数就是其对应直线的斜率.\\ \hline
K0702001X & D07001X & 知道截距的概念.\\ \hline
K0702002X & D07002X & 能根据确定一条直线的几何要素, 掌握直线的点斜式方程、斜截式方程及其使用范围, 并能在具体的实例中求直线的点斜式、斜截式方程.\\ \hline
K0703001X & D07002X & 能根据确定一条直线的几何要素, 掌握直线的两点式方程及其使用范围, 并能在具体的实例中求直线的两点式方程.\\ \hline
K0704001X & D07002X & 通过具体实例, 知道直线的方程是一个二元一次方程, 并且任意一个二元一次方程都能表示一条直线.\\ \hline
K0704002X & D07002X & 能根据确定一条直线的几何要素, 掌握直线的一般式方程及其使用范围, 并能在具体的实例中求直线的一般式方程.\\ \hline
K0705001X & D07002X & 能根据确定一条直线的几何要素, 掌握直线的点向式、点法式方程及其使用范围, 并能在具体的实例中求直线的点向式、点法式方程.\\ \hline
K0706001X & D07003X & 理解二元一次方程组的解与两条相交直线的交点坐标之间的对应关系, 能用解方程组的方法求两条直线的交点坐标.\\ \hline
K0706002X & D07003X & 能根据两条直线的方程的系数、斜率及法向量讨论两条直线的位置关系(相交、平行或重合).\\ \hline
K0707001X & D07003X & 能根据两条直线的方程的系数、斜率及法向量讨论两条直线是否垂直.\\ \hline
K0707002X & D07003X & 经历将两条直线的夹角转化为对应法向量的夹角的过程, 掌握两条直线的夹角公式.\\ \hline
K0708001X & D07003X & 通过具体实例, 探究并求解点到直线的距离, 掌握点到直线的距离公式.\\ \hline
K0708002X & D07003X & 根据点到直线距离公式, 推导及掌握两条平行线之间的距离公式.\\ \hline
K0709001X & D07004X & 回顾直线方程的概念, 结合具体的实例, 理解曲线与其对应方程的概念.\\ \hline
K0709002X & D07004X & 能在简单的情境中, 判断曲线与方程是否对应.\\ \hline
K0709003X & D07004X & 在平面直角坐标系中, 根据确定圆的几何要素, 探索并掌握圆的标准方程.\\ \hline
K0710001X & D07004X & 在平面直角坐标系中, 根据确定圆的几何要素, 探索并掌握圆的一般方程.\\ \hline
K0710002X & D07004X & 能利用配方法将圆的一般方程化为标准方程.\\ \hline
K0710003X & D07004X & 能在具体实例中, 选择合适的方法求圆的方程.\\ \hline
K0711001X & D07004X & 在平面直角坐标系中, 能根据给定直线、圆的方程, 通过代数方法(一元二次方程的判别式)判断直线与圆的位置关系.\\ \hline
K0711002X & D07004X & 在平面直角坐标系中, 能根据给定直线、圆的方程, 通过几何方法(点到直线的距离与半径的大小关系)判断直线与圆的位置关系.\\ \hline
K0711003X & D07004X & 探究圆心在原点的圆的切线方程并推广至一般情形.\\ \hline
K0712001X & D07004X & 在平面直角坐标系中, 会用圆心距与两圆半径的关系判断圆与圆的位置关系.\\ \hline
K0712002X & D07004X & 在平面直角坐标系中, 会用解方程组的方法判断圆与圆的位置关系.\\ \hline
K0712003X & D07004X & 能推导相交两圆的公共弦所在直线的方程, 体会设而不求的思想.\\ \hline
K0712004X & D07004X & 会利用直线与圆的方程解决简单的平面几何问题与实际问题.\\ \hline
K0713001X & D07005X & 经历从具体情境(天文学、数学史等方面)抽象出椭圆, 并借助信息技术等工具绘制出椭圆的这一过程, 掌握椭圆的定义.\\ \hline
K0713002X & D07005X & 能根据椭圆的定义, 推导椭圆的标准方程, 并掌握两种类型(中心在原点, 焦点在坐标轴上)椭圆的标准方程.\\ \hline
K0713003X & D07005X & 能利用椭圆的定义, 解决一些简单的与椭圆焦点有关的问题.\\ \hline
K0714001X & D07005X & 经历通过椭圆的标准方程研究椭圆的几何性质这一过程, 掌握椭圆的几何性质(对称性、顶点、范围、离心率), 初步领会可以用代数方法研究曲线的哪些方面.\\ \hline
K0714002X & D07005X & 通过具体例子, 会判断直线与椭圆的公共点个数, 从代数角度类比直线与圆的位置关系, 从形的角度掌握直线与椭圆的位置关系.\\ \hline
K0715001X & D07006X & 经历从具体情境(天文学、数学史等方面)抽象出双曲线, 并借助信息技术等工具绘制出双曲线的这一过程, 掌握双曲线的定义.\\ \hline
K0715002X & D07006X & 能根据双曲线的定义, 推导双曲线的标准方程, 掌握两种类型(中心在原点, 焦点在坐标轴上)双曲线的标准方程.\\ \hline
K0715003X & D07006X & 能利用双曲线的定义, 解决一些简单的与双曲线焦点有关的问题.\\ \hline
K0716001X & D07006X & 经历通过双曲线的标准方程研究双曲线的几何性质这一过程, 掌握双曲线的几何性质(对称性、顶点、范围、离心率), 进一步领会可以用代数方法研究曲线的哪些方面.\\ \hline
K0716002X & D07006X & 知道等轴双曲线的概念, 了解反比例函数的图像是等轴双曲线.\\ \hline
K0716003X & D07006X & 通过当横坐标一定时, 双曲线第一象限部分的点与其渐近线上的点纵坐标的关系, 理解双曲线的右支向右上方无限延伸时, 它总在其中一条渐近线的下方, 与该渐近线无限趋近, 但永不相交.\\ \hline
K0716004X & D07006X & 知道双曲线的渐近线方程, 会利用双曲线的渐近线解决一些简单的与极限有关的问题.\\ \hline
K0717001X & D07006X & 通过具体例子, 会判断直线与双曲线的公共点个数, 从形的角度掌握直线与双曲线的位置关系.\\ \hline
K0717002X & D07006X & 通过具体例子, 知道当直线与双曲线的渐近线平行时, 直线与双曲线有且只有一个公共点.\\ \hline
K0718001X & D07007X & 经历从具体情境(天文学、数学史等方面)抽象出抛物线, 并借助信息技术等工具绘制出抛物线的这一过程, 掌握抛物线的定义.\\ \hline
K0718002X & D07007X & 能根据抛物线的定义, 推导抛物线的标准方程, 包括证明以所求方程的任意一组解为坐标的点都在该抛物线上.\\ \hline
K0718003X & D07007X & 能利用抛物线的定义, 解决一些简单的与抛物线焦点、准线有关的问题.\\ \hline
K0718003X & D07007X & 知道抛物线的焦点、准线的概念, 掌握四种类型(顶点在原点, 焦点在坐标轴上)抛物线的标准方程.\\ \hline
K0718004X & D07007X & 通过回顾初中熟知的"二次函数的图像是抛物线"这一结论, 了解二次函数的图像符合抛物线的定义.\\ \hline
K0719001X & D07007X & 经历通过抛物线的标准方程研究抛物线的几何性质这一过程, 掌握抛物线的几何性质(对称性、顶点、离心率), 进一步领会如何用代数方法研究曲线的性质.\\ \hline
K0719002X & D07007X & 会判断直线与抛物线的公共点个数, 从形的角度掌握直线与抛物线的位置关系.\\ \hline
K0719003X & D07007X & 通过具体例子, 知道直线与抛物线的对称轴平行时, 直线与抛物线有且只有一个公共点.\\ \hline
K0801001B & D08001B & 通过具体事例, 认识随机现象在自然界、社会中普遍存在, 理解随机现象的概念. \\ \hline
K0801002B & D08001B & 通过具体事例, 理解随机试验的概念. \\ \hline
K0801003B & D08001B & 初步了解概率论的起源与发展历史, 了解概率的概念.\\ \hline
K0801004B & D08001B & 能够判断是随机现象还是确定性现象.\\ \hline
K0801005B & D08001B & 了解随机试验中含有的随机性.\\ \hline
K0802001B & D08001B & 了解样本空间, 基本事件(或样本点)的定义.\\ \hline
K0802002B & D08001B & 能够写出随机试验的样本空间, 理解随机事件的表达, 会用集合语言表达.\\ \hline
K0802003B & D08001B & 能够写出随机事件对应样本空间的子集.\\ \hline
K0802004B & D08001B & 结合具体的实例理解随机事件, 并了解随机事件与样本点之间的关系.\\ \hline
K0802005B & D08001B & 理解必然事件和不可能事件的概念, 了解它们对应的子集与样本空间的关系.\\ \hline
K0802006B & D08001B & 能够判断事件是必然事件还是不可能事件.\\ \hline
K0802007B & D08001B & 了解确定事件和不确定事件的概念, 会判断事件是确定事件还是不确定事件.\\ \hline
K0803001B & D08001B & 理解随机试验结果的等可能性.\\ \hline
K0803002B & D08001B & 通过具体实例, 理解组成古典概率模型的两个基本条件, 会计算古典概率模型中简单随机事件的概率.\\ \hline
K0803003B & D08001B & 通过实例, 理解概率性质1和概率性质2.\\ \hline
K0803004B & D08001B & 通过实例, 掌握随机事件概率的运算法则.\\ \hline
K0804001B & D08001B & 通过古典概型实例, 理解随着观察角度的不同, 并非所有的样本空间都有等可能性.\\ \hline
K0804002B & D08001B & 了解只有选取等可能的样本空间, 才能使得事件的概率如定义所示.\\ \hline
K0804003B & D08001B & 会对多步的等可能随机试验构造等可能的样本空间.\\ \hline
K0805001B & D08001B & 理解事件之间的子集关系, 会用集合语言表达.\\ \hline
K0805002B & D08001B & 通过具体实例, 掌握事件的交、并运算, 懂得事件的运算的含义, 并能够用集合语言表达.\\ \hline
K0805003B & D08001B & 通过具体实例, 理解互斥事件与对立事件的概念.\\ \hline
K0805004B & D08001B & 理解两个相互对立事之间的交、并运算关系: $A\cap\overline A=\varnothing$, $A\cup\overline A=\Omega$.\\ \hline
K0805005B & D08001B & 通过具体实例, 理解事件的否定形式, 并能写出简单的随机事件的否定形式.\\ \hline
K0805006B & D08001B & 掌握公式$\overline{A\cap B}=\overline A\cup\overline B$, 并理解对任意多个事件同样成立.\\ \hline
K0805007B & D08001B & 掌握公式$\overline{A\cup B}=\overline A\cap\overline B$, 并理解对任意多个事件同样成立.\\ \hline
K0806001B & D08001B & 能够推导两个不同时发生的事件至少有一个发生的概率是这两个事件的概率之和, 理解概率性质3(可加性).\\ \hline
K0806002B & D08001B & 基于概率性质3(可加性), 理解$B=\overline A$时的特殊情况, 掌握概率性质4.\\ \hline
K0806003B & D08001B & 能利用概率性质3与概率性质4解决简单的相关问题.\\ \hline
K0806004B & D08001B & 理解两个事件的可加性可以推出任意事件的可加性: $P(A_1\cup A_2\cup\cdots\cap A_n)=P(A_1)+P(A_2)+\cdots+P(A_n)$.\\ \hline
K0807001B & D08001B & 通过对实例的观察与分析初步理解伯努利试验中“独立地重复”的含义以及频率的意义.\\ \hline
K0807002B & D08001B & 结合试验实例, 归纳并抽象出伯努利大数定律, 了解其意义.\\ \hline
K0807003B & D08001B & 体会随机事件发生的不确定性以及频率的稳定性.\\ \hline
K0807004B & D08001B & 掌握事件频率的计算法则, 会用频率估计概率, 解决一些简单的实际问题.\\ \hline
K0808001B & D08002B & 结合有限样本空间, 通过具体事例, 经历由对事件独立的直观判断到事件独立的严格定义的形成过程, 理解随机事件独立性的含义.\\ \hline
K0808002B & D08002B & 结合古典概型, 掌握独立事件积的概率计算方法.\\ \hline
K0808003B & D08002B & 经历113页例题2的证明, 掌握事件独立性的性质: 如果$A$与$B$两个事件独立, 那么$A$与$\overline B$也独立. \\ \hline
K0808004B & D08002B & 掌握独立事件的条件, 并能利用独立性求相关的概率问题, 发展数学建模素养. \\ \hline
K0809001B & D08002B & 会用两个事件相互独立的充要条件判断两个事件是否独立. \\ \hline
K0809002B & D08002B & 结合$115$例题5, 掌握随机事件独立性性质的应用.\\ \hline
K0809003B & D08002B & 会利用事件的独立性解决较复杂的概率问题. \\ \hline
K0810001X & D08003X & 结合具体实例, 理解分步计数原理(乘法原理). \\ \hline
K0810002X & D08003X & 体会乘法原理的应用条件.\\ \hline
K0810003X & D08003X & 会利用乘法原理解决简单的相关计数问题. \\ \hline
K0811001X & D08003X & 结合具体实例, 理解分类计数原理(加法原理). \\ \hline
K0811002X & D08003X & 了解加法原理应用的条件, 体会分类讨论的思想方法. \\ \hline
K0811003X & D08003X & 能利用加法原理解决相关简单的计数问题. \\ \hline
K0811004X & D08003X & 能够区分相关计数问题是分步计数还是分类计数问题. \\ \hline
K0811005X & D08003X & 能利用加法原理与乘法原理解决较为复杂的计数问题. \\ \hline
K0812001X & D08003X & 基于乘法原理, 结合具体实例, 引出排列的定义. \\ \hline
K0812002X & D08003X & 理解排列的含义. \\ \hline
K0812003X & D08003X & 会利用乘法原理求解具体的排列问题. \\ \hline
K0813001X & D08003X & 结合具体实例, 理解排列数定义. \\ \hline
K0813002X & D08003X & 会利用乘法原理推导排列数公式, 体会乘法原理在推导排列数公式上的作用. \\ \hline
K0813003X & D08003X & 掌握排列数公式, 并能利用排列数公式求解相关的排列问题. \\ \hline
K0813004X & D08003X & 能利用排列数公式以及乘法原理和加法原理求解相关的计数问题. \\ \hline
K0813005X & D08003X & 掌握借助计算器求排列数的方法.\\ \hline
K0814001X & D08003X & 理解全排列的概念, 及全排列数的符号表示, 掌握全排列数的计算公式. \\ \hline
K0814002X & D08003X & 掌握阶乘的概念, 并能够用阶乘表示排列数公式. \\ \hline
K0814003X & D08003X & 定义$0!=1$, 领会全排列数$\mathrm{P}_n^n=n!$是排列数公式中$m=n$的特殊情况. \\ \hline
K0814004X & D08003X & 能用排列数表示连续的几个正整数相乘. \\ \hline
K0814005X & D08003X & 能用排列数公式证明公式: $\mathrm{P}_n^m=n\mathrm{P}_{n-1}^{m-1}$; $\mathrm{P}_n^m+m\mathrm{P}_n^{m-1}=\mathrm{P}_{n+1}^m$. \\ \hline
K0814006X & D08003X & 在具体问题中能够应用相关公式和性质. \\ \hline
K0815001X & D08003X & 基于排列定义, 理解组合定义. \\ \hline
K0815002X & D08003X & 理解排列与组合的区别, 能够判断问题是排列问题还是组合问题. \\ \hline
K0815003X & D08003X & 能够求解简单的组合问题. \\ \hline
K0816001X & D08003X & 结合排列数定义, 理解组合数定义, 并掌握组合数的符号表示. \\ \hline
K0816002X & D08003X & 会利用排列数公式和乘法原理推导组合数公式. \\ \hline
K0816003X & D08003X & 结合具体的实例, 理解组合数公式. \\ \hline
K0816004X & D08003X & 会利用组合数公式, 计算组合数. \\ \hline
K0816005X & D08003X & 遇到熟悉的环境, 能够利用组合数公式求解相关的问题. \\ \hline
K0816006X & D08003X & 掌握借助计算器计算组合数的方法. \\ \hline
K0817001X & D08003X & 会利用公式$\mathrm{P}_n^m=\frac{n!}{(n-m)!}$推导出组合数公式: $\mathrm{C}_n^m=\frac{n!}{m!(n-m)!}$. \\ \hline
K0817002X & D08003X & 规定$\mathrm{C}_n^0=1$, 理解公式: $\mathrm{C}_n^m=\frac{n!}{m!(n-m)!}$对$m=0$也成立. \\ \hline
K0817003X & D08003X & 会利用组合数公式证明: $\mathrm{C}_n^m=\frac{m+1}{n-m}\mathrm{C}_n^{m+1}$. \\ \hline
K0817004X & D08003X & 会利用组合数公式证明组合数的两个基本运算性质: $\mathrm{C}_n^m=\mathrm{C}_n^{n-m}$; $\mathrm{C}_{n+1}^m=\mathrm{C}_n^m+\mathrm{C}_n^{m-1}$. \\ \hline
K0817005X & D08003X & 理解组合数的两个基本运算性质的含义. \\ \hline
K0817006X & D08003X & 体会在计算$\mathrm{C}_n^m$时, 若$m>\frac n2$, 可利用基本运算性质转化为$\mathrm{C}_n^{n-m}$来计算. \\ \hline
K08017007X & D08003X & 会利用组合数公式及两个基本运算性质计算和求解相关问题. \\ \hline
K0818001X & D08003X & 在古典概率中, 能利用排列和组合求随机事件$A$包含的基本事件的个数$k$, 并能结合公式$P(A)=\frac kn$求概率. \\ \hline
K0818002X & D08003X & 在具体实例中, 会利用计数原理求解较为复杂的古典概率问题. \\ \hline
K0818003X & D08003X & 在具体实例中, 能够把所求事件转化为对立事件, 利用$P(A)=1-P(\overline A)$求相关概率. \\ \hline
K0819001X & D08003X & 通过具体的实例了解二项展开式的概念. \\ \hline
K0819002X & D08003X & 通过具体实例的展开, 体会二项展开式的规律, 归纳出对于任意正整数$n$, $(a+b)^n$的二项展开式的规律. \\ \hline
K0819003X & D08003X & 结合杨辉三角掌握二项展开式中各项系数的$3$个特点$(P66)$. \\ \hline
K0819004X & D08003X & 能够用组合数的公式表示二项展开式中各项系数的特点$1$, $2$. \\ \hline
K0819005X & D08003X & 掌握二项式定理, 并能够利用数学归纳法证明二项式定理. \\ \hline
K0819006X & D08003X & 能利用二项式定理展开具体的二项式. \\ \hline
K0819007X & D08003X & 能利用二项式定理求展开式中项的系数.\\ \hline
K0819008X & D08003X & 能利用二项式定理证明相关的数的整除问题. \\ \hline
K0820001X & D08003X & 在二项式定理中, 令$a=1$, $b=1$, 掌握恒等式$\mathrm{C}_n^0+\mathrm{C}_n^1+\mathrm{C}_n^2+\cdots+\mathrm{C}_n^n=2^n$. \\ \hline
K0820002X & D08003X & 在二项式定理中代入特殊的值, 得到一些与组合数有关的特殊恒等式, 从而掌握通过赋值解决相关问题的方法.\\ \hline
K0820003X & D08003X & 掌握并能够证明: 在$n+1$个组合数$\mathrm{C}_n^0,\mathrm{C}_n^1,\mathrm{C}_n^2,\cdots,\mathrm{C}_n^n$中, 当$n$为偶数时, 最大值是中间的一项, $n$为奇数时, 最大值是中间的两项.\\ \hline
K0820004X & D08003X & 会求具体二项展开式中系数的最大项.\\ \hline
K0821001X & D08004X & 理解条件概率的概念.\\ \hline
K0821002X & D08004X & 能分辨条件概率与概率的异同.\\ \hline
K0821003X & D08004X & 在熟悉的情境中能根据条件概率公式用除法计算条件概率.\\ \hline
K0821004X & D08004X & 知道概率的乘法公式.\\ \hline
K0821005X & D08004X & 能用概率的乘法公式求两事件积的概率.\\ \hline
K0821006X & D08004X & 了解条件概率与独立事件之间的联系.\\ \hline
K0822001X & D08004X & 了解加权平均的概念.\\ \hline
K0822002X & D08004X & 理解全概率公式, 会用概率乘法公式和可加性推导全概率公式.\\ \hline
K0822003X & D08004X & 在熟悉的情境中, 能合理地分拆事件, 用全概率公式计算概率.\\ \hline
K0823001X & D08004X & 会用概率乘法公式和条件概率公式推导贝叶斯公式.\\ \hline
K0823002X & D08004X & 会用贝叶斯公式计算形如$P(\Omega_k|A)$的条件概率.\\ \hline
K0823003X & D08004X & 了解先验概率和后验概率的概念.\\ \hline
K0823004X & D08004X & 知道贝叶斯公式与机器学习有联系.\\ \hline
K0824001X & D08005X & 理解随机变量是以样本空间的元素为自变量, 以实数为函数值得函数(这里推广了函数的概念).\\ \hline
K0824002X & D08005X & 能列举一些随机变量的例子.\\ \hline
K0824003X & D08005X & 理解随机变量的分布的概念, 知道分布中所有可能取值的概率之和为$1$, 取值互异.\\ \hline
K0824004X & D08005X & 能读懂用图或表来表示的分布.\\ \hline
K0824005X & D08005X & 会在简单的情境中计算分布, 并用图或表来表示.\\ \hline
K0824006X & D08005X & 了解等可能分布(均匀分布)的概念.\\ \hline
K0824007X & D08005X & 了解伯努利分布的概念.\\ \hline
K0825001X & D08005X & 理解期望是随机变量取值的加权平均(以概率为权), 也称数学期望或均值, 回规范地表示数学期望($E[X]$).\\ \hline
K0825002X & D08005X & 会根据分布列计算期望.\\ \hline
K0825003X & D08005X & 会用组合恒等式$k\mathrm{C}_n^k=n\mathrm{C}_{n-1}^{k-1}$计算二项分布的期望.\\ \hline
K0825004X & D08005X & 知道期望的实际意义与大数次试验有关, 是大数次试验的随机变量的平均值的趋势反映.\\ \hline
K0825005X & D08005X & 知道期望的线性性质及性质适用的条件(对事件之间的关系无要求).\\ \hline
K0825006X & D08005X & 会证明期望的数乘性质.\\ \hline
K0825007X & D08005X & 会用期望的线性性质计算随机事件的期望.\\ \hline
K0826001X & D08005X & 了解方差是随机变量与其均值的差的平方的期望(知道计算方法).\\ \hline
K0826002X & D08005X & 会推导方差的第二个计算公式$D[X]=\mathrm{E}(X^2)-(E[X])^2$.\\ \hline
K0826003X & D08005X & 了解方差越大, 分散程度越大, 不确定性越大.\\ \hline
K0826004X & D08005X & 会根据分布列计算方差.\\ \hline
K0826005X & D08005X & 知道方差的数乘性质, 并会证明与使用这一性质.\\ \hline
K0826006X & D08005X & 知道方差的可加性需要独立的条件, 能用该性质计算两独立随机变量的和与差的方差.\\ \hline
K0826007X & D08005X & 知道标准差是方差的算术根.\\ \hline
K0827001X & D08005X & 知道什么是二项分布$B(n,p)$, 会表示二项分布的分布列.\\ \hline
K0827002X & D08005X & 知道二项分布的概率与二项展开式有联系.\\ \hline
K0827003X & D08005X & 会利用期望的可加性计算二项分布的期望.\\ \hline
K0827004X & D08005X & 会利用独立事件方差的可加性计算二项分布的方差.\\ \hline
K0827005X & D08005X & 会计算符合二项分布模型的事件的概率.\\ \hline
K0828001X & D08005X & 知道超几何分布来源于不放回摸球模型.\\ \hline
K0828002X & D08005X & 理解超几何分布的定义及参数的实际意义.\\ \hline
K0828003X & D08005X & 会用组合数表示超几何分布中的概率.\\ \hline
K0828004X & D08005X & 经历将超几何分布模型分拆为多个二项分布模型, 进而用可加性计算期望的过程.\\ \hline
K0828005X & D08005X & 知道超几何分布的期望, 知道超几何分布的方差不好算.\\ \hline
K0828006X & D08005X & 了解二项分布与超几何分布的联系与区别.\\ \hline
K0829001X & D08005X & 了解自然语境下正态分布的含义.\\ \hline
K0829002X & D08005X & 知道数学意义下正态分布对应的概率密度函数.\\ \hline
K0829003X & D08005X & 知道正态分布密度函数中$\mu$表示随机变量的期望, $\sigma^2$表示随机变量的方差.\\ \hline
K0829004X & D08005X & 知道一个随机变量服从正态分布$X\sim N(\mu,\sigma^2)$的数学含义.\\ \hline
K0829005X & D08005X & 知道标准正态分布$N(0,1)$的概念及其密度函数.\\ \hline
K0829006X & D08005X & 会查表或用计算机根据$x$计算累积面积$\Phi(x)$的值, 并能根据$\Phi(x)$的值计算$x$.\\ \hline
K0829007X & D08005X & 理解$\Phi(x)=1-\Phi(-x)$的来源.\\ \hline
K0829008X & D08005X & 知道用$X'=\dfrac{X-\mu}{\sigma}$可将一般正态分布转化为标准正态分布.\\ \hline
K0829009X & D08005X & 知道$\mu,\sigma$对正态分布密度函数的图像的影响.\\ \hline
K0829010X & D08005X & 会根据$\Phi(x)$的值求服从正态分布的随机变量取值在某范围内的概率.\\ \hline
K0829011X & D08005X & 了解$3\sigma$原则, 知道对于服从正态分布的随机变量, 落在$[\mu-\sigma,\mu+\sigma]$, $[\mu-2\sigma,\mu+2\sigma]$, $[\mu-3\sigma,\mu+3\sigma]$内的概率的大致大小.\\ \hline
K0901001B & D09001B & 掌握总体、个体、总体的容量、样本和样本量(样本容量)的概念, 理解总体和样本的关系.\\ \hline
K0901002B & D09001B & 在具体的情境中能够准确表达出总体、样本、样本量.\\ \hline
K0901003B & D09001B & 知道‘达标率’、‘优秀率’等用来描述样本特征的概括性数字度量, 称为统计量, 了解统计量的相关概念.\\ \hline
K0901004B & D09001B & 了解统计活动的基本思想是通过分析样本的统计特征去推断总体的统计特征.\\ \hline
K0901005B & D09001B & 通过典型案例的研究, 初步感悟统计学研究对象的广泛性和不确定性.\\ \hline
K0902001B & D09002B & 能根据收集数据的不同方法, 判断所收集的数据类型是观测数据还是实验数据.\\ \hline
K0902002B & D09002B & 知道获取数据的基本途径, 包括统计报表和年鉴、社会调查、试验设计、普查和抽样、互联网等.\\ \hline
K0902003B & D09002B & 知道普查和抽样调查的优缺点. \\ \hline
K0902004B & D09002B & 会判断样本能否反映总体的特征, 即抽取的样本是否具有代表性. \\ \hline
K0903001B & D09003B & 了解简单随机抽样的含义, 了解简单随机抽样的特点, 并能够根据简单随机抽样的特点判断一个抽取样本的方法是否是简单随机抽样.\\ \hline
K0903002B & D09003B & 掌握两种简单随机抽样的方法: 抽签法和随机数法. 了解抽签法和随机数法的特点和适用范围.\\ \hline
K0903003B & D09003B & 会用抽签法进行简单随机抽样.\\ \hline
K0903004B & D09003B & 能够读懂随机数表, 掌握利用随机数表抽取样本的基本步骤.\\ \hline
K0903005B & D09003B & 会利用计算机或计算器产生随机数.\\ \hline
K0903006B & D09003B & 了解分层随机抽样的特点和适用范围.\\ \hline
K0903007B & D09003B & 掌握各层样本量比例分配的方法, 会根据总体情况制定分层抽样的方案.\\ \hline
K0903008B & D09003B & 能根据实际问题的特点, 选用恰当的抽样方法解决问题.\\ \hline
K0904001B & D09004B & 知道什么是极差, 会根据数据确定组距与组数, 并统计每组的频数及频率.\\ \hline
K0904002B & D09004B & 会将未经处理的统计数据制作成频率分布表, 掌握制作频数分布表的基本步骤.\\ \hline
K0904003B & D09004B & 能够根据频率分布表制作频率分布直方图以及频率分布折线图.\\ \hline
K0904004B & D09004B & 能够读懂频率分布直方图, 知道数据落在各小组内的频率可以用小矩形的面积来表示, 且这些面积的总和为1. \\ \hline
K0904005B & D09004B & 知道当组距取得足够小, 频率分布折线图将趋于一条光滑的曲线.\\ \hline
K0904006B & D09004B & 会用简单的语言描述统计图表呈现的信息. \\ \hline
K0904007B & D09004B & 能根据实际问题的特点, 选择恰当的统计图表对数据进行可视化描述.\\ \hline
K0905001B & D09004B & 理解茎叶图中“茎”、“叶”的具体含义, 了解茎叶图的适用范围, 会制作茎叶图.\\ \hline
K0905002B & D09004B & 能解读茎叶图中蕴含的数据分布信息, 体会其中的分组思想. \\ \hline
K0905003B & D09004B & 会制作散点图, 并会通过散点图发现数据之间的关系. \\ \hline
K0905004B & D09004B & 在对数据进行分析和整理时, 能够根据需要, 选择恰当的统计图表, 包括初中阶段学习的条形图、扇形图以及折线图等, 清楚各种统计图表的特点和适用范围.\\ \hline
K0905005B & D09004B & 会在借助计算机中的电子表格办公软件绘制统计图表.\\ \hline
K0906001B & D09005B & 知道总体的分布指的是总体中不同范围或类型的个体所占的比例.\\ \hline
K0906002B & D09005B & 能够根据样本的频率分布情况估计总体的大致分布.\\ \hline
K0906003B & D09005B & 知道什么是总体分布密度曲线.\\ \hline
K0906004B & D09005B & 理解集中趋势参数的统计含义, 会用平均数、中位数和众数描述样本的集中趋势, 从而估计总体的集中趋势. \\ \hline
K0906005B & D09005B & 理解离散程度参数统计含义, 会用方差、标准差等描述样本的离散程度, 从而估计总体的离散程度. \\ \hline
K0906006B & D09005B & 认识求和符号, 会求和符号表示下的线性运算.\\ \hline
K0906007B & D09005B & 熟悉使用求和符号, 会用求和符号表示平均数、方差、标准差等.\\ \hline
K0906008B & D09005B & 会计算只提供了区间及频数的样本数据的平均数、方差、标准差等, 知道此时可以用区间的中点值给区间内的每个数据赋值. \\ \hline
K0906009B & D09005B & 能根据多组样本的容量、平均数以及方差求全体样本数据的平均数及方差, 比如提供了各自调查的样本均值和方差, 如何得到所有数据的样本平均数和方差, 进而估计总体平均数和方差.\\ \hline
K0907001B & D09005B & 理解百分位数的定义, 学会计算一组数据的第百分位数.\\ \hline
K0907002B & D09005B & 知道四分位数的概念.\\ \hline
K0907003B & D09005B & 会用样本百分位数来估计总体百分位数, 体会样本估计总体的统计思想.\\ \hline
K0907004B & D09005B & 了解统计活动的基本步骤, 结合具体问题, 经历完整的统计过程, 积累统计活动经验.\\ \hline
K0908001X & D09006X & 知道成对数据和相关分析的概念, 并能够判断两组数据是否可以看作成对数据, 是否可以进行相关分析.\\ \hline
K0908002X & D09006X & 能够根据所给数据绘制数据的散点图, 并依据散点图观察和初步分析两组数据的相关性.\\ \hline
K0908003X & D09006X & 知道两组数据的线性相关系数是度量两个变量之间线性相关程度的统计量, 了解两组数据的线性相关系数的公式.\\ \hline
K0908004X & D09006X & 知道相关系数的取值范围, 并且知道相关系数的取值与两个变量的线性相关程度的关系.\\ \hline
K0908005X & D09006X & 知道正相关、负相关的概念.\\ \hline
K0908006X & D09006X & 会根据相关系数的公式计算相关系数.\\ \hline
K0908007X & D09006X & 理解相关系数描述的是两个变量之间线性关系的方向与程度, 是一种定量分析的方法, 了解相关系数的特点.\\ \hline
K0908008X & D09006X & 了解相关系数的几何意义.\\ \hline
K0909001X & D09006X & 了解离差的概念, 能够根据所给数据计算离差.\\ \hline
K0909002X & D09006X & 了解拟合误差的概念和公式, 能够根据所给数据计算离差, 知道拟合误差是描述数据与函数贴合程度的指标.\\ \hline
K0909003X & D09006X & 知道回归方程或回归模型的概念, 知道解释变量、反应变量的含义, 知道回归直线、回归系数、一元线性回归分析等概念.\\ \hline
K0909004X & D09006X & 了解回归系数公式及其推导过程, 能够根据所给数据计算回归系数.\\ \hline
K0909005X & D09006X & 知道最小二乘法、最小二乘估计的概念, 会利用最小二乘法估计线性方程中的参数, 进而得到回归方程.\\ \hline
K0910001X & D09006X & 了解建立一元线性回归模型的一般步骤, 针对实际问题, 会用一元线性回归模型进行预测.\\ \hline
K0910002X & D09006X & 知道相关分析和回归分析是处理成对数据的两种基本统计方法, 了解它们之间的联系与区别.\\ \hline
K0910003X & D09006X & 知道除了具有线性关系的散点图以外, 线性回归分析还可以处理呈指数分布性状的数据分布.\\ \hline
K0911001X & D09006X & 知道分类变量的概念.\\ \hline
K0911002X & D09006X & 知道2行×2列列联表(简称2×2列联表, 也称为四格表)的概念. \\ \hline
K0911003X & D09006X & 知道要检验两个随机变量是否有关, 统计上一般先假设它们相互独立, 再进行统计检验. \\ \hline
K0911004X & D09006X & 知道原假设(也称零假设)、备择假设的概念.\\ \hline
K0911005X & D09006X & 知道观察值、预期值的概念.\\ \hline
K0911006X & D09006X & 知道描述观察值与预期值之间的总体偏差的统计量$X^2$的公式, 并会在具体的情境中计算统计量$X^2$的值.\\ \hline
K0911007X & D09006X & 知道并会证明$2\times 2$列联表$X^2$检验的计算公式.\\ \hline
K0911008X & D09006X & 知道$2\times 2$列联表独立性检验的基本步骤.\\ \hline
K0911009X & D09006X & 会利用取自两类变量的样本来判断它们是否相互独立.\\ \hline
K0912001X & D09006X & 在具体的问题中, 会用独立性检验研究两个因素是否相互影响.\\ \hline
K0912002X & D09006X & 在具体的问题中, 会用独立性检验判断两个对象是否有显著差异.\\ \hline
K0912003X & D09006X & 进一步熟悉$2\times 2$列联表独立性检验的基本步骤, 掌握运用$2\times 2$列联表的方法解决独立性检验的简单实际问题, 培养估计思想和检验思想.\\ \hline

     & & \\ \hline
\end{longtable}

\end{document}