\documentclass[12pt,a4paper]{article}
\usepackage[UTF8,fontset = windows]{ctex}
\setCJKmainfont[BoldFont=黑体,ItalicFont=楷体]{等线}
\usepackage{amssymb,amsmath,amsfonts,amsthm,mathrsfs,dsfont,graphicx}
\usepackage{ifthen,indentfirst,enumerate,color,titletoc}
\usepackage{tikz}
\usetikzlibrary{arrows,calc,intersections}
\usepackage[bf,small,indentafter,pagestyles]{titlesec}
\usepackage[top=1in, bottom=1in,left=0.8in,right=0.8in]{geometry}
\renewcommand{\baselinestretch}{1.65}
\newtheorem{defi}{定义~}
\newtheorem{eg}{例~}
\newtheorem{ex}{~}
\newtheorem{rem}{注~}
\newtheorem{thm}{定理~}
\newtheorem{coro}{推论~}
\newtheorem{axiom}{公理~}
\newtheorem{prop}{性质~}
\newcommand{\blank}[1]{\underline{\hbox to #1pt{}}}
\newcommand{\bracket}[1]{(\hbox to #1pt{})}
\newcommand{\onech}[4]{\par\begin{tabular}{p{.9\textwidth}}
A.~#1\\
B.~#2\\
C.~#3\\
D.~#4
\end{tabular}}
\newcommand{\twoch}[4]{\par\begin{tabular}{p{.46\textwidth}p{.46\textwidth}}
A.~#1& B.~#2\\
C.~#3& D.~#4
\end{tabular}}
\newcommand{\vartwoch}[4]{\par\begin{tabular}{p{.46\textwidth}p{.46\textwidth}}
(1)~#1& (2)~#2\\
(3)~#3& (4)~#4
\end{tabular}}
\newcommand{\fourch}[4]{\par\begin{tabular}{p{.23\textwidth}p{.23\textwidth}p{.23\textwidth}p{.23\textwidth}}
A.~#1 &B.~#2& C.~#3& D.~#4
\end{tabular}}
\newcommand{\varfourch}[4]{\par\begin{tabular}{p{.23\textwidth}p{.23\textwidth}p{.23\textwidth}p{.23\textwidth}}
(1)~#1 &(2)~#2& (3)~#3& (4)~#4
\end{tabular}}
\begin{document}

第一章复习题A组
\begin{enumerate}[1.]
\item 用列举法表示下列集合:\\
(1) 十二生肖组成的集合;\\
(2) 中国国旗上所有颜色组成的集合.
\item 用描述法表示下列集合:\\
(1) 平面直角坐标系中第一象限的角平分线上的所有点组成的集合;\\
(2) $3$的所有倍数组成的集合.
\item (1) 若$\alpha$: $x^2-5x+6=0$, $\beta$: $x=2$, 则$\alpha$是$\beta$的\blank{50}条件;
(2) 若$\alpha$: 四边形$ABCD$是正方形, $\beta$: 四边形$ABCD$的两条对角线互相垂直平分, 则$\alpha$是$\beta$的\blank{50}条件.
\item 已知方程$x^2+px+4=0$的所有解组成的集合为$A$, 方程$x^2+x+q=0$的所有解组成的集合为$B$, 且$A\cap B=\{4\}$. 求集合$A\cup B$的所有子集.
\item 已知集合$A=(-2, 1)$, $B=(-\infty, -2)\cup [1, +\infty)$. 求: $A\cup B$, $A\cap B$.
\item 已知全集$U=(-\infty, 1)\cup [2, +\infty)$, 集合$A=(-1, 1)\cup [3, +\infty)$. 求$A$.
\item 已知集合$A=\{x|x^2+px+q=0\}$, $B=\{x|x^2-x+r=0\}$, 且$A\cap B=\{-1\}$, $A\cup B=\{-1, 2\}$. 求实数$p$、$q$、$r$的值.
\item 设$a$是实数. 若$x=1$是$x>a$的一个充分条件, 则$a$的取值范围为\blank{50}.
\item 已知陈述句$\alpha$是$\beta$的充分非必要条件. 若集合$M=\{x|x\text{满足}\alpha\}$, $N=\{x|x\text{满足}\beta\}$, 则$M$与$N$的关系为\bracket{20}.
\fourch{$M\subset N$}{$M\supset N$}{$M=N$}{$M\cap N=\varnothing$}
\item 证明: 若梯形的对角线不相等, 则该梯形不是等腰梯形.
\end{enumerate}

第一章复习题B组
\begin{enumerate}[1.]
\item 若集合$M=\{a|a=x+\sqrt2y, x,y\in \mathbf{Q}\}$, 则下列结论正确的是\bracket{20}.
\fourch{$M\subseteq \mathbf{Q}$}{$M=\mathbf{Q}$}{$M\supset \mathbf{Q}$}{$M\subset \mathbf{Q}$}
\item 若$\alpha$是$\beta$的必要非充分条件, $\beta$是$\gamma$的充要条件, $\gamma$是$\delta$的必要非充分条件, 则$\delta$是$\alpha$的\blank{50}条件, $\gamma$是$\alpha$的\blank{50}条件.
\item 已知全集$U=\{x|x\text{为不大于}20\text{的素数}\}$. 若$A\cap \overline{B}=\{3, 5\}$, $\overline{A}\cap B=\{7, 19\}$, $\overline{A\cup B}=\{2, 17\}$, 则A=\blank{50} , B=\blank{50}.
\item 已知集合$P=\{x|-2\le x\le 5\}$, $Q=\{x|x\ge k+1\text{且}x\le 2k-1\}$, 且$Q\subseteq P$. 求实数$k$的取值范围.
\item 已知全集$U=\mathbf{R}$, 集合$A=\{x|x\le a-1\}$, $B=\{x|x>a+2\}$, $C=\{x|x<0\text{或}x\ge 4\}$, 且$\overline{A\cup B}\subseteq C$. 求实数$a$的取值范围.
\item 已知集合$A=\{x|(a-1)x^2+3x-2=0\}$. 是否存在这样的实数$a$, 使得集合$A$有且仅有两个子集? 若存在, 求出实数$a$的值及对应的两个子集; 若不存在, 说明理由.
\item 证明: $\sqrt[3]{2}$是无理数. 
\end{enumerate}

第一章复习题B组
\begin{enumerate}[1.]
\item 若集合$M=\{a|a=x+\sqrt2y, x,y\in \mathbf{Q}\}$, 则下列结论正确的是\bracket{20}.
\fourch{$M\subseteq \mathbf{Q}$}{$M=\mathbf{Q}$}{$M\supset \mathbf{Q}$}{$M\subset \mathbf{Q}$}
\item 若$\alpha$是$\beta$的必要非充分条件, $\beta$是$\gamma$的充要条件, $\gamma$是$\delta$的必要非充分条件, 则$\delta$是$\alpha$的\blank{50}条件, $\gamma$是$\alpha$的\blank{50}条件.
\item 已知全集$U=\{x|x\text{为不大于}20\text{的素数}\}$. 若$A\cap \overline{B}=\{3, 5\}$, $\overline{A}\cap B=\{7, 19\}$, $\overline{A\cup B}=\{2, 17\}$, 则A=\blank{50} , B=\blank{50}.
\item 已知集合$P=\{x|-2\le x\le 5\}$, $Q=\{x|x\ge k+1\text{且}x\le 2k-1\}$, 且$Q\subseteq P$. 求实数$k$的取值范围.
\item 已知全集$U=\mathbf{R}$, 集合$A=\{x|x\le a-1\}$, $B=\{x|x>a+2\}$, $C=\{x|x<0\text{或}x\ge 4\}$, 且$\overline{A\cup B}\subseteq C$. 求实数$a$的取值范围.
\item 已知集合$A=\{x|(a-1)x^2+3x-2=0\}$. 是否存在这样的实数$a$, 使得集合$A$有且仅有两个子集? 若存在, 求出实数$a$的值及对应的两个子集; 若不存在, 说明理由.
\item 证明: $\sqrt[3]{2}$是无理数. 
\end{enumerate}

第一章拓展与思考
\begin{enumerate}[1.]
\item 设$a,b$是正整数. 求证: 若$ab-1$是$3$的倍数, 则$a$与$b$被$3$除的余数相同.
\item 已知非空数集$S$满足: 对任意给定的$x,y\in S$($x,y$可以相同), 有$x+y\in S$且$x-y\in S$.\\
(1) 哪个数一定是$S$中的元素? 说明理由;\\
(2) 若$S$是有限集, 求$S$;\\
(3) 若$S$中最小的正数为$5$, 求$S$.
\end{enumerate}
\end{document}