\documentclass[10pt,a4paper]{article}
\usepackage[UTF8,fontset = none]{ctex}
\setCJKmainfont[BoldFont=黑体,ItalicFont=楷体]{华文中宋}
\usepackage{amssymb,amsmath,amsfonts,amsthm,mathrsfs,dsfont,graphicx}
\usepackage{ifthen,indentfirst,enumerate,color,titletoc}
\usepackage{tikz}
\usepackage{multicol}
\usepackage{makecell}
\usepackage{longtable}
\usetikzlibrary{arrows,calc,intersections,patterns,decorations.pathreplacing,3d,angles,quotes}
\usepackage[bf,small,indentafter,pagestyles]{titlesec}
\usepackage[top=1in, bottom=1in,left=0.8in,right=0.8in]{geometry}
\renewcommand{\baselinestretch}{1.65}
\newtheorem{defi}{定义~}
\newtheorem{eg}{例~}
\newtheorem{ex}{~}
\newtheorem{rem}{注~}
\newtheorem{thm}{定理~}
\newtheorem{coro}{推论~}
\newtheorem{axiom}{公理~}
\newtheorem{prop}{性质~}
\newcommand{\blank}[1]{\underline{\hbox to #1pt{}}}
\newcommand{\bracket}[1]{(\hbox to #1pt{})}
\newcommand{\onech}[4]{\par\begin{tabular}{p{.9\textwidth}}
A.~#1\\
B.~#2\\
C.~#3\\
D.~#4
\end{tabular}}
\newcommand{\twoch}[4]{\par\begin{tabular}{p{.46\textwidth}p{.46\textwidth}}
A.~#1& B.~#2\\
C.~#3& D.~#4
\end{tabular}}
\newcommand{\vartwoch}[4]{\par\begin{tabular}{p{.46\textwidth}p{.46\textwidth}}
(1)~#1& (2)~#2\\
(3)~#3& (4)~#4
\end{tabular}}
\newcommand{\fourch}[4]{\par\begin{tabular}{p{.23\textwidth}p{.23\textwidth}p{.23\textwidth}p{.23\textwidth}}
A.~#1 &B.~#2& C.~#3& D.~#4
\end{tabular}}
\newcommand{\varfourch}[4]{\par\begin{tabular}{p{.23\textwidth}p{.23\textwidth}p{.23\textwidth}p{.23\textwidth}}
(1)~#1 &(2)~#2& (3)~#3& (4)~#4
\end{tabular}}
\begin{document}

\begin{enumerate}[1.]
\item 判断下列命题是否正确:\\
(1) 终边重合的两个角相等;\\
(2) 锐角是第一象限的角;\\
(3) 第二象限的角是钝角;\\
(4) 小于$90^\circ$的角都是锐角.
\item 分别用集合的形式表示终边位于第三象限的所有角和终边位于$y$轴正半轴上的所有角.
\item 在$0^\circ-360^\circ$范围内, 分别找出终边与下列各角的终边重合的角, 并判断它们是第几象限的角:\\
(1) $-315^\circ$\\
(2) $905.3^\circ$;\\
(3) $-1090^\circ$;\\
(4) $530^\circ$.
\item 分别将下列角度化为弧度:\\
(1) $15^\circ$;\\
(2) $-108^\circ$;\\
(3) $22^\circ 30'$.
\item 分别将下列弧度化为角度:
(1) $\dfrac{11}{12}\pi$;\\
(2) $-\dfrac 25\pi$;\\
(3) $-3$(结果精确到$0.01^\circ$).
\item 已知扇形的弧所对的圆心角为$54^\circ$, 且半径为$10\text{cm}$. 求该扇形的弧长和面积.
\item 如果$\alpha$是第三象限的角, 判断$\dfrac\alpha 2$是哪个象限的角.
\item 已知角$\alpha$的终边过点$P(2a, -3a)$($a<0$), 求角$\alpha$的正弦、余弦、正切及余切值.
\item 已知角$\alpha$的终边过点$P(0, -3)$, 则下列值不存在的是\bracket{20}.
\fourch{$\sin \alpha$}{$\cos \alpha$}{$\tan \alpha$}{$\cot \alpha$}
\item 根据下列条件, 分别判断角$\theta$属于第几象限:\\
(1) $\sin \theta =-\dfrac 12$且$\cos \theta =-\dfrac{\sqrt 3}2$;\\
(2) $\sin \theta <0$且$\tan \theta >0$.
\item 求角$\dfrac 53\pi$的正弦、余弦、正切及余切值.
\item 分别求$\sin k\pi$($k\in \mathbf{Z}$)和$\cos k\pi$($k\in \mathbf{Z}$)的值.
\item 已知$\alpha$为第三象限的角, $\cos \alpha=-\dfrac{\sqrt 5}5$. 求$\sin \alpha$、$\tan \alpha$及$\cot \alpha$.
\item 已知$\cot \alpha=\dfrac 13$, 求$\sin \alpha$、$\cos \alpha$及$\tan \alpha$.
\item 已知$\tan \alpha=3$, 求$\dfrac{2\sin \alpha+\cos \alpha}{\sin \alpha-\cos \alpha}$的值.
\item 化简:\\
(1) $\sin ^2\alpha+\sin ^2\alpha\cos ^2\alpha+\cos^4\alpha$;\\
(2) $\sin \alpha\cos \alpha(\tan \alpha+\cot \alpha)$.
\item 证明: $\cot ^2\alpha-\cos ^2\alpha=\cot ^2\alpha\cdot \cos ^2\alpha$.
\item 证明:\\
(1) $\sin (2\pi -\alpha)=-\sin \alpha$;\\
(2) $\cos (2\pi -\alpha)=\cos \alpha$;\\
(3) $\tan (2\pi -\alpha)=-\tan \alpha$;\\
(4) $\cot (2\pi -\alpha)=-\cot \alpha$.
\item 利用诱导公式求值:\\
(1) $\sin \dfrac{11}4\pi$;\\
(2) $\cos (-\dfrac 56\pi)$;\\
(3) $\tan (-\dfrac{14}3\pi)$.
\item 化简:\\
(1) $\dfrac{\sin (180^\circ -\alpha)}{\sin (180^\circ +\alpha)}+\dfrac{\cos (360^\circ -\alpha)}{\cos (180^\circ +\alpha)}+\dfrac{\tan (180^\circ +\alpha)}{\tan (-\alpha)}$;\\
(2) $\dfrac{\sin (\pi -\alpha)}{\cos (\pi +\alpha)}\cdot \dfrac{\sin (2\pi -\alpha)}{\tan (\pi +\alpha)}$.
\item 证明:\\
(1) $\sin (\dfrac{3\pi}2-\alpha)=-\cos \alpha$;\\
(2) $\cos (\dfrac{3\pi}2-\alpha)=-\sin \alpha$;\\
(3) $\tan (\dfrac{3\pi}2-\alpha)=\cot \alpha$;\\
(4) $\cot (\dfrac{3\pi}2-\alpha)=\tan \alpha$.
\item 化简: $\dfrac{\sin (\dfrac\pi 2+\alpha)\cot (\dfrac{3\pi}2-\alpha)\cos (3\pi +\alpha)}{\cot (\dfrac\pi 2-\alpha)\cos (\dfrac{3\pi} 2+\alpha)\cot (\pi -\alpha)}$.
\item 已知点$A$的坐标为$(3, 4)$, 将$OA$绕坐标原点$O$顺时针旋转$\dfrac\pi 2$至$OA'$. 求点$A'$的坐标.
\item 根据下列条件, 分别求角$x$:\\
(1) 已知$\sin x=-\dfrac{\sqrt 3}2$;\\
(2) 已知$\cos x=-\dfrac 12$;\\
(3) 已知$\tan x=-\sqrt 3$.
\item 分别求满足下列条件的角$x$的集合:\\
(1) $2\sin (x+\dfrac\pi 3)=1$, $x\in [0, 2\pi ]$;\\
(2) $\cos (2x+\dfrac \pi 4)=-\dfrac 12$;\\
(3) $\tan (3x+\dfrac \pi 4)=-1$.
\item 化简:\\
(1) $\cos (22^\circ -x)\cos (23^\circ +x)-\sin (22^\circ -x)\sin (23^\circ +x)$;\\
(2) $\cos (\dfrac\pi 6+\alpha)\cos \alpha+\sin (\dfrac \pi 6+\alpha)\sin \alpha$.
\item 已知$\sin \theta =-\dfrac 5{13}$, $\theta \in (\pi , \dfrac3 2\pi )$. 求$\cos (\theta +\dfrac \pi 4)$的值.
\item 证明:\\
(1) $\dfrac{2\cos A\cos B-\cos (A-B)}{\cos (A-B)-2\sin A\sin B}=1$;\\
(2) $\cos (\alpha+\beta)\cos (\alpha-\beta)=\cos^2\beta-\sin ^2\alpha$.
\item 求下列各式的值:\\
(1) $\sin \dfrac{5\pi}{12}\cos \dfrac\pi{12}-\cos \dfrac{5\pi}{12}\sin \dfrac\pi {12}$;
(2) $\dfrac{1+\tan 15^\circ}{1-\tan 15^\circ}$.
\item 已知$\cos \theta =-\dfrac 35$, $\theta \in (0, \pi)$. 求$\sin (\theta +\dfrac\pi 4)$和$\tan (\theta -\dfrac\pi 4)$的值.
\item 证明下列恒等式:\\
(1) $\dfrac{\sin (\alpha+\beta)\sin (\alpha-\beta)}{\cos^2\alpha\cos^2\beta}=\tan^2\alpha-\tan^2\beta$;\\
(2) $\tan (\theta +\dfrac\pi 4)=\dfrac{1+\tan \theta}{1-\tan \theta}$.
\item 在$\triangle ABC$中, 已知$\cos A=\dfrac{12}{13}$, $\cos B=\dfrac8{17}$. 求$\sin C$和$\cos C$的值.
\item 已知$\cos \alpha=\dfrac 45$, $\alpha\in (0, \dfrac \pi 2)$, $\sin \beta=\dfrac{12}{13}$, $\beta\in (\dfrac\pi 2, \pi )$. 求$\sin (\alpha+\beta)$和$\cos (\alpha+\beta)$的值, 并判断$\alpha+\beta$是第几象限的角.
\item 把下列各式化为$A\sin (\alpha+\varphi )$($A>0$)的形式:\\
(1) $\sin \alpha+\cos \alpha$;\\
(2) $-\sin \alpha+\sqrt 3\cos \alpha$.
\item 利用二倍角公式, 求下列各式的值:\\
(1) $\sin \dfrac{5\pi}{12}\cos\dfrac{5\pi}{12}$;\\
(2) $\cos^222.5^\circ -\sin^222. 5^\circ$;\\
(3) $\dfrac{\tan 15^\circ}{1-\tan^215^\circ}$.
\item 已知$\cos \alpha=-\dfrac{\sqrt 5}5$, $\alpha\in (\dfrac\pi 2, \pi)$. 求$\sin 2\alpha$, $\cos 2\alpha$和$\tan 2\alpha$的值.
\item 证明下列恒等式:\\
(1) $(\sin \alpha+\cos \alpha)^2=1+\sin 2\alpha$;\\
(2) $\cos^4\alpha-\sin^4\alpha=\cos 2\alpha$;\\
(3) $\sin 3\alpha=3\sin \alpha-4\sin^3\alpha$.
\item 证明: $\cos \alpha\cos \beta=\dfrac 12[\cos (\alpha+\beta)+\cos (\alpha-\beta)]$.
\item 证明: $\cos \alpha+\cos \beta=2\cos \dfrac{\alpha+\beta}
2 \cos \dfrac{\alpha-\beta}2$.
\item 证明: $\tan \dfrac\alpha2=\dfrac{1-\cos \alpha}{\sin \alpha}$.
\item 在$\triangle ABC$中, 已知$a=7$, $B=30^\circ$, $C=85^\circ$. 求$c$. (结果精确到$0. 01$)
\item 在$\triangle ABC$中, 已知$a=5$, $A=40^\circ$, $B=80^\circ$. 求$b$、$c$和面积$S$. (结果精确到$0. 01$)
\item 在$\triangle ABC$中, 如果$\sin^2A+\sin^2B=\sin^2C$, 试判断该三角形的形状. 
练习6. 3(2)
\item 在$\triangle ABC$中, 已知$a=3$, $b=4$, $C=60^\circ$. 求$c$.
\item 在$\triangle ABC$中, 已知$A=45^\circ$, $a=2\sqrt 6$, $b=2\sqrt 3$. 求$B$、$C$及$c$.
\item 在$\triangle ABC$中, 已知三边之比为$2:3:4$. 求该三角形的最大角的余弦值. 
\end{enumerate}

\end{document}