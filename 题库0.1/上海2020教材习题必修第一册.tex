\documentclass[10pt,a4paper]{article}
\usepackage[UTF8,fontset = windows]{ctex}
\setCJKmainfont[BoldFont=黑体,ItalicFont=楷体]{华文中宋}
\usepackage{amssymb,amsmath,amsfonts,amsthm,mathrsfs,dsfont,graphicx}
\usepackage{ifthen,indentfirst,enumerate,color,titletoc}
\usepackage{tikz}
\usepackage{multicol}
\usepackage{makecell}
\usepackage{longtable}
\usetikzlibrary{arrows,calc,intersections,patterns,decorations.pathreplacing,3d,angles,quotes}
\usepackage[bf,small,indentafter,pagestyles]{titlesec}
\usepackage[top=1in, bottom=1in,left=0.8in,right=0.8in]{geometry}
\renewcommand{\baselinestretch}{1.65}
\newtheorem{defi}{定义~}
\newtheorem{eg}{例~}
\newtheorem{ex}{~}
\newtheorem{rem}{注~}
\newtheorem{thm}{定理~}
\newtheorem{coro}{推论~}
\newtheorem{axiom}{公理~}
\newtheorem{prop}{性质~}
\newcommand{\blank}[1]{\underline{\hbox to #1pt{}}}
\newcommand{\bracket}[1]{(\hbox to #1pt{})}
\newcommand{\onech}[4]{\par\begin{tabular}{p{.9\textwidth}}
A.~#1\\
B.~#2\\
C.~#3\\
D.~#4
\end{tabular}}
\newcommand{\twoch}[4]{\par\begin{tabular}{p{.46\textwidth}p{.46\textwidth}}
A.~#1& B.~#2\\
C.~#3& D.~#4
\end{tabular}}
\newcommand{\vartwoch}[4]{\par\begin{tabular}{p{.46\textwidth}p{.46\textwidth}}
(1)~#1& (2)~#2\\
(3)~#3& (4)~#4
\end{tabular}}
\newcommand{\fourch}[4]{\par\begin{tabular}{p{.23\textwidth}p{.23\textwidth}p{.23\textwidth}p{.23\textwidth}}
A.~#1 &B.~#2& C.~#3& D.~#4
\end{tabular}}
\newcommand{\varfourch}[4]{\par\begin{tabular}{p{.23\textwidth}p{.23\textwidth}p{.23\textwidth}p{.23\textwidth}}
(1)~#1 &(2)~#2& (3)~#3& (4)~#4
\end{tabular}}
\begin{document}

\begin{enumerate}[1.]
\item 用列举法表示下列集合:\\
(1) $10$以内的所有素数组成的集合;\\
(2) $\{y|y=x-1,\  0\le x\le 3,\ x\in \mathbf{Z}\}$.
\item 用描述法表示下列集合:\\
(1) 被$3$除余$1$的所有自然数组成的集合;\\
(2) 比$1$大又比$10$小的所有实数组成的集合;\\
(3) 平面直角坐标系中坐标轴上所有点组成的集合.
\item 集合$\{(x, y)|xy>0, \ x,y\text{为实数}\}$是指\bracket{20}.
\twoch{第一象限内的所有点组成的集合}{第三象限内的所有点组成的集合}{第一象限和第三象限内的所有点组成的集合}{不在第二象限也不在第四象限内的所有点组成的集合}
\item 用符号``$\subset$''``$=$''或``$\supset$''连接集合$A$与$B$:\\
(1) $A=\{x|x^2-2x+1=0\}$, $B=\{x|x^2-1=0\}$;\\
(2) $A=\{1, 2, 4, 8\}$, $B=\{x|x$是$8$的正约数$\}$.
\item 已知集合$A=\{1\}$, $B=\{x|x^2-3x+a=0\}$. 是否存在实数$a$, 使得$A\subset B$?  若存在, 求$a$的值; 若不存在, 说明理由.
\item 已知集合$A=\{x, y\}$, $B=\{2x, 2x^2\}$, 且$A=B$. 求集合$A$.
\item 已知集合$A=\{x|x\le 7\}$, $B=\{x|x<2\}$, $C=\{x|x>5\}$. 求: $A\cap B$, $A\cap C$, $A\cap (B\cap C)$.
\item 已知集合$A=\{(x, y)|y=-x+1\}$, $B=\{(x, y)|y=x^2-1\}$. 求$A\cap B$.
\item 已知全集$U=\mathbf{R}$, 集合$A=\{x|4-x>2x+1\}$. 求$A$.
\item 已知集合$A=\{2, (a+1)^2, a^2+3a+3\}$, 且$1\in A$. 求实数$a$的值.
\item 已知集合$A=\{x|x=2n+1,\ n\in \mathbf{Z}\}$, $B=\{x|x=4n-1,\ n\in \mathbf{Z}\}$. 判断集合$A$与$B$的包含关系, 并证明你的结论.
\item 设$a$是实数, 集合$M=\{x|x^2+x-6=0\}$, $N=\{y|ay+2=0\}$. 是否存在$a$, 使得$N\subset M$? 若存在, 求这些$a$的值; 若不存在, 说明理由.
\item 已知集合$A=\{1, 4, x\}$, $B=\{1, x^2\}$, 且$A\cup B=A$. 求$x$的值及集合$A$、$B$.
\item 判断下列语句是否为命题:\\
(1) 有的正方形是三角形;\\
(2) 任意一个三角形的内角和都为$180^\circ$;\\
(3) $1$是自然数吗?\\
(4) $3>\pi$;\\
(5) $2\in (0, 5)$, 且$2\in \mathbf{Z}$.
\item 判断下列命题的真假, 并说明理由:\\
(1) 如果$a$、$b$都是奇数, 那么$a+b$是偶数;\\
(2) 一组对边平行且两对角线等长的四边形是平行四边形;\\
(3) 如果$A\cap B=A$, 那么$A\cup B=B$.
\item 如果$a$、$b$、$c$为实数, 设$\alpha$: $a=b=c=0$; $\beta$: $a$、$b$、$c$中至少有一个为$0$; $\gamma$: $a^2+\sqrt b+|c|=0$. 那么$\alpha$\blank{20}$\beta$; $\alpha$\blank{20}$\gamma$; $\beta$ \blank{20}$\gamma$. (用符号``$\Leftarrow$''``$\Rightarrow$''或``$\Leftrightarrow$''填空)
\item 下列各组中, $\alpha$是$\beta$的什么条件?\\
(1) $\alpha$: 四边形$ABCD$的四条边等长, $\beta$: 四边形$ABCD$是正方形;\\
(2) $\alpha$: $\triangle ABC$与$\triangle DEF$全等, $\beta$: $\triangle ABC$与$\triangle DEF$的周长相等;\\
(3) $\alpha$: $x$是$2$的倍数, $\beta$: $x$是$6$的倍数;\\
(4) $\alpha$: 集合$A\subseteq B$, $B\subseteq C$, $C\subseteq A$, $\beta$: 集合$A=B=C$;\\
(5) $\alpha$: $A\cap B=A\cap C$, $\beta$: $B=C$.
\item 已知$l$、$m$都是自然数, 试判断``$l+m$是偶数''与``$l$、$m$都是偶数''是否等价, 并说明理由.
\item 证明: ``四边形$ABCD$是平行四边形''是``四边形$ABCD$的对角线互相平分''的充要条件.
\item 判断下列命题的真假, 并说明理由:\\
(1) 若$A\cap B=\varnothing$, $C\subset B$, 则$A\cap C=\varnothing$;\\
(2) 若$a$、$b\in \mathbf{R}$, 则关于$x$的方程$(a+1)x+b=0$的解为$x=- \dfrac b{a+1}$.
\item 已知$a$为实数. 写出关于$x$的方程$ax^2+2x+1=0$至少有一个实根的一个充要条件、一个充分非必要条件和一个必要非充分条件.
\item 若$\alpha$: $\{2\}\subset B\subseteq \{2, 3, 4\}$, $\beta$: $B=\{2, 4\}$, 则$\alpha$是$\beta$的\bracket{20}.
\twoch{充分非必要条件}{必要非充分条件}{充要条件}{既非充分又非必要条件}
\item 已知$\alpha$: $x<3m-1$或$x>-m$, $\beta$: $x<2$或$x\ge 4$.\\
(1) 若$\alpha$是$\beta$的充分条件, 求实数$m$的取值范围;\\
(2) 若$\alpha$是$\beta$的必要条件, 求实数$m$的取值范围. 
\item 设$a\in \mathbf{R}$, 求关于$x$的方程$ax=2$的解集.
\item 设$k\in \mathbf{R}$, 求关于$x$与$y$的二元一次方程组$\begin{cases}y=-2x+1,\\  y=kx-3\end{cases}$的解集.
\item 设$a\in \mathbf{R}$, 求一元二次方程$x^2-2ax+a^2-4=0$的解集.
\item 已知等式$2x^2+3x+5=a(2x+1)(x+1)+c$恒成立, 求常数$a$、$c$的值.
\item 已知一元二次方程$ax^2+bx+c=0$($a\ne 0$)的两实根为$x_1$、$x_2$, 求证: $|x_2-x_1| = \dfrac{\sqrt{b^2-4 ac}}{|a|}$.
\item 已知一元二次方程$x^2+3x-3=0$的两个实根分别为$x_1$、$x_2$, 求作二次项系数是$1$, 且分别以下列数值为根的一元二次方程:\\
(1) $-x_1, -x_2$;\\
(2) $2x_1+1, 2x_2+1$;\\
(3) $\dfrac 1{x_1}, \dfrac 1{x_2}$;\\
(4) $x_1^2, x_2^2$.
\item 设$a$、$b$、$c$、$d$为实数, 判断下列命题的真假:\\
(1) 若$a>b\ge 0$, 则$a^2>b^2$;\\
(2) 若$\sqrt a>\sqrt b$, 则$a>b$;\\
(3) 若$a>b>0, c>d>0$, 则$ac>bd$;\\
(4) 若$\dfrac ba>0$, 则$ab>0$;\\
(5) 若$a>b>0$, 则$a^2>ab>b^2$;\\
(6) 若$\sqrt a>b$, 则$a>b^2$.
\item 如果$a^2>b^2$, 那么下列不等式中成立的是\bracket{20}.
\fourch{$a>0>b$}{$a>b>0$}{$|a|>|b|$}{$a>|b|$}
\item 如果$a<b<0$, 那么下列不等式中成立的是\bracket{20}.
\fourch{$\dfrac ab<1$}{$a^2>ab$}{$\dfrac1{b^2}<\dfrac 1{a^2}$}{$\dfrac 1a<\dfrac 1b$}
\item 如果$a<0<b$, 那么下列不等式中成立的是\bracket{20}.
\fourch{$\sqrt{-a}<\sqrt{-b}$}{$a^2<b^2$}{$a^3<b^3$}{$ab>b^2$}
\item 证明: ``$a>0$且$b>0$''是``$a+b>0$且$ab>0$''的充要条件.
\item 设$x$是实数, 比较$(x+1)(x^2-x+1)$与$(x-1)(x^2+x+1)$的值的大小.
\item 试比较下列各数的大小, 并说明理由:\\
(1) $3+\sqrt 3$与$2+\sqrt 5$;\\
(2) $\sqrt 3+\sqrt 5$与$\sqrt 2+\sqrt 6$.
\item 设$a$、$b$为实数, 比较$a^2+b^2$与$2a-2b-2$的值的大小.
\item 已知$a>b$, $c>d$. 求证: $ac+bd>ad+bc$.
\item 已知$a\ge -1$, 求证$: a^3+1\ge a^2+a$.
\item 已知$a$、$b$为任意给定的正数, 求证: $a^3+b^3\ge ab^2+ba^2$, 并指出等号成立的条件.
\item 设$a$为实数, 求关于$x$的方程$2x+a^2=ax+4$的解集.
\item 设$m$为实数, 求关于$x$的方程$(m+1)x^2+6mx+9m=1$的解集.
\item 已知等式$2x^2-3x-1=a(x-1)^2+b(x-1)+c$恒成立, 其中$a$、$b$、$c$为常数. 求$a-b+c$的值.
\item 对一元二次方程$ax^2+bx+c=0$($a\ne 0$), 证明: $ac<0$是该方程有两个异号实根的充要条件.
\item 已知一元二次方程$2x^2+x-3=0$的两个实根分别为$x_1$、$x_2$, 求作二次项系数是$1$, 且分别以下列数值为根的一元二次方程:\\
(1) $x_1+x_2, x_1x_2$;\\
(2) $2x_1^2+1, 2x_2^2+1$;\\
(3) $\dfrac{x_2}{x_1}$, $\dfrac{x_1}{x_2}$;\\
(4) $x_1^4$, $x_2^4$.
\item 已知一元二次方程$x^2-2mx+m-1=0$的两实根为$x_1$、$x_2$, 且$x_1^2+x_2^2=4$. 求实数$m$的值.
\item 已知实数$a$、$b$、$c$满足$a+b+c=0$, 且$a>b>c$. 求证: $a>0$且$c<0$.
\item 设$s=a+b$, $p=ab$($a$、$b\in\mathbf{R}$), 写出``$a>1$且$b>1$''用$s$、$p$表示的一个充要条件, 并证明.
\item 原有酒精溶液$a$(单位: $\text{g}$), 其中含有酒精$b$(单位: $\text{g}$), 其酒精浓度为$\dfrac ba$. 为增加酒精浓度, 在原溶液中加入酒精$x$(单位: $\text{g}$), 新溶液的浓度变为$\dfrac{b+x}{a+x}$. 根据这一事实, 可提炼出如下关于不等式的命题:若$a>b>0$, $x>0$, 则$\dfrac ba<\dfrac{b+x}{a+x}<1$. 试加以证明. 
\end{enumerate}

\end{document}