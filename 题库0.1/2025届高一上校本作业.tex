\documentclass[10pt,a4paper]{article}
\usepackage[UTF8,fontset = windows]{ctex}
\setCJKmainfont[BoldFont=黑体,ItalicFont=楷体]{华文中宋}
\usepackage{amssymb,amsmath,amsfonts,amsthm,mathrsfs,dsfont,graphicx}
\usepackage{ifthen,indentfirst,enumerate,color,titletoc}
\usepackage{tikz}
\usepackage{multicol}
\usepackage{makecell}
\usepackage{longtable}
\usepackage{ifthen}
\usetikzlibrary{arrows,calc,intersections,patterns,decorations.pathreplacing,3d,angles,quotes}
\usepackage[bf,small,indentafter,pagestyles]{titlesec}
\usepackage[top=1in, bottom=1in,left=0.8in,right=0.8in]{geometry}
\renewcommand{\baselinestretch}{1.65}
\newtheorem{defi}{定义~}
\newtheorem{eg}{例~}
\newtheorem{ex}{~}
\newtheorem{rem}{注~}
\newtheorem{thm}{定理~}
\newtheorem{coro}{推论~}
\newtheorem{axiom}{公理~}
\newtheorem{prop}{性质~}
\newcommand{\blank}[1]{\underline{\hbox to #1pt{}}}
\newcommand{\bracket}[1]{(\hbox to #1pt{})}
\newcommand{\onech}[4]{\par\begin{tabular}{p{.9\textwidth}}
A.~#1\\
B.~#2\\
C.~#3\\
D.~#4
\end{tabular}}
\newcommand{\twoch}[4]{\par\begin{tabular}{p{.46\textwidth}p{.46\textwidth}}
A.~#1& B.~#2\\
C.~#3& D.~#4
\end{tabular}}
\newcommand{\vartwoch}[4]{\par\begin{tabular}{p{.46\textwidth}p{.46\textwidth}}
(1)~#1& (2)~#2\\
(3)~#3& (4)~#4
\end{tabular}}
\newcommand{\fourch}[4]{\par\begin{tabular}{p{.23\textwidth}p{.23\textwidth}p{.23\textwidth}p{.23\textwidth}}
A.~#1 &B.~#2& C.~#3& D.~#4
\end{tabular}}
\newcommand{\varfourch}[4]{\par\begin{tabular}{p{.23\textwidth}p{.23\textwidth}p{.23\textwidth}p{.23\textwidth}}
(1)~#1 &(2)~#2& (3)~#3& (4)~#4
\end{tabular}}
\begin{document}

\begin{enumerate}[1.]
\item 判断下列各组对象能否组成集合, 若能组成集合, 指出是有限集还是无限集.\\
(1) 上海市控江中学$2022$年入学的全体高一年级新生;\\
(2) 中国现有各省的名称;\\
(3) 太阳、$2$、上海市;\\
(4) 大于$10$且小于$15$的有理数;\\
(5) 末位是$3$的自然数;\\
(6) 影响力比较大的中国数学家;\\
(7) 方程$x^2+x+3=0$的所有实数解;\\ 
(8) 函数$y=\dfrac 1x$图像上所有的点;\\ 
(9) 在平面直角坐标系中, 到定点$(0, 0)$的距离等于$1$的所有点;\\
(10) 不等式$3x-10<0$的所有正整数解;\\
(11) 所有的平面四边形.
\item 用``$\in$''或`` $\notin$''填空:\\
(1) $-3$\blank{20}$\mathbf{N}$;\\
(2) $3.14$\blank{20}$\mathbf{Q}$;\\
(3) $5$\blank{20}$\mathbf{Z}$;\\
(4) $\dfrac 12$\blank{20}$\mathbf{N}$;\\
(5) $-2$\blank{20}$\mathbf{Q}$;\\
(6) $\pi$\blank{20}$\mathbf{R}$; 
(7) $0.\dot{1}\dot{3}$\blank{20}$\mathbf{Q}$;\\ 
(8) $\dfrac 1{\sqrt 2-1}-\sqrt 2$\blank{20}$\mathbf{Z}$;\\
(9) $\dfrac{\pi}2$\blank{20}$\mathbf{Q}$;\\
(10) $\dfrac 1{1-\dfrac 1{1-\dfrac 12}}$\blank{20}$\mathbf{N}$;\\
(11) $0$\blank{20}$\varnothing$;\\
(12) $0$\blank{20}$\mathbf{N}$.
\item 对于一个确定的实数$x$, 由$x$, $-x$, $|x|$, $-\sqrt{x^2}$中的一个值或几个值组成的所有集合中, 元素的个数最多有多少个? 
\item 已知关于$x$的方程$\sqrt {x^2+4x+a}=x+2$, 若以该方程的所有解为元素组成的集合是无限集, 求实数$a$满足的条件.
\item 用列举法表示下列集合:\\
(1) $12$以内的素数组成的集合;\\
(2) 绝对值小于$3$的所有整数的集合;\\
(3) $\{x|\dfrac 6{3-x}\in\mathbf{N}, \ x\in\mathbf{Z}\}$;\\
(4) $\{y|y=x^2-1 , \ |x| \le 2, \ x\in\mathbf{Z}\}$;\\
(5) $\{( x,y)|y=x^2-1,\ |x|\le 2, \ x\in\mathbf{Z}\}$;\\
(6) $\{( x,y)|x +y=5, \ x\in\mathbf{N}, \ y\in\mathbf{N}\}$.
\item 用描述法表示下列集合:\\
(1) 所有奇数组成的集合;\\
(2) 被$3$除余数等于$2$的正整数的集合;\\
(3) 不小于$10$的实数组成的集合;\\
(4) 绝对值大于$4$的所有整数组成的集合;\\
(5) 平面直角坐标系内$y$轴上的点的坐标组成的集合;\\
(6) 在直线$y=2x+1$上所有的点的坐标组成的集合.
\item 用区间表示下列集合:\\
(1) $\{x|-2<x<7\}$;\\
(2) $\{x|-2\le\ x\le7\}$;\\
(3) $\{x|-2\le\ x<7\}$;\\
(4) 不等式$2x<5$的解集;\\
(5) 不等式$-x<5$的解集; \\
(6) 非负实数集.
\item 用适当的方法表示下列集合:\\
(1) 能整除$10$的所有正整数组成的集合;\\
(2) 能整除$10$的所有正整数组成的集合;\\
(3) 方程$x^2+2=0$的实数解组成的集合;\\
(4) 方程组$\begin{cases}2x+y=0, \\ x-y+3=0\end{cases}$的所有解组成的集合;\\
(5) 两直线$y=2x+1$和$y=x-2$的交点组成的集合.
\item 下面写法正确的有\blank{50}.\\
\textcircled{1} $\varnothing\in\{a\}$; \textcircled{2} $(0, 1)\in\{0, 1\}$; \textcircled{3} $1 \in \{(0,1)\}$; \textcircled{4} $(0,1) \in \{(0,1)\}$; \textcircled{5} $0\in \{0,1\}$; \textcircled{6} $0 \notin \{0,1\}$.
\item 集合$\{(x, y)|xy\ge 0,\  x\in\mathbf{R},\  y\in\mathbf{R}\}$是指\bracket{20}.
\twoch{第一象限内的所有点}{第三象限内的所有点}{第一象限和第三象限内的所有点}{不在第二象限、第四象限内的所有点}
\item 若集合$M=\{0,2,3,7\}$, $P=\{x|x=ab,\ a,b\in M, \ a\ne b\}$. 用列举法写出集合$P$.
\item 已知集合$A={2, a^2, a}$, 且$1\in A$, 求实数$a$的值.
\item 设集合$M=\{a|a=x^2-y^2, \ x,y\in\mathbf{Z}\}$, 下列数中不属于$M$的为\bracket{20}.
\fourch{$3$}{$6$}{$9$}{$12$}
\item 已知集合$A=\{x|x=a+\sqrt 2b,\ a,b\in \mathbf{Z}\}$, 若$x_1,x_2\in A$, 证明: $x_1x_2\in A$.
\item 已知集合$A=\{x|(k+1)x^2+x-k=0\}$中只有一个元素, 求实数$k$的值.
\item 用符号``$\subset$''、``$=$''或``$\supset$''填空:\\
(1) $\{a\}$\blank{50}$\{a, b, c\}$;\\
(2) $\{a, b, c\}$\blank{50}$\{a, c\}$;\\
(3) $\{1, 2\}$\blank{50}$\{x|x^2-3x+2=0\}$;\\
(4) $A=\{x|x^2-2x+1=0\}$\blank{50}$B=\{x|x^2+2x-3=0\}$;\\
(5) $A=\{1, 2\}$\blank{50}$B=\{x|x$是$2$的正约数$\}$;\\
(6) $A=\{(x, y)|xy>0\}$\blank{50}$B=\{(x, y)|x>0, \ y>0\}$.

\iffalse




\item (1) 集合$\{1,2,3\}$的子集共有$____________$个.
(2) 已知集合$A=\{12\}$, 集合$B=\{1\$2\ 3\ 4\$5\}$.若集合$M$满足$A\subset M$且$M\subseteq B$, 则这样的集合$M$有$________________________________________________________________$.
(3) 已知满足$\{a$, b\}\subset\ $M\subset{a$, b, c, d, e}的集合$M$有$_________$个. $3$.不定项选择题.
(1)下列写法正确的是$()$
\fourch{\emptyset\subset\{0\}}{\emptyset=\emptyset}{\emptyset\in\{0\}}{$0\in\emptyset$}
(2)下列各题中的$M$与$P$表示同一个集合的是$()$
\fourch{M=\{$(1$, -3)$\}$, P=\{(-3, 1)$\}       }{ M=\{1$, -3\}, P=\{-3, 1\} $}{ M=\emptyset$, P=\{\emptyset\} $}{ M=\{y|y=x^2+1$, x\in\mathbf{R}\}, P=\{(x, y)$|y=x^2+1$, x\in\mathbf{R}\}}
(E)$M=\{y|y=x^2+1$, x\in\mathbf{R}\}, P=\{t|t=y^2+1, y\in\mathbf{R}\}
(F)$M=\{y|y=x^2+1$, x\in\mathbf{R}\}, P=\{x|y=\sqrt{x-1}, x\in\mathbf{R}\}
(3)下列说法正确的是$()$
\fourch{若$a\in\ A$且$A\subseteq\ B$, 则$a\in\ B}{$若$A\subseteq\ B$且$A\subseteq\ C$, 则$B=C}{$若$A\subset\ B$且$B\subseteq\ C$, 则$A\subset\ C
\item$设常数$x$, y\in\{\mathbf{R}}, 已知集合$A=\{x$, y\}, $B=\{2x$, x^2\}, 且$A=B$, 求集合$A$.
\item (1) 证明:集合$A=\{1,2,3\}$是集合$B=\{0,1,2,3,4,5,6\}$的子集.
(2) 判断集合$A=\{n|n=2k-1,k\in \mathbf{Z}\}, B=\{n|n=2m+1,m\in \mathbf{Z}\}$的关系, 并说明理由;
(3) 证明集合$A=\{n|n=2k-1,k\in \mathbf{N}\}$不是集合$B=\{n|n=2m+1,m\in \mathbf{N}\}$的子集, 且集合$A$真包含集合$B$; $
(B$组)$
\item$已知集$B={0$, 2, 4}, $C={0$, 2, 6}, 若集合$A$满足A⊆B, A⊆C, 写出所有满足条件的集合$A$.2.已知集合$A=\{1\}, B=\{\ x|x\subseteq A\}$, 用列举法表示集合$B$. 并指出$A$与B的关系. 1.1.3集合之间的关系$-2
(A$组)$
\item$若集合$A=\{2,a,a+3\}, B=\{2,3,5,8\}$, 且$B\supset A$, 则$a$的值为$__________$.
\item 设常数$a\in \mathbf{R}$. 若集合$A=(-\infty ,5)$与$B=(-\infty ,a]$满足$A\subseteq B$, 则$a$的取值范围是$_____________$.证明$: 1^\mathrmo$当$a______$时, 任取$x\in\ A$, 则$____________________$, 所以$x\in\ B$, 即$A\subseteq\ B$; $2^\mathrmo$当$a______$时, 取$x_1=__________$, 则$____________________$, 所以$x_1\in\ A$且x1∉B, 由$1^o$、$2^o$可得结论.
\item 设常数$p\in\\mathbf{R}$, 已知$A=xx<-1$或$x>2, B=\{\ x|4x+p=0\}$, 若$B\subset A$, 则$p$的取值范围是$_____________$.
\item (1) 已知集合$A=\{1\}$, 集合$B=\{x|x^2-2x+a=0\}$, 且$A\subset\ B$, 求实数$a$的取值范围.
(2) 已知集合$S=\{1$, 2\}, 集合$T=\{x|ax^2-3x+2=0\}$, 且$S=T$, 求实数$a$的取值范围.
(3) 已知集合$S=\{1$, 2\}, 集合$T=\{x|ax^2-3x+2=0\}$, 且$S\supseteq\ T$, 求实数$a$的取值范围.
\item 证明:集合$A=\{x|x=6n-1$, n\in\mathbf{Z}\}是$B=\{x|x=3n+2$, n\in\mathbf{Z}\}的真子集.
\item 设常数$a\in \mathbf{R}$, 已知集合$A=x|x^2-1=0$, 集合$B=x|(x-1)(x-a)=0$.
(1) 若$B\subset\ A$, 求$a$值的集合; (2) 若$B$不是$A$的子集, 求$a$值的集合.
(B组$)1$.(1) 已知集合$A\text=\{x|0<x<a\}, B\text=\{x|1<x<2\}$, 若$B\subseteq A$, 则实数$a$的取值范围$_____________$.
(2) 已知集合$A=[-2,5], B=[m+1,2m-1]$, 满足$B\subseteq A$, 则实数$m$的取值范围是$_____________$.
(3) 已知非空集合$P$满足$: \textcircled1$ $P\subseteq$\{1$,2,3,4,5\}$; $\textcircled2$若$a\in P$, 则$6-a\in P$, 符合上述要求的集合$P$的个数是$________$.
\item 已知集合$A=\{1$, 1+d, 1+3d\}, 集合$B=\{1$, q, q^2\}, 其中$d$、$q\in\\mathbf{R}$, 且$d\neq0$. 若$A=B$, 求$q$的值.
\item 已知$A\text=\{x|x\text=a+\sqrt 2b,ab\in \mathbf{N}\}$, 若集合$B\text=\{x|x=\sqrt 2x_1,\$\ {{x}_{1}}\in$A\}$, 证明$B\subset A$. 1.1.4集合的运算$-1
(A$组)$
\item$已知$A={1$, 2, 3, 4}, B={3, 4, 5, 6}求:(1)$A\cap\ B=___________$;  (2)$A\cup\ B=_____________$;
(3)$A\cap\emptyset=____________$; (4)$A\cup\emptyset=_____________$.
\item 已知任一集合$A$, 则
(1)$A\cap\ A\mathrm{\mathrm{=}}____$; (2)$A\cap\emptyset\mathrm{\mathrm{=}}____$;
(3)$A\cup\ A\mathrm{\mathrm{=}}____$; (4)$A\cup\emptyset\mathrm{\mathrm{=}}____$.
\item (1) 已知$A=x|x^2-4=0$, B={x|x^2+2x-8=0}, 则$A\cap\ B=___________, A\cup\ B=___________$.
(2) 已知$A={y|y=x^2-4$, x\in\{\mathbf{R}}}, B={y|y=x^2+2x-8, x\in\{\mathbf{R}}}, 则$A\cap\ B=_______, A\cup\ B=___________$.
(3) 已知$A={(x$, y)$|y=x^2-4$, x\in\{\mathbf{R}}}, B={(x, y)$|y=x^2+x-6$, x\in\{\mathbf{R}}}, 则$A\cap\$B$=____________________, A\cup\$B$=__________________$.
(4) 已知$A={x|$存在$y\in \mathbf{R}$, 使得$y=x+1}, B={x|$存在$y\in \mathbf{R}$, 使得$y=x}$, 则$A\cap\ B= __________$.5.已知$A=\{x|x\le6\}, B=\{x|x<1\}, C=\{x|x>5\}$, 则$A\cap\ B=_______________,  B\cap\ C=______________, A\cap(B\cap C)=_______________, (A\cap B)$\cap\$C=______________, A\cap(B\cup C)=_______________________, (A\cap\ B)$\cup(A\cap\$C)=_______________________, A\cup(B\cap C)=_______________________, (A\cup\ B)$\cap(A\cup\$C)=_______________________$.
\item 用``$\subset$''、`` $\subseteq$''或``$=$''填空.$A\cap B$____$A, A\cap B$____$B\cap A, \varnothing$____$B\cap A$.7.已知集合$A=\{$x| x\le $1$ $\}$, 集合 $B=\{$x| x\ge $a$ $\}$, 且$A\cup B=R$, 则$a$的取值范围为$_____$.
\item 设常数$a\in\\mathbf{R}$. 已知集合$A=\{x|x^2-3x+2=0$, x\in\mathbf{R}\}, 集合$B=\{x|2x^2-x+2a=0$, x\in\mathbf{R}\}. (1) 若$A\cup\ B=B$, 求$a$的值的集合.
(2) 若$A\cap\ B=B$, 求$a$的值的集合.
(B组$)$
\item已知集合$A=(-\infty$, -1)$\cup(6$, +\infty), 集合$B=(5-a$, 5+a). 若$11\in\ B$, 则$A\cup\ B=______$.
\item 已知集合$P=\{\ x|-2\le x\le5\}, Q=xx>k+1$且$x<2k-1$, 若$P\cap\ Q=\emptyset$, 求实数$k$的取值范围.
\item 已知集合$A={(x$, y)$|x+y=0}$, 集合$B=\{(x$, y)$|y=x-2\}$, 集合$C=\{(x$, y)$|y=x+b\}$. 若$(A\cup C)$\cap(B\cup$C)=C$, 求实数$b$.1.1.4集合的运算$-2
(A$组)$
\item$设常数$m\in \mathbf{R}$. 若集合$A=\{1,2,3\}$, 集合$B=\{m^2,3\}$, 且$A\cup$B=\{1$,2,3,m\}$, 则$m$的值是$_________$. $
\item$设常数$a\in \mathbf{R}$. 已知集合$A=\{$x| x\le $1$ $\}$, 集合$B=\{$x| x>a $\}$, 且$A\cap$B=\varnothing, 则$a$的取值范围为$__________$.
\item (1)设全集$U=\{x|x9\}, A=\{1,2,3\}, B=\{3,4,5,6\}$, 则$
\overlineA=________________$; $\overlineB=_______________$; $\overlineA\cap\overlineB=______________$; $
\overlineA\cup\overlineB=______________$; $\overline{A\cup B}=__________________$; $\overline{A\cap B}=___________________$.
(2) 已知$A=x|x<2, \textcircled1$若$U=\\mathbf{R}$, 则$\overlineA=____________$; $
\textcircled2$若$U=x|x\geq0$, 则$\overlineA=____________$. $\textcircled3$若$U=\\mathbf{N}$, 则$\overlineA=____________$.
(3)已知全集$U=\\mathbf{R}, A=\{x|-1<x<2\}$, 则$\overlineA=____________$; $\overline{\overlineA}=____________$; $\overlineA\cap\ U=____________$; $\overlineA\cup\ U=____________$; $\overlineA\cap\ A=____________$; $\overlineA\cup\ A=____________$.
(4) 已知集合$U=\{x|x\geq2\}$, 集合$A=\{y|3\le y<4\}$, 集合$B=\{z|2\le z<5\}$, 则$
\overlineA\cap\ B=____________$; $\overlineB\cup\ A=_____________$.
(5) 设全集$U=\\mathbf{N}, A=xx$为正奇数, $B={x|x$是$5$的倍数}, 则$B\cap\overlineA=_______________4$.(1) 设常数$ab\in \mathbf{R}$, 已知全集$U=\{2$, 4, b\}, $B=\{a+1$, 2\}.  若$\overlineB=\{7\}$, 则$a=___________$;
(2) 设常数$a\in \mathbf{R}$, 已知全集$U=\mathbf{R}$, 集合$A=\{x|-2<x<2\}$, 集合$B=\{x|x>a\}$.  若$A\cap\overlineB=A$, 则$a$的取值范围为$_________$.
(3) 设常数$a\in \mathbf{R}$, 全集$U=\mathbf{R}$. 集合$A=\{$x| x<2 $\}, B=\{$x| x>a $\}$. 若$\overlineA\subseteq\ B$, 则$a$的取值范围为$_________$.
\item 用集合$A$、$B$的运算式表示图中的阴影部分:(1)(2)(3)$6$.不定项选择题:设全集为$U$, 且$M\subseteq\ N$, 则$					()$
(A) M\cup\ N=N   (B) M\cup\ N=M    (C) \overline{N}\subseteq\overline{M}}{\overline{M}\subseteq\overline{N}}
(E) \overline{M}\cup\overline{N}=U   (F) M\cap\overline{N}=\emptyset\blank{50}(G)\ \overline{M}\cap\ N=\emptyset$(B$组)$
\item$已知全集$U=A\cup\ B={x|0\le\ x\le10$, \ $x\in\\mathbf{N}}, A\cap\barB={1$, 3, 5, 7}. 则集合$B=_________$.
\item 若全集$U=\{(x$, y)$|x\in\mathbf{R}$, y\in\mathbf{R}\}, 集合$A=\{(x$, y)$|\dfrac yx=1\}$, 集合$B=\{(x$, y)$|y\neq x\}$, 则$\overline{A\cup B}= _______$.
\item 如图, 已知集合$U$为全集, 分别用集合$A$、B、$C$的运算式表示下列图中的阴影部分.1.2常用逻辑用语$1$.2.1命题$
(A$组)$
\item$判断下列语句是否为命题, 并在相应的括号内填入``是$''$或``否$''$.
(1) 正方形和四边形. $		 ()$\blank{50}(2)正方形是四边形吗? $	()$(3)$\pi>3                 ()$\blank{50}(4)正方形好美!. $	    ()$(5)$2x>4                ()$\blank{50}(6)$968$能被$11$整除. $	()$
\item判断下列命题的真假, 并在相应的括号内填入``真$''$或``假$''$.
(1)$2\sqrt 3>3\sqrt 2$或$1\le1$; $()$\blank{50}(2)$2\sqrt 3>3\sqrt 2$且$1\le1$; $()$(3) 如果$a$、$b$都是奇数, 那么$ab$也是奇数. $			()$(4)$1$是${0$, 1, 2}的真子集. $()$(5)$1$是${0$, 1, 2}的真子集. $()$(6) 若$x<-2$或$x>2$, 则$x^2>1$. $					()$(7) 如果$|$a$|<2$, 那么$a<2$.						( $)$(8) 对任意实数$a,b$, 方程$($a+1$)x+b=0$的解为$x=-\dfrac b{a+1}$.	 $()$(9) 若命题$\alpha$、$\beta$、$\gamma$满足$\alpha$\Rightarrow \beta$, \beta$\Rightarrow \gamma$, \gamma$\Rightarrow \alpha, 则$\alpha$\Leftrightarrow \gamma.	 $()$(10)若关于$x$的方程$ax^2+bx+c=0($a\ne 0$)$的两实数根之积是正数, 则$ac>0$. $	()$(11) 若某个整数不是偶数, 则这个数不能被$4$整除. $		()$(12) 合数一定是偶数. $								()$(13) 所有的偶数都是素数或合数. $()$(14) 所有的偶数都是素数或所有的偶数都是合数. $()$(15) 如果$A\subset$B$,B\supset C$, 那么$A=C$. $					()$(16) 空集是任何集合的真子集. $							()$(17) 若$x\in \mathbf{R}$, 则方程$x^2-x+1=0$不成立. $			()$(18) 若$A\cap$B\ne \varnothing$, B\subset C$, 则$A\cap$C\ne \varnothing.				( $)$(19) 存在一个三角形, 它的任意两边的平方和小于第三边的平方; $()$(20) 任意一个三角形, 存在一组两边的平方和不等于第三边的平方; $()$
\item在下列各题中, 用符号``$\Rightarrow$\Leftarrow \Leftrightarrow''把$\alpha$和$\beta$联系起来:(1)$\alpha:a=0$, \mathrm{\vthicksp}\beta:ab=0.  $\alpha$\_\_\_\_\beta						
(2)$\alpha:x^2=4, \beta:x=2$. .  $\alpha$\_\_\_\_\beta
(3)$\alpha:$实数$x$适合$x^2-5x+6=0$, \mathrm{\vthicksp}\beta:x=2.  $\alpha$\_\_\_\_\beta (4)$\alpha:\sqrt {x^2}=x$, \mathrm{\vthicksp}\beta:x>0.  $\alpha$\_\_\_\_\beta
(5)$\alpha:$实数$x$适合$\dfrac{x-3}{x+1}=-1, \beta:x=1$.  $\alpha$\_\_\_\_\beta
(6)$\alpha:k$除以$4$余1, $\beta:k$除以$2$余1.  $\alpha$\_\_\_\_\beta
(7)$\alpha: \{2\}\subset\ B\subseteq\{2$, 3, 5\}, $\beta:B=\{2$, 5\}.  $\alpha$\_\_\_\_\beta $
(B$组)$
\item$已知命题``非空集合$M$的元素都是集合$P$的元素$''$是假命题, 给出下列命题$: \textcircled1$ $M$中的元素都不是$P$的元素; $\textcircled2$ $M$中有不属于$P$的元素; $\textcircled3$ $M$中有$P$的元素; $\textcircled4$ $M$中的元素不都是$P$的元素.其中真命题是$__________
\item$已知$\alpha: 2\le\ x<4, \beta: 3m-1\le\ x\le-m$, 且$\alpha\Rightarrow\beta$, 求实数$m$的取值范围.
\item 已知$a$是常数, 命题$\alpha :-1<a<3, \beta :$关于$x$的方程$x+a=0(x\in \mathbf{R}\text)$没有正根, 若命题$\alpha$\beta有且只有一个是真命题, 求实数$a$的取值范围.1.2.2充分条件与必要条件$
(A$组)$
\item$下列各题中$P$是$Q$的什么条件? (充分非必要、必要非充分、充要、既非充分又非必要$)$(1) $P$: $x$是$2$的倍数  	$Q$: $x$是$6$的倍数.
(2) $P$: $x$不是$2$的倍数 $Q$: $x$不是$6$的倍数.
(3) $P$: $x\in A$或$x\in B$  	$Q$: $x\in$A\bigcap$B$.
(4)$P:$f$(x)=ax^2+bx+c$的图像过原点.\ $Q: c=0$.
\item (填写``充分非必要、必要非充分、充要、既非充分又非必要$''$之一$)$若$x,y,z$都是实数, 则
(1) ``$xy=0$''是``$x=0$''的$__________$条件.
(2) ``$x\cdot$y=y\cdot$z$''是``$x=z$''的$__________$条件.
(3) ``$\dfrac xy=\dfrac yz$''是``$xz=y^2$''的$__________$条件.
(4) ``$|$x |>| y$|$''是``$x>y>0$''的$__________$条件.
(5) ``$x^2>4$''是``$x>2$'' 的$__________$条件.
(6) ``$x=-3$''是``$x^2+x-6=0$'' 的$__________$条件.
(7) ``$|x+y|<2$''是``$|x|<1$且$|y|<1$'' 的$__________$条件.
(8) ``$|x|<3$''是``$x^2<9$'' 的$__________$条件.
(9) ``$x^2+y^2>0$''是``$x\ne 0$'' 的$__________$条件.
(10) ``$\dfrac{x^2+x+1}{3x+2}<0$''是``$3x+2<0$'' 的$__________$条件.
(11) ``$0<x<3$''是``$|x-1|<2$'' 的$__________$条件.
\item 如果$A$是$B$的必要条件, $C$是$B$的充分条件, $A$是$C$的充分条件, 那么$B$、$C$分别是$A$的$__________$和$__________$条件.4.写出使得``$x>3$''成立的一个充分条件$:__________$和一个必要条件$: _________$.
\item 一次函数$y=kx+b$的图像经过第二、三、四象限的充要条件是$__________$.
\item 关于$x$的方程$ax^2=0$至少有一个实数根的充要条件是$____________$.7.单选题
(1)已知$xy\in \mathbf{R}$, ``$x^2+y^2>0$''是``$x\ne 0$或$y\ne 0$''的$()$
\fourch{充分而不必要条件$			}{$必要而不充分条件$}{$充要条件$					}{$既不充分又不必要条件$}$(2)三个数$a$、$b$、$c$不全为零的充要条件是$			($ )
\fourch{$abc$都不是零}{$abc$中最多一个零}{$abc$中只有一个是零}{$abc$中至少有一个不是零$}9$.证明: $x_1>2$且$x_2>2$是$x_1+x_2>4$且$x_1\cdot x_2>4$的充分非必要条件.
(B组$)$
\item单选题有限集合$S$中元素的个数记作$\mathrm{card}(S)$, 设$A,B$都是有限集合, 给出下列命题$:
\textcircled1$ $A\cap$B=\varnothing的充要条件是$\mathrm{card}\mathrm{card}$; $
\textcircled2$ $A\subseteq B$的必要不充分条件是$\mathrm{card}$; $
\textcircled3$ $A$不是$B$的子集的充分不必要条件是$\textcard($A )$>\textcard($ B$)$; $
\textcircled4$ $A=B$的充要条件是$\mathrm{card}$.其中真命题的个数是$									()$
\fourch{$0$}{$1$}{$2$}{$3}2$.设$\alpha,\beta$是方程$x^2-ax+b=0$的两个实数根. 试分析$a>2$且$b>1$是``两个实数根$\alpha,\beta$均大于$1$''的什么条件? 并证明你的结论.
\item 设$x,y\in \mathbf{R}$, 求证: $|x+y|=|x|+|y|$成立的充要条件是$xy\ge 0$.1.2.3反证法$-1
(A$组)$1$、已知下列字母均为常实数, 写出下列命题的否定形式;
(1)$x>0$; $______________________________$(2)$1>x>0$; $_______________________________$(3)$x>0$且$y\le 1$; $_______________________________$(4)$x>0$或$x\le -2$; $__________________________$(5)$x\neq\ y$或$y\neq\ z$; $__________________________$(6)$a,b,c,d$中至多有$2$个0; $___________________________________$(7)$a,b,c,d$中至少有$2$个1; $___________________________________$(8)$a,b,c,d$都大于$1$; $___________________________________$(9)$a,b,c,d$不都大于$1$; $___________________________________$(10)$a,b,c,d$都不大于$1$; $___________________________________
\item$在横线上写出下列命题的否定形式, 并判断命题真假, 在相应的括号中填入``真$''$或``假$''$.
(1)$\pi$是无理数; $()$\blank{50}(\blank{50}$)$(2)$2+1=4$; $()$\blank{50}(\blank{50}$)$(3)任何实数是正数或负数; $()$\blank{50}(\blank{50}$)$(4)任何实数是正数或任何实数是负数; $()$\blank{50}(\blank{50}$)$(5)对一切实数$x, x^3+1=0$; $()$\blank{50}(\blank{50}$)$(6)存在实数$x, x^3+1=0$; $()$\blank{50}(\blank{50}$)$(7)任意实数$k$, 都有关于$x$的方程$x^2+x+k=0$有实数根; $()$\blank{50}(\blank{50}$)$(8)任何三角形中至多有一个钝角; $()$\blank{50}(\blank{50}$)$(9)若$a>1$, b>1, 则$ab>1$; $()$\blank{50}(\blank{50}$)$(10)能被$2$整除的整数是质数. $()$\blank{50}\bracket{20}$(B$组)$
\item$写出下列陈述的否定形式.
(1)在平面上, 过定点$P$有且只有一条直线垂直于给定直线$l$\blank{50}
(2)任意两个有理数之间存在一个无理数; $______________________________________________________________$(3)存在实数$a,$使得关于$x$的不等式$x^2+(a-2)$x+a-1\ge$0$至少有一个正数解$______________________________________________________________$(4)存在实数$a,$关于$x$的不等式$x^2+(a-2)$x+a-1\ge$0$恒成立$______________________________________________________________$(5)存在实数$a,$关于$x$的不等式$x^2+(a-2)$x+a-1\ge$0$有解$______________________________________________________________1$.2.3反证法$-2
(A$组)$1$.单选题:已知甲$\Rightarrow$乙, 下列说法一定正确的是$________$(1) 甲不成立, 可推出乙成立(2) 甲不成立, 可推出乙不成立
(3) 乙不成立, 可推出甲成立(4) 乙不成立, 可推出甲不成立$2$.``a\neq1且$b\neq2''$是``a+b\neq3''的$	_______	$条件$	
($填: ``充分非必要、必要非充分、充要、非充分非必要$''$之一$)$
\item证明:若$x+2y+z>0$, 则$x,y,z$中至少有一个大于$0$.
\item 证明:对于三个实数$a$, b, c, 若$a\neq\ c$, 则$a\neq\ b$或$b\neq\ c$.
\item 证明: $\sqrt 3$是无理数.
(B组$)
\item$ ``若$x\ne 3$或$x\ne 4$'' 是``$x^2-7x+12\ne 0$''的$______$条件.
(填: ``充分非必要、必要非充分、充要、非充分非必要$''$之一$)2$.证明:若$x^2\ne y^2$, 则$x\ne y$或$x\ne -y$.3.若$a^3+b^3=2$, 证明: $a+b\le 2$.
\fi

\end{enumerate}

\end{document}