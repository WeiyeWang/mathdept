\documentclass[10pt,a4paper]{article}
\usepackage[UTF8,fontset = windows]{ctex}
\setCJKmainfont[BoldFont=黑体,ItalicFont=楷体]{华文中宋}
\usepackage{amssymb,amsmath,amsfonts,amsthm,mathrsfs,dsfont,graphicx}
\usepackage{ifthen,indentfirst,enumerate,color,titletoc}
\usepackage{tikz}
\usepackage{makecell}
\usepackage{longtable}

\usetikzlibrary{arrows,calc,intersections,patterns}
\usepackage[bf,small,indentafter,pagestyles]{titlesec}
\usepackage[top=1in, bottom=1in,left=0.8in,right=0.8in]{geometry}
\renewcommand{\baselinestretch}{1.65}
\newtheorem{defi}{定义~}
\newtheorem{eg}{例~}
\newtheorem{ex}{~}
\newtheorem{rem}{注~}
\newtheorem{thm}{定理~}
\newtheorem{coro}{推论~}
\newtheorem{axiom}{公理~}
\newtheorem{prop}{性质~}
\newcommand{\blank}[1]{\underline{\hbox to #1pt{}}}
\newcommand{\bracket}[1]{(\hbox to #1pt{})}
\newcommand{\onech}[4]{\par\begin{tabular}{p{.9\textwidth}}
A.~#1\\
B.~#2\\
C.~#3\\
D.~#4
\end{tabular}}
\newcommand{\twoch}[4]{\par\begin{tabular}{p{.46\textwidth}p{.46\textwidth}}
A.~#1& B.~#2\\
C.~#3& D.~#4
\end{tabular}}
\newcommand{\vartwoch}[4]{\par\begin{tabular}{p{.46\textwidth}p{.46\textwidth}}
(1)~#1& (2)~#2\\
(3)~#3& (4)~#4
\end{tabular}}
\newcommand{\fourch}[4]{\par\begin{tabular}{p{.23\textwidth}p{.23\textwidth}p{.23\textwidth}p{.23\textwidth}}
A.~#1 &B.~#2& C.~#3& D.~#4
\end{tabular}}
\newcommand{\varfourch}[4]{\par\begin{tabular}{p{.23\textwidth}p{.23\textwidth}p{.23\textwidth}p{.23\textwidth}}
(1)~#1 &(2)~#2& (3)~#3& (4)~#4
\end{tabular}}
\begin{document}
\begin{enumerate}[1.]

\item 写出集合$\{1,2\}$的所有子集.
\item 已知集合$A=\{x|1 \le x<3,\ x\in \mathbf{R}\}$, $B=\{x|x>2,\ x\in \mathbf{R}\}$. 求$A\cap B$, $A\cup B$.
\item 已知集合$U =\{x|x\text{取不大于}30\text{的质数}\}$, $A$, $B$是$U$的两个子集, 且满足$A\cap \complement_UB=\{5,13,23\}$, $\complement_A\cap B=\{11,19,29\}$, $\complement_UA\cap \complement_UB=\{3,7\}$, 求$A$, $B$.
\item 已知集合$A=\{x|x^2- ax+a^2-19=0\}$, $B=\{x|x^2-5x+6=0\}$, $C=\{ x|x^2+2x-8=0\}$满足$A\cap B\ne \varnothing$, $A\cap C=\varnothing$, 求实数$a$的值.
\item 已知集合$A=\{x|x^2-5x+4\le 0\}$与$B=\{x|x^2-2ax+a+2\le 0,\ a\in \mathbf{R}\}$满足$B\subseteq A$, 求$a$的取值范围.
\item 已知集合$A=\{x|x^2 +(\rho +2)x+1=0, \ x\in \mathbf{R}\}$, 且$A\cap \mathbf{R}^+=\varnothing$, 求实数$\rho$的取值范围.
\item 在``\textcircled{1} 难解的题目, \textcircled{2} 方程$x^2+1=0$在实数集内的解, \textcircled{3} 直角坐标平面内第四象限的一些点, \textcircled{4} 很多多项式''中, 能够组成集合的是\bracket{20}.
\fourch{\textcircled{2}}{\textcircled{1}\textcircled{3}}{\textcircled{2}\textcircled{4}}{\textcircled{1}\textcircled{2}\textcircled{4}}
\item 集合$M=\{(x,y)|xy\ge 0,\ x\in \mathbf{R},\ y\in \mathbf{R}\}$是指\bracket{20}.
\twoch{第一象限内的点集}{第三象限内的点集}{在第一、三象限内的点集}{不在第二、四象限内的点集}
\item 下列四个关系中, 正确的是\bracket{20}.
\fourch{$\varnothing \in \{a\}$}{$a\notin \{a\}$}{$\{a\}\in \{a,b\}$}{$a\in \{a,b\}$}
\item 方程组$\begin{cases} 2x+y=0, \\ x-y+3=0 \end{cases}$的解集是\bracket{20}.
\fourch{$\{-1,2\}$}{$(-1,2)$}{$\{(-1,2)\}$}{$\{(x,y)|x=-1, \ y=2\}$}
\item 下列各题中的$M$与$P$表示同一个集合的是\bracket{20}.
\onech{$M=\{(1,-3)\}$, $P=\{(-3,1)\}$}{$M=\varnothing$, $P=\{0\}$}{$M=\{y|y=x^2+1, \ x\in \mathbf{R}\}$, $P=\{(x,y)|y=x^2+1, \ x\in \mathbf{R}\}$}{$M=\{y|y=x^2+1,\ x\in \mathbf{R}\},P\{t|t=(y-1)^2+1, \ y\in \mathbf{R}\}$}
\end{enumerate}
\end{document}