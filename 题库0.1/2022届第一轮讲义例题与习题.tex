\documentclass[10pt,a4paper]{article}
\usepackage[UTF8,fontset = windows]{ctex}
\setCJKmainfont[BoldFont=黑体,ItalicFont=楷体]{等线}
\usepackage{amssymb,amsmath,amsfonts,amsthm,mathrsfs,dsfont,graphicx}
\usepackage{ifthen,indentfirst,enumerate,color,titletoc}
\usepackage{tikz}
\usetikzlibrary{arrows,calc,intersections}
\usepackage[bf,small,indentafter,pagestyles]{titlesec}
\usepackage[top=1in, bottom=1in,left=0.8in,right=0.8in]{geometry}
\renewcommand{\baselinestretch}{1.65}
\newtheorem{defi}{定义~}
\newtheorem{eg}{例~}
\newtheorem{ex}{~}
\newtheorem{rem}{注~}
\newtheorem{thm}{定理~}
\newtheorem{coro}{推论~}
\newtheorem{axiom}{公理~}
\newtheorem{prop}{性质~}
\newcommand{\blank}[1]{\underline{\hbox to #1pt{}}}
\newcommand{\bracket}[1]{(\hbox to #1pt{})}
\newcommand{\onech}[4]{\par\begin{tabular}{p{.9\textwidth}}
A.~#1\
B.~#2\
C.~#3\
D.~#4
\end{tabular}}
\newcommand{\twoch}[4]{\par\begin{tabular}{p{.46\textwidth}p{.46\textwidth}}
A.~#1& B.~#2\
C.~#3& D.~#4
\end{tabular}}
\newcommand{\vartwoch}[4]{\par\begin{tabular}{p{.46\textwidth}p{.46\textwidth}}
(1)~#1& (2)~#2\
(3)~#3& (4)~#4
\end{tabular}}
\newcommand{\fourch}[4]{\par\begin{tabular}{p{.23\textwidth}p{.23\textwidth}p{.23\textwidth}p{.23\textwidth}}
A.~#1 &B.~#2& C.~#3& D.~#4
\end{tabular}}
\newcommand{\varfourch}[4]{\par\begin{tabular}{p{.23\textwidth}p{.23\textwidth}p{.23\textwidth}p{.23\textwidth}}
(1)~#1 &(2)~#2& (3)~#3& (4)~#4
\end{tabular}}
\begin{document}
\begin{enumerate}[1.]
% 第一讲 8+3+7
\item 用适当符号($\in$,$\notin$,$=$,$\subsetneq$)填空:$\pi$\blank{10}$\mathbf{Q}$; $\{x|x=2k+1, \ k\in \mathbf{Z}\}$\blank{10}$\{x|x=2k-1,k\in \mathbf{Z}\}$; $\{3.14\}$\blank{10}$\mathbf{Q}$; $\{y|y=x^2\}$\blank{10}$\{x|y=x^2\}$.  
\item 已知$P=\{y=x^2+1\}$, $Q=\{y|y=x^2+1, \ x\in \mathbf{R}\}$, $E=\{x|y=x^2+1, \  x\in \mathbf{R}\}$, $F=\{(x,y)|y=x^2+1, \ x\in \mathbf{R}\}$, $G=\{x|x\ge 1\}$, $H=\{x|x^2+1=0, \ x\in \mathbf{R}\}$, 则各集合间关系正确的有\blank{50}. (答案可能不唯一)\\
(A) $P=F$   (B) $Q=E$   (C) $E=F$   (D) $Q\subseteq G$  (E) $H\subsetneq P$
\item 设全集是实数集$\mathbf{R}$, $M=\{x|-2 \le x\le 2\}$, $N=\{x|x<1\}$, 则$\complement_U M\cap N=$\blank{50}.
\item 设$A=\{x|-4<x<4, \ x\in \mathbf{R}\}$, $B=(-\infty,1]\cup [3,+\infty)$, 则$\{x|x\in A, \ x\notin A\cap B  \}$=\blank{50}.
\item 设$A=\{x|x=\sqrt k, \ k\in \mathbf{N}\}$,$B=\{x|x\le 3,\ x\in \mathbf{Q}\}$, 则$A\cap B=$\blank{50}.
\item 设全集$U=\{2,3,a^2+2a-3\}$, 集合$A=\{|2a-1|,2\}$, $\complement_U A=\{5\}$, 则实数$a=$\blank{50}.
\item (1) 设$M=\{y|y=x^2, x\in \mathbf{R}\}$, $N=\{x|x=t,\ t\in \mathbf{R}\}$, 则$M\cap N=$\blank{50}.\\
(2) 设$M=\{(x,y)|y=x^2,\ x\in \mathbf{R}\}$, $N=\{(t,x)|x=t,\ t\in \mathbf{R}\}$, 则$M\cap N=$\blank{50}.
\item 设全集$U=\{1,2,3,4\}$, $\complement_U A\cap B=\{3\}$, $A\cap \complement_U B=\{2\}$, $\complement_U A\cup \complement_U B=\{2,3,4\}$, 则$\complement_U A\cap \complement_U B$=\blank{50}.
\item 集合$C=\{x|x=\dfrac k2\pm \dfrac14, \ k\in \mathbf{Z}\},D=\{x|x=\dfrac k4,\ k\in \mathbf{Z}\}$, 试判断$C$与$D$的关系, 并证明.
\item 集合$A=\{x|x^2+4x=0\}$, $B=\{x|x^2+2(a+1)x+a^2-1=0,\ x\in \mathbf{R}\}$.\\
(1) 若$A\cap B=A$, 求实数$a$的取值范围;\\
(2) 若$A\cup B=A$, 求实数$a$的取值范围.
\item 若集合$A=[2,3]$, 集合$B=[a,2a+1]$.\\
(1) 若$A\subsetneq B$, 求实数$a$的取值范围;\\
(2) 若$A\cap B\ne \varnothing$, 求实数$a$的取值范围.
\item 设全集$U=\mathbf{R}$, 集合$A=\{x|f(x)=0\}$, $B=\{x|g(x)=0\}$, $C=\{x|h(x)=0, \ x\in \mathbf{R}\}$, 则方程$\dfrac{f^2(x)+g^2(x)}{h(x)}=0$的解集是\blank{50}(用$U,A,B,C$表示).
\item (1) 已知集合$A=\{y|y=x^2, \ x\in \mathbf{R}\}, B=\{y|y=4-x^2, \ x\in \mathbf{R}\}$, 则$A\cap B=$\blank{50}.\\
(2) 已知集合$A=\{(x,y)|y={x^2},\ x\in \mathbf{R}\}$, $B=\{(x,y)|y=4-x^2, \ x\in \mathbf{R}\}$, 则$A\cap B=$\blank{50}.
\item 设$m\in \mathbf{R}$, 已知$A=\{x|x^2-3x+2=0\}$, $B=\{x|mx+1=0\}$, 且$B\subsetneq A$, 则$m=$\blank{50}.
\item (1) 集合$A$满足$\{1\}\subseteq A \subsetneq \{1,2,3,4\}$, 则满足条件的集合$A$有\blank{50}个.
(2) 若$A\cup B=\{1,2\}$, 将满足条件的集合$A$, $B$写成有序集合对$(A,B)$, 则有序集合对$(A,B)$有\blank{50}个.
\item 已知$A=\{x|x^2-3x+2=0\}$, $B=\{x|x^2-ax+a=0, \ x\in \mathbf{R}\}$, 若$B\subsetneq A$, 求满足题意的实数$a$.
\item 设集合$A=\{x|x^2+px+1=0,\ x\in \mathbf{R}\}$, 若$A\cap \mathbf{R}^+=\varnothing$. 求实数p的取值范围.
\item 设函数$f(x)=\lg (\dfrac2{x+1}-1)$的定义域为集合$A$, 函数$g(x)=\sqrt{1-|x+a|}$的定义域为集合$B$.\\
(1) 当$a=1$时, 求集合$B$.\\
(2) 问: $a\ge 2$是$A\cap B=\varnothing$的什么条件(在``充分非必要条件、必要非充分条件、充要条件、既非充分也非必要条件''中选一)?并证明你的结论.

\end{enumerate}

\end{document}