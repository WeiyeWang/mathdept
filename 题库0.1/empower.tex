\documentclass[12pt,a4paper]{article}
\usepackage[UTF8,fontset = windows]{ctex}
\setCJKmainfont[BoldFont=黑体,ItalicFont=楷体]{等线}
\usepackage{amssymb,amsmath,amsfonts,amsthm,mathrsfs,dsfont,graphicx}
\usepackage{ifthen,indentfirst,enumerate,color,titletoc}
\usepackage{tikz}
\usetikzlibrary{arrows}
\usepackage[bf,small,indentafter,pagestyles]{titlesec}
\usepackage[top=1in, bottom=1in,left=0.8in,right=0.8in]{geometry}
\renewcommand{\baselinestretch}{1.65}
\newtheorem{defi}{定义~}
\newtheorem{eg}{例~}
\newtheorem{ex}{~}
\newtheorem{rem}{注~}
\newtheorem{thm}{定理~}
\newtheorem{coro}{推论~}
\newtheorem{axiom}{公理~}
\newtheorem{prop}{性质~}

\newcommand{\blank}[1]{\underline{\hbox to #1pt{}}}
\newcommand{\bracket}[1]{(\hbox to #1pt{})}

\begin{document}

\begin{enumerate}[1.]

%赋能1

\item  若``$a>b$'', 则``$a^3>b^3$''是\blank{50}命题(填: 真、假).
\item  已知$A=(-\infty ,0]$, $B=(a,+\infty )$, 若$A\cup B=\mathbf{R}$, 则$a$的取值范围是\blank{50}.
\item  $z+2\bar{z}=9+4\mathrm{i}$($\mathrm{i}$为虚数单位), 则$|z|=$\blank{50}.
\item  若$\triangle ABC$中, $a+b=4$, $\angle C=30^\circ$, 则$\triangle ABC$面积的最大值是\blank{50}.
\item  若函数$f(x)=\log_2\dfrac{x-a}{x+1}$的反函数的图像过点$(-2,3)$, 则$a=$\blank{50}.
\item  若半径为2的球$O$表面上一点$A$作球$O$的截面, 若$OA$与该截面所成的角是$60^\circ$, 则该
截面的面积是\blank{50}.
\item  抛掷一枚均匀的骰子(刻有1、2、3、4、5、6)三次, 得到的数字依次记作$a$、$b$、$c$, 则$a+b\mathrm{i}$($\mathrm{i}$为虚数单位)是方程$x^2-2x+c=0$的根的概率是\blank{50}.
\item  设常数$a>0$, $(x+\dfrac{a}{\sqrt{x}})^9$展开式中$x^6$的系数为$4$, 则$\displaystyle\lim_{n\to \infty}(a+a^2+\cdots+a^n)=$\blank{50}.
\item  已知直线$l$经过点$(-\sqrt{5},0)$且方向向量为$(2,-1)$, 则原点$O$到直线$l$的距离为\blank{50}.
\item  若双曲线的一条渐近线为$x+2y=0$, 且双曲线与抛物线$y=x^2$的准线仅有一个公共点, 则此双曲线的标准方程为\blank{50}.

%赋能2

\end{enumerate}

\end{document}