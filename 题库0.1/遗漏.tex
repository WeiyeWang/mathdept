\documentclass[10pt,a4paper]{article}
\usepackage[UTF8,fontset = windows]{ctex}
\setCJKmainfont[BoldFont=黑体,ItalicFont=楷体]{华文中宋}
\usepackage{amssymb,amsmath,amsfonts,amsthm,mathrsfs,dsfont,graphicx}
\usepackage{ifthen,indentfirst,enumerate,color,titletoc}
\usepackage{tikz}
\usepackage{multicol}
\usepackage{makecell}
\usepackage{longtable}
\usepackage{diagbox}
\usetikzlibrary{arrows,calc,intersections,patterns,decorations.pathreplacing,3d,angles,quotes,positioning}
\usepackage[bf,small,indentafter,pagestyles]{titlesec}
\usepackage[top=1in, bottom=1in,left=0.8in,right=0.8in]{geometry}
\renewcommand{\baselinestretch}{1.65}
\newtheorem{defi}{定义~}
\newtheorem{eg}{例~}
\newtheorem{ex}{~}
\newtheorem{rem}{注~}
\newtheorem{thm}{定理~}
\newtheorem{coro}{推论~}
\newtheorem{axiom}{公理~}
\newtheorem{prop}{性质~}
\newcommand{\blank}[1]{\underline{\hbox to #1pt{}}}
\newcommand{\bracket}[1]{(\hbox to #1pt{})}
\newcommand{\onech}[4]{\par\begin{tabular}{p{.9\textwidth}}
A.~#1\\
B.~#2\\
C.~#3\\
D.~#4
\end{tabular}}
\newcommand{\twoch}[4]{\par\begin{tabular}{p{.46\textwidth}p{.46\textwidth}}
A.~#1& B.~#2\\
C.~#3& D.~#4
\end{tabular}}
\newcommand{\vartwoch}[4]{\par\begin{tabular}{p{.46\textwidth}p{.46\textwidth}}
(1)~#1& (2)~#2\\
(3)~#3& (4)~#4
\end{tabular}}
\newcommand{\fourch}[4]{\par\begin{tabular}{p{.23\textwidth}p{.23\textwidth}p{.23\textwidth}p{.23\textwidth}}
A.~#1 &B.~#2& C.~#3& D.~#4
\end{tabular}}
\newcommand{\varfourch}[4]{\par\begin{tabular}{p{.23\textwidth}p{.23\textwidth}p{.23\textwidth}p{.23\textwidth}}
(1)~#1 &(2)~#2& (3)~#3& (4)~#4
\end{tabular}}
\begin{document}

\begin{enumerate}[1.]

\item 若$x,y,z$都是实数, 则:(填写``充分非必要、必要非充分、充要、既非充分又非必要''之一)\\
(1) ``$xy=0$''是``$x=0$''的\blank{50}条件;\\
(2) ``$x\cdot y=y\cdot z$''是``$x=z$''的\blank{50}条件;\\
(3) ``$\dfrac xy=\dfrac yz$''是``$xz=y^2$''的\blank{50}条件;\\
(4) ``$|x |>| y|$''是``$x>y>0$''的\blank{50}条件;\\
(5) ``$x^2>4$''是``$x>2$'' 的\blank{50}条件;\\
(6) ``$x=-3$''是``$x^2+x-6=0$'' 的\blank{50}条件;\\
(7) ``$|x+y|<2$''是``$|x|<1$且$|y|<1$'' 的\blank{50}条件;\\
(8) ``$|x|<3$''是``$x^2<9$'' 的\blank{50}条件;\\
(9) ``$x^2+y^2>0$''是``$x\ne 0$'' 的\blank{50}条件;\\
(10) ``$\dfrac{x^2+x+1}{3x+2}<0$''是``$3x+2<0$'' 的\blank{50}条件;\\
(11) ``$0<x<3$''是``$|x-1|<2$'' 的\blank{50}条件.
出处在这里 2025届高一校本作业必修第一章


\item 设$ab>0$, 且$\dfrac ca>\dfrac db$, 则下列各式中, 恒成立的是\bracket{20}.
\fourch{$bc<ad$}{$bc>ad$}{$\dfrac ac>\dfrac bd$}{$\dfrac ac<\dfrac bd$}
出处在这里 代数精编第二章不等式

\item 若集合$M=\{x |x\le 6\},a=\sqrt 5$, 则下面结论正确的是\bracket{20}.
\fourch{$\{ a\}\subset M$}{$a\subset M$}{$\{ a\}\notin M$}{$a\notin M$}
出处在这里 代数精编第一章集合与命题

\item 已知集合$M=\{y |y=x^2-2x-1,x\in \mathbf{R}\},P=\{x |-2\le x\le 4,x\in \mathbf{R}\}$, 则$M$与$P$之间的关系是\bracket{20}.
\fourch{$M=P$}{$M\subset P$}{$M\supset P$}{$M\not\subset P$且$M\not\supset P$}
出处在这里 代数精编第一章集合与命题

\end{enumerate}

\end{document}