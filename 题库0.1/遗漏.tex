\documentclass[10pt,a4paper]{article}
\usepackage[UTF8,fontset = windows]{ctex}
\setCJKmainfont[BoldFont=黑体,ItalicFont=楷体]{华文中宋}
\usepackage{amssymb,amsmath,amsfonts,amsthm,mathrsfs,dsfont,graphicx}
\usepackage{ifthen,indentfirst,enumerate,color,titletoc}
\usepackage{tikz}
\usepackage{multicol}
\usepackage{makecell}
\usepackage{longtable}
\usepackage{diagbox}
\usetikzlibrary{arrows,calc,intersections,patterns,decorations.pathreplacing,3d,angles,quotes,positioning}
\usepackage[bf,small,indentafter,pagestyles]{titlesec}
\usepackage[top=1in, bottom=1in,left=0.8in,right=0.8in]{geometry}
\renewcommand{\baselinestretch}{1.65}
\newtheorem{defi}{定义~}
\newtheorem{eg}{例~}
\newtheorem{ex}{~}
\newtheorem{rem}{注~}
\newtheorem{thm}{定理~}
\newtheorem{coro}{推论~}
\newtheorem{axiom}{公理~}
\newtheorem{prop}{性质~}
\newcommand{\blank}[1]{\underline{\hbox to #1pt{}}}
\newcommand{\bracket}[1]{(\hbox to #1pt{})}
\newcommand{\onech}[4]{\par\begin{tabular}{p{.9\textwidth}}
A.~#1\\
B.~#2\\
C.~#3\\
D.~#4
\end{tabular}}
\newcommand{\twoch}[4]{\par\begin{tabular}{p{.46\textwidth}p{.46\textwidth}}
A.~#1& B.~#2\\
C.~#3& D.~#4
\end{tabular}}
\newcommand{\vartwoch}[4]{\par\begin{tabular}{p{.46\textwidth}p{.46\textwidth}}
(1)~#1& (2)~#2\\
(3)~#3& (4)~#4
\end{tabular}}
\newcommand{\fourch}[4]{\par\begin{tabular}{p{.23\textwidth}p{.23\textwidth}p{.23\textwidth}p{.23\textwidth}}
A.~#1 &B.~#2& C.~#3& D.~#4
\end{tabular}}
\newcommand{\varfourch}[4]{\par\begin{tabular}{p{.23\textwidth}p{.23\textwidth}p{.23\textwidth}p{.23\textwidth}}
(1)~#1 &(2)~#2& (3)~#3& (4)~#4
\end{tabular}}
\begin{document}

\begin{enumerate}[1.]

\item 某油库的设计容量为$30$万吨, 年初储量为$10$万吨, 从年初起计划每月先购进石油$m$万吨, 以满足区域内和区域外的需求, 若区域内每月用石油$1$万吨, 区域外前$x$个月的需求量$y$(万吨)与$x$的函数关系为$y=\sqrt {2px}$($p>0$, $1\le x\le 16$, $x\in \mathbf{N}$), 并且前$4$个月, 区域外的需求量为$20$万吨.\\
(1) 试写出第$x$个月石油调出后, 油库内储油量$M$(万吨)与$x$的函数关系式;\\
(2) 要使$16$个月内每月按计划购进石油之后, 油库总能满足区域内和区域外的需求, 且每月石油调出后, 油库的石油剩余量不超过油库的容量, 求$m$的取值范围.

\item 在不考虑空气阻力的情况下, 某型号火箭的最大速度$v$(单位: $m/s$)和燃料的质量$M$(单位: $kg$), 火箭(除燃料外)的质量$m$(单位: $kg$)满足$\mathrm{e}^v=(1+\dfrac Mm)^{2000}$($\mathrm{e}$为自然对数的底).\\
(1) 当燃料质量$M$为火箭(除燃料外)质量$m$的两倍时, 求火箭的最大速度(单位: $\text{m}/\text{s}$, 结果精确到$0.1$);\\
(2) 当燃料质量$M$为火箭(除燃料外)质量$m$的多少倍时, 火箭的最大速度可以达到$8000\text{m}/\text{s}$(结果精确到$0.1$).

\item 某网店有$3$万件商品, 计划在元旦旺季售出商品$x$万件. 经市场调查测算, 花费$t$万元进行促销后, 商品的剩余量$3-x$(万件)与促销费$t$(万元)之间的关系为$3-x=\dfrac k{t+1}$(其中$k$为常数), 如果不搞促销活动, 只能售出$1$万件商品.\\
(1) 要使促销后商品的剩余量不大于$0.1$万件, 促销费至少为多少万元?\\
(2) 已知商品的进价为$32$元/件, 另有固定成本$3$万元. 定义每件售出商品的平均成本为$32+\dfrac 3x$元. 若将商品售价定为: ``每件售出商品平均成本的$1.5$倍''与``每件售出商品平均促销费的一半''之和, 则当促销费$t$为多少时, 该网店售出商品的总利润最大? 此时商品的剩余量为多少?

\item 勤俭节约是中华民族的传统美德. 为避免舌尖上的浪费, 各地各部门采取了精准供应的措施.某学校食堂经调查分析预测, 从年初开始的前$n$($n=1,2,3,\cdots ,12$)个月对某种食材的需求总量$S_n$(公斤)近似地满足$S_n=\begin{cases}   635n, & 1\le n\le 6,\\ -6n^2+774n-618, & 7\le n\le 12.  \end{cases}$ 为保证全年每一个月该食材都够用, 食堂前$n$个月的进货总量须不低于前$n$个月的需求总量.\\
(1) 如果每月初进货$646$公斤, 那么前$7$个月每月该食材是否都够用?\\
(2) 若每月初等量进货$p$公斤, 为保证全年每一个月该食材都够用, 求$p$的最小值.

\item 提高隧道的车辆通行能力可改善附近路段高峰期间的交通状况. 在一般情况下, 隧道内的车流速度$v$(单位: 千米/小时)和车流密度$x$(单位: 辆/千米)满足关系式:
$v=\begin{cases} 50, & 0<x\le 20, \\ 60-\dfrac k{140-x}, & 20<x\le 120 \end{cases}$ ($k\in \mathbf{R}$).\\
研究表明: 当隧道内的车流密度达到$120$辆/千米时造成堵塞, 此时车流速度是$0$千米/小时.\\
(1) 若车流速度$v$不小于$40$千米/小时, 求车流密度$x$的取值范围;\\
(2) 隧道内的车流量$y$(单位时间内通过隧道的车辆数, 单位: 辆/小时)满足$y=x\cdot v$, 求隧道内车流量的最大值(精确到$1$辆/小时), 并指出当车流量最大时的车流密度(精确到$1$辆/千米).

\end{enumerate}




\end{document}