\documentclass[10pt,a4paper]{article}
\usepackage[UTF8,fontset = windows]{ctex}
\setCJKmainfont[BoldFont=黑体,ItalicFont=楷体]{华文中宋}
\usepackage{amssymb,amsmath,amsfonts,amsthm,mathrsfs,dsfont,graphicx}
\usepackage{ifthen,indentfirst,enumerate,color,titletoc}
\usepackage{tikz}
\usepackage{multicol}
\usepackage{makecell}
\usepackage{longtable}
\usetikzlibrary{arrows,calc,intersections,patterns,decorations.pathreplacing,3d,angles,quotes}
\usepackage[bf,small,indentafter,pagestyles]{titlesec}
\usepackage[top=1in, bottom=1in,left=0.8in,right=0.8in]{geometry}
\renewcommand{\baselinestretch}{1.65}
\newtheorem{defi}{定义~}
\newtheorem{eg}{例~}
\newtheorem{ex}{~}
\newtheorem{rem}{注~}
\newtheorem{thm}{定理~}
\newtheorem{coro}{推论~}
\newtheorem{axiom}{公理~}
\newtheorem{prop}{性质~}
\newcommand{\blank}[1]{\underline{\hbox to #1pt{}}}
\newcommand{\bracket}[1]{(\hbox to #1pt{})}
\newcommand{\onech}[4]{\par\begin{tabular}{p{.9\textwidth}}
A.~#1\\
B.~#2\\
C.~#3\\
D.~#4
\end{tabular}}
\newcommand{\twoch}[4]{\par\begin{tabular}{p{.46\textwidth}p{.46\textwidth}}
A.~#1& B.~#2\\
C.~#3& D.~#4
\end{tabular}}
\newcommand{\vartwoch}[4]{\par\begin{tabular}{p{.46\textwidth}p{.46\textwidth}}
(1)~#1& (2)~#2\\
(3)~#3& (4)~#4
\end{tabular}}
\newcommand{\fourch}[4]{\par\begin{tabular}{p{.23\textwidth}p{.23\textwidth}p{.23\textwidth}p{.23\textwidth}}
A.~#1 &B.~#2& C.~#3& D.~#4
\end{tabular}}
\newcommand{\varfourch}[4]{\par\begin{tabular}{p{.23\textwidth}p{.23\textwidth}p{.23\textwidth}p{.23\textwidth}}
(1)~#1 &(2)~#2& (3)~#3& (4)~#4
\end{tabular}}
\begin{document}

\begin{enumerate}[1.]
\item 自由落体运动中, 物体下落的距离$d$(单位 :$\text{m}$)与时间$t$(单位: $\text{s}$)近似满足函数关系$d=5t^2$.\\
(1) 求物体在$[2, 4]$时间段内的平均速度;\\
(2) 求物体在$t=3$时的瞬时速度;\\
(3) 求物体在$t=a$($a>0$)时的瞬时速度.
\item 将石子投入水中, 水面产生的圆形波纹不断扩散.\\
(1) 当半径$r$从$a$增加到$a+h$($h>0$)时, 求圆周长相对于半径的平均变化率;\\
(2) 当半径$r=a$时, 求圆周长相对于半径的瞬时变化率. 
\item 已知$f(x)=3x^2$, 分别求曲线$y=f(x)$在点$P(-1,3)$和点$Q(1,3)$处的切线方程.
\item 借助函数图像, 判断下列导数的正负(可利用信息技术工具):\\
(1) $f'(\dfrac\pi 4)$, 其中$f(x)=\sin x$;\\
(2) $f'(0)$, 其中$f(x)=(\dfrac 12)^x$. 
\item 用导数的定义求函数 $y=x^2+3x-5$的导数.
\item 用公式求下列函数$y=f(x)$的导数, 其中:\\
(1) $f(x)=\sqrt[3]{x^2}$;\\
(2) $f(x)=x^\pi$.
\item 求余弦函数$y=\cos x$在$x=\dfrac \pi2$处的导数.
\item 证明函数$y=\ln x$与$y=\mathrm{e}^x$没有驻点. 
\item 求下列函数$y=f(x)$的导数, 其中:\\
(1) $f(x)=3\mathrm{e}^x-x^{\mathrm{e}}+\mathrm{e}$;\\
(2) $f(x)=\cos x-\dfrac 2x$;\\
(3) $f(x)=(2x+1)^3$;\\
(4) $f(x)=\sqrt x\sin x$;\\
(5) $f(x)=x\ln x-\dfrac1{x^2}$;\\
(6) $f(x)=\dfrac{x^2-1}x$;\\
(7) $f(x)=\dfrac{x^2-1}{x^2+1}$;\\
(8) $f(x)=\tan x$. 
\item 利用$f(ax+b)$型复合函数的求导法则求下列函数的导数:\\
(1) $y=(3-2x)^2$;\\
(2) $y=\sin 2x$.
\item 尝试用两种不同的方法求$f(x)=\dfrac1{2x-1}$的导数.
\item 求曲线$y=2^{1-3x}$在点$(0,2)$处的切线方程.
\item 求下列函数的导数:\\
(1) $y=3x \sqrt{2-x}$;\\
(2) $y=\dfrac{\ln(2x+1)}x$. 
\item 利用导数研究下列函数的单调性, 并说明所得结果与你之前的认识是否一致:\\
(1) $y=\mathrm{e}^x$;\\
(2) $y=\ln x$;\\
(3) $y=ax^2+bx+c$, 其中$a\ne 0$.
\item 确定下列函数的单调区间:\\
(1) $y=x\mathrm{e}^x$;\\
(2) $y=4x^3-9x^2+6x+7$.
\item 求余弦函数$y=\cos x$的单调区间和极值.
\item 求函数$y=x^3-3x$的单调区间和极值.
\item 判断下列说法是否正确, 并说明理由:\\
(1) 函数在某区间上的极大值不会小于它的极小值;\\
(2) 函数在某区间上的最大值不会小于它的最小值;\\
(3) 函数在某区间上的极大值就是它在该区间上的最大值;\\
(4) 函数在某区间上的最大值就是它在该区间上的极大值.
\item 求函数$y=x^2-6x+5$, $x\in [1, 4]$的值域.
\item 求函数$y=x^3-3x$在区间$[-\dfrac 32,0]$上的最大值与最小值. 
\item 商品的成本$C$和产量$q$满足函数关系$C=50000+200q$, 该商品的销售单价$p$和产量$q$满足函数关系$p=24200-\dfrac 15q^2$. 问: 要使利润最大, 应如何确定产量?
\item 采矿、采石或取土时, 常用炸药包进行爆破, 部分爆破呈圆锥漏斗形状(如图), 已知圆锥的母线长是炸药包的爆破半径$R$, 它的值是固定的. 问: 炸药包埋多深可使爆破体积最大? 
\begin{center}
\begin{tikzpicture}[>=latex,scale = 1.6]
\draw (-1,0) -- (1,0) (0.5,0) node [above] {$r$};
\draw (-1,0) -- (0,-2) -- (1,0);
\draw (0,0) ellipse (1 and 0.25);
\draw [dashed] (0,0) -- (0,-2) (0,-1) node [left] {$h$};
\draw (0.5,-1) node [below right] {$R$};
\filldraw (0,-2) circle (0.08) (0,-2.3) node {炸药包};
\end{tikzpicture}
\end{center}
\item 公园有$4$个门, 从一个门进, 再从另一个门出, 共有多少种不同的走法?
\item $4$名学生报名参加两项体育比赛, 每名学生可参加的比赛数目不限, 每项比赛参加的人数不限, 共有多少种不同的报名结果? 
\item 在平面直角坐标系中, 以$1$、$2$、$3$、$4$、$5$这五个数中的两个分别作为一个点的横坐标和纵坐标, 可以组成多少个位于直线$y=x$下方的点?
\item 书架上放有$6$本不同的数学书和$5$本不同的语文书. 从中任取一本, 有多少种不同的取法? 
\item 写出从$a$、$b$、$c$、$d$、$e$这五个不同元素中任意取出两个元素的所有排列.
\item 已知$M=\{1,2,3,4\}$, 且$m\in M$, $n\in M$, 方程$\dfrac{x^2}m+\dfrac{y^2}n=1$表示的曲线是椭圆. 问: 可以有多少个不同的椭圆? 
\item $5$名篮球队员甲、乙、丙、丁、戊, 排成一排.\\
(1) 共有多少种不同的排法?\\
(2) 若甲必须站在排头, 有多少种不同的排法?\\
(3) 若甲不能站排头, 也不能站排尾, 有多少种不同的排法?
\item (1) 配制某种染色剂, 需要加入$3$种有机染料、$2$种无机染料和$2$种添加剂, 其中有机染料的添加顺序不可以相邻. 为研究所有不同的添加顺序对染色效果的影响, 总共要试验多少次?\\
(2) 某展览馆计划展出$10$幅不同的画, 其中水彩画$1$幅、油画$4$幅、国画$5$幅. 现排成一排陈列, 要求同一品种的画必须连在一起, 并且水彩画不放在两端. 问: 有多少种不同的陈列方式?
\item 已知$n$是正整数, 且$\dfrac{\mathrm{P}_n^7-\mathrm{P}_n^5}{\mathrm{P}_n^5} =89$. 求$n$的值.
\item 已知$n$为不小于$2$的正整数, 求证: $\mathrm{P}_{n+1}^{n+1}-\mathrm{P}_n^n=n^2\mathrm{P}_{n-1}^{
n-1}$. 
\item (1) 写出从$a$、$b$、$c$、$d$、$e$五个元素中任取两个不同元素的所有组合;\\
(2) 写出从$a$、$b$、$c$、$d$、$e$五个元素中任取两个不同元素的所有排列.
\item 平面上的$6$个点$A$、$B$、$C$、$D$、$E$、$F$中的任意$3$个点都不在同一条直线上, 写出所有以其中$3$个点为顶点的三角形. 
\item 某班有$20$名男生、$18$名女生, 现从中任选$5$人组成一个宣传小组, 其中男生和女生都有的选法有多少种?
\item 从$1$、$2$、$3$、$4$、$5$这五个数字中任取两个不同的奇数和两个不同的偶数.\\
(1) 一共有多少种不同的选法?
(2) 可以组成多少个没有重复数字的四位奇数? 
\item 解关于正整数$x$的方程:\\
(1) $\mathrm{C}_{16}^{x^2-x}=\mathrm{C}_{16}^{5x-5}$;\\
(2) $\mathrm{C}_{x+2}^{x-2}+\mathrm{C}_{x+2}^{x-3}=\dfrac 14\mathrm{P}_{x+3}^3$.
\item 观察下列等式及其所示的规律:\\
\begin{align*}
\mathrm{C}_3^0+\mathrm{C}_4^1=&\mathrm{C}_4^0+\mathrm{C}_4^1=\mathrm{C}_5^1,\\
\mathrm{C}_3^0+\mathrm{C}_4^1+\mathrm{C}_5^2=&\mathrm{C}_5^1+\mathrm{C}_5^2=\mathrm{C}_6^2,\\
\mathrm{C}_3^0+\mathrm{C}_4^1+\mathrm{C}_5^2+\mathrm{C}_6^3=&\mathrm{C}_6^2+\mathrm{C}_6^3=\mathrm{C}_7^3.
\end{align*}
并据此化简$\mathrm{C}_3^0+\mathrm{C}_4^1+\mathrm{C}_5^2+\mathrm{C}_6^3+\cdots+\mathrm{C}_{n+3}^n$, 其中$n$为正整数.
\item 袋中装有$4$个红球、$3$个黄球、$3$个白球, 所有小球的大小与质地完全相同. 从中同时任取$2$个小球, 求取出的$2$个球颜色相同的概率.
\item 某校要从$2$名男生和$4$名女生中任选$4$人担任一项赛事的志愿者工作, 每个人被选中的可能性相同. 求在选出的志愿者中, 男生和女生都有的概率.
\item (1) 求$(x-\sqrt2y)^8$的二项展开式;\\
(2) 求$(x-x^{-\frac 13})^{12}$的二项展开式中的常数项;\\
(3) 求$(x-\dfrac 2x)^9$的二项展开式中$x^3$的系数;\\
(4) 在$(1-x^2)^{20}$的二项展开式中, 如果第$4r$项和第$r+2$项的系数的绝对值相等, 求此展开式的第$4r$项.
\item 利用二项式定理证明: $7^{100}-1$是$8$的倍数. 
\item (1) 若$(1-x)^6=a_0+a_1x+a_2x^2+\cdots+a_6x^6$, 求$a_0+a_1+a_2+\cdots+a_6$的值;\\
(2) 已知$(x+1)^n=a_0+a_1(x-1)+a_2(x-1)^2+a_3(x-1)^3+\cdots+a_n(x-1)^n$($n\ge 2$, $n$为正整数), 求$a_0+a_1+a_2+\cdots+a_n$的值.
\item (1) 求$(1+2x)^7$的二项展开式中系数最大的项;\\
(2) 求$(1-2x)^7$的二项展开式中系数最大的项. 
\item 一个家庭有两个孩子.
(1) 已知年龄大的是女孩, 求年龄小的也是女孩的概率;
(2) 已知其中一个是女孩, 求另一个也是女孩的概率.
\item 掷一颗骰子, 令事件$A=\{2,3,5\}$, $B=\{1,2,4,5,6\}$. 求$P(A)$、$P(B)$、$P(A\cap B)$及$P(A|B)$.
\item 在一个盒子中有大小与质地相同的$20$个球, 其中$10$个红球, $10$个白球. 两人依次不放回地各摸$1$个球, 求:
(1) 在第一个人摸出$1$个红球的条件下, 第二个人摸出$1$个白球的概率;\\
(2) 第一个人摸出$1$个红球, 且第二个人摸出$1$个白球的概率.
\item 公司库房中的某个零件的$70\%$来自$A$公司, $30\%$来自$B$公司, 两个公司的正品率分别是$95\%$和$90\%$. 从库房中任取一个零件, 求它是正品的概率.
\item 盒子中有大小与质地相同的$5$个红球和$4$个白球, 从中随机取$1$个球, 观察其颜色后放回, 并同时放入与其相同颜色的球$3$个, 再从盒子中取$1$个球. 求第二次取出的球是白色的概率.
\item 从一个放有大小与质地相同的$3$个黑球、$2$个白球的袋子里摸出$2$个球并放入另外一个空袋子里, 再从后一个袋子里摸出$1$个球. 求该球是黑色的概率. 
\item 设某公路上经过的货车与客车的数量之比为$2: 1$, 货车中途停车修理的概率为$0.02$, 客车中途停车修理的概率为$0.01$. 今有一辆汽车中途停车修理, 求该汽车是货车的概率.
\item 已知在所有男子中有$5\%$患有色盲症, 在所有女子中有$0.25\%$患有色盲症. 现随机抽取一人发现患有色盲症, 问: 其为男子的概率是多少? (设男子和女子的人数相等)
\item 掷两颗骰子, 用$X$表示两点数差的绝对值. 求$X$的分布.
\item 以下是分布的为\bracket{20}.
\fourch{$\begin{pmatrix}0 & 1 \\ 1 & 1\end{pmatrix}$}{$\begin{pmatrix}-1 & 0 & 1\\ \dfrac 12 & \dfrac 13 & \dfrac 16\end{pmatrix}$}{$\begin{pmatrix}1 & 2 & 3\\ \dfrac 12 & \dfrac 14 & \dfrac 18\end{pmatrix}$}{$\begin{pmatrix}1 & 1.2 & 2 & 2.4 \\ -0.5 & 0.5 & 0.3 & 0.7\end{pmatrix}$}
\item 抛掷$4$枚硬币, 用$X$表示正面朝上的枚数. 求$X$的期望.
\item 从一个放有大小与质地相同的$5$个白球、$4$个黑球的罐子中不放回地摸$3$个球, 用$X$表示摸到的白球数. 求$X$的期望. 
\item 设$X$是一个随机变量, $c$是常数. 求证: $X+c$的方差与$X$的方差相等.
\item 已知随机变量$X$的分布为$\begin{pmatrix}1 & 2 & 3 \\ 0.4 & 0.2 & 0.4\end{pmatrix}$, 求$X$的方差. 
\item 已知随机变量$X$服从二项分布$B(n,p)$, 若$E[X]=30$, $D[X]=20$, 求$p$的值.
\item 一批产品的二等品率为$0.3$. 从这批产品中每次随机取一件, 并有放回地抽取$20$次. 用$X$表示抽到二等品的件数, 求$D[X]$. 
\item 盒子中有大小与质地相同的$3$个白球、$1$个黑球, 若从中随机地摸出$2$个球, 求它们颜色不同的概率.
\item 从放有$6$黑$2$白共$8$颗珠子的袋子中抓$3$颗珠子, 分别求黑珠颗数$X$与白珠颗数$Y$的分布、期望与方差.
\item 从一副去掉大小王牌的$52$张扑克牌中任取$5$张牌, 求:\\
(1) 至少有一张黑桃的概率;\\
(2) 至少有一个对子(两张牌的数字一样)的概率. 
\item 已知随机变量$X$服从正态分布$N(-2, \sigma^2)$, 且$P(X\le -1)=k$. 求$P(X\le -3)$的值.
\item 某校高中三年级$1600$名学生参加了区第一次高考模拟统一考试, 已知数学考试成绩$X$服从正态分布$N(100, \sigma^2)$(试卷满分为$150$分). 统计结果显示, 数学考试成绩在$80$分到$120$分之间的人数约为总人数的$\dfrac 34$, 则此次统考中成绩不低于$120$分的学生人数约为\bracket{20}.
\fourch{$80$}{$100$}{$120$}{$200$} 
\item 若已知下列各组数据, 它们是否可以看作成对数据?是否可以进行相关分析? 判断并简要说明理由.\\
(1) $A$校学生的身高与$B$校学生的体重;\\
(2) 人体内的脂肪含量与体重;\\
(3) 某班学生的物理成绩与数学成绩.
\item 《国家学生体质健康标准($2014$年修订)》中, 体能监测包含身高、体重、肺活量、$50$米跑、坐位体前屈、引体向上(女: 仰卧起坐)、立定跳远、$1000$米跑(女: $800$米跑),
据此得到的每项指标都可以按照相应的单项指标评分表进行测量和计分, 分别得到相应的数据.\\
(1) 这些数据中的任意两组是否都可以作为成对数据进行相关分析?\\
(2) 依据你的经验, 哪两组数据的相关程度可能最高? 哪两组数据的相关程度可能最低? 如何通过统计方法检验你的判断?
\item 某市$104$路公交车上午$7:05-8:55$时段在起点站每$9$分钟发一班次. 公交公司为了了解早高峰时段各班次上客情况, 某日上午$7:14—8:35$记录了在起点站各班次车辆上客的人数:
\begin{center}
\begin{tabular}{|c|c|c|c|c|c|c|c|c|c|c|}
\hline
发车时刻 & 7:14  & 7:23  & 7:32  & 7:41  & 7:50  & 7:59  & 8:08  & 8:17  & 8:26  & $8:35$ \\ \hline
上车乘客数/人 & $10$ & $13$ & $13$ & $18$ & $17$ & $15$ & $12$ & $9$ & $3$ & $3$ \\ \hline
\end{tabular}
\end{center}
请绘制这组成对数据的散点图, 并通过观察散点图大致判断客车发车时刻与上车乘客人数之间的相关性. 
\item 用经过匿名处理的本班同学最近一次期中或期末测验的各科成绩表, 考察不同科目测验成绩之间的相关性.
\item 为了研究豆类脂肪含量与其产生的热量的关系, 选取了$5$种豆类进行实验测定. 下面是$0. 1\text{kg}$豆类中脂肪含量(单位: $\text{kg}$)与相应热量(单位: $\text{kJ}$)的对照表.
\begin{center}
\begin{tabular}{|c|c|c|c|c|c|}
\hline
豆类 & 黄豆 & 豇豆 & 青毛豆 & 豌豆(鲜) & 四季豆 \\ \hline
脂肪含量/$\text{kg}$ & $0.0184$ & $0.0002$ & $0.0057$ & $0.0003$ & $0.0004$ \\ \hline
热量/$\text{kJ}$ & $1726$ & $108$ & $527$ & $336$ & $130$ \\ \hline
\end{tabular}
\end{center}
(1) 根据表中的数据绘制散点图;\\
(2) 观察散点图的趋势, 如果能看成线性关系, 请在图中画出一条直线来近似地表示这种关系, 并计算豆类脂肪含量与热量的相关系数. 
\item 将学生甲所给的线性方程$y=-0.5x+5.0$作为下表中数据的线性拟合, 计算各数据点的离差, 再计算拟合误差. 
\begin{center}
\begin{tabular}{|c|c|c|c|c|c|c|c|c|c|c|}
\hline
每千克价格/百元  & $4.0$ & $4.0$ & $4.6$ & $5.0$ & $5.2$ & $5.6$ & $6.0$ & $6.6$ & $7.0$ & $10.0$ \\ \hline
年需求量/千克  & $3.5$ & $3.0$ & $2.7$ & $2.4$ & $2.5$ & $2.0$ & $1.5$ & $1.2$ & $1.2$ & $1.0$ \\ \hline
\end{tabular}
\end{center}
把结果与下表的数据相比较, 
\begin{center}
\begin{tabular}{|c|c|c|c|c|c|c|c|c|c|c|}
\hline
$x$ & $4.0$ & $4.0$ & $4.6$ & $5.0$ & $5.2$ & $5.6$ & $6.0$ & $6.6$ & $7.0$ & $10$ \\ \hline
$y$ & $3.5$ & $3.0$ & $2.7$ & $2.4$ & $2.5$ & $2.0$ & $1.5$ & $1.2$ & $1.2$ & $1.0$ \\ \hline
$\hat{y}$ & $2.8$ & $2.8$ & $2.6$ & $2.4$ & $2.3$ & $2.2$ & $2.0$ & $1.8$ & $1.6$ & $0.4$ \\ \hline
离差$y_i-\hat{y_i}$ & $0.7$ & $0.2$ & $0.1$ & $0.0$ & $0.2$ & $-0.2$ & $-0.5$ & $-0.6$ & $-0.4$ & $0.6$ \\ \hline
\end{tabular}
\end{center}
说说你对``最佳拟合$''$有什么新的理解和体会.
\item 两个变量$x$与$y$之间的回归方程\bracket{20}.
\onech{表示$x$与$y$之间的函数关系}{表示$x$与$y$之间的不确定关系}{反映$x$与$y$之间的真实关系}{是反映$x$与$y$之间的真实关系的一种最佳拟合}
\item 用最小二乘法求回归方程是为了使\bracket{20}.
\fourch{$\displaystyle\sum_{i=1}^n(y_i-\overline{y})=0$}{$\displaystyle\sum_{i=1}^n(y_i-\hat{y_i})=0$}{$\displaystyle\sum_{i=1}^n(y_i-\hat{y_i})$最小}{$\displaystyle\sum_{i=1}^n(y_i-\hat{y_i})^2$最小} 
\item 某公司为了解用电量$y$(单位: $\text{kW}\cdot \text{h}$)与气温$x$(单位: $^\circ\text{C}$)之间的关系, 随机统计了$4$天的用电量与当天气温, 并制作了如下对照表: 
\begin{center}
\begin{tabular}{|c|c|c|c|c|}
\hline
气温$x$/$^\circ\text{C}$ & $18$ & $13$ & $10$ & $-1$ \\ \hline
用电量$y$/($\text{kW}\cdot \text{h}$) & $24$ & $34$ & $38$ & $64$ \\ \hline
\end{tabular}
\end{center}
由表中数据可得回归方程$y=ax+b$中$a=-2$, 试预测当气温为$-4^\circ\text{C}$时, 用电量约为\blank{50}$\text{kW}\cdot \text{h}$.
\item 用计算器或计算机软件建立下列观测数据的回归方程:
\begin{center}
\begin{tabular}{|c|c|c|c|c|c|c|c|}
\hline
$x_i$ & $70$ & $115$ & $130$ & $190$ & $195$ & $400$ & $450$ \\ \hline
$y_i$ & $1.10$ & $1.00$ & $0.85$ & $0.75$ & $0.85$ & $0.67$ & $0.50$ \\ \hline
\end{tabular}
\end{center}
\item 为了研究长江口滨海湿地乡土植物芦苇高度(单位: $\text{cm}$)与干重(单位: $\text{g}$)之间的关系, 观察芦苇高度与干重的数据(见下表), 其中干重为植物收获并烘干到一定标准后的质量. 试建立芦苇干重关于芦苇高度的回归方程.
\begin{center}
\begin{longtable}{|c|c|c|c|c|c|}
\hline
编号 & 高度/$\text{cm}$ & 干重/$\text{g}$ & 编号 & 高度/$\text{cm}$ & 干重/$\text{g}$ \\ \hline
\endhead
$1$ & $136$ & $15.01$ & $13$ & $147$ & $16.87$ \\ \hline
$2$ & $136$ & $14.88$ & $14$ & $150$ & $17.13$ \\ \hline
$3$ & $135$ & $15.12$ & $15$ & $148$ & $17.26$ \\ \hline
$4$ & $138$ & $14.99$ & $16$ & $150$ & $18.13$ \\ \hline
$5$ & $139$ & $15.54$ & $17$ & $149$ & $17.66$ \\ \hline
$6$ & $138$ & $15.24$ & $18$ & $152$ & $17.84$ \\ \hline
$7$ & $141$ & $15.68$ & $19$ & $151$ & $18.17$ \\ \hline
$8$ & $143$ & $15.88$ & $20$ & $154$ & $18.36$ \\ \hline
$9$ & $142$ & $18.16$ & $21$ & $155$ & $17.95$ \\ \hline
$10$ & $144$ & $16.33$ & $22$ & $155$ & $18.65$ \\ \hline
$11$ & $148$ & $15.99$ & $23$ & $157$ & $18.89$ \\ \hline
$12$ & $146$ & $16.57$ & $24$ & $156$ & $19.26$ \\ \hline
\end{longtable}
\end{center}
\item 某初中调查了该校$1000$名初三学生最近一次数学测试成绩与课堂注意力表现情况, 得到下表:
\begin{center}
\begin{tabular}{|c|c|c|c|}
\hline
& 数学成绩$\ge 80$分 & 数学成绩$< 80$分 & 总计 \\ \hline
上课注意力集中 & $418$ & $279$ & $697$ \\ \hline
上课注意力不集中 & $43$ & $260$ & $303$ \\ \hline
总计 & $461$ & $539$ & $1000$ \\ \hline
\end{tabular}
\end{center}
请根据表中提供的数据判断: 上课注意力集中与否对学习成绩有影响吗?
\item 为了调查髋关节保护器在减少老年人髋部骨折中的作用, 随机选择一些老年人, 其中一部分佩戴髋关节保护器, 而另一部分不佩戴, 作为对照组. 得到如下列联表:
\begin{center}
\begin{tabular}{|c|c|c|c|}
\hline
& 佩戴髋关节保护器 & 对照组 & 总计 \\ \hline
髋部骨折 & $13$ & $67$ & $80$ \\ \hline
无髋部骨折 & $640$ & $1081$ & $1721$ \\ \hline
总计 & $653$ & $1148$ & $1801$ \\ \hline
\end{tabular}
\end{center}
根据表中的数据回答: 髋关节保护器是否可以降低老年人髋部骨折的可能性?
\item 下表是$A$、$B$两所中学的学生对报考某类大学的意愿的列联表:
\begin{center}
\begin{tabular}{|c|c|c|c|}
\hline
& 愿意报考某类大学 & 不愿意报考某类大学 & 总计 \\ \hline
$A$中学 & $18$ & $37$ & $55$ \\ \hline
$B$中学 & $38$ & $57$ & $95$ \\ \hline
总计 & $56$ & $94$ & $150$ \\ \hline
\end{tabular}
\end{center}
根据表中的数据回答: $A$、$B$两所中学的学生对报考某类大学的态度是否有显著差异?
 
\end{enumerate}



















\end{document}