\documentclass[10pt,a4paper]{article}
\usepackage[UTF8,fontset = windows]{ctex}
\setCJKmainfont[BoldFont=黑体,ItalicFont=楷体]{华文中宋}
\usepackage{amssymb,amsmath,amsfonts,amsthm,mathrsfs,dsfont,graphicx}
\usepackage{ifthen,indentfirst,enumerate,color,titletoc}
\usepackage{tikz}
\usetikzlibrary{arrows,calc,intersections}
\usepackage[bf,small,indentafter,pagestyles]{titlesec}
\usepackage[top=1in, bottom=1in,left=0.8in,right=0.8in]{geometry}
\renewcommand{\baselinestretch}{1.65}
\newtheorem{defi}{定义~}
\newtheorem{eg}{例~}
\newtheorem{ex}{~}
\newtheorem{rem}{注~}
\newtheorem{thm}{定理~}
\newtheorem{coro}{推论~}
\newtheorem{axiom}{公理~}
\newtheorem{prop}{性质~}

\newcommand{\blank}[1]{\underline{\hbox to #1pt{}}}
\newcommand{\bracket}[1]{(\hbox to #1pt{})}

\newcommand{\onech}[4]{\par\begin{tabular}{p{.9\textwidth}}
A.~#1\\
B.~#2\\
C.~#3\\
D.~#4
\end{tabular}}

\newcommand{\twoch}[4]{\par\begin{tabular}{p{.46\textwidth}p{.46\textwidth}}
A.~#1& B.~#2\\
C.~#3& D.~#4
\end{tabular}}

\newcommand{\vartwoch}[4]{\par\begin{tabular}{p{.46\textwidth}p{.46\textwidth}}
(1)~#1& (2)~#2\\
(3)~#3& (4)~#4
\end{tabular}}


\newcommand{\fourch}[4]{\par\begin{tabular}{p{.23\textwidth}p{.23\textwidth}p{.23\textwidth}p{.23\textwidth}}
A.~#1 &B.~#2& C.~#3& D.~#4
\end{tabular}}

\newcommand{\varfourch}[4]{\par\begin{tabular}{p{.23\textwidth}p{.23\textwidth}p{.23\textwidth}p{.23\textwidth}}
(1)~#1 &(2)~#2& (3)~#3& (4)~#4
\end{tabular}}
    

\begin{document}
2022届高三暑假概率校本作业

\begin{enumerate}[1.]
\item 某地区气象台统计, 该地区下雨的概率是$\dfrac 4{15}$, 刮风的概率是$\dfrac 25$, 既刮风又下雨的概率为$\dfrac 1{10}$, 设事件$A$表示``该地区下雨'', 事件$B$表示``该地区刮风'', 那么$P(B|A)$等于\blank{50}.
\item 已知盒中装有$3$只螺口灯泡与$7$只卡口灯泡, 这些灯泡的外形都相同且灯口向下放着, 现需要安装一只卡口灯泡, 电工师傅每次从盒中任取一只并且不放回, 则在他第$1$次抽到的是螺口灯泡的条件下, 第$2$次抽到的是卡口灯泡的概率为\blank{50}.
\item 近年来, 新能源汽车技术不断推陈出新, 新产品不断涌现, 在汽车市场上影响力不断增大. 动力蓄电池技术作为新能源汽车的核心技术, 它的不断成熟也是推动新能源汽车发展的主要动力. 假定现在市售的某款新能源汽车上, 车载动力蓄电池充放电循环次数达到$2000$次的概率为$85\%$, 充放电循环次数达到$2500$次的概率为$35\%$. 若某用户的自用新能源汽车已经经过了$2000$次充电, 那么他的车能够充电$2500$次的概率为\blank{50}.
\item 将三颗骰子各掷一次, 记事件$A$为``三个点数都不相同'', $B$为``至少出现一个$6$点'', 则条件概率$P(A|B)$=\blank{50}, $P(B|A)$=\blank{50}.
\item 袋中有大小完全相同的$2$个白球和$3$个黄球, 逐个不放回地摸出$2$个球, 设``第一次摸到白球''为事件$A$, ``摸到的$2$个球同色''为事件$B$, 则$P(B|A)$=\blank{50}.
\item 已知$P(A)>0$, $P(B)>0$, $P(B|A)=P(B)$, 证明: $P(A|B)=P(A)$.
\item *甲、乙、丙三人互相作传球训练, 第$1$次由甲将球传出, 每次传球时, 传球者都等可能地将球传给另外两个人中的任何一个, 求$4$次传球后球在甲手中的概率.
\item 现在有$12$道四选一的单选题, 学生张三对其中$9$道题有思路, $3$道题完全没有思路. 有思路的题做对的概率为$0.9$, 没有思路的题只好任意猜一个答案, 猜对的概率为$0.25$, 张三从这$12$道题中随机选择$1$题, 则他做对该题的概率是\blank{50}.
\item 两批同种规格的产品, 第一批占$40\%$, 次品率为$5\%$;第二批占$60\%$, 次品率为$4\%$, 将这两批产品混合, 从混合的产品中任取一件. 则这件产品时合格品的概率是\blank{50}.
\item 甲和乙两个箱子中各装有$10$个球, 其中甲箱中有$5$个红球、$5$个白球, 乙箱中有$8$个红球、$2$个白球. 掷一枚质地均匀的骰子, 如果点数为$1$或$2$, 从甲箱子随机摸出$1$个球; 如果点数为$3, 4, 5, 6$, 从乙箱子中随机摸出$1$个球, 则摸到红球的概率是\blank{50}.
\item 在$A$、$B$、$C$三个地区暴发了流感, 这三个地区分别有$6\%$, $5\%$, $4\%$的人患了流感, 假设这三个地区的人口数的比为$5: 7: 8$, 现从这三个地区中任意选取一个人. 则这个人患流感的概率是\blank{50}.
\item 甲、乙两人独立地向同一目标各射击一次, 已知甲命中目标的概率为$0.6$, 乙命中目标的概率为$0.5$, 则目标至少被命中一次时, 甲命中目标的概率是\blank{50}.
\item 设$B$和$\overline B$是对立事件, 求证: $P(\overline B|A)=1-P(B|A)$.
\item 一批产品共有$100$件, 其中$5$件为不合格品, 收获方从中不放回地随机抽取产品进行检验, 并按以下规则判断是否接受这批产品; 如果抽检地第$1$件产品不合格, 则拒绝整批产品; 如果抽检的第一件产品合格, 则再抽$1$件, 如果抽检的第$2$件产品合格, 则接受整批产品, 否则拒绝整批产品, 求这批产品被拒绝的概率.
\item 在孟德尔豌豆试验中, 子二代的基因型为DD, Dd, dd, 其中D为显性基因, d为隐性基因, 且这三种基因型的比为$1: 2: 1$. 如果在子二代中任意选取$2$颗豌豆作为父代进行杂交试验, 那么第三代中基因型为dd的概率有多大?
\item 长时间玩手机可能影响视力, 据调查, 某校学生大约$40\%$的人近视, 而该校大约有$20\%$的学生每天玩手机超过$1\text{h}$, 这些人的近视率为$50\%$. 现从每天玩手机不超过$1\text{h}$的学生中任意调查一名学生, 求他的近视概率.
\item 设随机变量$X$的概率分布列如下, 则$P(|X-2|=1)=$\blank{50}.
\begin{center}
    \begin{tabular}{|c|c|c|c|c|}
        \hline
        $X$ & $1$ & $2$ & $3$ & $4$\\ \hline
        $P$ & $\dfrac 16$ & $\dfrac 14$ & $m$ & $\dfrac 13$\\ \hline       
    \end{tabular}
\end{center}
\item 已知离散型随机变量$X$的分布列为
\begin{center}
    \begin{tabular}{|c|c|c|c|}
        \hline
        $X$ & $0$ & $1$ & $2$ \\ \hline
        $P$ & $0.5$ & $1-2q$ & $q^2$ \\ \hline
    \end{tabular}
\end{center}
则常数$q=$\blank{50}.
\item 一盒中有$12$个乒乓球, 其中$9$个新的, $3$个旧的, 从盒子中任取$3$个球来用, 用完即为旧的, 用完后装回盒中, 此时盒中旧球个数$X$是一个随机变量, 则$P(X=4)$的值为\blank{50}.
\item 离散型随机变量$X$的概率分布规律为$P(X=n)=\dfrac{a}{n(n+1)}$($n=1, 2, 3, 4$), 其中$a$是常数, 则$P(\dfrac 12<X<\dfrac 52)$的值为\blank{50}.
\item 设离散型随机变量X的分布列如下表, 求$|X-1|$的分布列.
\begin{center}
    \begin{tabular}{|c|c|c|c|c|c|}
        \hline
        $X$	& $0$ & $1$ & $2$ & $3$ & $4$ \\ \hline
        $P$	& $0.2$ & $0.1$ & $0.1$ & $0.3$ & $m$ \\ \hline
    \end{tabular}
\end{center}

\item 某射手有$5$发子弹, 射击一次命中目标的概率为$0.9$, 如果命中就停止射击, 否则一直到子弹用尽, 求耗用子弹数$X$的分布列.
\item 某汽车美容公司为吸引顾客, 推出优惠活动: 对首次消费的顾客, 按$200$元/次收费, 并注册成为会员, 对会员逐次消费给予相应优惠, 标准如下:\\
\begin{center}
    \begin{tabular}{|c|c|c|c|c|c|}
        \hline
        消费次第 & 第$1$次 & 第$2$次 & 第$3$次 & 第$4$次 & $\ge 5$次 \\ \hline
        收费比率 & $1$ & $0.95$ & $0.90$ & $0.85$ & $0.80$\\ \hline
    \end{tabular}
\end{center}
该公司注册的会员中没有消费超过$5$次的, 从注册的会员中, 随机抽取了$100$位进行统计, 得到的统计数据如下:
\begin{center}
    \begin{tabular}{|c|c|c|c|c|c|}
        \hline
        消费次数 & $1$ & $2$ & $3$ & $4$ & $5$ \\ \hline
        人数 & $60$ & $20$ & $10$ & $5$ & $5$\\ \hline
    \end{tabular}
\end{center}
假设汽车美容$1$次, 公司成本为$150$元, 根据所给数据, 解答下列问题:\\
(1) 某会员仅消费$2$次, 求这$2$次消费中, 公司获得的平均利润;\\
(2) 以事件发生的频率作为相应事件发生的概率, 设该公司为$1$位会员服务的平均利润为$X$元, 求$X$的分布列.
\item 习近平总书记在$2020$年新年贺词中勉励大家:``让我们只争朝夕, 不负韶华, 共同迎接$2020$年的到来.'' 其中``只争朝夕, 不负韶华''旋即成了网络热词, 成了大家互相砥砺前行的铮铮誓言, 激励着广大青年朋友奋发有为, 积极进取, 不负青春, 不负时代.\\
``只争朝夕, 不负韶华''用英文可翻译为:``Seize the day and live it to the full.''\\
(1) 求上述英语译文中, e, i, t, a $4$个字母出现的频率(小数点后面保留两位有效数字), 并比较$4$个频率的大小(用``>''连接);\\
(2) 在上面的句子中随机取一个单词, 用$X$表示取到的单词所包含的字母个数, 写出$X$的分布列;\\
(3) 从上述单词中任选$2$个单词, 求其字母个数之和为$6$的概率.
\item 已知$X$的分布列为
\begin{center}
    \begin{tabular}{|c|c|c|c|}
        \hline
        $X$ & $-1$ & $0$ & $1$ \\ \hline
        $P$ & $\dfrac 12$ & $\dfrac 13$ & $\dfrac 16$ \\ \hline
    \end{tabular}
\end{center}
两个随机变量$X$, $Y$满足$X+2Y=4$, 则$E[X]=$\blank{50}, $E[Y]=$\blank{50}.
\item ``过大年, 吃水饺''是我国不少地方过春节的一大习俗. $2021$年春节前夕, $A$市某质量检测部门随机抽取了$100$包某种品牌的速冻水饺, 检测其某项质量指标值, 所得频率分布直方图如图.
\begin{center}
    \begin{tikzpicture}[>=latex]
        \draw [->] (0,0) -- (7,0) node [below] {质量指标值};
        \draw [->] (0,0) -- (0,4) node [left] {$\dfrac{\text{频率}}{\text{组距}}$};
        \draw (0,0) node [below left] {$O$};
        \draw (0,1) node [left] {$0.010$} -- (1,1) -- (1,0) node [below] {$10$};
        \draw (1,1) -- (1,2) -- (2,2) -- (2,0) node [below] {$20$};
        \draw (2,2) -- (2,3) -- (3,3) -- (3,0) node [below] {$30$};
        \draw (3,2.5) -- (4,2.5) -- (4,0) node [below] {$40$};
        \draw (4,1.5) -- (5,1.5) -- (5,0) node [below] {$50$};
        \draw [dashed] (0,1.5) node [left] {$0.015$} -- (4,1.5);
        \draw [dashed] (0,2) node [left] {$0.020$} -- (1,2);
        \draw [dashed] (0,2.5) node [left] {$0.025$} -- (3,2.5);
        \draw [dashed] (0,3) node [left] {$0.030$} -- (2,3);
    \end{tikzpicture}
\end{center}
(1) 求所抽取的$100$包速冻水饺该项质量指标值的样本平均数$\overline x$(同一组中的数据用该组区间的中点值作代表);\\
(2) 将频率视为概率, 若某人从该市某超市购买了$4$包这种品牌的速冻水饺, 记这$4$包速冻水饺中该项质量指标值位于$(10,30]$内的包数为$X$, 求$X$的分布列和期望.
\item 近年来, 祖国各地依托本地自然资源, 打造旅游产业, 旅游业正蓬勃发展. 景区与游客都应树立尊重自然、顺应自然、保护自然的生态文明理念, 合力使旅游市场走上规范有序且可持续的发展轨道. 某景区有一个自愿消费的项目: 在参观某特色景点入口处会为每位游客拍一张与景点的合影, 参观后, 在景点出口处会将刚拍下的照片打印出来, 游客可自由选择是否带走照片, 若带走照片则需支付$20$元, 没有被带走的照片会收集起来统一销毁. 该项目运营一段时间后, 统计出平均只有$30\%$游客会选择带走照片. 为改善运营状况, 该项目组就照片收费与游客消费意愿关系做了市场调研, 发现收费与消费意愿有较强的线性相关性, 并统计出在原有的基础上, 价格每下调$1$元, 游客选择带走照片的可能性平均增加$0.05$. 假设平均每天约有$5000$人参观该特色景点, 每张照片的综合成本为$5$元, 假设每位游客是否购买照片相互独立.\\
(1) 若调整为支付$10$元就可带走照片, 该项目每天的平均利润比调整前多还是少?\\
(2) 要使每天的平均利润达到最大值, 应如何定价?
\item 某种大型医疗检查机器生产商, 对一次性购买$2$台机器的客户, 推出$2$种超过质保期后$2$年内的延保维修优惠方案.\\
方案一: 交纳延保金$7000$元, 在延保的$2$年内可免费维修$2$次, 超过$2$次每次收取维修费$2000$元;\\
方案二: 交纳延保金$10000$元, 在延保的$2$年内可免费维修$4$次, 超过$4$次每次收取维修费$1000$元.\\
某医院准备一次性购买$2$台这种机器. 现需决策在购买机器时应购买哪种延保方案, 为此搜集并整理了$50$台这种机器超过质保期后延保$2$年内维修的次数, 得下表:
\begin{center}
    \begin{tabular}{|c|c|c|c|c|}
        \hline
        维修次数 & $0$ & $1$ & $2$ & $3$\\ \hline
        台数 & $5$ & $10$ & $20$ & $15$\\ \hline
    \end{tabular}
\end{center}
以这$50$台机器维修次数的频率代替$1$台机器维修次数发生的概率. 记$X$表示这$2$台机器超过质保期后延保的$2$年内共需维修的次数.\\
(1) 求$X$的分布列;\\
(2) 以方案一与方案二所需费用(所需延保金及维修费用之和)的期望值为决策依据, 医院选择哪种延保方案更合算?
\item 已知$X$的分布列为
\begin{center}
    \begin{tabular}{|c|c|c|c|}
        \hline
        $X$ & $-1$ & $0$ & $1$ \\ \hline
        $P$ & $\dfrac 12$ & $\dfrac 13$ & $\dfrac 16$ \\ \hline       
    \end{tabular}
\end{center}
两个随机变量$X$, $Y$满足$X+2Y=4$, 则$D[X]=$\blank{50}, $D[Y]=$\blank{50}.
\item 五个自然数$1, 2, 3, 4, 5$按照一定的顺序排成一排.\\
(1) 求$2$和$4$不相邻的概率;\\
(2) 定义: 若两个数的和为$6$且相邻, 称这两个数为一组``友好数''. 随机变量X表示上述五个自然数组成的一个排列中``友好数''的组数, 求$X$的分布列、数学期望$E[X]$和方差$D[X]$.
\item 为推广滑雪运动, 某滑雪场开展滑雪促销活动. 该滑雪场的收费标准是: 滑雪时间不超过$1$小时免费, 超过$1$小时的部分每小时收费标准为$40$元(不足$1$小时的部分按$1$小时计算). 有甲、乙两人相互独立地来该滑雪场运动, 设甲、乙不超过$1$小时离开的概率分别为$\dfrac 14$, $\dfrac 16$; $1$小时以上且不超过$2$小时离开的概率分别为$\dfrac 12$, $\dfrac 23$; 两人滑雪时间都不会超过$3$小时.\\
(1) 求甲、乙两人所付滑雪费用相同的概率;\\
(2) 设甲、乙两人所付的滑雪费用之和为随机变量$X$(单位: 元), 求$X$的分布列与数学期望$E[X]$, 方差$D[X]$.
\item 甲、乙两人各射击$1$次, 击中目标的概率分别是$\dfrac 23$和$\dfrac 12$, 假设两人击中目标与否相互之间没有影响, 每人各次击中目标与否相互之间也没有影响, 若两人各射击$4$次, 则甲恰好有$2$次击中目标且乙恰好有$3$次击中目标的概率为\blank{50}.
\item 在一次招聘中, 主考官要求应聘者从$18$道备选题中一次性随机抽取$9$道题, 并独立完成所抽取的$3$道题. 甲能正确完成每道题的概率为$\dfrac 23$, 且每道题完成与否互不影响. 记甲能答对的题数为$X$, 则$X$的期望为\blank{50}.
\item 设$X$为随机变量, 且$X\sim B(n,p)$, 若随机变量$X$的数学期望$E[X]=4$, $D[X]=\dfrac 43$, 则$P(X=2)=$\blank{50}(结果用分数表示).
\item 某地区为贯彻习近平总书记关于``绿水青山就是金山银山''的理念, 鼓励农户利用荒坡种植果树. 某农户考察三种不同的果树苗$A,B,C$, 经引种试验后发现, 引种树苗$A$的自然成活率为$0.8$, 引种树苗$B,C$的自然成活率均为$p$($0.7\le p\le 0.9$).\\
(1) 任取树苗$A,B,C$各一棵, 估计自然成活的棵数为$X$, 求$X$的分布列及数学期望$E[X]$;\\
(2) 将(1)中的$E[X]$取得最大值时$p$的值作为$B$种树苗自然成活的概率. 该农户决定引种$n$棵$B$种树苗, 引种后没有自然成活的树苗中有$75\%$的树苗可经过人工栽培技术处理, 处理后成活的概率为$0.8$, 其余的树苗不能成活.\\
(1) 求一棵$B$种树苗最终成活的概率;
(2) 若每棵树苗最终成活后可获利$300$元, 不成活的每棵亏损$50$元, 该农户为了获利不低于$20$万元, 问至少引种$B$种树苗多少棵?
\item 一款小游戏的规则如下: 每轮游戏要进行三次, 每次游戏都需要从装有大小相同的$2$个红球、$3$个白球的袋中随机摸出$2$个球, 若``摸出的两个球都是红球''出现$3$次获得$200$分, 若``摸出的两个球都是红球''出现$1$次或$2$次获得$20$分, 若``摸出的两个球都是红球''出现$0$次, 则扣除$10$分(即获得负$10$分).\\
(1) 设每轮游戏中出现``摸出的两个球都是红球''的次数为$X$, 求$X$的分布列;\\
(2) 许多玩过这款游戏的人发现, 若干轮游戏后, 与最初的分数相比, 分数没有增加, 反而减少了, 请运用概率统计的相关知识解释上述现象.
\item 一年之计在于春, 一日之计在于晨, 春天是播种的季节, 是希望的开端. 某种植户对一块地的$n$($n\in \mathbf{N}^*$, $n>0$)个坑进行播种, 每个坑播$3$粒种子, 每粒种子发芽的概率均为$\dfrac 12$, 且每粒种子是否发芽相互独立. 对每一个坑而言, 如果至少有$2$粒种子发芽, 则不需要进行补播种, 否则要补播种.\\
(1) 当$n$取何值时, 有$3$个坑要补播种的概率最大? 最大概率为多少?\\
(2) 当$n=4$时, 用$X$表示要补播种的坑的个数, 求$X$的分布列与数学期望.
\item $2019$年$3$月$5$日, 国务院总理李克强作的政府工作报告中, 提到要``惩戒学术不端, 力戒浮躁之风''. 教育部$2014$年印发的《博士硕士学位论文抽检办法》通知中规定: 每篇抽检的学位论文送$3$位同行专家进行评议, $3$位专家中有$2$位以上(含$2$位)专家评议意见为``不合格''的学位论文, 将认定为``存在问题学位论文'', 有且仅有$1$位专家评议意见为``不合格''的学位论文, 将再送另外$2$位同行专家(不同于前$3$位专家)进行复评, $2$位复评专家中有$1$位以上(含$1$位)专家评议意见为``不合格''的学位论文, 将认定为``存在问题学位论文''. 设每篇学位论文被每位专家评议为``不合格''的概率均为$p$($0<p<1$), 且各篇学位论文是否被评议为``不合格''相互独立.\\
(1) 若$p=\dfrac 12$, 求抽检一篇学位论文, 被认定为``存在问题学位论文''的概率;\\
(2) 现拟定每篇抽检论文不需要复评的评审费用为$900$元, 需要复评的总评审费用为$1500$元, 若某次评审抽检论文总数为$3000$篇, 求该次评审费用期望的最大值及对应$p$的值.
\item 某市有一家大型共享汽车公司, 在市场上分别投放了黄、蓝两种颜色的汽车, 已知黄、蓝两种颜色的汽车的投放比例为$3:1$. 监管部门为了了解这两种颜色汽车的质量, 决定从投放到市场上的汽车中随机抽取$5$辆汽车进行试驾体验, 假设每辆汽车被抽取的可能性相同.\\
(1) 求抽取的$5$辆汽车中恰有$2$辆是蓝色汽车的概率.\\
(2) 在试驾体验过程中, 发现蓝色汽车存在一定质量问题, 监管部门决定从投放的汽车中随机地抽取一辆送技术部门作进一步抽样检测, 并规定: 若抽到的是黄色汽车, 则将其放回市场, 并继续随机地抽取下一辆汽车; 若抽到的是蓝色汽车, 则抽样结束. 抽样的次数不超过$n$($n\in \mathbf{N}$, $n>0$)次. 在抽样结束时, 若已抽到的黄色汽车数以$X$表示, 求$X$的分布列和数学期望.
\item 河南省三门峡市成功入围``十佳魅力中国城市'', 吸引了大批投资商的目光, 一些投资商积极准备投入到``魅力城市''的建设之中. 某投资公司准备在$2022$年年初将$400$万元投资到三门峡下列两个项目中的一个之中.\\
项目一: 天坑院是黄土高原地域独具特色的民居形式, 是人类``穴居''发展史演变的实物见证. 现准备投资建设$20$个天坑院, 每个天坑院投资$20$万元, 假设每个天坑院是否盈利是相互独立的, 据市场调研, 到$2024$年底每个天坑院盈利的概率为$p$($0<p<1$), 若盈利则盈利投资额的$40\%$, 否则盈利额为$0$.\\
项目二: 天鹅湖国家湿地公园是一处融生态、文化和人文地理于一体的自然山水景区. 据市场调研, 投资到该项目上, 到$2024$年底可能盈利投资额的$50\%$, 也可能亏损投资额的$30\%$, 且这两种情况发生的概率分别为$p$和$1-p$.\\
(1) 若投资项目一, 记$X_1$为盈利的天坑院的个数, 求$E[X_1]$(用$p$表示);\\
(2) 若投资项目二, 记投资项目二的盈利为$X_2$百万元, 求$E[X_2]$(用$p$表示);\\
(3) 在(1)(2)两个条件下, 针对以上两个投资项目, 请你为投资公司选择一个项目, 并说明理由.
\item 一箱$24$罐的饮料中$4$罐有奖券, 每张奖券奖励饮料一罐, 从中任意抽取$2$罐, 则这$2$罐中有奖券的概率是\blank{50}.
\item 学校要从$12$名候选人中选$4$名学生组成学生会, 已知有$4$名候选人来自甲班. 假设每名候选人都有相同的机会被选到, 则甲班恰有$2$名同学被选到的概率\blank{50}.
\item 从一副不含大小王的$52$张扑克牌中任意抽取出$5$张, 则至少有两张$A$牌的概率\blank{50}(精确到$0.001$).
\item 有一个摸奖游戏, 在一个口袋中装有$10$个红球和$20$个白球, 这些球除了颜色外完全相同, 一次从中摸出$5$个球.\\
(1) 至少摸到$3$个红球就中奖, 求中奖的概率(精确到0.001);\\
(2) 设摸到红球的颗数为$X$, 求$X$的期望.
\item 在测试中, 客观题难度的计算公式为$P_i=\dfrac{R_i}N$, 其中$P_i$为第$i$题的难度, $R_i$为答对该题的人数, $N$为参加测试的总人数. 现对某校高三年级$240$名学生进行一次测试, 共$5$道客观题, 测试前根据对学生的了解, 预估了每道题的难度, 如下表所示:
\begin{center}
    \begin{tabular}{|c|c|c|c|c|c|}
        \hline
        题号 & $1$ & $2$ & $3$ & $4$ & $5$ \\ \hline
        考前预估难度$P_i$ & $0.9$ & $0.8$ & $0.7$ & $0.6$ & $0.4$ \\ \hline       
    \end{tabular}
\end{center}
测试后, 随机抽取了$20$名学生的答题数据进行统计, 结果如下:
\begin{center}
    \begin{tabular}{|c|c|c|c|c|c|}
        \hline
        题号 & $1$ & $2$ & $3$ & $4$ & $5$ \\ \hline
        实测答对人数 & $16$ & $16$ & $14$ & $14$ & $4$ \\ \hline       
    \end{tabular}
\end{center}
(1) 根据题中数据, 估计这$240$名学生中第$5$题的实测答对人数;\\
(2) 从抽样的$20$名学生中随机抽取$2$名学生, 记这$2$名学生中答对第$5$题的人数为$X$, 求$X$的分布列和数学期望;\\
(3) 试题的预估难度和实测难度之间会有偏差, 设$P_i'$为第$i$题的实测难度, 并定义统计量$S=\dfrac 1n[(P_1'-P_1)^2+(P_2'-P_2)^2+\cdots+(P_n'-P_n)^2]$, 若$S<0.05$, 则本次测试的难度预估合理, 否则不合理, 试检验本次测试对难度的预估是否合理.
\item 在中华人民共和国成立$70$周年时, 《我和我的祖国》《中国机长》《攀登者》三大主旋律电影在国庆期间集体上映. 据统计, 《我和我的祖国》票房收入为$31.46$亿元, 《中国机长》票房收入为$28.84$亿元, 《攀登者》票房收入为$10.88$亿元. 已知国庆过后某城市文化局统计得知大量市民至少观看了一部国庆档电影, 在已观影的市民中随机抽取了$100$人进行调查, 其中观看了《我和我的祖国》的有$49$人, 观看了《中国机长》的有$46$人, 观看了《攀登者》的有$34$人, 统计图如图所示.
\begin{center}
    \begin{tikzpicture}
        \draw (0,0) circle (2) node [above] {《中国机长》} node [below] {$(27)$};
        \draw (3.5,0) circle (2.5) node [above] {《我和我的祖国》} node [below] {$(30)$};
        \draw (1.2,-2.6) circle (1.8) node [above] {《攀登者》} node [below] {$(18)$};
        \draw (1.45,-1) node {$(4)$};
        \draw (1.5,0) node {$(a)$};
        \draw (0.6,-1.3) node {$(b)$};
        \draw (2.2,-1.6) node {$(c)$};
    \end{tikzpicture}
\end{center}
(1) 计算图中$a, b, c$的值;\\
(2) 文化局从只观看了两部电影的观众中采用分层抽样的方法抽取了$7$人进行观影体验的访谈, 了解到他们均表示要观看第三部电影, 现从这$7$人中随机选出$4$人, 用$X$表示这$4$人中将要观看《我和我的祖国》的人数, 求$X$的分布列.
\item 某大学为了调查该校学生性别与身高(单位: 厘米)的关系, 对该校$1000$名学生按照$10:1$的比例进行抽样调查, 得到身高频数分布表如下:
\begin{center}
男生身高频数分布表
    \begin{tabular}{|c|c|c|c|c|c|c|}
        \hline 
        男生身高/厘米 & $[160,165)$ & $[165,170)$ & $[170,175)$ & $[175,180)$ & $[180,185)$ & $[185,190]$\\ \hline
        频数 & $7$ & $10$ & $19$ & $18$	& $4$ & $2$\\ \hline
    \end{tabular}
\end{center}
\begin{center}
    女生身高频数分布表
    \begin{tabular}{|c|c|c|c|c|c|c|}
        \hline 
        女生身高/厘米 &  $[150,155)$ & $[155,160)$ & $[160,165)$ & $[165,170)$ & $[170,175)$ & $[175,180]$ \\ \hline
        频数 & $3$ & $10$ & $15$ & $6$	& $3$ & $3$\\ \hline
    \end{tabular}
 \end{center}
(1) 估计这$1000$名学生中女生的人数;\\
(2) 估计这$1000$名学生的身高在$[170, 190]$的概率;\\
(3) 在样本中, 从身高在$[170, 180]$的女生中任取$3$名进行调查, 设$X$表示所选$3$名学生中身高在$[170, 175)$的人数, 求$X$的分布列和期望.

\end{enumerate}
\end{document}