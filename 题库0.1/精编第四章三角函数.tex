\documentclass[10pt,a4paper]{article}
\usepackage[UTF8,fontset = windows]{ctex}
\setCJKmainfont[BoldFont=黑体,ItalicFont=楷体]{华文中宋}
\usepackage{amssymb,amsmath,amsfonts,amsthm,mathrsfs,dsfont,graphicx}
\usepackage{ifthen,indentfirst,enumerate,color,titletoc}
\usepackage{tikz}
\usepackage{makecell}
\usepackage{longtable}
%\usepackage{mathptmx}

\usetikzlibrary{arrows,calc,intersections,patterns,decorations.pathreplacing}
\usepackage[bf,small,indentafter,pagestyles]{titlesec}
\usepackage[top=1in, bottom=1in,left=0.8in,right=0.8in]{geometry}
\renewcommand{\baselinestretch}{1.65}
\newtheorem{defi}{定义~}
\newtheorem{eg}{例~}
\newtheorem{ex}{~}
\newtheorem{rem}{注~}
\newtheorem{thm}{定理~}
\newtheorem{coro}{推论~}
\newtheorem{axiom}{公理~}
\newtheorem{prop}{性质~}
\newcommand{\blank}[1]{\underline{\hbox to #1pt{}}}
\newcommand{\bracket}[1]{(\hbox to #1pt{})}
\newcommand{\onech}[4]{\par\begin{tabular}{p{.9\textwidth}}
A.~#1\\
B.~#2\\
C.~#3\\
D.~#4
\end{tabular}}
\newcommand{\twoch}[4]{\par\begin{tabular}{p{.46\textwidth}p{.46\textwidth}}
A.~#1& B.~#2\\
C.~#3& D.~#4
\end{tabular}}
\newcommand{\vartwoch}[4]{\par\begin{tabular}{p{.46\textwidth}p{.46\textwidth}}
(1)~#1& (2)~#2\\
(3)~#3& (4)~#4
\end{tabular}}
\newcommand{\fourch}[4]{\par\begin{tabular}{p{.23\textwidth}p{.23\textwidth}p{.23\textwidth}p{.23\textwidth}}
A.~#1 &B.~#2& C.~#3& D.~#4
\end{tabular}}
\newcommand{\varfourch}[4]{\par\begin{tabular}{p{.23\textwidth}p{.23\textwidth}p{.23\textwidth}p{.23\textwidth}}
(1)~#1 &(2)~#2& (3)~#3& (4)~#4
\end{tabular}}
\begin{document}
\begin{enumerate}[1.]


\item 已知$\cos \alpha =\dfrac{24}{25}$, 求$\sin \alpha$.		
解答在这里  利用``勾股数'', 若$\alpha$在第一象限, 则$\sin \alpha =\dfrac 7{25}$; 若$\alpha$在第四象限, 则$\sin \alpha =-\dfrac 7{25}$.
\item 已知$\tan \alpha =-\sqrt 5$, 求$\cos \alpha$.
解答在这里 如图, 若$\alpha$在第二象限, 则$\cos \alpha =\dfrac{-1}{\sqrt 6}=-\dfrac{\sqrt 6}6$; 若$\alpha$在第四象限, 则$\cos \alpha =\dfrac 1{\sqrt 6}=\dfrac{\sqrt 6}6$.
(图1)
(2)若角$\alpha$的一个三角函数值是以字母形式给出的.通常按照``倒、平、倒、商、倒''的顺序求解, 还应注意象限的分类, 即从已知的三角函数有关的平方关系中对另一个三角函数的符号进行分类, 具体见下表:
已知	平方关系的变式	分类
$\sin \alpha$	$\cos \alpha =\pm \sqrt {1-\sin ^2\alpha }$	一、四象限和二、三象限
$\tan \alpha$	$\sec \alpha =\pm \sqrt {1+\tan ^2\alpha }$	
$\cos \alpha$	$\sin \alpha =\pm \sqrt {1-\cos ^2\alpha }$	一、二象限和三、四象限
$\cot \alpha$	$\csc \alpha =\pm \sqrt {1+\cot ^2\alpha }$	
$\sec \alpha$	$\tan \alpha =\pm \sqrt {\sec ^2\alpha -1}$	一、三象限和二、四象限
$\csc \alpha$	$\cot \alpha =\pm \sqrt {\csc ^2\alpha -1}$	
\item 已知$\cos \alpha =m$($m\ne 0$, $m\ne \pm 1$), 求$\alpha$的其他三角函数值.
解  因为$\sin ^2\alpha +\cos ^2\alpha =1$, 故可按$\sin \alpha$的符号划分象限.
(1)若$\alpha$在第一、二象限, 则$\sec \alpha =\dfrac 1m$(倒), $\sin \alpha =\sqrt {1-m^2}$(平), $\csc \alpha =\dfrac 1{\sqrt {1-m^2}}$(倒), $\tan \alpha =\dfrac{\sin \alpha }{\cos \alpha }=\dfrac{\sqrt {1-m^2}}m$(商), $\cot \alpha =\dfrac m{\sqrt {1-m^2}}$(倒).
(2)若$\alpha$在第二、四象限, 则$\sec \alpha =\dfrac 1m$, $\sin \alpha =-\sqrt {1-m^2}$, $\csc \alpha =-\dfrac 1{\sqrt {1-m^2}}$, $\tan \alpha =-\dfrac{\sqrt {1-m^2}}m$, $\cot \alpha =-\dfrac m{\sqrt {1-m^2}}$.
\item 三角恒等式的证明.
证明三角恒等式的常用方法有:
(1)切、割化弦.将正切和余切、正割和余割化为正弦和余弦.
\item 求证: $\dfrac{1-\tan ^2x}{1+\tan ^2x}=\cos ^2x-\sin ^2x$.
证明  $\because \dfrac{1-\tan ^2x}{1+\tan ^2x}=\dfrac{1-\dfrac{\sin ^2x}{\cos ^2x}}{1+\dfrac{\sin ^2x}{\cos ^2x}}=\dfrac{\cos ^2x-\sin ^2x}{\cos ^2x+\sin ^2x}=\cos ^2x-\sin ^2x$, 左边$=$右边,
$\therefore$原式成立.
(2)正、余互化.若要证之式仅含正切、余切, 则常可将正切(余切)化为余切(正切).
\item 求证: $\dfrac{\tan \alpha }{\tan \alpha -\tan \beta }=\dfrac{\cot \beta }{\cot \beta -\cot \alpha }$.
证明  $\because \dfrac{\cot \beta }{\cot \beta -\cot \alpha }=\dfrac{\cot \beta (\tan \alpha \tan \beta)}{(\cot \beta -\cot \alpha)(\tan \alpha \tan \beta)}=\dfrac{\tan \alpha (\cot \beta \tan \beta)}{\tan \alpha (\cot \beta \tan \beta)-(\cot \alpha \tan \alpha)\tan \beta }$
$=\dfrac{\tan \alpha }{\tan \alpha -\tan \beta }$, 左边$=$右边,
$\therefore$原式成立.
有时利用公式$\sin (90^\circ -\alpha)=\cos \alpha$, $\cos (90^\circ -\alpha)=\sin \alpha$, $\tan (90^\circ -\alpha)=\cot \alpha$, $\cot (90^\circ -\alpha)=\tan \alpha$, 也可进行``正、余互化''.
\item 求证: $\sin ^21^\circ +\sin ^22^\circ +\cdots +\sin ^289^\circ =\dfrac{89}2$.
证明  $\because \sin ^289^\circ =\cos ^21^\circ$, $\sin ^288^\circ =\cos ^22^\circ$, …, $\sin ^246^\circ =\cos ^244^\circ$,
$\therefore$左边$=(\sin ^21^\circ +\cos ^21^\circ)+(\sin ^22^\circ +\cos ^22^\circ)+\cdots +(\sin ^244^\circ +\cos ^244^\circ)+\sin ^245^\circ$
\blank{50}$=44+\dfrac 12=\dfrac{89}2$.
$\therefore$原式成立.
(3)以``1''逆代.
由于$1=\sin ^2\alpha +\cos ^2\alpha =\sec ^2\alpha -\tan ^2\alpha =\csc ^2\alpha -\cot ^2\alpha$
\blank{50}$=\tan \alpha \cdot \cot \alpha =\sin \alpha \cdot \csc \alpha =\cos \alpha \cdot \sec \alpha =\tan \dfrac{\pi }4=\cdots$,
故常可将``隐含的1''用以上各式逆代.
\item 求证$\dfrac{1+2\sin \alpha \cos \alpha \cos ^2\alpha -\sin ^2\alpha =\dfrac 1+\tan \alpha }{1-\tan \alpha }$.
证明  $\because$左边$=\dfrac{\sin ^2\alpha +2\sin \alpha \cos \alpha +\cos ^2\alpha \cos ^2\alpha -\sin ^2\alpha =\dfrac (\cos \alpha +\sin \alpha)^2}{(\cos \alpha +\sin \alpha)(\cos \alpha -\sin \alpha)}=\dfrac{\cos \alpha +\sin \alpha }{\cos \alpha -\sin \alpha }$
$=\dfrac{1+\tan \alpha }{1-\tan \alpha }$,
$\therefore$原式成立.
\item 求证$\dfrac{1+\sec \alpha +\tan \alpha }{1+\sec \alpha -\tan \alpha }=\dfrac{1+\sin \alpha }{\cos \alpha }$.
证明  $\because \dfrac{1+\sec \alpha +\tan \alpha }{1+\sec \alpha -\tan \alpha }=\dfrac{(\sec ^2\alpha -\tan ^2\alpha)+(\sec \alpha +\tan \alpha)}{\sec \alpha +1-\tan \alpha }$
\blank{50}$=\dfrac{(\sec \alpha +\tan \alpha)(\sec \alpha -\tan \alpha)+(\sec \alpha +\tan \alpha)}{\sec \alpha +1-\tan \alpha }$
\blank{50}$=\dfrac{(\sec \alpha +\tan \alpha)(\sec \alpha -\tan \alpha +1)}{\sec \alpha +1-\tan \alpha }=\sec \alpha +\tan \alpha =\dfrac 1{\cos \alpha }+\dfrac{\sin \alpha }{\cos \alpha }=\dfrac{1+\sin \alpha }{\cos \alpha }$,
$\therefore$原式成立.
\item 关于正弦、余弦的``齐次题型''的解题方法, 形如
$a\sin \theta +b\cos \theta$, 							\textcircled{1}
$a\sin ^2\theta +b\sin \theta \cos \theta +c\cos ^2\theta$, 				\textcircled{2}
$\dfrac{a\sin \theta +b\cos \theta }{a'\sin \theta +b'\cos \theta }$, 							\textcircled{3}
$\dfrac{a\sin ^2\theta +b\sin \theta \cos \theta +c\cos ^2\theta }{a'\sin ^2\theta +b'\sin \theta \cos \theta +c'\cos ^2\theta }$				\textcircled{4}
的式子, 分别称为正弦、余弦的齐次式或齐次分式.
已知$\tan \theta$的值, 欲求以上各式的值, 可按下述方法(实质是以``1''代换〉求得:
\textcircled{1} 式乘以$\dfrac{\cos \theta }{\cos \theta }$, 得到$(a\tan \theta +b)\cos \theta$, 然后求$\cos \theta$即可;
\textcircled{3} 式分子、分母同除以$\cos \theta$, 得到$\dfrac{a\tan \theta +b}{a'\tan \theta +b'}$;
\textcircled{4} 式分子、分母同除以$\cos ^2\theta$, 得到$\dfrac{a\tan ^2\theta +b\tan \theta +c}{a'\tan ^2\theta +b'\tan \theta +c'}$;
\textcircled{2} 式可先化为$\dfrac{a\sin ^2\theta +b\sin \theta \cos \theta +c\cos ^2\theta }{\sin ^2\theta +\cos ^2\theta }$, 然后得到$\dfrac{a\tan ^2\theta +b\tan \theta +c}{\tan ^2\theta +1}$.
\item 已知$\tan \theta =-3$, 求下列各式的值:
(1)$3\sin \theta +\cos \theta$.						(2)$\sin ^2\theta -2\sin \theta \cos \theta +1$.
解  (1)$3\sin \theta +\cos \theta =\cos \theta (3\tan \theta +1)=\pm \dfrac 1{\sqrt {10}}(-9+1)=\pm \dfrac 8{\sqrt {10}}=\pm \dfrac 45\sqrt {10}$.
(2)$\sin ^2\theta -2\sin \theta \cos \theta +1=\dfrac{2\sin ^2\theta -2\sin \theta \cos \theta +\cos ^2\theta }{\sin ^2\theta +\cos ^2\theta }=\dfrac{2\tan ^2\theta -2\tan \theta +1}{\tan ^2\theta +1}$
\blank{50}$=\dfrac{18+6+1}{9+1}=\dfrac 52$.
【训练题】
(一)角的概念的推广
1在``\textcircled{1} 160°, \textcircled{2} 480°, \textcircled{3} -960°, \textcircled{4} -1600°''这四个角中, 属于第二象限的角有\bracket{20}.
\fourch{\textcircled{1} }{\textcircled{1} \textcircled{2} }{\textcircled{1} \textcircled{2} \textcircled{3} }{\textcircled{1} \textcircled{2} \textcircled{3} \textcircled{4} }
\item 集合$M=\{\alpha|\alpha =k\cdot 90^\circ ,k\in \mathbf{N}\}$中各角的终边都在\bracket{20}.
\fourch{$x$轴的正半轴上}{$y$轴的正半轴上}{$x$轴或$y$轴上}{$x$轴正半轴或$y$轴的正半轴上}
\item 若$\alpha$是第四象限的角, 则$\pi -\alpha$是\bracket{20}.
\fourch{第一象限的角}{第二象限的角}{第三象限的角}{第四象限的角}
\item 若一圆弧长等于其所在圆的内接正三角形的边长, 则其圆心角的弧度数为\bracket{20}.
\fourch{$\dfrac{\pi }3$}{$\dfrac 23\pi$.			(D)$\sqrt 3$.		(D)2.
\item 若$\alpha$和$\beta$的终边关于$y$轴对称, 则必有\bracket{20}.
(A)$\alpha +\beta =\dfrac{\pi }2$						(B)$\alpha +\beta =(2k+\dfrac 12)\pi$($k\in \mathbf{Z}$)}{$\alpha +\beta =2k\pi$($k\in \mathbf{Z}$)}{$\alpha +\beta =(2k+1)\pi$($k\in \mathbf{Z}$)}
\item 若$-\dfrac{\pi }2<\alpha <\beta <\dfrac{\pi }2$, 则$\alpha -\beta$的取值范围是\bracket{20}.
\fourch{$(-\dfrac{\pi }2,0)$}{$(-\dfrac{\pi }2,\dfrac{\pi }2)$}{$(-\pi ,0)$}{$(-\pi ,\pi)$)}
\item 集合$M=\{x|x=\dfrac{k\pi }2\pm \dfrac{\pi }4,k\in \mathbf{Z}\}$与$P=\{x|x=\dfrac{k\pi }4,k\in \mathbf{Z}\}$之间的关系是\bracket{20}.
\fourch{$M\subset P$}{$M\supset P$}{$M=P$}{$M\cap P=\varnothing$}
\item (1)与-45°角终边相同的角的集合是\blank{50}.
(2)若$\alpha$是第四象限的角, 则$\alpha$的取值范围是\blank{50}.
(3)终边落在$x$轴负半轴上的角的集合为\blank{50}.
(4)终边落在第一、三象限角平分线上的角的集合为\blank{50}.
(5)若角$\alpha$与$\beta$的终边是互为反向延长线, 则$\alpha$, $\beta$之间满足关系式是\blank{50}.
(6)若角$\alpha$的终边和函数$y=-|x|$的图象重合, 则$\alpha$的集合是\blank{50}.
\item (1)若$\alpha$是第二象限的角, 则$\dfrac{\alpha }2$是第\blank{50}象限的角, $2\alpha$是第\blank{50}象限的角.
(2)若$\alpha =-4$, 则$\alpha$是第\blank{50}象限的角.
\item (1)在-720°与720°之间, 与60°角终边相同的角是\blank{50}.
(2)设角$\alpha$的终边与$\dfrac 75\pi$的终边关于$y$轴对称, 且$\alpha \in (-2\pi ,2\pi)$, 则$\alpha =$\blank{50}.
\item (1)在扇形$OAB$中, 已知半径$OA=8$cm, $\overset\frown{AB}=12$cm, 则圆心角$\angle AOB=$		\blank{50}弧度, 扇形$OAB$的面积为\blank{50}cm2.
(2)若3弧度的圆心角所对的弧长为9cm, 则此圆心角所夹的扇形面积为\blank{50}cm2.
(3)若圆中的一条弦长等于其半径$r$, 则此弦和劣弧所组成的弓形的面积等于\blank{50}.
(4)若1弧度的圆心角所对的弦长为2, 则此圆心角所夹的扇形的面积等于\blank{50}.
\item 若集合$A=\{x|k\pi +\dfrac{\pi }3\le x<k\pi +\dfrac{\pi }2,k\in \mathbf{Z}\}$, $B=\{x|4-x^2\ge 0\}$, 则$A\cap B=$\blank{50}.
\item 已知扇形的周长为30cm, 当它的半径和圆心角各取什么值时, 扇形的面积最大? 最大面积是多少?
\item 已知一扇形的圆心角是120°, 求此扇形面积与其内切圆面枳之比.
\item 在1时15分时, 时针和分针所成的最小正角是多少弧度?
(二)任意角的三角函数
\item 若角$\alpha$的终边落在直线$y=2x$上, 则$\sin \alpha$的値等于\bracket{20}.
\fourch{$\pm \dfrac 15$}{$\pm \dfrac{\sqrt 5}5$}{$\pm \dfrac 25\sqrt 5$}{$\pm \dfrac 12$}
\item 若点$P(3,y)$在角$\alpha$的终边上, 且满足$y<0$, $\cos \alpha =\dfrac 35$, 则$\tan \alpha$的值等于\bracket{20}.
\fourch{$-\dfrac 34$}{$\dfrac 43$}{$\dfrac 34$}{$-\dfrac 43$}
\item 若三角形的两内角$\alpha ,\beta$满足$\sin \alpha \cdot \cos \beta <0$, 则此三角形的形状\bracket{20}.
\fourch{是锐角三角形}{是钝角三角形}{是直角三角形}{不能确定}
\item 若$\alpha$是第三象限角, 则下列各式中不成立的是\bracket{20}.
\fourch{$\sin \alpha +\cos \alpha <0$}{$\tan \alpha -\sin \alpha <0$}{$\cos \alpha -\cot \alpha <0$}{$\cot \alpha \cdot \csc \alpha <0$}
\item 下列四个命题中, 正确的是\bracket{20}.
\onech{终边相同的角的三角函数值相等}{$\{\alpha|\alpha =k\pi +\dfrac{\pi }6,k\in \mathbf{Z}\}\ne \{\beta|\beta =-k\pi +\dfrac{\pi }6,k\in \mathbf{Z}\}$}{若$\alpha$是第二象限角, 则$\sin 2\alpha <0$}{第四象限的角可表示为$\{\alpha|2k\pi +\dfrac 32\pi <\alpha <2k\pi ,k\in \mathbf{Z}\}$}
\item 若$\theta$是第三象限角.且$\cos \dfrac{\theta }2<0$.则$\theta$是\bracket{20}.
\fourch{第一象限角}{第二象限角}{第二象限角}{第四象限角}
\item 若$(\dfrac 12)^{\sin 2\theta }<1$, 则$\theta$是\bracket{20}.
\fourch{第一或第二象限角}{第二或第四象限角}{第一或第三象限角}{第二或第三象限角}
\item (1)直角坐标平面内, 终边过点$(1-\sqrt 3)$的所有角组成的集合可表示成\blank{50}.
(2)若角$\alpha$的终边上有一点$P(-3,a)$, 且$\cos \alpha =-\dfrac 35$, 则$a=$\blank{50}.
(3)若点$P(-\sqrt 3,m)$是角$\theta$终边上一点, 且$\sin \theta =\dfrac{\sqrt {13}}{13}$, 则$m=$\blank{50}.
(4)若点$P(-\sqrt 2,-\sqrt 3)$在角$\alpha$的终边上, 则$\sin \alpha -\cos \alpha =$\blank{50}.
\item (1)$\dfrac{\sin x}{|\sin x|}+\dfrac{|\cos x|}{\cos x}+\dfrac{\tan x}{|\tan x|}+\dfrac{|\cot x|}{\cot x}$的取值范围是\blank{50}.
(2)若$\sin \alpha \cdot \cos \alpha >0$, 则$\alpha$的取值范围(用区间表示)是\blank{50}.
\item (1)若$x$为三角形的内角, 则当$x=$\blank{50}时, $\dfrac{\sin \dfrac x2}{1-\tan x}$无意义.
(2)若函数$f(x)$的定义域是[0, 1], 则$f(\sin x)$的定义域是\blank{50}.
(3)函数$y=\sqrt {\cos x}$的定义域是\blank{50}.
(4)函数$y=\sqrt {-\cot x}+\lg \cos x$的定义域是\blank{50}.
(5)函数$y=\sqrt {\sin x}+\sqrt {-\tan x}$的定义域是\blank{50}.
\item 若实数$\alpha ,\beta$满足$|\cos \alpha -\cos \beta|=|\cos \alpha|+|\cos \beta|$, 且$\alpha \in (\dfrac{\pi }2,\pi)$, 则化简$\sqrt {(\cos \alpha -\cos \beta)^2}$结果是\bracket{20}.
\fourch{$\cos \alpha -\cos \beta$}{$|\cos \alpha|-|\cos \beta|$}{$\cos \beta -\cos \alpha$}{$|\cos \beta|-|\cos \alpha|$}
\item (1)已知角$\alpha$终边上—点$P$的坐标是$(5a,12a)$($a<0$), 求角$\alpha$的各二角函数值.
(2)已知角$\alpha$终边上一点$P$与$x$轴的距离和与轴的距离之比为$4:3$, 且$\cos \alpha <0$.求$\sin \alpha$和$\tan \alpha$.
\item 求下列函数的定义域:
(1)$y=\sqrt {\sin (\cos x)}$.					(2)$y=\sqrt {\cos (\sin x)}$.
(三)同角三角函数的基本关系式
\item 下列四个命题中.能够成立的是\bracket{20}.
\fourch{$\sin \alpha =\dfrac 12$且$\cos \alpha =\dfrac 12$}{$\sin \alpha =\dfrac 13$且$\csc \alpha =2$}{$\sin \alpha =0$且$\cos \alpha =-1$}{$\cos \alpha =\dfrac 12$且$\sec \alpha =-2$}
\item 已知$\sin \alpha =\dfrac 45$, 且$\alpha$是第二象限的角, 那么$\tan \alpha$的值等于\bracket{20}.
\fourch{$-\dfrac 34$}{$-\dfrac 43$}{$\dfrac 34$			(〇)$\dfrac 43$.
\item 若$1+\sin \theta \sqrt {1-\cos ^2\theta }+\cos \theta \sqrt {1-\sin ^2\theta }=0$.则$\theta$的取值范围是\bracket{20}.
(A)第三象限角.							(B)第四象限角.
(C)$2k\pi \le \theta \le 2k\pi +\dfrac 32\pi$($k\in \mathbf{Z}$)}{$2k\pi +\dfrac 32\pi \le \theta \le 2k\pi +2\pi$($k\in \mathbf{Z}$)}
\item 若$\alpha$是二角形的一个内角, 且$\sin \alpha +\cos \alpha =\dfrac 23$, 则这个三角形的形状是\bracket{20}.
\fourch{锐角三角形}{钝角三角形}{不等腰的直角三角形}{等腰直角三角形}
\item 化简$(\dfrac 1{\sin \alpha }+\dfrac 1{\tan \alpha })(1-\cos \alpha)$的结果是\bracket{20}.
\fourch{$\sin \alpha$}{$\cos \alpha$}{$1+\sin \alpha$}{$1+\cos \alpha$}
\item 若$\theta \ne \dfrac{k\pi }2$($k\in \mathbf{Z}$), 则$\dfrac{\sin \theta +\tan \theta }{\cos \theta +\cot \theta }$\bracket{20}.
\fourch{恒取正值}{恒取负值}{恒取非正值}{恒取非负值}
\item 若$0<\alpha <\dfrac{\pi }2$, 且$\lg (1+\cos \alpha)=m$, $\lg \dfrac 1{1-\cos \alpha }=n$, 则$\lg \sin \alpha =$的值等于\bracket{20}.
\fourch{$m+\dfrac 1n$}{$m-n$}{$\dfrac 12(m+\dfrac 1n)$}{$\dfrac 12(m-n)$}
36若$\dfrac{\sin ^2\theta +4}{\cos \theta +1}=2$, 则$(\cos \theta +3)(\sin \theta +1)$的值是\bracket{20}.
\fourch{6}{4}{2}{0}
\item 若$\sin \theta \cdot \cos \theta <0$, $|\cos \theta|=\cos \theta$, 则点$P(\tan \theta ,\sec \theta)$—定在\bracket{20}.
\fourch{第一象限}{第二象限}{第三象限}{第四象限}
\item 若$\sqrt {\dfrac{1-\sin x}{1+\sin x}}=\tan x-\sec x$, 则$x$的取值范围是\bracket{20}.
\fourch{$2k\pi +\dfrac{\pi }2<x<2k\pi +\dfrac{3\pi }2$($k\in \mathbf{Z}$)}{$k\pi +\dfrac{\pi }2<x<k\pi +\dfrac{3\pi }2$($k\in \mathbf{Z}$)}{$2k\pi <x<2k\pi +\pi$($k\in \mathbf{Z}$)}{$2k\pi -\dfrac{\pi }2<x<2k\pi +\dfrac{\pi }2$($k\in \mathbf{Z}$)}
\item 若$\alpha \in (0,2\pi)$, 则适合$\sqrt {\dfrac{1+\cos \alpha }{1-\cos \alpha }}-\sqrt {\dfrac{1-\cos \alpha }{1+\cos \alpha }}=2\cot \alpha$的角$\alpha$的集合是\bracket{20}.
\fourch{$\{\alpha|0<\alpha <\pi\}$}{$\{\alpha|0<\alpha <\dfrac{\pi }2\pi <\alpha <\dfrac{3\pi }2\}$}{$\{\alpha|0<\alpha <\pi \alpha =\dfrac{3\pi }2\}$}{$\{\alpha|0<\alpha <\dfrac{\pi }2\dfrac{3\pi }2<\alpha <2\pi\}$}
\item (1)若角$\alpha$的终边过点$(1,\tan \theta)$, 且$\theta \in (\dfrac{\pi }2,\pi)$, 则$\sin \alpha =$\blank{50}.
(2)若$\sin \alpha +\cos \alpha =\dfrac 13$, 则$\sin \alpha \cos \alpha =$\blank{50}.
(3)化简$\sin ^2\alpha +\cos ^2\alpha \sin ^2\beta +\cos ^2\alpha \cos ^2\beta =$\blank{50}.
(4)化简$\sin ^2\alpha +\sin ^2\beta -\sin ^2\alpha \sin ^2\beta +\cos ^2\alpha \cos ^2\beta =$\blank{50}.
(5)化简$\sin ^6\alpha +\cos ^6\alpha +3\sin ^2\alpha \cos ^2\alpha =$\blank{50}.
(6)若$\theta$是第二象限角, 且$\sin \theta =\dfrac{m-3}{m+5}$, $\cos \theta =\dfrac{1-2m}{m+5}$, 则$m=$\blank{50}.
\item (1)计算下列各式:
\textcircled{1} $\tan \alpha (1-\cot ^2\alpha)+\cot \alpha (1-\tan ^2\alpha)=$\blank{50};
\textcircled{2} $(\sec ^2\beta -1)(1-\csc ^2\beta)+\tan \beta \cot \beta =$\blank{50};
\textcircled{3} $(\sec \alpha -\cos \alpha)(\csc \alpha -\sin \alpha)(\tan \alpha +\cot \alpha)\text=$\blank{50}.
(2)若$\alpha \in (-\dfrac 43\pi -\dfrac 54\pi)$, 则$\dfrac{\sin \alpha }{|\sin \alpha|}+\dfrac{|\cos \alpha|}{\cos \alpha }+\tan \alpha|\cot \alpha|=$\blank{50}.
(3)若$\theta$是第四象限的角, 则$\dfrac 1{\cos \theta \sqrt {1+\tan ^2\theta }}+\dfrac{2\cot \theta }{\sqrt {\dfrac 1{\sin ^2\theta }-1}}=$\blank{50}
(4)若$\cot \theta +\csc \theta =5$, 则$\sin \theta =$\blank{50}.
(5)若$\sin \alpha +\cos \alpha =\dfrac{\sqrt 3}3$, 则$\tan \alpha +\cot \alpha =$\blank{50}.
(6)若$\cot \alpha +\tan \alpha =\dfrac{25}{12}$, 则$\tan \alpha -\cot \alpha =$\blank{50}.
\item (1)$\tan x=2$, 则\textcircled{1} $\dfrac 1{1-\sin x}+\dfrac 1{1+\sin x}=$\blank{50};
\textcircled{2} $\dfrac 1{(\sin x-3\cos x)(\cos x-\sin x)}=$\blank{50};
\textcircled{3} $\dfrac 14\sin ^2x+\dfrac 23\cos ^2x=$\blank{50}.
(2)若$\dfrac{2\sin ^2\alpha -3\cos ^2\alpha }{\cos ^2\alpha -\sin ^2\alpha }=-4$, 则$\tan \alpha =$\blank{50}.
(3)若$(\sin \alpha +\cos \alpha)^2=\dfrac 85$, 则$\tan \alpha =$\blank{50}.
\item 若$\tan \alpha$和$\tan \beta$是关于$x$的方程$x^2-px+q=0$的两根, $\cot \alpha$和$\cot \beta$是关于$x$的方程$x^2-rx+s=0$的两根, 则$rs$等于\bracket{20}.
\fourch{$pq$}{$\dfrac 1{pq}$}{$\dfrac p{q^2}$}{$\dfrac q{p^2}$}
\item 若$\sin x=\dfrac{a-b}{a+b}$($0<a<b$), 则$\sqrt {\cot ^2x-\cos ^2x}$的结果是\bracket{20}.
\fourch{$\dfrac{4ab}{a^2-b^2}$}{$-\dfrac{4ab}{a^2-b^2}$}{$\dfrac{4ab}{a^2+b^2}$}{$-\dfrac{4ab}{a^2+b^2}$}
\item 若$\alpha$在第一象限, 且$\dfrac{1+\tan \alpha }{1-\tan \alpha }=3+2\sqrt 2$, 则$\cos \alpha$的值是\bracket{20}.
\fourch{$\dfrac{\sqrt 6}2$}{$\dfrac{\sqrt 6}3$}{$\dfrac{\sqrt 3}2$}{$\dfrac{\sqrt 3}3$}
\item 求下列各式的值:
(1)$(1+\cot \alpha -\csc \alpha)(1+\tan \alpha +\sec \alpha)$.		(2)$\dfrac{1-\sin ^6\alpha -\cos ^6\alpha }{\sin ^2\alpha -\sin ^4\alpha }$.
(3)$\dfrac{1-\sin ^4\alpha -\cos ^4\alpha }{1-\sin ^6\alpha -\cos ^6\alpha }$.
\item 求证下列各式:
(1)$\dfrac{\tan \alpha -\cot \alpha }{\sec \alpha -\csc \alpha }=\sin \alpha +\cos \alpha$.		(2)$\dfrac{\sin ^2\alpha }{1+\cot \alpha }+\dfrac{\cos ^2\alpha }{1+\tan \alpha }=1-\sin \alpha \cos \alpha$
(3)$(\dfrac{\sin \theta +\tan \theta }{\csc \theta +\cot \theta })^2=\dfrac{\sin ^2\theta +\tan ^2\theta }{\csc ^2+\cot ^2\theta }$.
\item 利用``1''的代换证明下列各题:
(1)$\dfrac{1-2\cos ^2\alpha }{\sin \alpha \cos \alpha }=\tan \alpha -\cot \alpha$.			(2)$\dfrac{\cot \alpha +\csc \alpha -1}{\cot \alpha -\csc \alpha +1}=\cot \alpha +\csc \alpha$.
(3)$\tan \alpha \cdot \dfrac{1-\sin \alpha }{1+\cos \alpha }=\cot \alpha \cdot \dfrac{1-\cos \alpha }{1+\sin \alpha }$.
\item (1)已知$\sin \theta +\cos \theta =\sqrt 2$, 求$\sin \theta -\cos \theta$的值.
(2)已知$\sin \theta -\cos \theta =\dfrac{\sqrt 2}3$($0<\theta <\dfrac{\pi }2$), 求$\sin \theta +\cos \theta$的值.
(3)已知$\sin \theta +m\cos \theta =n$, 求$m\sin \theta -\cos \theta$的值.
(4)已知$\sin \theta +\sin ^2\theta =1$, 求$\cos ^2\theta +\cos ^4\theta =1$的值.
(5)已知$\cos A=\cos \theta \cdot \sin C$, $\cos B=\sin \theta \cdot \sin C$($C\ne k\pi$, $k\in \mathbf{Z}$), 求$\sin ^2A+\sin ^2B+\sin ^2C$的值.
\item (1)已知$\tan \theta =\sqrt {\dfrac{1-a}a}$($0<a<1$), 求$\dfrac{\sin ^2\theta }{a+\cos \theta }+\dfrac{\sin ^2\theta }{a-\cos \theta }$的值.
(2)已知锐角$\theta$满足$\log _{(\tan \theta +\cot \theta)}\sin \theta =-\dfrac 34$, 求$\log _{\tan \theta }\cos \theta$的值.
(四)诱导公式
\item 若$\sin (\pi +\alpha)=-\dfrac 35$, 则\bracket{20}.
\fourch{$\cos \alpha =\dfrac 45$}{$\tan \alpha =\dfrac 34$}{$\sec \alpha =-\dfrac 54$}{$\sin (\pi -\alpha)=\dfrac 35$}
\item 若$4\pi <\alpha <5\pi$, $\cos \alpha =-\dfrac 13$, 则$\tan \alpha$的值为\bracket{20}.
\fourch{$-2\sqrt 2$}{$\pm 2\sqrt 2$}{$\pm \dfrac{\sqrt 2}4$}{$-\dfrac{\sqrt 2}4$}
\item 下列各式正确的是\bracket{20}.
\fourch{$\cos ^3(-\alpha -\pi)=\cos ^3\alpha$}{$\sin (\alpha -3\pi)=\sin \alpha$}{$\sec (3\pi -\alpha)=\dfrac 1{\cos \alpha }$}{$-\cot (5\pi -2\alpha)=\cot 2\alpha$}
\item 若$\alpha ,\beta ,\gamma$是一个三角形的三个内角, 则在``\textcircled{1} $\sin (\alpha +\beta)-\sin \gamma$, \textcircled{2} $\cos (\alpha +\beta)+\cos \gamma$, \textcircled{3} $\tan \dfrac{\alpha +\beta }2\cdot \tan \dfrac{\gamma }2$, \textcircled{4} $\tan (\alpha +\beta)-\tan \gamma$''这四个式子中, 其值为常数的有\bracket{20}.
\fourch{1个}{2个}{3个}{4个}
\item 函数$y=\cos (\tan x)$\bracket{20}.
\fourch{是奇函数, 但不是偶函数}{是偶函数, 但不是奇函数}{既不是奇函数, 也不是偶函数}{奇偶性无法确定}
\item 若函数$f(x)=a\sin x+b\tan x+1$满足$f(5)=7$, 则$f(-5)$的值等于\bracket{20}.
\fourch{5}{-5}{6}{-6}
\item 化简$\tan (\dfrac{k\pi }2+\alpha)$($k\in \mathbf{Z}$)的结果是\bracket{20}.
\fourch{$\tan \alpha$}{$\pm \tan \alpha$}{$\tan \alpha$或$-\cot \alpha$}{$\tan \alpha$或$\cot \alpha$}
\item 计算下列各题:
(1)$\sin ^220^\circ +\sin ^270^\circ -\cos ^220^\circ \cdot \cot ^270^\circ \cdot \csc ^220^\circ =$\blank{50}.
(2)$\tan 1^\circ \cdot \tan 2^\circ \cdot \tan 3^\circ \cdot \cdots \cdot \tan 87^\circ \cdot \tan 88^\circ \cdot \tan 89^\circ =$\blank{50}.
(3)$\sin ^2(42^\circ +\alpha)+\cot (25^\circ +\beta)\cdot \cot (\beta -65^\circ)+\sin ^2(48^\circ -\alpha)=$\blank{50}.
(4)$\log _4\sin \dfrac 34\pi +\log _9\tan (-\dfrac{5\pi }6)=$\blank{50}.
(5)$\tan \dfrac{\pi }5+\tan \dfrac{2\pi }5+\tan \dfrac{3\pi }5+\tan \dfrac{4\pi }5=$\blank{50}.
\item 若锐角$\alpha$终边上一点$A$的坐标为$(2\sin 3,-2\cos 3)$, 则角$\alpha$的弧度数为\blank{50}.
\item 化简下列各式:
(1)$\dfrac{\sin (\pi +\alpha)\cos (\pi -\alpha)\tan (-\alpha +3\pi)}{\sin (5\pi -\alpha)\tan (8\pi -\alpha)\cot (\alpha -3\pi)}$.	
(2)$\dfrac{\sin (\theta -\pi)\cos (\theta -\dfrac 32\pi)\cot (-\theta -\pi)}{\tan (\theta +3\pi)\sec (-\theta -2\pi)\csc (\dfrac{\pi }2-\theta)}$.
\item 若三角形中的两内角$\alpha ,\beta$满足$\sin 2\alpha =\sin 2\beta$, 则这个三角形的形状\bracket{20}.
\fourch{只可能是等腰三角形.不可能是直角三角形}{只可能是直角三角形, 不可能是等腰三角形}{只可能是等腰直角三角形}{既可能是等腰三角形, 也可能是直角三角形}
\item 若函数$f(x)$满足, $f(\cos x)=\dfrac x2$($0\le x\le \pi$), 则$f(-\dfrac 12)$等于\bracket{20}.
\fourch{$\cos \dfrac 12$}{$\dfrac{\pi }3$}{$\dfrac{\pi }4$}{$\dfrac{\pi }2$}
\item 若函数$f(x)=a\sin (\pi x+\alpha)+b\cos (\pi x+\beta)$, 其中$a,b,\alpha ,\beta$都是非零实数, 且满足$f(1997)=-1$, 则$f(1998)$等于\bracket{20}.
\fourch{-1}{0}{1}{2}
\item (1)已知$\cos (\dfrac{\pi }6-\theta)=a$($|a|\le 1$), 求$\cos (\dfrac{5\pi }6+\theta)$和$\sin (\dfrac{2\pi }3-\theta)$的值.
(2)已知$\tan (\pi -\alpha)=a^2$, $|\cos (\pi -\alpha)|=-\cos \alpha$, 求$\sec (\pi +\alpha)$的值.
\item (1)求满足$\sin (\dfrac{\pi }4-\alpha)=\dfrac{\sqrt 2}2$, $\alpha \in (0,2\pi)$的角$\alpha$.
(2)求$\dfrac{\sin (k\pi -x)}{\sin x}-\dfrac{\cos x}{\cos (k\pi -x)}+\dfrac{\tan (k\pi -x)}{\tan x}-\dfrac{\cot x}{\cot (k\pi -x)}$($k\in \mathbf{Z}$)的取值范围.
二、三角函数的图象和性质
【典型题型和解题技巧】
\item $f(x)=a\sin ^2x+b\sin x+c$($a\ne 0$)型函数的值域.
\item 求函数$y=-2\sin ^2x+2\sin x+1$的值域.
解  $y=-2(\sin x-\dfrac 12)^2+\dfrac 32$.
考虑到$-1\le \sin x\le 1$, 因此, 若以$\sin x$为横轴, 则函数图象应足拋物线夹在两直线$\sin x=\pm 1$之间的一段(如图2).观察图象易知$y_{\max }=\dfrac 32$, $y_{\min }=-3$
$\therefore$函数的值域是$-3\le y\le \dfrac 32$.
(图2)
注意  此例属于复合函数的问题.请读者注意, 高中阶段有关的函数问题, 常常以与二次函数有关的复合函数的题型出现, 解此类问题时, 应记住``配方, 画图, 截断''三个步驟.
\item 已知$0\le x\le \dfrac{\pi }2$, 求函数$y=\cos ^2x-2a\cos x$的最大值$M(a)$与最小值$m(a)$.
解  函数$y=f(\cos x)=(\cos x-a)^2-a^2$, 又$0\le x\le \dfrac{\pi }2$,
$\therefore 0\le \cos x\le 1$, 画出函数的图象如下:
\blank{50}(1)\blank{50}(2)\blank{50}(3)\blank{50}(4)
(图3)
(1)如图3(1), 此时$a<0$, $m(a)=f(0)=0$, $M(a)=f(1)=1-2a$.
(2)如图3(2), 此时$0\le a\le \dfrac 12$, $m(a)=f(a)=-a^2$, $M(a)=f(1)1-2a$, .
(3)如图3(3), 此时$\dfrac 12\le a<1$, $m(a)=f(a)=-a^2$, $M(a)=f(a)=0$.
(4)如图3(4), 此时$a\ge 1$, $m(a)=f(1)=1-2a$, $M(a)=f(0)=0$.
综上所述, 可得$M(a)=\begin{cases}
    1-2a  (a<\dfrac 12),  \\0  (a\ge \dfrac 12),  \end{cases} m(a)=\begin{cases}
    0  (a<0),  \\-a^2  (0\le a<1),  \\1-2a  (a\ge 1).  \end{cases}$
\item $f(x)=\dfrac{a\sin x+b}{a'\sin x+b'}$型函数的值域.
求此类函数的值域, 可按去分母、反表示(即表示成$\sin x=\dfrac{cy+d}{c'y+d'}$和解不等式$(\dfrac{cy+d}{c'y+d'}\le 1)$三个步骤求解.
\item 求函数$y=\dfrac{2\sin x-1}{\sin x+3}$的值域.
解  由已知, 得$\sin x=\dfrac{3y+1}{2-y}$, 而$|\sin x|\le 1$, 故$|\dfrac{3y+1}{2-y}|\le 1$,
即$8y^2+10y-3\le 0$, $(4y-1)(2y+3)\le 0$.  $\therefore$函数的值域是$y\in [-\dfrac 32,\dfrac 14]$.
\item $f(x)=\dfrac{a\tan ^2x+b\tan x+c}{a'\tan ^2x+b'\tan x+c'}$型函数的值域.
由于$\tan x\in \mathbf{R}$, 故此类问题与$\dfrac{at^2+bt+c}{a't^2+b't+c'}$($t\in \mathbf{R}$)类问题相同, 可去分母、移项, 然后利用$\triangle \ge 0$解之.
\item 求函数$y=\dfrac{\sec ^2x-\tan x}{\sec ^2x+\tan x}$的值域.
解  因为$\sec ^2x=\tan ^2x+1$, 故原式时变形为$(y-1)\tan ^2x+(y+1)\tan x+(y-1)=0$.
(1)若$y=1$, 则$\tan x=0$.
(2)若$y\ne 1$, 则$\tan x\in \mathbf{R}$, 得$\triangle =(y+1)^2-4(y-1)^2\ge 0$, 于是$\dfrac 13\le y\le 3$且$y\ne 1$.
综含(1), (2)知, 函数的值域是$y\in [\dfrac 13,3]$.
\item 解简单的三角不等式.
\item 解不等式$\sin x\le \dfrac 12$.
解  在单位圆内绘出$\sin x=\dfrac 12$的正弦线(如图4), 并结合$y=\sin x$的单调性, 可得$2k\pi -\dfrac{7\pi }6\le x\le 2k\pi +\dfrac{\pi }6$($k\in \mathbf{Z}$).
\blank{50}(图4)\blank{50}(图5)
\item 解不等式$|\cos 2x|\le \dfrac 12$.
解  原不等式为$-\dfrac 12\le \cos 2x\le \dfrac 12$.由图5, 可得$k\pi +\dfrac{\pi }3\le 2x\le k\pi +\dfrac{2\pi }3$,
于是$\dfrac{k\pi }2+\dfrac{\pi }6\le x\le \dfrac{k\pi }2+\dfrac{\pi }3$($k\in \mathbf{Z}$).
\item 解不等式$\tan \dfrac x2\ge \sqrt 3$.
解  由图6, 可得$k\pi +\dfrac{\pi }3\le \dfrac x2\le k\pi +\dfrac{\pi }2$,
$\therefore 2k\pi +\dfrac{2\pi }3\le x<2k\pi +\pi$($k\in \mathbf{Z}$).
(图6)
注意  有关$\sin x,\cos x,\tan x$等的简单不等式, 通常可在单位圆中利用三角函数线的知识求解.
\item 函数图象的``平移''和坐标的``伸缩''
(1)图象的``平移''.
容易证明, 函数$y=f(x-m)$($m>0$)的图象可由函数$y=f(x)$的图象向右平移$m$个单位长度得到, 而函数$y=f(x+m)$($m>0$)的图象可由函数$y=f(x)$的图象向左平移$m$个单位长度得到.
\item 在同一个坐标系内, 为了得到$y=3\sin (2x+\dfrac{\pi }4)$的图象, 只需将$y=3\cos 2x$的图象\bracket{20}.
\fourch{向左平移$\dfrac{\pi }4$}{向右平移$\dfrac{\pi }4$}{向左平移$\dfrac{\pi }8$}{向右平移$\dfrac{\pi }8$}
解  令$f(x)=3\cos 2x$, 则
$\begin{cases} f(x-m)=3\cos 2(x-m)=3\cos (2x-2m)=3\cos (2m-2x)=3\sin [\dfrac{\pi }2-(2m-2x)] \\ =3\sin (2x+\dfrac{\pi }2-2m).
\end{cases}$按题意应有$3\sin (2x+\dfrac{\pi }2-2m)=3\sin (2x+\dfrac{\pi }4)$.
令$\dfrac{\pi }2-2m=\dfrac{\pi }4$, 得$m=\dfrac{\pi }8$, 故选D.
也可以这样解:
$\because f(x)=3\sin (2x+\dfrac{\pi }4)=3\cos [(2x+\dfrac{\pi }4)-\dfrac{\pi }2]$
\blank{50}$=3\cos (2x-\dfrac{\pi }4)=3\cos [2(x-\dfrac{\pi }8)]=f(x-\dfrac{\pi }8)$,
$\therefore$选D.
(2)坐标的``伸缩''.
容易证明, 函数$y=f(\dfrac xk)$($k>0$)的图象, 可由将$y=f(x)$图象上每—点的横坐标伸长到原来的$k$倍(纵坐标不变)而得到.
\item 将函数$y=\cos x$图象上每一点的纵坐标保持不变, 横坐标缩小为原来的一半, 再将所得图象沿$x$轴向左平移$\dfrac{\pi }4$个单位长度, 则与所得新图象对应的函数的解析式为\bracket{20}.
\fourch{$y=\cos (2x+\dfrac{\pi }4)$}{$y=\cos (2x-\dfrac{\pi }4)$}{$y=\sin 2x$}{$y=-\sin 2x$}
解  横坐标缩小为原来的一半, 可理解为伸长到原来的$\dfrac 12$, 故先得到函数$y=\cos \dfrac x{\dfrac 12}=\cos 2x$.再向左平移$\dfrac{\pi }4$后, 得$y=\cos 2(x+\dfrac{\pi }4)$, 即$y=\cos (2x+\dfrac{\pi }2)=-\sin 2x$, 故选D.
\item 函数$y=3\sin x$的图象经过怎样的变换后, 可得到$y=3\sin (\dfrac x2-\dfrac{\pi }4)$的图象?
解法一  先``伸缩'', 后``平移''.
第一步: 将函数$y=3\sin x$的图象上的每一点, 纵坐标保持不变, 横坐标伸长到原来的2倍.得到函数$y=3\sin \dfrac x2$的图象.
第二步: 将函数$y=3\sin \dfrac x2$的图象向右平移$\dfrac{\pi }2$个单位长度, 便得到函数$y=3\sin \dfrac 12(x-\dfrac{\pi }2)=3\sin (\dfrac x2-\dfrac{\pi }4)$的图象.
解法二  先``平移'', 后``伸缩''.
第一步: 将函数$y=3\sin x$的图象, 向右平移$\dfrac{\pi }4$个单位长度, 得到函数$y=3\sin (x-\dfrac{\pi }4)$的图象.
第二步: 将函数$y=3\sin (x-\dfrac{\pi }4)$的每一点, 纵坐标保持不变, 横坐标伸长到原来的2倍, 得到函数$y=3\sin (\dfrac x2-\dfrac{\pi }4)$的图象.
\item 函数$y=A\sin (\omega x+\varphi)=\pm 1$($A\ne 0$)的对称轴.
观察图7, 易求出$\sin (\omega x+\varphi)=\pm 1$的解$x_0$, 则直线$x=x_0$便是函数$y=A\sin (\omega x+\varphi)$图象的对称轴.
(图7)
\item 函数$y=\sin (2x+\dfrac{\pi }4)$图象的一条对称轴是直线\bracket{20}.
\fourch{$x=\dfrac{3\pi }4$}{$x=-\dfrac{3\pi }4$}{$x=\dfrac{3\pi }8$}{$x=-\dfrac{3\pi }8$}
解  以$x=-\dfrac{3\pi }8$代入, 得$\sin [1(-\dfrac{3\pi }8)+\dfrac{\pi }4]=\sin (-\dfrac{\pi }2)=-1$, 故选D.
注意  若令$\sin (2x+\dfrac{\pi }4)=\pm 1$, 可得$2x+\dfrac{\pi }4=k\pi +\dfrac{\pi }2$, 即得$x=\dfrac{k\pi }2+\dfrac{\pi }8$($k\in \mathbf{Z}$), 故函数$y=\sin (2x+\dfrac{\pi }4)$图象的对称轴直线的一般形式是$x=\dfrac{k\pi }2+\dfrac{\pi }8$($k\in \mathbf{Z}$).
请读者思考: 若函数$f(x)=3\cos (2x+\varphi)$是偶函数, 则$\varphi$的值应取什么?
【训练题】
(一)正弦函数、余弦函数的图象和性质
\item 若$MP,OM,AT$分别是60°角的正弦线、余弦线和正切线, 则\bracket{20}.
\fourch{$MP<OM<AT$}{$OM<MP<AT$}{$AT<OM<MP$}{$OM<AT<MP$}
\item 在同一坐标系内, 曲线$y=\sin x$与$y=\cos x$的交点坐标是\bracket{20}.
\fourch{$(2k\pi +\dfrac{\pi }2,1)$}{$(k\pi +\dfrac{\pi }2,(-1)^k)$}{$(k\pi +\dfrac{\pi }4,\dfrac{(-1)^k}{\sqrt 2})$}{$(k\pi ,0)$($k\in \mathbf{Z}$)}
\item 函数$y=\log _{\dfrac 12}(\sin 2x)$为减函数的区间是\bracket{20}.
\fourch{$(k\pi ,k\pi +\dfrac{\pi }4]$, $k\in \mathbf{Z}$}{$(k\pi ,k\pi +\dfrac{\pi }2]$, $k\in \mathbf{Z}$}{$(2k\pi ,2k\pi +\dfrac{\pi }4]$, $k\in \mathbf{Z}$}{$(2k\pi ,2k\pi +\dfrac{\pi }2]$, $k\in \mathbf{Z}$}
\item 函数$y=\lg (1-\sin x)-\lg (1+\sin x)$(.)
\fourch{是奇函数, 但非偶函数}{是偶函数, 但非奇函数}{既不是奇函数, 也不是偶函数}{奇偶性无法确定}
\item 若$0<x<\dfrac 12$, 则下列各式不成立的是\bracket{20}.
\fourch{$\sin (1+x)>\sin x$}{$\cos (1+x)<\cos x$}{$(1+x)^x>x^x$}{$\log _x(1+x)>\log _xx$}
\item 若函数$y=\cos (\sin x)$, 则下列结论正确的是\bracket{20}.
\fourch{它的定义域是[-1, 1]}{它是奇函数}{它的值域是$[\cos 1,1]$}{它不是周期函数}
\item 下列四个函数中, 是偶函数且在$[0,\dfrac{\pi }2]$上为增函数, 但不是周期函数的函数是\bracket{20}.
\fourch{$y=|\sin x|$($x\in \mathbf{R}$)}{$y=|\cos x|$($x\in \mathbf{R}$)}{$y=\sin|x|$($x\in \mathbf{R}$)}{$y=|\sin x|+|\cos x|$($x\in \mathbf{R}$)}
\item 下列函数中, 既在$(0,\dfrac{\pi }2)$上是增函数, 又是以$\pi$为最小正周期的偶函数是\bracket{20}.
\fourch{$y=x^2|\cos x|$}{$y=\cos 2x$}{$y=|\sin x|$}{$y=|\sin 2x|$}
\item 要使$\sqrt {(1+2\sin \theta)^2}=-(1+2\sin \theta)$, 则$\theta$的取值范围是\bracket{20}.
\fourch{第三、四象限}{$[2k\pi -\dfrac{5\pi }6,2k\pi -\dfrac{\pi }6]$}{$[2k\pi -\dfrac{\pi }6,2k\pi +\dfrac{7\pi }6]$}{$[2k\pi -\dfrac{7\pi }6,2k\pi -\dfrac{\pi }6]$($k\in \mathbf{Z}$)}
\item (1)设$\cos ^2x+4\sin x-a=0$($a,x\in \mathbf{R}$), 则$a$的取值范围是\blank{50}.
(2)函数$y=1-2\sin x+3\cos ^2x$的值域是\blank{50}.
(3)函数$y=\sin ^2x+2\cos x(-\dfrac{\pi }3\le x\le \dfrac 23\pi)$的值域是\blank{50}.
(4)函数$y=\dfrac{3\cos x+1}{\cos x+2}$的值域是\blank{50}.
(5)函数$f(x)=\log _{\dfrac 12}(2\sin x)$的最小值是\blank{50}.
\item 将下列各数由小到大排列:
(1)$\sin 46^\circ ,\cos 46^\circ ,\cos 36^\circ$:\blank{50}.
(2)$\sin 2,\cos 2,\tan 2$:\blank{50}.
(3)$\log _x\sin \dfrac x2,\log _x\cos \dfrac x2$($0<x<1$):\blank{50}.
(4)$\cos 1^\circ ,\sin 1^\circ ,\cos 1,\sin 1$:\blank{50}.
\item (1)在$[0,2\pi]$中, 满足$\sin x\ge \dfrac 12$的$x$的取值范围是\blank{50}.
(2)不等式$\sin x\le \dfrac 12$的解为\blank{50}.
(3)不等式$|\cos 2x|\le \dfrac 12$的解为\blank{50}.
(4)若集合$M=\{\theta|\sin \theta \ge \dfrac 12,0\le \theta \le \pi\}$, $P=\{\theta|\cos \theta \le \dfrac 12,0<\theta \le \pi\}$, 则$M\cap P=$\blank{50}.
(5)若$-\pi \le x\le \pi$, 则不等式$\log _2(1+2\cos x)<1$的解为\blank{50}.
\item 若锐角$\alpha ,\beta$满足$\sin \alpha <\cos \beta$则\bracket{20}.
\fourch{$\alpha >\beta$}{$\alpha <\beta$}{$\alpha +\beta <\dfrac{\pi }2$}{$\alpha +\beta >\dfrac{\pi }2$}
\item 方程$2^x=\cos x$的解有\bracket{20}.
\fourch{0个}{1个}{2个}{无穷多个}
\item 函数$f(x)=(\sin \alpha)^{|\log _{\sin \alpha }x|}$($2k\pi <\alpha <2k\pi +\pi$且$\alpha \ne 2k\pi +\dfrac{\pi }2$, $k\in \mathbf{Z}$)的图象是\bracket{20}.
\blank{50}\fourch{}{}{}{}
\item 设$x\in (0,\dfrac{\pi }2)$, 则下列各式中正确的是\bracket{20}.
\fourch{$\sin (\sin x)<\cos x<\cos (\cos x)$}{$\sin (\cos x)<\cos x<\cos (\sin x)$}{$\cos (\sin x)<\cos x<\sin (\cos x)$}{$\cos (\cos x)<\cos x<\sin (\sin x)$}
\item 求下列函数的定义域:
(1)$y=\log _{\sin x}(2\cos x+1)$.				(2)$y=\sqrt {1-2\cos x}+\lg (2\sin x-\sqrt 2)$.
(3)$y=\sqrt {\sin x}+\dfrac 1{\sqrt {16-x^2}}$.
\item 画出下列函数的图象:
(1)$y=|\sin x|$.							(2)$y=|\cos x|+\cos x$.
(3)$y=(\sin \alpha)^{|\log _{\sin \alpha }x|}$($\alpha$为锐角).		(4)$y=\dfrac{|\sin x|}{\sin x}$.
(5)$y=f(\sin x)$, 其中$f(x)=\begin{cases}
    2  (x\ge 0),  \\-1  (x<0).  \end{cases}$
\item (1)若$0<\alpha <\dfrac{\pi }4$, 且$\lg \sin \alpha +\log \cos \alpha +\lg 9=\lg \tan \alpha +\lg \cot \alpha +\dfrac 12\lg 8$, 求$\sin \alpha -\cos \alpha$的值.
(2)设$x$是第二象限角, 且满足$\cos \dfrac x2+\sin \dfrac x2=-\dfrac{\sqrt 5}2$, 求$\sin \dfrac x2-\cos \dfrac x2$的值.
\item 若$0<\theta <\dfrac{\pi }2$, 比较$M=\log _{\sin \theta }\cos \theta$与$N=\log _{\cos \theta }\sin \theta$的大小.
\item 若$\alpha ,\beta$是关于$x$的二次方程$x^2+2(\cos \theta +1)x+\cos ^2\theta =0$的两实根, 且$|\alpha -\beta|\le 2\sqrt 2$, 求$\theta$的范围.
\item (1)求函数$f(x)=a\sin x-\sin ^2x$的最大值$g(a)$, 并画出$g(a)$的图象.
(2)若函数$f(x)=\cos ^2x-a\sin x+b$的最大值为0, 最小值为-4, 实数$a>0$, 求$a,b$的值.
(二)函数$y=A\sin (\omega x+\varphi)$的图象
\item 函数$y=3\sin (2x+\dfrac{\pi }6)$的图象的一条对称轴是直线\bracket{20}.
\fourch{$x=0$}{$x=\dfrac{\pi }6$}{$x=-\dfrac{\pi }6$}{$x=\dfrac{\pi }3$}
\item 先将函数$y=\sin 2x$的图象向右平移$\dfrac{\pi }3$个单位长度, 再将所得图象作关于$y$轴的对称变换, 则与最后所得图象对应的函数的解析式是\bracket{20}.
\fourch{$y=\sin (-2x+\dfrac{\pi }3)$}{$y=\sin (-2x-\dfrac{\pi }3)$}{$y=\sin (-2x+\dfrac 23\pi)$}{$y=\sin (-2x-\dfrac 23\pi)$}
\item 将函数$y=\sin x$的图象上所有点向左平移$\dfrac{\pi }3$个单位长度, 再把所得图象上各点横坐标伸长到原来的2倍, 则与最后得到的图象对应的函数的解析式为\bracket{20}.
\fourch{$y=\sin (\dfrac x2-\dfrac{\pi }3)$}{$y=\sin (\dfrac x2+\dfrac{\pi }6)$}{$y=\sin (\dfrac x2+\dfrac{\pi }3)$}{$y=\sin (2x+\dfrac{\pi }3)$}
\item 函数$y=A\sin (\omega x+\varphi)$($A>0$, $\omega >0$, $|\varphi|<\dfrac{\pi }2$)的图象如图所示, 则$y$的表达式是\bracket{20}.
\fourch{$2\sin (\dfrac{10}{11}x+\dfrac{\pi }6)$}{$2\sin (\dfrac{10}{11}x-\dfrac{\pi }6)$}{$2\sin (2x+\dfrac{\pi }6)$}{$2\sin (2x-\dfrac{\pi }6)$}
(第91题)
\item 函数$y=2\sin (\dfrac 12x+\dfrac{\pi }3)$在一个周期内的简图是\bracket{20}.
\blank{50}\fourch{\blank{50}
\blank{50}(C)\blank{50}(D)
\item 要得到函数$y=\sin (\dfrac x2-\dfrac{\pi }6)$的图象.只需将函数$y=\sin \dfrac x2$的图象\bracket{20}.
(A)向右平移$\dfrac{\pi }6$}{向左平移$\dfrac{\pi }6$}{向右平移$\dfrac{\pi }3$}{向左平移$\dfrac{\pi }3$}
\item (1)$f(x)=\log _{\dfrac{\pi }4}\cos (2x+\dfrac{\pi }4)$为增函数的区间是\blank{50}.
(2)函数$f(x)=2\sin (3-2x)$为增喊数的区间是\blank{50}.
(3)函数$y=\cos (2x-\dfrac{\pi }5)$为减函数的区间是\blank{50}.
\item (1)函数$y=\sin (2x+\dfrac{\pi }3)$的图象可由$y=\sin 2x$的图象向\blank{50}平移\blank{50}个单位长度得到.
(2)将奇函数$y=f(x)$($x\in \mathbf{R}$)的图象沿$x$轴正向平移1个单位长度后, 所得的图象为$C'$, 而图象$C'$与$C$关于原点对称, 那么$C$所对应的函数应为\blank{50}.
(3)先将函数$f(x)=\sin x$的图象向右平移$\dfrac{\pi }5$个单位长度, 再改变各点的横坐标(纵坐标不变), 得到最小正周期为$\dfrac{2\pi }3$的函数$y=\sin (\omega x+\varphi)$($\omega >0$)的图象, 则$\omega =$\blank{50}, $\varphi =$\blank{50}.
\item 若函数$f(x)=2\cos (\dfrac k4x+\dfrac{\pi }3)-5$的最小正周期不大于2, 则正整数$k$的最小值为\bracket{20}.
\fourch{10}{11}{12}{13}
\item (1)若函数$f(x)=\sin (2x+\varphi)$($-\pi <\varphi <0$)是偶函数, 则$\varphi =$\blank{50}.
(2)若函数$f(x)=\cos (x+\varphi)$的图象关于坐标原点对称, 则$\varphi =$\blank{50}.
\item 根据周期涵数的定义, 求函数$y=2\cos (4x-\dfrac{\pi }3)$的最小正周期.
\item (1)若奇函数$f(x)$是姑小正周期为3的周期函数, 且$f(1)=-1$, 则$f(101)=$\blank{50}.
(2)若偶函数$y=f(x)$是最小正周期为2的周期函数.且$2\le x\le 3$时, $f(x)=x$, 则当$-2\le x\le 0$时, $f(x)$的表达式为\blank{50}.
\item (1)已知函数$f(x)=A\sin (\omega x+\varphi)$($A>0$, $\omega >0$)在同一周期内, 当$x=\dfrac{\pi }9$时取得最大值$\dfrac 12$, 当$x=\dfrac{4\pi }9$时取得最小值$-\dfrac 12$, 求此函数的解析式.
(2)已知函数$f(x)=A\sin (\omega x+\varphi)$($A>0$, $\omega >0$)的图象上一个最高点的坐标为$(2,\sqrt 2)$, 由这个最高点到其相邻的最低点间, 图象与$x$轴交于点(6, 0), 求此函数的解析式.
(三)正切函数、余切函数的图象和性质
\item 函数$y=\tan 3\pi x$的最小正周期为\bracket{20}.
\fourch{$\dfrac 13$}{$\dfrac 23$}{$\dfrac 6{\pi }$}{$\dfrac 3{\pi }$}
\item 下列函数中, 以$\pi$为最小正周期的偶函数是\bracket{20}.
\fourch{$y=\sin x\cdot \cos x$}{$y=\cot x$}{$y=\cos \dfrac x2$}{$y=\cos ^2x$}
\item 若$a=\sin \dfrac 34$, $b=\cos \dfrac 34$, $c=\cot \dfrac 34$, 则$a,b,c$之间的大小关系是\bracket{20}.
\fourch{$a>b>c$}{$b>c>a$}{$c>a>b$}{$c>b>a$}
\item 若$\tan (2x-\dfrac{\pi }3)\le 1$, 则$x$的取值范围是\bracket{20}.
\fourch{$\dfrac{k\pi 2-\dfrac{\pi }{12}\le x\le \dfrac k\pi }2+\dfrac 7{24}\pi$($k\in \mathbf{Z}$)}{$k\pi -\dfrac{\pi }{12}\le x<k\pi +\dfrac 7{24}\pi$($k\in \mathbf{Z}$)}{$\dfrac{k\pi 2-\dfrac{\pi }{12}<x\le \dfrac k\pi }2+\dfrac 7{24}\pi$($k\in \mathbf{Z}$)}{$k\pi -\dfrac{\pi }{12}<x<k\pi +\dfrac 7{24}\pi$($k\in \mathbf{Z}$)}
\item 下列函数中, 同时满足条件``\textcircled{1} 在$(0,\dfrac{\pi }2)$为增函数, \textcircled{2} 为奇函数, \textcircled{3} 以$\pi$为最小正周期''的函数是\bracket{20}.
\fourch{$y=\tan x$}{$y=\cot x$}{$y=\tan \dfrac x2$}{$y=|\sin x|$}
\item 函数$y=\cot x(-\dfrac{\pi }4\le x\le \dfrac{\pi }4)$的值域是\bracket{20}.
\fourch{[-1.1]}{$(-\infty ,-1]\cup [1,+\infty)$}{$(-\infty ,-1]$}{$[1,+\infty)$}
\item 根据要求, 求$x$的取值范围:
(1)$\tan \dfrac x2\ge \sqrt 3$:\blank{50}.
(2)$\cot 2x\le -\sqrt 3$:\blank{50}.
(3)$|\sin x|\le|\cos x|$:\blank{50}.
(4)$\log _x\tan x>0$:\blank{50}.
(5)$\log _{\sqrt 3}\sin \dfrac x2-\log _{\sqrt 3}\cos \dfrac x2>-1$, 且$-2\pi <x<2\pi$:\blank{50}.
\item 将下列各题中的数由小到大排列:
(1)$\tan 1,\tan 2,\tan 3$:\blank{50}.
(2)$1,\sin 1,\cos 1,\tan 1$:\blank{50}.
\item 在``\textcircled{1} $y=|\sin 2x|$, \textcircled{2} $y=|\cos x|$, \textcircled{3} $y=|\tan 2x|$, \textcircled{4} $y=|\tan x|+|\cot x|$''这四个函数中, 最小正周期为$\dfrac{\pi }2$的偶函数有\bracket{20}.
\fourch{0个}{1个}{2个}{3个}
\item $\sin \dfrac{2\pi }3,\cos 1,\tan 2,\cot 3$的大小关系为\bracket{20}.
\fourch{$\sin \dfrac{2\pi }3>\cos 1>\cot 3>\tan 2$}{$\sin \dfrac{2\pi }3>\cos 1>\tan 2>\cot 3$}{$\cos 1>\sin \dfrac{2\pi }3>\tan 2>\cot 3$}{$\cos 1>\sin \dfrac{2\pi }3>\cot 3>\tan 2$}
\item 若$0<\alpha <2\pi$, 且满足$\sin \alpha <\cos \alpha <\cot \alpha <\tan \alpha$, 则有\bracket{20}.
\fourch{$0<\alpha <\dfrac{\pi }4$}{$\dfrac{\pi }4<\alpha <\dfrac{\pi }2$}{$\pi <\alpha <\dfrac 54\pi$}{$\dfrac{5\pi }4<\alpha <\dfrac{3\pi }2$}
\item 求下列函数的定义域:
(1)$y=\sqrt {\sqrt 3-\cot \dfrac x2}$.
(2)$y=\dfrac{\lg (\tan x-1)}{\sqrt {1-2\sin x}}$.
(3)$y=\lg (\tan x-1)+\sqrt {\sin 2x}$.
\item (1)求函数$y=\dfrac{\sec ^2x+\tan x}{\sec ^2x-\tan x}$的值域.
(2)已知$\theta \in [-\dfrac{\pi }3,\dfrac{\pi }4]$, 求函数$y=\sec ^2\theta +2\tan \theta +1$的最大值与最小值.
\item 根据条件比较下列各组数的大小:
(1)已知$\dfrac{\pi }3<\theta <\dfrac{\pi }2$, 比较$\sin \theta ,\cot \theta ,\cos \theta$的大小.
(2)已知$0<\alpha <\dfrac{\pi }4$, 比较$\sin \alpha ,\sin (\sin \alpha),\sin (\tan \alpha)$的大小.
(3)已知$0<\theta <\dfrac{\pi }2$, 比较$\cos \theta ,\sin (\cos \theta),\cos (\sin \theta)$的大小.
\item 利用锐角三角函数的定义解下列各题:
(1)若$\alpha ,\beta \in (0,\dfrac{\pi }2)$, 且$17\cos \alpha +13\cos \beta =17$, $17\sin \alpha =13\sin \beta$, 求$\dfrac{\alpha }2+\beta$.
(2)设$x\in [\dfrac{\pi }4,\dfrac{\pi }2]$, 求证: $\csc x-\cot x\ge \sqrt 2-1$.
\item 已知$a\cos \alpha +b\sin \alpha =c$, $a\cos \beta +b\sin \beta =c$($0<\alpha ,\beta <\pi$, $\alpha \ne \beta$), 且$\cos \alpha +\cos \beta =\cos \alpha \cdot \cos \beta$, 求证: $c^2-b^2=2ac$.
\item 已知函数$f(x)$满足$af(\sin x)+bf(-\sin x)=c\sin x\cos x(-\dfrac{\pi }2\le x\le \dfrac{\pi }2,a^2-b^2\ne 0)$, 求$f(x)$的解析式.
\item (1)设$\dfrac{\sin \alpha}{a^2-1}=\dfrac{\cos \alpha }{2a\sin 2\beta }=\dfrac 1{1+2a\cos 2\beta +a^2}$, 求证: $\sin \alpha =\dfrac{a^2-1}{a^2+1}$.
(2)已知$a\sec ^2\alpha -b\cos \alpha =2a$, $b\cos ^2\alpha -a\sec \alpha =2b$, 求$a,b$的关系式.
(3)已知$a\sin ^2\theta +b\cos ^2\theta =m$, $b\sin ^2\varphi +a\cos ^2\varphi =n$, $a\tan \theta =b\tan \varphi$($a,b,m,n$互不相等), 求证: $\dfrac 1m+\dfrac 1n=\dfrac 1a+\dfrac 1b$.
\item 利用单位圆和三角函数线证明:
(1)若$\alpha$为锐角, 则\textcircled{1} $\sin \alpha +\cos \alpha >1$; \textcircled{2} $\sin \alpha <\alpha <\tan \alpha$; \textcircled{3} $\alpha \cdot \sin \alpha +\cos \alpha >1$.
(2)若$0<\beta <\alpha <\dfrac{\pi }2$, 则$\sin \alpha -\sin \beta <\alpha -\beta <\tan \alpha -\tan \beta$.
\item 若$\alpha$是锐角, 求证: $\cos (\sin \alpha)>\sin (\cos \alpha)$.
\item (1)已知函数$f(x)$满足$f(x+a)=\dfrac{1-f(x)}{1+f(x)}$($a$为常数, 且$a\ne 0$), 求证: $f(x)$是一个以$2a$为周期的周期函数.
(2)已知$f(x)$为偶函数, 其图象关于直线$x=a$($a\ne 0$)对称, 求证: $f(x)$是一个以$2a$为周期的周期函数.
\item 已知$f(x)$, $g(x)$是定义在$R$上的两个函数, 且$g(x)$为奇函数.并满足: \textcircled{1} $f(0)=1$; \textcircled{2} 对任何$x,y\in \mathbf{R}$都有$f(x-y)=f(x)f(y)+g(x)g(y)$.求证:
(1)对任何$x\in \mathbf{R}$都有$f^2(x)+g^2(x)=1$.
(2)$f(x)$是偶函数.
(3)若存在非零实数$a$满足$f(a)=1$, 则$f(x)$是周期函数.
\item 利用图象求方程$\sin x=\tan \dfrac x2$在区间$[0,8\pi]$上解的个数.
\item 设$0\le x\le \pi$, $f_1(x)=\sin (\cos x)$, $f_2(x)=\cos (\sin x)$.
(1)求$f_1(x)$, $f_2(x)$的最大值和最小值.
(2)比较$f_1(x)$与$f_2(x)$的大小.
    
\end{enumerate}
\end{document}