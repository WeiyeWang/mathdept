\documentclass[10pt,a4paper]{article}
\usepackage[UTF8,fontset = windows]{ctex}
\setCJKmainfont[BoldFont=黑体,ItalicFont=楷体]{华文中宋}
\usepackage{amssymb,amsmath,amsfonts,amsthm,mathrsfs,dsfont,graphicx}
\usepackage{ifthen,indentfirst,enumerate,color,titletoc}
\usepackage{tikz}
\usepackage{makecell}
\usepackage{longtable}

\usetikzlibrary{arrows,calc,intersections,patterns}
\usepackage[bf,small,indentafter,pagestyles]{titlesec}
\usepackage[top=1in, bottom=1in,left=0.8in,right=0.8in]{geometry}
\renewcommand{\baselinestretch}{1.65}
\newtheorem{defi}{定义~}
\newtheorem{eg}{例~}
\newtheorem{ex}{~}
\newtheorem{rem}{注~}
\newtheorem{thm}{定理~}
\newtheorem{coro}{推论~}
\newtheorem{axiom}{公理~}
\newtheorem{prop}{性质~}
\newcommand{\blank}[1]{\underline{\hbox to #1pt{}}}
\newcommand{\bracket}[1]{(\hbox to #1pt{})}
\newcommand{\onech}[4]{\par\begin{tabular}{p{.9\textwidth}}
A.~#1\\
B.~#2\\
C.~#3\\
D.~#4
\end{tabular}}
\newcommand{\twoch}[4]{\par\begin{tabular}{p{.46\textwidth}p{.46\textwidth}}
A.~#1& B.~#2\\
C.~#3& D.~#4
\end{tabular}}
\newcommand{\vartwoch}[4]{\par\begin{tabular}{p{.46\textwidth}p{.46\textwidth}}
(1)~#1& (2)~#2\\
(3)~#3& (4)~#4
\end{tabular}}
\newcommand{\fourch}[4]{\par\begin{tabular}{p{.23\textwidth}p{.23\textwidth}p{.23\textwidth}p{.23\textwidth}}
A.~#1 &B.~#2& C.~#3& D.~#4
\end{tabular}}
\newcommand{\varfourch}[4]{\par\begin{tabular}{p{.23\textwidth}p{.23\textwidth}p{.23\textwidth}p{.23\textwidth}}
(1)~#1 &(2)~#2& (3)~#3& (4)~#4
\end{tabular}}
\begin{document}
\begin{enumerate}[1.]

\item 当$a>b>0$时, 比较$\dfrac{2a+b}{a+2b}$和$\dfrac ab$的大小.
\item 已知$a>0$, $a\ne 1$, $m>n>0$, 比较$A=a^m+\dfrac 1{a^m}$和$B=a^n+\dfrac 1{a^n}$的大小.
\item 若$a>b$, 则下列各式中正确的是\bracket{20}.
\fourch{$a\lg x>b\lg x$($x>0$)}{$ax^2>bx^2$}{$a^2>b^2$}{$2^x\cdot a>2^x\cdot b$}
item 设$ab>0$, 且$\dfrac ca>\dfrac db$, 则下列各式中, 恒成立的是\bracket{20}.
\fourch{$bc<ad$}{$bc>ad$}{$\dfrac ac>\dfrac bd$}{$\dfrac ac<\dfrac bd$}
\item 下列命题中, 不正确的一个是\bracket{20}.
\twoch{若$\sqrt[3]a>\sqrt[3]b$, 则$a>b$}{若$a>b$, $c>d$, 则$a-d>b-c$}{若$a>b>0$, $c>d>0$, 则$\dfrac ad>\dfrac bc$}{若$a>b>0$, $ac>bd$, 则$c>d$}
\item 若$x<y<0$, 则有\bracket{20}.
\fourch{$0<x^2<xy$}{$y^2<xy<x^2$}{$xy<y^2<x^2$}{$y^2>x^2>0$}
\item 若$a=\log_{0.2}0.3$, $b=\log_{0.3}0.2$, $c=1$, 则$a,b,c$的大小关系是\bracket{20}.
\fourch{$a>b>c$}{$b>a>c$}{$b>c>a$}{$c>b>a$}
\item 用不等号(``$>$''或``$<$'')填空:\\
(1) 若$a\ne b$, 则$a^2+3b^2$\blank{50}$2b(a+b)$;\\
(2) 若$c>1$, 则$\sqrt{c+1}-\sqrt c$\blank{50}$\sqrt c-\sqrt{c-1}$;\\
(3) 若$a>b$, $c>d$, 且$a$与$d$都是负数, 则$ac$\blank{50}$bd$.
\item 若``$a>b$, $a-\dfrac 1a>b-\dfrac 1b$''同时成立, 则$ab$应满足的条件是\blank{50}.
\item 已知$a>0$, $b>0$, 且$a\ne b$, 比较$\dfrac{a^2}b+\dfrac{b^2}a$与$a+b$的大小.
\item 已知$0<\dfrac ab<\dfrac cd$, 比较$\dfrac b{a+b}$与$\dfrac d{c+d}$的大小.
\item 若$x>y>1$, $0<a<1$, 则下列各式中正确的一个是\bracket{20}.
\fourch{${x^{-a}}>{y^{-a}}$}{$(\sin a)^x>(\sin a)^y$}{$\log_{\frac 1a}x<\log_{\frac 1a}y$}{$1+a^{x+y}>a^x+a^y$}
\item 已知$a\in \mathbf{R}$, 比较$\dfrac 1{1+a}$与$1-a$的大小.
\item 设$a>0$, $a\ne 1$, $t>0$, 比较$\dfrac 12\log_at$和$\log_a\dfrac{t+1}2$的大小.
\item 已知$x>y>0$, 比较$\sqrt{\dfrac{y^2+1}{x^2+1}}$与$\dfrac yx$的大小.
\item 已知$a$, $b$, $m$, $n$都是正实数, 且$m+n=1$, 比较$\sqrt{ma+nb}$和$m\sqrt a+n\sqrt b$的大小.
\item 解下列不等式:\\
(1) $6x^2-5x-1>0$;\\
(2) $6x^2-5x-1<0$;\\
(3) $5x^2-2x+3>0$;\\
(4) $9x^2+6x+1>0$;\\
(5) $3x^2-4x+5<0$.
\item 已知关于$x$的不等式$ax^2+bx+c<0$的解集是$\{x|x<-2\text{或}x>-\dfrac 12\}$, 求$ax^2-bx+c>0$的解集.
\item 已知集合$A=\{x|x^2+(a-1)x-a>0\}$, $B=\{x|(x+a)(x+b)>0\}$, $a\ne b$, $M=\{x|x^2-2x-3\le 0\}$.\\
(1) 若$\complement_UB=M$, 求$a$, $b$的值;\\
(2) 若$-1<b<a<1$, 求$A\cap B$;\\
(3) 若$-3<a<-1$, 且$a^2-1\in \complement_UA$, 求实数$a$的取值范围.
\item 已知函数$y=(k^2+4k-5)x^2+4(1-k)x+3$的图象都在$x$轴的上方, 求实数$k$的取值范围.
\item 已知$a<b$, 则下列各式中恒成立的是\bracket{20}.
\fourch{$a^2<b^2$}{$c-a>c-b$}{$|a|<|b|$}{$a-1>b-2$}
\item 若$|x|>2$, 则\bracket{20}.
\fourch{$x>2$}{$x>\pm 2$}{$-2<x<2$}{$x>2$或$x<-2$}
\item 不等式$|x|-3<0$的解集是\bracket{20}.
\fourch{$\{x|x<\pm 3\}$}{$\{x|-3<x<3\}$}{$\{x|x>3\}$}{$\{x|x<-3\}$}
\item 已知集合$M=\{x||x|>2\},N=\{x|x<3\}$, 则下列结论正确的是\bracket{20}.
\twoch{$M\cup N=M$}{$M\cap N=\{x|2<x<3\}$}{$M\cup N=R$}{$M\cap N=\{x|x<-2\}$}
\item 已知集合$M=\{x||x+1|\le 2\},P=\{x|x\le 2$或$x\ge 3\}$, 则$M$, $P$之间的关系是\bracket{20}.
\fourch{$M\supseteq P$}{$M\supset P$}{$M\subseteq P$}{$M\subset P$}
\item 已知$|1-x|+\sqrt{x^2-4x+4}=1$, 则$x$的取值范围是\bracket{20}.
\fourch{$1\le x\le 2$}{$x\le 1$}{$x<1$或$x>2$}{$x\ge 2$}
\item 不等式$2x+3-x^2>0$的解集是\bracket{20}.
\fourch{$\{x|-\dfrac 32\le x<1\}$}{$\{x|-1<x<3\}$}{$\{x|1\le x<3\}$}{$\{x|-\dfrac 32\le x<3\}$}
\item 不等式$6x^2+5x<4$的解集是\bracket{20}.
\fourch{$\{x|x<-\dfrac 43\text{或}x>\dfrac 12\}$}{$\{x|-\dfrac 43<x<\dfrac 12\}.$}{$\{x|-\dfrac 12<x<\dfrac 43\}.$}{$\{x|x<-\dfrac 12\text{或}x>\dfrac 43\}$}
\item 当$a<0$时, 关于$x$的不等式$x^2-4ax-5a^2>0$的解集是\bracket{20}.
\fourch{$\{x|x>5a\text{或}x<-a\}$}{$\{x|x<5a\text{或}x>-a\}$}{$\{x|-a<x<5a\}$}{$\{x|5a<x<-a\}$}
\item 若$x$为实数, 则下列命题正确的是\bracket{20}.
\onech{$x^2\ge 2$的解集是$\{x|x\ge \pm \sqrt 2\}$}{$(x-1)^2<2$的解集是$\{x|1-\sqrt 2<x<1+\sqrt 2\}$}{$x^2-9<0$的解集是$\{x|x<3\}$}{设$x_1,x_2$为$ax^2+bx+c=0$的两个实根, 且$x_1>x_2$, 则$ax^2+bx+c>0$的解集是$\{x|x_2<x<x_1\}$}
\item 在\textcircled{1} $x^2-2x-3<0$与$\dfrac{x^2-2x}{x-1}<\dfrac 3{x-1}$; \textcircled{2} $x^2+3x-4>0$与$x^2+3x+\sqrt x>4+\sqrt x$; \textcircled{3} $\dfrac{(x+2)(x^2-1)}{x+2}>0$与$x^2-1>0$''三组不等式中, 解集相同的组数是\bracket{20}.
\fourch{$0$}{$1$}{$2$}{$3$}
\item 若$x^2+x<0$, 则$x^2,x,-x^2,-x$的大小关系是\bracket{20}.
\fourch{$x^2>x>-x^2>-x$}{$-x>x^2>-x^2>x$}{$-x>x^2>x>-x^2$}{$x^2>-x>x>-x^2$}
\item 直接写出下列不等式的解集:\\
(1) $(x-1)^2>0$:\blank{50};\\
(2) $(2-x)(3x+1)>0$:\blank{50};\\
(3) $1-3x^2>2x$:\blank{50};\\
(4) $1-2x-x^2\ge 0$:\blank{50};\\
(5) $x+\sqrt x-6<0$:\blank{50}.
\item 直接写出下列不等式的解集:\\
(1) $\dfrac{3x+4}{x-2}\ge 0$:\blank{50};\\
(2) $\dfrac{4-2x}{1+3x}>0$:\blank{50};\\
(3) $\dfrac 1x>x$:\blank{50};\\	
(4) $x^2-2|x|-3>0$:\blank{50};\\
(5) $x^2-x-5>|2x-1|$:\blank{50}.
\item 若$\sqrt{x^2-x-6}\in \mathbf{R}$, 则$x$的取值范围为\blank{50}.	
\item 要使代数式$\dfrac{\sqrt{x-3}}{\sqrt{x^2-3x+2}}$有意义, 实数$x$的取值范围是\blank{50}.
\item 若代数式$6x^2+x-2$的值恒取非负实数, 则实数$x$的取值范围是\blank{50}.
\item 不等式$4\le x^2-3x<18$的整数解集是\blank{50}.
\item 已知实数$x$满足$4x^2-4x-15\le 0$, 化简$\sqrt{x^2-8x+16}-|x-3|$.
\item 已知$a>b$, 直接写出下列不等式的解集:\\
(1) $\dfrac{x-a}{x-b}\ge 0$:\blank{50};\\
(2) $\dfrac{x-a}{x-b}<0$:\blank{50};\\
(3) $x^2-(a-b)x+ab>0$:\blank{50};\\
(4) $x^2-(a-b)x+ab<0$:\blank{50}.
\item 若关于$x$的方程$2kx^2+(8k+1)x+8k=0$有两个不等实根, 则实数$k$的取值范围是\blank{50}.	
\item 已知$a\ne 0$, 若关于$x$的不等式$ax^2-2ax+2a+3>0$无实数解, 则$a$的取值范围是\blank{50}.
\item 不等式$\dfrac{x-1}{2x}\le 1$的解集是\bracket{20}.
\fourch{$\{x|x\ge -1\}$	}{$\{x|x\le -1\}$}{$\{x|-1\le x<0\}$}{$\{x|x\le -1\text{或}x>0\}$}
\item 若关于$x$的二次不等式$mx^2+8mx+21<0$的解集是$\{x|-1<x<-1\}$, 则实数$m$的值等于\bracket{20}.
\fourch{$1$}{$2$}{$3$}{$4$}
\item 若关于$x$的不等式$(a^2-3)x^2+5x-2>0$的解集是$\{x|\dfrac 12<x<2\}$, 则实数$a$的值等于\bracket{20}.
\fourch{$1$}{$-1$}{$\pm 1$}{$0$}
\item 若关于$x$的不等式$ax^2+bx+c<0(a\ne 0)$的解集是空集, 则\bracket{20}.
\fourch{$a<0$且$b^2-4ac>0$}{$a<0$且$b^2-4ac\le 0$}{$a>0$且$b^2-4ac\le 0$}{$a>0$且$b^2-4ac>0$}
\item 若对任何实数$x$, 二次函数$y=ax^2-x+c$的值恒为负, 则$a,c$应满足\bracket{20}.
\fourch{$\begin{cases}
   a>0,  \\ ac\le \dfrac 14  \end{cases}$}{$\begin{cases}
   a<0,  \\ ac<\dfrac 14  \end{cases}$}{$\begin{cases}
   a<0,  \\ ac>\dfrac 14  \end{cases}$}{$\begin{cases}
   a<0,  \\ ac<0  \end{cases}$}
\item 若对任意实数$x$, 不等式$x^2+2(1+k)x+3+k>0$恒成立, 则$k$的取值范围是\bracket{20}.
\fourch{$-1<k<2$}{$-1\le k\le 2$}{$-2<k<1$}{$-2\le k\le 1$}
\item 若关于$x$的二次方程$2(k+1)x^2+4kx+3k-2=0$的两根同号, 则$k$的取值范围是\bracket{20}.
\twoch{$-2<k<1$}{$-2\le k<-1$或$\dfrac 23<k\le 1$}{$k<-1$或$k>\dfrac 23$}{$-2<k<1$或$\dfrac 23<k<1$}
\item 已知关于$x$的方程$(m+3)x^2-4mx+2m-1=0$的两根异号, 且负根的绝对值比正根大, 那么实数$m$的取值范围是\bracket{20}.
\fourch{$-3<m<0$}{$0<m<3$}{$m<-3$或$m>0$}{$m<0$或$m>3$}
\item 若$\alpha ,\beta$是关于$x$的方程$x^2-(k-2)x+k^2+3k+5=0$($k$为实数)的两个实根, 则${{\alpha }^2}+{{\beta }^2}$的最大值等于\bracket{20}.
\fourch{$19$}{$18$}{$\dfrac{50}9$}{$-6$}
\item 不等式$(x-1)(x-2)(x-3)(x-4)>120$的解为\bracket{20}.
\fourch{$x>6$}{$x<-1$或$x>6$}{$x<-1$}{$-1<x<6$}
\item 在三个关于$x$的方程$x^2-ax+4=0$, $x^2+(a-1)x+16=0$和$x^2+2ax+3a+10=0$中, 已知至少有一个方程有实根, 则实数$a$的取值范围是\bracket{20}.
\fourch{$-4\le a\le 4$}{$-2<a<4$}{$a\le -2$或$a\ge 4$}{$a<0$}
\item 若关于$x$的二次方程$x^2-2mx+4x+2m^2-4m-2=0$有实根, 则其两根之积的最大值等于\blank{50}.
\item 使关于$x$的方程$x^2-kx+2k-3=0$的两实根的平方和取最小值, 实数$k$的值等于\blank{50}.
\item 若关于$x$的不等式$x^2-mx+n\le 0$的解集是$\{x|-5\le x\le 1\}$, 则实数$m=$\blank{50}, $n=$\blank{50}.
\item 若关于$x$的不等式$ax^2+bx+1\ge 0$的解集是$\{x|-5\le x\le 1\}$, 则实数$a=$\blank{50}, $b=$\blank{50}.
\item 若关于$x$的不等式$ax^2+bx+2>0$的解集是$\{x|-\dfrac 12<x<\dfrac 13\}$, 则实数$a=$\blank{50}, $b=$\blank{50}.
\item 若关于$x$的不等式$ax^2+bx-6>0$的解集是$\{x|2<x<3\}$, 则实数$a=$\blank{50}, $b=$\blank{50}.
\item 若关于$x$的不等式$(a+b)x+(2a-3b)<0$的解集是$\{x|x>3\}$, 则不等式$(a-3b)x+b-2a>0$的解集是\blank{50}.
\item 若关于$x$的不等式$ax^2+bx+c<0$的解集是$\{x|x<-2\text{或}x>-\dfrac 12\}$, 则关于$x$的不等式$ax^2-bx+c>0$的解集是\blank{50}.
\item 解不等式$x^4-2x^2+1>x^2-1$.
\item 已知关于$x$的不等式$kx^2-2x+6k<0(k\ne 0)$.\\
(1) 若不等式的解集是$\{x|x<-3\text{或}x>-2\}$, 求实数$k$的值;\\
(2) 若不等式的解集是$\{x|x\ne \dfrac 1k\}$, 求实数$k$的值;\\
(3) 若不等式的解集是实数集, 求实数$k$的值.
\item 已知关于$x$的方程$m(x-1)=3(x+2)$的解是正实数, 求实数$m$的取值范围.
\item 已知关于$x$的方程$\dfrac 14x^2-kx+5k-6=0$无实数解, 求实数$k$的取值范围.
\item 已知关于$x$的方程$kx^2-(3k-1)x+k=0$有两个正实数根, 求实数$k$的取值范围.
\item 已知集合$M=\{x|x^2-7x+10\le 0\}$, $N=\{x|x^2-(2-m)x+5-m\le 0\}$, 且$N\subseteq M$, 求实数$m$的取值范围.
\item 已知集合$A=\{x|x^2+4x+p<0\}$, $B=\{x|x^2-x-2>0\}$, 且$A\subseteq B$, 求实数$p$的取值范围.
\item 已知集合$A=\{x|x^2+ax+1\le 0\}$, $B=\{x|x^2-3x+2\le 0\}$, 且$A\subseteq B$, 求实数$a$的取值范围.
\item 已知集合$A=\{x|x^2-2x-3\le 0\}$, $B=\{x|x^2+px+q<0\}$, 且$A\cap B=\{x|-1\le x<2\}$, 求实数$p,q$的关系式及其取值范围.
\item 已知集合$A=\{x|-2<x<-1\text{或}x>\dfrac 12\}$, $B=\{x|x^2+ax+b\le 0\}$, 且$A\cup B=\{x|x+2>0\}$, $A\cap B=\{x|\dfrac 12<x\le 3\}$, 求$a,b$的值.
\item 要使代数式$mx^2+(m-1)x+(m-1)$的值恒为负值, 求实数$m$的取值范围.
\item 已知关于$x$的不等式$(a^2-4)x^2+(a+2)x-1\ge 0$的解集是空集, 求实数$a$的取值范围.
\item 若关于$x$的不等式$\dfrac{x^2-8x+20}{mx^2+2(m+1)x+9m+4}<0$的解集为$\mathbf{R}$, 求实数$m$的取值范围.
\item 当$0^\circ <\varphi <90^\circ$时, 要使$\dfrac{x^2-6x+8}{x^2+2}=\sin \varphi$恒成立, 求实数$x$的取值范围.
\item 既要使关于$x$的不等式$x^2+(m-\dfrac 12)x-\dfrac 7{16}\le 0$有实数解, 又要使关于$x$的方程$(2m+3){x^2}+mx+\dfrac{m-2}4=0$有实数解, 求实数$m$的取值范围.
\item 为长$80\text{cm}$、宽$60\text{cm}$的工作台做一块台布, 使台布的面积是台面面积的两倍以上, 并使台子四边垂下的长度相等, 问: 垂下的长度至少是多少(精确到$0.1\text{cm}$)?
\item 已知非零实数$x,y,z$, 满足$x+y+z=xyz$, $x^2=yz$, 求证: $x^2\ge 3$.
\item 已知$a+b\ge 0$, 求证: $a^3+b^3\ge a^2b+ab^2$.
\item 设$a,b\in \mathbf{R}^+$, 且$a\ne b$, 求证: $a^ab^b>a^bb^a$.
\item 已知$a,b,c\in \mathbf{R}$, 求证: $a^2+b^2+c^2\ge ab+bc+ca$.
\item 已知$a,b,c>0$, 求证:
(1) $(a+b)(\dfrac 1a+\dfrac 1b)\ge 4$;\\
(2) $(a+b+c)(\dfrac 1a+\dfrac 1b+\dfrac 1c)\ge 9$.
\item 已知正数$a,b$满足$a+b=1$, 求证: $\sqrt{2a+1}+\sqrt{2b+1}\le 2\sqrt 2$.
\item 已知$\alpha ,\beta \in (0,\dfrac{\pi}2)$, 且$\alpha \ne \beta$, 求证: $\tan \alpha +\tan \beta >2\tan \dfrac{\alpha +\beta}2$.
\item 记$f(x)=x^2+ax+b$, 求证: $|f(1)|,|f(2)|,|f(3)|$中至少有一个不小于$\dfrac 12$.
\item 已知$-1\le x\le 1$, $n\ge 2$, $n\in \mathbf{N}$, 求证: $(1-x)^n+(1+x)^n\le 2^n$.
\item 已知$x+2y+3z=12$, 求证: $x^2+2y^2+3z^2\ge 24$.
\item 已知$a,b,c\in \mathbf{R}^+$, 求证: $a^3+b^3+c^3\ge 3abc$(当且仅当$a=b=c$时取等号).
\item 已知$a>0$, 求证: $x+\dfrac 1x+\dfrac 1{x+\dfrac 1x}\ge \dfrac 52$.
\item 已知实数$a,b,c$满足$a+b+c=0$和$abc=2$, 求证: $a,b,c$中至少有一个不小于2.
\item 已知$0<a<1$, $0<b<1$, 求证: $\sqrt{a^2+b^2}+\sqrt{(a-1)^2+b^2}+\sqrt{a^2+(b-1)^2}+\sqrt{(a-1)^2+(b-1)^2}\ge 2\sqrt 2$.
\item 已知实数$x,y,z$不全为零, 求证: $\sqrt{x^2+xy+y^2}+\sqrt{y^2+yz+z^2}+\sqrt{z^2+zx+x^2}>\dfrac 32(x+y+z)$.
\item 已知$x\ge 0$, $y\ge 0$, 求证: $\dfrac 12(x+y)^2+\dfrac 14(x+y)\ge x\sqrt y+y\sqrt x$.
\item 求证: $1+\dfrac 14+\dfrac 19+\dfrac 1{16}+\cdots +\dfrac 1{n^2}<\dfrac 74(n\in \mathbf{N}^*)$.
\item 已知$x>0$, $y>0$, $a,b$是正常数, 且满足$\dfrac ax+\dfrac by=1$, 求证: $x+y\ge (\sqrt a+\sqrt b)^2$.
\item 已知正数$a,b$满足$a^2b=1$, 求$a+b$的最小值.
\item 求$\sin^2\alpha\cos^2\alpha +\dfrac 1{\sin^2\alpha \cos^2\alpha }$的最小值.
\item 已知直角三角形的周长为定值$l$, 求它面积的最大值.
\item 已知圆柱的体积为定值$V$, 求圆柱全面积的最小值.
\item 从半径为$R$的圆形铁片里剪去一个扇形, 然后把剩下部分卷成一个圆锥形漏斗, 要使漏斗有最大容量, 剪去扇形的圆心角$\theta$应是多少弧度?
\item 在Rt$\triangle ABC$中, 已知$\angle C=90^\circ$, $\angle A,\angle B,\angle C$的对边$a,b,c$满足$a+b=cx$. 设$\triangle ABC$绕直线$AB$旋转一周所得的旋转体的侧面积为$S_1$, $\triangle ABC$的内切圆面积为$S_2$. 求:\\
(1) 函数$f(x)=\dfrac{S_1}{S_2}$的解析式和定义域;\\
(2) 函数$f(x)$的最小值.
\end{enumerate}
\end{document}