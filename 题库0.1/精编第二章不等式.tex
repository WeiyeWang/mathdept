\documentclass[10pt,a4paper]{article}
\usepackage[UTF8,fontset = windows]{ctex}
\setCJKmainfont[BoldFont=黑体,ItalicFont=楷体]{华文中宋}
\usepackage{amssymb,amsmath,amsfonts,amsthm,mathrsfs,dsfont,graphicx}
\usepackage{ifthen,indentfirst,enumerate,color,titletoc}
\usepackage{tikz}
\usepackage{makecell}
\usepackage{longtable}

\usetikzlibrary{arrows,calc,intersections,patterns}
\usepackage[bf,small,indentafter,pagestyles]{titlesec}
\usepackage[top=1in, bottom=1in,left=0.8in,right=0.8in]{geometry}
\renewcommand{\baselinestretch}{1.65}
\newtheorem{defi}{定义~}
\newtheorem{eg}{例~}
\newtheorem{ex}{~}
\newtheorem{rem}{注~}
\newtheorem{thm}{定理~}
\newtheorem{coro}{推论~}
\newtheorem{axiom}{公理~}
\newtheorem{prop}{性质~}
\newcommand{\blank}[1]{\underline{\hbox to #1pt{}}}
\newcommand{\bracket}[1]{(\hbox to #1pt{})}
\newcommand{\onech}[4]{\par\begin{tabular}{p{.9\textwidth}}
A.~#1\\
B.~#2\\
C.~#3\\
D.~#4
\end{tabular}}
\newcommand{\twoch}[4]{\par\begin{tabular}{p{.46\textwidth}p{.46\textwidth}}
A.~#1& B.~#2\\
C.~#3& D.~#4
\end{tabular}}
\newcommand{\vartwoch}[4]{\par\begin{tabular}{p{.46\textwidth}p{.46\textwidth}}
(1)~#1& (2)~#2\\
(3)~#3& (4)~#4
\end{tabular}}
\newcommand{\fourch}[4]{\par\begin{tabular}{p{.23\textwidth}p{.23\textwidth}p{.23\textwidth}p{.23\textwidth}}
A.~#1 &B.~#2& C.~#3& D.~#4
\end{tabular}}
\newcommand{\varfourch}[4]{\par\begin{tabular}{p{.23\textwidth}p{.23\textwidth}p{.23\textwidth}p{.23\textwidth}}
(1)~#1 &(2)~#2& (3)~#3& (4)~#4
\end{tabular}}
\begin{document}
\begin{enumerate}[1.]

\item 当$a>b>0$时, 比较$\dfrac{2a+b}{a+2b}$和$\dfrac ab$的大小.
\item 已知$a>0$, $a\ne 1$, $m>n>0$, 比较$A=a^m+\dfrac 1{a^m}$和$B=a^n+\dfrac 1{a^n}$的大小.
\item 若$a>b$, 则下列各式中正确的是\bracket{20}.
\fourch{$a\lg x>b\lg x$($x>0$)}{$ax^2>bx^2$}{$a^2>b^2$}{$2^x\cdot a>2^x\cdot b$}
item 设$ab>0$, 且$\dfrac ca>\dfrac db$, 则下列各式中, 恒成立的是\bracket{20}.
\fourch{$bc<ad$}{$bc>ad$}{$\dfrac ac>\dfrac bd$}{$\dfrac ac<\dfrac bd$}
\item 下列命题中, 不正确的一个是\bracket{20}.
\twoch{若$\sqrt[3]a>\sqrt[3]b$, 则$a>b$}{若$a>b$, $c>d$, 则$a-d>b-c$}{若$a>b>0$, $c>d>0$, 则$\dfrac ad>\dfrac bc$}{若$a>b>0$, $ac>bd$, 则$c>d$}
\item 若$x<y<0$, 则有\bracket{20}.
\fourch{$0<x^2<xy$}{$y^2<xy<x^2$}{$xy<y^2<x^2$}{$y^2>x^2>0$}
\item 若$a=\log_{0.2}0.3$, $b=\log_{0.3}0.2$, $c=1$, 则$a,b,c$的大小关系是\bracket{20}.
\fourch{$a>b>c$}{$b>a>c$}{$b>c>a$}{$c>b>a$}
\item 用不等号(``$>$''或``$<$'')填空:\\
(1) 若$a\ne b$, 则$a^2+3b^2$\blank{50}$2b(a+b)$;\\
(2) 若$c>1$, 则$\sqrt{c+1}-\sqrt c$\blank{50}$\sqrt c-\sqrt{c-1}$;\\
(3) 若$a>b$, $c>d$, 且$a$与$d$都是负数, 则$ac$\blank{50}$bd$.
\item 若``$a>b$, $a-\dfrac 1a>b-\dfrac 1b$''同时成立, 则$ab$应满足的条件是\blank{50}.
\item 已知$a>0$, $b>0$, 且$a\ne b$, 比较$\dfrac{a^2}b+\dfrac{b^2}a$与$a+b$的大小.
\item 已知$0<\dfrac ab<\dfrac cd$, 比较$\dfrac b{a+b}$与$\dfrac d{c+d}$的大小.
\item 若$x>y>1$, $0<a<1$, 则下列各式中正确的一个是\bracket{20}.
\fourch{${x^{-a}}>{y^{-a}}$}{$(\sin a)^x>(\sin a)^y$}{$\log_{\frac 1a}x<\log_{\frac 1a}y$}{$1+a^{x+y}>a^x+a^y$}
\item 已知$a\in \mathbf{R}$, 比较$\dfrac 1{1+a}$与$1-a$的大小.
\item 设$a>0$, $a\ne 1$, $t>0$, 比较$\dfrac 12\log_at$和$\log_a\dfrac{t+1}2$的大小.
\item 已知$x>y>0$, 比较$\sqrt{\dfrac{y^2+1}{x^2+1}}$与$\dfrac yx$的大小.
\item 已知$a$, $b$, $m$, $n$都是正实数, 且$m+n=1$, 比较$\sqrt{ma+nb}$和$m\sqrt a+n\sqrt b$的大小.
\item 解下列不等式:\\
(1) $6x^2-5x-1>0$;\\
(2) $6x^2-5x-1<0$;\\
(3) $5x^2-2x+3>0$;\\
(4) $9x^2+6x+1>0$;\\
(5) $3x^2-4x+5<0$.
\item 已知关于$x$的不等式$ax^2+bx+c<0$的解集是$\{x|x<-2\text{或}x>-\dfrac 12\}$, 求$ax^2-bx+c>0$的解集.
\item 已知集合$A=\{x|x^2+(a-1)x-a>0\}$, $B=\{x|(x+a)(x+b)>0\}$, $a\ne b$, $M=\{x|x^2-2x-3\le 0\}$.\\
(1) 若$\complement_UB=M$, 求$a$, $b$的值;\\
(2) 若$-1<b<a<1$, 求$A\cap B$;\\
(3) 若$-3<a<-1$, 且$a^2-1\in \complement_UA$, 求实数$a$的取值范围.
\item 已知函数$y=(k^2+4k-5)x^2+4(1-k)x+3$的图象都在$x$轴的上方, 求实数$k$的取值范围.
\item 已知$a<b$, 则下列各式中恒成立的是\bracket{20}.
\fourch{$a^2<b^2$}{$c-a>c-b$}{$|a|<|b|$}{$a-1>b-2$}
\item 若$|x|>2$, 则\bracket{20}.
\fourch{$x>2$}{$x>\pm 2$}{$-2<x<2$}{$x>2$或$x<-2$}
\item 不等式$|x|-3<0$的解集是\bracket{20}.
\fourch{$\{x|x<\pm 3\}$}{$\{x|-3<x<3\}$}{$\{x|x>3\}$}{$\{x|x<-3\}$}
\item 已知集合$M=\{x||x|>2\},N=\{x|x<3\}$, 则下列结论正确的是\bracket{20}.
\twoch{$M\cup N=M$}{$M\cap N=\{x|2<x<3\}$}{$M\cup N=R$}{$M\cap N=\{x|x<-2\}$}
\item 已知集合$M=\{x||x+1|\le 2\},P=\{x|x\le 2$或$x\ge 3\}$, 则$M$, $P$之间的关系是\bracket{20}.
\fourch{$M\supseteq P$}{$M\supset P$}{$M\subseteq P$}{$M\subset P$}
\item 已知$|1-x|+\sqrt{x^2-4x+4}=1$, 则$x$的取值范围是\bracket{20}.
\fourch{$1\le x\le 2$}{$x\le 1$}{$x<1$或$x>2$}{$x\ge 2$}
\item 不等式$2x+3-x^2>0$的解集是\bracket{20}.
\fourch{$\{x|-\dfrac 32\le x<1\}$}{$\{x|-1<x<3\}$}{$\{x|1\le x<3\}$}{$\{x|-\dfrac 32\le x<3\}$}
\item 不等式$6x^2+5x<4$的解集是\bracket{20}.
\fourch{$\{x|x<-\dfrac 43\text{或}x>\dfrac 12\}$}{$\{x|-\dfrac 43<x<\dfrac 12\}.$}{$\{x|-\dfrac 12<x<\dfrac 43\}.$}{$\{x|x<-\dfrac 12\text{或}x>\dfrac 43\}$}
\item 当$a<0$时, 关于$x$的不等式$x^2-4ax-5a^2>0$的解集是\bracket{20}.
\fourch{$\{x|x>5a\text{或}x<-a\}$}{$\{x|x<5a\text{或}x>-a\}$}{$\{x|-a<x<5a\}$}{$\{x|5a<x<-a\}$}
\item 若$x$为实数, 则下列命题正确的是\bracket{20}.
\onech{$x^2\ge 2$的解集是$\{x|x\ge \pm \sqrt 2\}$}{$(x-1)^2<2$的解集是$\{x|1-\sqrt 2<x<1+\sqrt 2\}$}{$x^2-9<0$的解集是$\{x|x<3\}$}{设$x_1,x_2$为$ax^2+bx+c=0$的两个实根, 且$x_1>x_2$, 则$ax^2+bx+c>0$的解集是$\{x|x_2<x<x_1\}$}
\item 在\textcircled{1} $x^2-2x-3<0$与$\dfrac{x^2-2x}{x-1}<\dfrac 3{x-1}$; \textcircled{2} $x^2+3x-4>0$与$x^2+3x+\sqrt x>4+\sqrt x$; \textcircled{3} $\dfrac{(x+2)(x^2-1)}{x+2}>0$与$x^2-1>0$''三组不等式中, 解集相同的组数是\bracket{20}.
\fourch{$0$}{$1$}{$2$}{$3$}
\item 若$x^2+x<0$, 则$x^2,x,-x^2,-x$的大小关系是\bracket{20}.
\fourch{$x^2>x>-x^2>-x$}{$-x>x^2>-x^2>x$}{$-x>x^2>x>-x^2$}{$x^2>-x>x>-x^2$}
\item 直接写出下列不等式的解集:\\
(1) $(x-1)^2>0$:\blank{50};\\
(2) $(2-x)(3x+1)>0$:\blank{50};\\
(3) $1-3x^2>2x$:\blank{50};\\
(4) $1-2x-x^2\ge 0$:\blank{50};\\
(5) $x+\sqrt x-6<0$:\blank{50}.
\item 直接写出下列不等式的解集:\\
(1) $\dfrac{3x+4}{x-2}\ge 0$:\blank{50};\\
(2) $\dfrac{4-2x}{1+3x}>0$:\blank{50};\\
(3) $\dfrac 1x>x$:\blank{50};\\	
(4) $x^2-2|x|-3>0$:\blank{50};\\
(5) $x^2-x-5>|2x-1|$:\blank{50}.
\item 若$\sqrt{x^2-x-6}\in \mathbf{R}$, 则$x$的取值范围为\blank{50}.	
\item 要使代数式$\dfrac{\sqrt{x-3}}{\sqrt{x^2-3x+2}}$有意义, 实数$x$的取值范围是\blank{50}.
\item 若代数式$6x^2+x-2$的值恒取非负实数, 则实数$x$的取值范围是\blank{50}.
\item 不等式$4\le x^2-3x<18$的整数解集是\blank{50}.
\item 已知实数$x$满足$4x^2-4x-15\le 0$, 化简$\sqrt{x^2-8x+16}-|x-3|$.
\item 已知$a>b$, 直接写出下列不等式的解集:\\
(1) $\dfrac{x-a}{x-b}\ge 0$:\blank{50};\\
(2) $\dfrac{x-a}{x-b}<0$:\blank{50};\\
(3) $x^2-(a-b)x+ab>0$:\blank{50};\\
(4) $x^2-(a-b)x+ab<0$:\blank{50}.
\item 若关于$x$的方程$2kx^2+(8k+1)x+8k=0$有两个不等实根, 则实数$k$的取值范围是\blank{50}.	
\item 已知$a\ne 0$, 若关于$x$的不等式$ax^2-2ax+2a+3>0$无实数解, 则$a$的取值范围是\blank{50}.
\item 不等式$\dfrac{x-1}{2x}\le 1$的解集是\bracket{20}.
\fourch{$\{x|x\ge -1\}$	}{$\{x|x\le -1\}$}{$\{x|-1\le x<0\}$}{$\{x|x\le -1\text{或}x>0\}$}
\item 若关于$x$的二次不等式$mx^2+8mx+21<0$的解集是$\{x|-1<x<-1\}$, 则实数$m$的值等于\bracket{20}.
\fourch{$1$}{$2$}{$3$}{$4$}
\item 若关于$x$的不等式$(a^2-3)x^2+5x-2>0$的解集是$\{x|\dfrac 12<x<2\}$, 则实数$a$的值等于\bracket{20}.
\fourch{$1$}{$-1$}{$\pm 1$}{$0$}
\item 若关于$x$的不等式$ax^2+bx+c<0(a\ne 0)$的解集是空集, 则\bracket{20}.
\fourch{$a<0$且$b^2-4ac>0$}{$a<0$且$b^2-4ac\le 0$}{$a>0$且$b^2-4ac\le 0$}{$a>0$且$b^2-4ac>0$}
\item 若对任何实数$x$, 二次函数$y=ax^2-x+c$的值恒为负, 则$a,c$应满足\bracket{20}.
\fourch{$\begin{cases}
   a>0,  \\ ac\le \dfrac 14  \end{cases}$}{$\begin{cases}
   a<0,  \\ ac<\dfrac 14  \end{cases}$}{$\begin{cases}
   a<0,  \\ ac>\dfrac 14  \end{cases}$}{$\begin{cases}
   a<0,  \\ ac<0  \end{cases}$}
\item 若对任意实数$x$, 不等式$x^2+2(1+k)x+3+k>0$恒成立, 则$k$的取值范围是\bracket{20}.
\fourch{$-1<k<2$}{$-1\le k\le 2$}{$-2<k<1$}{$-2\le k\le 1$}
\item 若关于$x$的二次方程$2(k+1)x^2+4kx+3k-2=0$的两根同号, 则$k$的取值范围是\bracket{20}.
\twoch{$-2<k<1$}{$-2\le k<-1$或$\dfrac 23<k\le 1$}{$k<-1$或$k>\dfrac 23$}{$-2<k<1$或$\dfrac 23<k<1$}
\item 已知关于$x$的方程$(m+3)x^2-4mx+2m-1=0$的两根异号, 且负根的绝对值比正根大, 那么实数$m$的取值范围是\bracket{20}.
\fourch{$-3<m<0$}{$0<m<3$}{$m<-3$或$m>0$}{$m<0$或$m>3$}
\item 若$\alpha ,\beta$是关于$x$的方程$x^2-(k-2)x+k^2+3k+5=0$($k$为实数)的两个实根, 则${{\alpha }^2}+{{\beta }^2}$的最大值等于\bracket{20}.
\fourch{$19$}{$18$}{$\dfrac{50}9$}{$-6$}
\item 不等式$(x-1)(x-2)(x-3)(x-4)>120$的解为\bracket{20}.
\fourch{$x>6$}{$x<-1$或$x>6$}{$x<-1$}{$-1<x<6$}
\item 在三个关于$x$的方程$x^2-ax+4=0$, $x^2+(a-1)x+16=0$和$x^2+2ax+3a+10=0$中, 已知至少有一个方程有实根, 则实数$a$的取值范围是\bracket{20}.
\fourch{$-4\le a\le 4$}{$-2<a<4$}{$a\le -2$或$a\ge 4$}{$a<0$}
\item 若关于$x$的二次方程$x^2-2mx+4x+2m^2-4m-2=0$有实根, 则其两根之积的最大值等于\blank{50}.
\item 使关于$x$的方程$x^2-kx+2k-3=0$的两实根的平方和取最小值, 实数$k$的值等于\blank{50}.
\item 若关于$x$的不等式$x^2-mx+n\le 0$的解集是$\{x|-5\le x\le 1\}$, 则实数$m=$\blank{50}, $n=$\blank{50}.
\item 若关于$x$的不等式$ax^2+bx+1\ge 0$的解集是$\{x|-5\le x\le 1\}$, 则实数$a=$\blank{50}, $b=$\blank{50}.
\item 若关于$x$的不等式$ax^2+bx+2>0$的解集是$\{x|-\dfrac 12<x<\dfrac 13\}$, 则实数$a=$\blank{50}, $b=$\blank{50}.
\item 若关于$x$的不等式$ax^2+bx-6>0$的解集是$\{x|2<x<3\}$, 则实数$a=$\blank{50}, $b=$\blank{50}.
\item 若关于$x$的不等式$(a+b)x+(2a-3b)<0$的解集是$\{x|x>3\}$, 则不等式$(a-3b)x+b-2a>0$的解集是\blank{50}.
\item 若关于$x$的不等式$ax^2+bx+c<0$的解集是$\{x|x<-2\text{或}x>-\dfrac 12\}$, 则关于$x$的不等式$ax^2-bx+c>0$的解集是\blank{50}.
\item 解不等式$x^4-2x^2+1>x^2-1$.
\item 已知关于$x$的不等式$kx^2-2x+6k<0(k\ne 0)$.\\
(1) 若不等式的解集是$\{x|x<-3\text{或}x>-2\}$, 求实数$k$的值;\\
(2) 若不等式的解集是$\{x|x\ne \dfrac 1k\}$, 求实数$k$的值;\\
(3) 若不等式的解集是实数集, 求实数$k$的值.
\item 已知关于$x$的方程$m(x-1)=3(x+2)$的解是正实数, 求实数$m$的取值范围.
\item 已知关于$x$的方程$\dfrac 14x^2-kx+5k-6=0$无实数解, 求实数$k$的取值范围.
\item 已知关于$x$的方程$kx^2-(3k-1)x+k=0$有两个正实数根, 求实数$k$的取值范围.
\item 已知集合$M=\{x|x^2-7x+10\le 0\}$, $N=\{x|x^2-(2-m)x+5-m\le 0\}$, 且$N\subseteq M$, 求实数$m$的取值范围.
\item 已知集合$A=\{x|x^2+4x+p<0\}$, $B=\{x|x^2-x-2>0\}$, 且$A\subseteq B$, 求实数$p$的取值范围.
\item 已知集合$A=\{x|x^2+ax+1\le 0\}$, $B=\{x|x^2-3x+2\le 0\}$, 且$A\subseteq B$, 求实数$a$的取值范围.
\item 已知集合$A=\{x|x^2-2x-3\le 0\}$, $B=\{x|x^2+px+q<0\}$, 且$A\cap B=\{x|-1\le x<2\}$, 求实数$p,q$的关系式及其取值范围.
\item 已知集合$A=\{x|-2<x<-1\text{或}x>\dfrac 12\}$, $B=\{x|x^2+ax+b\le 0\}$, 且$A\cup B=\{x|x+2>0\}$, $A\cap B=\{x|\dfrac 12<x\le 3\}$, 求$a,b$的值.
\item 要使代数式$mx^2+(m-1)x+(m-1)$的值恒为负值, 求实数$m$的取值范围.
\item 已知关于$x$的不等式$(a^2-4)x^2+(a+2)x-1\ge 0$的解集是空集, 求实数$a$的取值范围.
\item 若关于$x$的不等式$\dfrac{x^2-8x+20}{mx^2+2(m+1)x+9m+4}<0$的解集为$\mathbf{R}$, 求实数$m$的取值范围.
\item 当$0^\circ <\varphi <90^\circ$时, 要使$\dfrac{x^2-6x+8}{x^2+2}=\sin \varphi$恒成立, 求实数$x$的取值范围.
\item 既要使关于$x$的不等式$x^2+(m-\dfrac 12)x-\dfrac 7{16}\le 0$有实数解, 又要使关于$x$的方程$(2m+3)x^2+mx+\dfrac{m-2}4=0$有实数解, 求实数$m$的取值范围.
\item 为长$80\text{cm}$、宽$60\text{cm}$的工作台做一块台布, 使台布的面积是台面面积的两倍以上, 并使台子四边垂下的长度相等, 问: 垂下的长度至少是多少(精确到$0.1\text{cm}$)?
\item 已知非零实数$x,y,z$, 满足$x+y+z=xyz$, $x^2=yz$, 求证: $x^2\ge 3$.
\item 已知$a+b\ge 0$, 求证: $a^3+b^3\ge a^2b+ab^2$.
\item 设$a,b\in \mathbf{R}^+$, 且$a\ne b$, 求证: $a^ab^b>a^bb^a$.
\item 已知$a,b,c\in \mathbf{R}$, 求证: $a^2+b^2+c^2\ge ab+bc+ca$.
\item 已知$a,b,c>0$, 求证:
(1) $(a+b)(\dfrac 1a+\dfrac 1b)\ge 4$;\\
(2) $(a+b+c)(\dfrac 1a+\dfrac 1b+\dfrac 1c)\ge 9$.
\item 已知正数$a,b$满足$a+b=1$, 求证: $\sqrt{2a+1}+\sqrt{2b+1}\le 2\sqrt 2$.
\item 已知$\alpha ,\beta \in (0,\dfrac{\pi}2)$, 且$\alpha \ne \beta$, 求证: $\tan \alpha +\tan \beta >2\tan \dfrac{\alpha +\beta}2$.
\item 记$f(x)=x^2+ax+b$, 求证: $|f(1)|,|f(2)|,|f(3)|$中至少有一个不小于$\dfrac 12$.
\item 已知$-1\le x\le 1$, $n\ge 2$, $n\in \mathbf{N}$, 求证: $(1-x)^n+(1+x)^n\le 2^n$.
\item 已知$x+2y+3z=12$, 求证: $x^2+2y^2+3z^2\ge 24$.
\item 已知$a,b,c\in \mathbf{R}^+$, 求证: $a^3+b^3+c^3\ge 3abc$(当且仅当$a=b=c$时取等号).
\item 已知$a>0$, 求证: $x+\dfrac 1x+\dfrac 1{x+\dfrac 1x}\ge \dfrac 52$.
\item 已知实数$a,b,c$满足$a+b+c=0$和$abc=2$, 求证: $a,b,c$中至少有一个不小于2.
\item 已知$0<a<1$, $0<b<1$, 求证: $\sqrt{a^2+b^2}+\sqrt{(a-1)^2+b^2}+\sqrt{a^2+(b-1)^2}+\sqrt{(a-1)^2+(b-1)^2}\ge 2\sqrt 2$.
\item 已知实数$x,y,z$不全为零, 求证: $\sqrt{x^2+xy+y^2}+\sqrt{y^2+yz+z^2}+\sqrt{z^2+zx+x^2}>\dfrac 32(x+y+z)$.
\item 已知$x\ge 0$, $y\ge 0$, 求证: $\dfrac 12(x+y)^2+\dfrac 14(x+y)\ge x\sqrt y+y\sqrt x$.
\item 求证: $1+\dfrac 14+\dfrac 19+\dfrac 1{16}+\cdots +\dfrac 1{n^2}<\dfrac 74(n\in \mathbf{N}^*)$.
\item 已知$x>0$, $y>0$, $a,b$是正常数, 且满足$\dfrac ax+\dfrac by=1$, 求证: $x+y\ge (\sqrt a+\sqrt b)^2$.
\item 已知正数$a,b$满足$a^2b=1$, 求$a+b$的最小值.
\item 求$\sin^2\alpha\cos^2\alpha +\dfrac 1{\sin^2\alpha \cos^2\alpha }$的最小值.
\item 已知直角三角形的周长为定值$l$, 求它面积的最大值.
\item 已知圆柱的体积为定值$V$, 求圆柱全面积的最小值.
\item 从半径为$R$的圆形铁片里剪去一个扇形, 然后把剩下部分卷成一个圆锥形漏斗, 要使漏斗有最大容量, 剪去扇形的圆心角$\theta$应是多少弧度?
\item 在Rt$\triangle ABC$中, 已知$\angle C=90^\circ$, $\angle A,\angle B,\angle C$的对边$a,b,c$满足$a+b=cx$. 设$\triangle ABC$绕直线$AB$旋转一周所得的旋转体的侧面积为$S_1$, $\triangle ABC$的内切圆面积为$S_2$. 求:\\
(1) 函数$f(x)=\dfrac{S_1}{S_2}$的解析式和定义域;\\
(2) 函数$f(x)$的最小值.
\item 用比较法证明以下各题:\\
(1) 已知$a>0$, $b>0$, 求证: $\dfrac 1a+\dfrac 1b\ge \dfrac 2{\sqrt{ab}}$;\\
(2) 已知$a>0$, $b>0$, 求证: $\dfrac b{\sqrt a}+\dfrac a{\sqrt b}\ge \sqrt a+\sqrt b$;\\
(3) 已知$a>0$, $b>0$, 求证: ${a^2}+{b^2}\ge (a+b)\sqrt{ab}$;\\
(4) 已知$0<x<1$, 求证: $\dfrac{a^2}x+\dfrac{b^2}{1-x}\ge (a+b)^2$.
\item 已知$a\ge 0$, $b\ge 0$, 求证: $a^3+b^3\ge a^2b+b^2a$.
\item 已知$x\in \mathbf{R}^+$, $y\in \mathbf{R}^+$, $n\in \mathbf{N}$, 求证: $x^{n+1}+y^{n+1} \ge x^ny+xy^n$.
\item 已知$a>0$, $b>0$, $c>0$, 求证: $a(b^2+c^2)+b(c^2+a^2)+c(a^2+b^2)\ge 6abc$.
\item 求证: $a^5+b^5\ge \dfrac 12(a^3+b^3)(a^2+b^2)$($a>0$, $b>0$).
\item 求证: $a^2+b^2+c^2\ge ab+bc+ca$($a,b,c$是实数).
\item 已知$a>b>c$, 求证: $a^2b+b^2c+c^2a>ab^2+bc^2+ca^2$.
\item 在$\triangle ABC$中, 记$a,b,c$分别是角$A,B,C$的对边, $S$是$\triangle ABC$的面积, 求证: ${c^2}-{a^2}-{b^2}+4ab\ge 4\sqrt 3S$.
\item 设$a,b\in \mathbf{N}$, 则$\sqrt 2$在$\dfrac ba$与$\dfrac{2a+b}{a+b}$之间.
\item 已知$a,b,c$都是正数, 求证: $a^{2a}b^{2b}\ge a^{b+c}b^{c+a}c^{a+b}$.
\item 下列命题中, 正确的一个是\bracket{20}.
\twoch{若$a,b,c\in \mathbf{R}$, 且$a>b$, 则$ac^2>bc^2$}{若$a,b\in \mathbf{R}$, 且$a\cdot b\ne 0$, 则$\dfrac ab+\dfrac ba\ge 2$}{若$a,b\in \mathbf{R}$, 且$a>|b|$, 则$a^n>b^n$($n\in \mathbf{N}$)}{若$a>b$, $c<d$, 则$\dfrac ac>\dfrac bd$}
\item 下列各式中, 对任何实数$x$都成立的一个是\bracket{20}.
\fourch{$\lg (x^2+1)\ge \lg 2x$}{$x^2+1>2x$}{$\dfrac 1{x^2+1}\le 1$}{. $x+\dfrac 1x\ge 2$}
\item 已知, $a,b\in \mathbf{R}$, 且$a,b\ne 0$, 则在\textcircled{1} $\dfrac{a^2+b^2}2\ge ab$; \textcircled{2} $\dfrac ba+\dfrac ab\ge 2$;  \textcircled{3} $ab\le (\dfrac{a+b}2)^2$; \textcircled{4} $(\dfrac{a+b}2)^2\le \dfrac{a^2+b^2}2$这四个式子中, 恒成立的个数是\bracket{20}.
\fourch{$1$}{$2$}{$3$}{$4$}
\item 若$a>0$, $b>0$, $c>0$, $d>0$, 则$\dfrac ba+\dfrac ab$的最小值为\blank{50}, $\dfrac ba+\dfrac cb+\dfrac ac$的最小值为\blank{50}, $\dfrac{b+c}a+\dfrac{c+a}b+\dfrac{a+b}c$的最小值为\blank{50}, $(a+b)(\dfrac 1a+\dfrac 1b)$的最小值为\blank{50}, $(\dfrac ba+\dfrac dc)(\dfrac cb+\dfrac ad)$的最小值为\blank{50}.
\item 若$x>0$, 则$x+\dfrac 1x$的最小值为\blank{50}; 若$x<0$, 则$(-x)+\dfrac 1{-x}$的最小值为\blank{50}, $x+\dfrac 1x$的最大值为\blank{50}.
\item 若$a>1$, $b>1$, $c>1$, 则$\log_ab+\log_ba$的最小值为\blank{50}, $\log_ab+\log_bc+\log_ca$的最小值为\blank{50}.
\item 若$0<a<1$, $0<b<1$, 则$\log_ab+\log_ba$的最小值为\blank{50}.
\item 若$a>1$, $0<b<1$, 则$\log_ab+\log_ba$的最大值为\blank{50}.
\item 设$a,b$为正数, 且$a+b\le 4$, 则下列各式中, 一定正确的是\bracket{20}.
\fourch{$\dfrac 1a+\dfrac 1b\le \dfrac 14$}{$\dfrac 14\le \dfrac 1a+\dfrac 1b\le \dfrac 12$}{$\dfrac 12\le \dfrac 1a+\dfrac 1b\le 1$}{$\dfrac 1a+\dfrac 1b\ge 1$}
\item 若$a,b,c$均大于1, 且$\log_ac\cdot \log_bc=4$, 则下列各式中, 一定正确的是\bracket{20}.
\fourch{$ac\ge b$}{$ab\ge c$}{$bc\ge a$}{$ab\le c$}
\item 若$a>0$, $b>0$, 且$a\ne b$, 则下列各式恒成立的是\bracket{20}.
\fourch{$\dfrac{2ab}{a+b}<\dfrac{a+b}2<\sqrt{ab}$}{$\sqrt{ab}<\dfrac{2ab}{a+b}<\dfrac{a+b}2$}{$\dfrac{2ab}{a+b}<\sqrt{ab}<\dfrac{a+b}2$}{$\sqrt{ab}<\dfrac{a+b}2<\dfrac{2ab}{a+b}$}
\item 利用公式$a^2+b^2\ge 2ab$或$a+b\ge 2\sqrt{ab}$($a,b\ge 0$), 求证:
若$x>0$, $y>0$, 则$\sqrt{(1+x)(1+y)}\ge 1+\sqrt{xy}$.
\item 利用公式$a^2+b^2\ge 2ab$或$a+b\ge 2\sqrt{ab}$($a,b\ge 0$), 求证: 若$a>0$, $b>0$, $c>0$, 则$ab(a+b)+bc(b+c)+ca(c+a)\ge 6abc$.
\item 利用公式$a^2+b^2\ge 2ab$或$a+b\ge 2\sqrt{ab}$($a,b\ge 0$), 求证: 若$a>0$, $b>0$, 则$a+b+\dfrac 1{\sqrt{ab}}\ge 2\sqrt 2$.
\item 利用公式$a^2+b^2\ge 2ab$或$a+b\ge 2\sqrt{ab}$($a,b\ge 0$), 求证: 若$m=x{{\cos }^2}\theta +y{{\sin }^2}\theta$, $n=x{{\sin }^2}\theta +y{{\cos }^2}\theta$, 则$mn\ge xy$.
\item 利用公式$a^2+b^2\ge 2ab$或$a+b\ge 2\sqrt{ab}$($a,b\ge 0$), 求证: 若$x+3y-1=0$, 则${2^x}+{8^y}\ge 2\sqrt 2$.
\item 利用公式$a^2+b^2\ge 2ab$或$a+b\ge 2\sqrt{ab}$($a,b\ge 0$), 求证: $\log_{0.5}(\dfrac 1{4^a}+\dfrac 1{4^b})\le a+b-1$.
\item 已知$x>0$, $y>0$, $x+y=1$, 求证:\\
(1) $(1+\dfrac 1x)(1+\dfrac 1y)\ge 9$;\\
(2) $(\dfrac 1{x^2}-1)(\dfrac 1{y^2}-1)\ge 9$.
\item 已知$a>0$, $b>0$, $c>0$, $a+b+c=1$, 求证: $(1-a)(1-b)(1-c)\ge 8abc$.
\item 已知$a>0$, $b>0$, $c>0$, $a+b+c=1$, 求证: $(\dfrac 1a-1)(\dfrac 1b-1)(\dfrac 1c-1)\ge 8$.
\item 已知$a>0$, $b>0$, $c>0$, $a+b+c=1$, 求证: $\dfrac 1a+\dfrac 1b+\dfrac 1c\ge 9$.
\item 已知$a>0$, $b>0$, $c>0$, $a+b+c=1$, 求证: $\dfrac 1{abc}\ge 27$.
\item 已知$a>0$, $b>0$, $c>0$, $a+b+c=1$, 求证: $(1+\dfrac 1a)(1+\dfrac 1b)(1+\dfrac 1c)\ge 64$.
\item 利用公式$\dfrac{a+b+c}3\le \sqrt{\dfrac{a^2+b^2+c^2}3}$, 求证: $\sqrt{a^2}+{b^2}+\sqrt{b^2}+{c^2}+\sqrt{c^2}+{a^2}\ge \sqrt 2(a+b+c)$.
\item 利用公式$\dfrac{a+b}2\le \sqrt{\dfrac{a^2+b^2}2}$, 求证: 若$a+b=1(a,b\ge 0)$, 则$\sqrt{2a+1}+\sqrt{2b+1}\le 2\sqrt 2$.
\item 利用公式$\dfrac{a+b+c}3\le \sqrt{\dfrac{a^2+b^2+c^2}3}$, 求证: 若$a+b+c=1(a,b,c\ge 0)$, 则$\sqrt{13a+1}+\sqrt{13b+1}+\sqrt{13c+1}\le 4\sqrt 3$.
\item 利用公式$\dfrac{a+b}2\le \sqrt{\dfrac{a^2+b^2}2}$, 求证: $a\cos \varphi +b\sin \varphi +c\le \sqrt{2(a^2+b^2+c^2)}$.
\item 利用$a^2+b^2+c^2\ge ab+bc+ca(a,b,c\in \mathbf{R})$, 证明: 若$a>0$, $b>0$, $c>0$, 则$\dfrac{a^2}{b^2}+{b^2}{c^2}+{c^2}{a^2}{a+b+c}\ge abc$.
\item 利用$a^2+b^2+c^2\ge ab+bc+ca(a,b,c\in \mathbf{R})$, 证明: 若半径为$1$的圆内接$\triangle ABC$的而积为$\dfrac 14$, 二边长分别为$a,b,c$, 则\\(1) $abc=1$;\\
(2) $\sqrt b+\sqrt c<\dfrac 1a+\dfrac 1b+\dfrac 1c$.
\item 利用$a^2+b^2+c^2\ge ab+bc+ca(a,b,c\in \mathbf{R})$, 证明: 若$a,b,c>0$, $n\in \mathbf{N}$, $f(n)=\lg \dfrac{a^n+b^n+c^n}3$, 则$2f(n)\le f(2n)$.
\item 利用放缩法并结合公式$ab\le (\dfrac{a+b}2)^2$, 证明: $\lg 9\cdot \lg 11<1$.
\item 利用放缩法并结合公式$ab\le (\dfrac{a+b}2)^2$, 证明: $\log_a(a-1)\cdot \log_a(a+1)<1$($a>1$).
\item 利用放缩法并结合公式$ab\le (\dfrac{a+b}2)^2$, 证明: 若$a>b>c$, 则$\dfrac 1{a-b}+\dfrac 1{b-c}+\dfrac 4{c-a}\ge 0$.
\item 利用放缩法证明: $\dfrac 1n+\dfrac 1{n+1}+\dfrac 1{n+2}+\dfrac 1{n+3}+\dfrac 1{n+4}+\cdots +\dfrac 1{n^2}>1$($n\in \mathbf{N}$, $n\ge 2$).
\item 利用放缩法证明: $\dfrac 12\le \dfrac 1{n+1}+\dfrac 1{n+2}+\cdots +\dfrac 1{2n}<1$($n\in \mathbf{N}$).
\item 利用放缩法证明: 已知$a>0$, $b>0$, $c>0$, 且$a^2+b^2=c^2$, 求证: $a^n+b^n<c^n$($n\ge 3$, $n\in \mathbf{N}$).
\item 利用拆项法证明: 若$x>y$, $xy=1$, 则$\dfrac{x^2+y^2}{x-y}\ge 2\sqrt 2$.
\item 利用拆项法证明: $\dfrac 12({a^2}+{b^2})+1\ge \sqrt{{a^2}+1}\cdot \sqrt{{b^2}+1}$.
\item 利用拆项法证明: 若$a>0$, $b>0$, $c>0$, 则$2(\dfrac{a+b}2-\sqrt{ab})\le 3(\dfrac{a+b+c}3-\sqrt[3]{abc})$.
\item 利用拆项法证明: $2(\sqrt{n+1}-1)<1+\dfrac 1{\sqrt 2}+\dfrac 1{\sqrt 3}+\cdots +\dfrac 1{\sqrt n}<2\sqrt n$($n\in \mathbf{N}$).
\item 利用逆代法证明: 若正数$x,y$满足$x+2y=1$, 则$\dfrac 1x+\dfrac 1y\ge 3+2\sqrt 2$.
\item 利用逆代法证明: $\dfrac 1{\sin ^2\alpha}+\dfrac 3{\cos^2\alpha}\ge 4+2\sqrt 3$.
\item 利用逆代法证明: 若$x,y>0$, $a,b$为正常数, 且$\dfrac ax+\dfrac ay=1$, 则$x+y\ge (\sqrt a+\sqrt b)^2$.
\item 利用判别式法证明: $\dfrac 13\le \dfrac{x^2-x+1}{x^2+x+1}\le 3$.
\item 利用判别式法证明: 若关于$x$的不等式$(a^2-1)x^2-(a-1)x-1<0(a\in \mathbf{R})$对仟意实数$x$恒成立, 则$-\dfrac 35<a\le 1$.
\item 利用函数的单调性证明: 若$x>0$, $y>0$, $x+y=1$, 则$(x+\dfrac 1x)(y+\dfrac 1y)\ge \dfrac{25}4$.
\item 利用函数的单调性证明: 若$0<a<\dfrac 1k(k\ge 2,k\in \mathbf{N})$, 且$a^2<a-b$, 则$b<\dfrac 1{k+1}$.
\item 利用三角换元法证明: 若$a^2+b^2=1$, 则$a\sin x+b\cos x\le 1$.
\item 利用三角换元法证明: 若$|a|<1$, $|b|<1$, 则$|ab\pm \sqrt{(1-{a^2})(1-{b^2})}|\le 1$.
\item 利用三角换元法证明: 若$x^2+y^2\le 1$, 则$-\sqrt 2\le x^2+2xy-y^2\le \sqrt 2$.
\item 利用三角换元法证明: 若$|x|\le 1$, 则$(1+x)^n+(1-x)^n\le 2^n$.
\item 利用三角换元法证明: 若$a>0$, $b>0$, 且$a-b=1$, 则$0<\dfrac 1a(\sqrt a-\dfrac 1{\sqrt a})(\sqrt b+\dfrac 1{\sqrt b})<1$.
\item 利用三角换元法证明: $0<\sqrt{1+x}-\sqrt x\le 1$.
\item 试构造几何图形证明: 若$f(x)=\sqrt{1+x^2}$, $x>b>0$, 则$|f(a)-f(b)|<|a-b|$.
\item 试构造几何图形证明: 若$x,y,z>0$, 则$\sqrt{x^2+y^2+xy}+\sqrt{y^2+z^2+yz}>\sqrt{z^2+x^2+zx}$.
\item 利用均值换元证明: 若$a>0$, $b>0$, 且$a+b=1$, 则$\dfrac 43\le \dfrac 1{a+1}+\dfrac 1{b+1}<\dfrac 32$.
\item 利用均值换元证明: 若$a+b+c=1$, 则${a^2}+{b^2}+{c^2}\ge \dfrac 13$.
\item 利用设差换元证明: 若$x\ge y\ge 0$, 则$\sqrt{2xy-{y^2}}+\sqrt{x^2-y^2}\ge x$.
\item 已知$a,b,c$都是正数, 求证: $a^ab^bc^c\ge (abc)^{\frac{a+b+c}3}$.
\item 已知正数$a,b$满足$a+b=1$, 求证: $(ax+by)(ay+bx)\ge xy$.
\item 已知正数$a,b$满足$a+b=1$, 求证: $(a+\dfrac 1a)^2+(b+\dfrac 1b)^2\ge \dfrac{25}2$.
\item 已知正数$a,b$满足$a+b=1$, 求证: $(a+\dfrac 1a)(b+\dfrac 1b)\ge \dfrac{25}4$.
\item 已知正数$a,b,c$满足$a+b+c=1$, 求证: $(a+\dfrac 1a)+(b+\dfrac 1b)+(c+\dfrac 1c)\ge 10$.
\item 已知正数$a,b,c$满足$a+b+c=1$, 求证: $(a+\dfrac 1a)^2+(b+\dfrac 1b)^2+(c+\dfrac 1c)^2\ge \dfrac{100}3$
\item 已知正数$a,b,c$满足$a+b+c=1$, 求证: $\dfrac 1{\sqrt a}+\dfrac 1{\sqrt b}+\dfrac 1{\sqrt c}\ge 3\sqrt 3$.
\item 已知$a^2+b^2+c^2=1$, 求证: $-\dfrac 12\le ab+bc+ca\le 1$.					
\item 已知$a^2+b^2+c^2=1$, 求证: $|abc|\le \dfrac{\sqrt 3}9$.
\item 已知$x>1$, 求证: $\sqrt x-\sqrt{x-1}>\sqrt{x+1}-\sqrt x$.
\item 已知$a>0$, $b>0$, $c>0$, 求证: $\dfrac 1a+\dfrac 1b+\dfrac 1c\ge 2(\dfrac 1{a+b}+\dfrac 1{b+c}+\dfrac 1{c+a})$.
\item 已知$a>0$, $b>0$, $c>0$, 求证: $\dfrac c{a+b}+\dfrac a{b+c}+\dfrac b{c+a}\ge \dfrac 32$.
\item 已知$\alpha ,\beta \in (0,\dfrac{\pi}2)$, 求证: $\dfrac 1{\cos^2\alpha}+\dfrac 1{\sin^2\alpha \sin^2\beta\cos^2\beta}\ge 9$.
\item 已知$a>0$, $b>0$, $c>0$, 求证: $\dfrac 1{a+b}+\dfrac 1{b+c}+\dfrac 1{c+a}\ge \dfrac 9{2(a+b+c)}$.
\item 己知$\tan \alpha,\tan \beta$是关于$x$的方程$mx^2+(2m-3)x+(m-2)=0(m\ne 0)$的两根, 求证: $\tan (\alpha +\beta)\ge -\dfrac 34$.
\item 已知长方体的对角线长为定长$l$, 求证: 它的体积$V\le \dfrac{\sqrt 3l^3}9$.
\item 在$\triangle ABC$中, 求证: $\cos A+\cos B+\cos C\le \dfrac 32$.
\item 在$\triangle ABC$中, 求证: $\sin \dfrac A2\sin \dfrac B2\sin \dfrac C2\le \dfrac 18$.
\item 在$\triangle ABC$中, 求证: $\tan A\tan B\tan C\ge 3\sqrt 3$, 其中三内角$A,B,C$都是锐角.
\item 在$\triangle ABC$中, 求证: $a^2+b^2+c^2\ge 4\sqrt 3S$, 其中三内角$A,B,C$的对边分别为$a,b,c$, 三角形的面积为$S$.
\item 已知$f(x)=\lg \dfrac{1+2^x+a\cdot 4^x}3$($a\in \mathbf{R}$).\\
(1) 如果$x\le 1$时$f(x)$有意义, 求$a$的取值范围;\\
(2) 如果$0<a\le 1$, 求证: $x\ne 0$时, $2f(x)<f(2x)$.
\item 求证: $2+\sin \theta +\cos \theta \ge \dfrac 2{2-\sin \theta -\cos \theta }$.
\item 求证: $-1<\dfrac{4\sin \theta +3}{\sin^2\theta+1}\le 4$.
\item 求证: $\dfrac{x+b+c+abc}{1+ab+bc+ca}\le 1$, 其中$0\le a\le 1$, $0\le b\le 1$, $0\le c\le 1$.
\item 求证: $2\sin 2\alpha \le \cot \dfrac{\alpha}2$, 其中$0<\alpha <\pi$.
\item 求证: 若$x>-1$, 则$(\dfrac 13)^{x+\frac 32}<(\dfrac 13)^{\sqrt{(x+1)(x+2)}}$.
\item 求证: 若$a>b>0$, $c>d>0$, 则$\sqrt{ac}-\sqrt{bd}>\sqrt{(a-b)(c-d)}$.
\item 求证: $ac+bd\le \sqrt{a^2+b^2}\cdot \sqrt{c^2+d^2}$.
\item 求证: 若$x>y>0$, $\theta \in (0,\dfrac{\pi}2)$, 则$x\sec \theta -y\tan \theta \ge \sqrt{x^2-y^2}$.
\item 求证: 若$-1<x<1$, $-1<y<1$, 则$|\dfrac{x+y}{1+xy}|<1$.
\item 求证: $16^{18}>18^{16}$.
\item 求证: $(\sqrt 2)^{\sqrt 3}<(\sqrt 3)^{\sqrt 2}$.
\item 求证: 若$a>0$, $b>0$, $a+b=1$, 则$3^a+3^b<4$.
\item 利用反证法证明: 若$0<a<1$, $0<b<1$, $0<c<1$, 则$(1-a)b$, $(1-b)c$, $(1-c)a$不能都大于$\dfrac 14$.
\item 利用反证法证明: 若$0<a<2$, $0<b<2$, $0<c<2$, 则$a(2-b)$, $b(2-c)$, $c(2-a)$不可能都大于$1$.
\item 利用反证法证明: 若$x,y>0$, 且$x+y>2$, 则$\dfrac{1+y}x$和$\dfrac{1+x}y$中至少有一个小于$2$.
\item 利用反证法证明: 若$0<a<1$, $b>0$, 且$a^b=b^a$, 则$a=b$.
\item 若$a>0$, $b>0$, 且$a^3+b^3=2$, 试分别利用$x^3+y^3+z^3\ge 3xyz$($x,y,z\ge 0$)构造方程, 并利用判别式以及反证法证明: $a+b\le 2$.
\item 下列函数中, 最小值为$2$的是\bracket{20}.
\twoch{$x+\dfrac 1x$}{$\dfrac{x^2+2}{\sqrt{x^2+1}}$}{$\log_ax+\log_xa$($a>0$, $x>0$, $a\ne 1$, $x\ne 1$)}{$3^x+3^{-x}$($x>0$)}
\item 若$\log_{\sqrt 2}x+\log_{\sqrt 2}y=4$, 则$x+y$的最小值是\bracket{20}.
\fourch{$8$}{$4\sqrt 2$}{$4$}{$2$}
\item 若$a,b$均为大于$1$的正数, 且$ab=100$, 则$\lg a\cdot \lg b$的最大值是\bracket{20}.
\fourch{$0$}{$1$}{$2$}{$\dfrac 52$}
\item 若实数$x$与$y$满足$x+y-4=0$, 则$x^2+y^2$的最小值是\bracket{20}.
\fourch{$4$}{$6$}{$8$}{$10$}
\item 若非负实数$a,b$满足$2a+3b=10$, 则$\sqrt{3b}+\sqrt{2a}$的最大值是\bracket{20}.
\fourch{$\sqrt{10}$}{$2\sqrt 5$}{$5$}{$10$}
\item 若$x>1$, 则$\dfrac{x^2-2x+2}{2x-2}$有\bracket{20}.
\fourch{最小值$1$}{最大值$1$}{最小值$-1$}{最大值$-1$}
\item 若$x,y\in \mathbf{R}^+$, 且$x^2+y^2=1$, 则$x+y$的最大值是\blank{50}.
\item 若$x+2y=2\sqrt 2a$($x>0$, $y>0$, $a>1$), 则$\log_ax+\log_ay$的最大值是\blank{50}.
\item 若$x>1$, 则$2+3x+\dfrac 4{x-1}$的最小值\blank{50}, 此时$x=$\blank{50}.
\item 若$x>0$, 则$x+\dfrac 1x+\dfrac{16x}{x^2+1}$的最小值是\blank{50}, 此时$x=$\blank{50}.
\item 若正数$a,b$满足$a^2+\dfrac{b^2}2=1$, 则$a\sqrt{1+b^2}$的最大值为\blank{50}, 此时$a=$\blank{50}, $b=$\blank{50}.
\item 若$x>0$, 则$3x+\dfrac{12}{x^2}$的最小值是\blank{50}, 此时$x=$\blank{50}.
\item 若$0<x<\dfrac 13$, 则$x^2(1-3x)$的最大值是\blank{50}, 此时$x=$\blank{50}.
\item 若$xy>0$, 且$x^2y=2$, 则$xy+x^2$的最小值是\blank{50}.
\item $\sin^4\alpha \cos^2\alpha$的最大值是\blank{50}, 此时$\sin \alpha =$\blank{50}, $\cos \alpha =$\blank{50}.
\item 若正数$x,y,z$满足$5x+2y+z=100$, 则$\lg x+\lg y+\lg z$的最大值是\blank{50}.
\item 若$\dfrac{x^2}4+{y^2}=x$, 则$x^2+y^2$有\bracket{20}.
\fourch{最小值$0$, 最大值$16$}{最小值$-\dfrac 13$, 最大值$0$}{最小值$0$, 最大值$1$}{最小值$1$, 最大值$2$}
\item $|\sin x|+|\cos x|$的最大值是\bracket{20}.
\fourch{$2$}{$\sqrt 2$}{$\dfrac{\sqrt 2}2$}{$\dfrac 12$}
\item 若$x>0$, 则$\dfrac x{x^3+2}$的最大值是\bracket{20}.
\fourch{$5$}{$3$}{$1$}{$\dfrac 13$}
\item 若正数$a,b$满足$ab-(a+b)=1$, 则$a+b$的最小值是\bracket{20}.
\fourch{$2+2\sqrt 2$}{$2\sqrt 2-2$}{$\sqrt 5+2$}{$\sqrt 5-2$}
\item 已知$a>1$且$a^{\lg b}=\sqrt[4]2$, 求$\log_2(ab)$的最小值.
\item 求函数$y=\dfrac{x^4+3x^2+3}{x^2+1}$的最小值.
\item 求$f(x)=4x^2+\dfrac{16}{(x^2+1)^2}$的最小值.
\item 求$f(x)=x^2-3x-2-\dfrac 3x+\dfrac 1{x^2}$($x>0$)的最小值.
\item 若$x,y>0$, 求$\dfrac{\sqrt x+\sqrt y}{\sqrt{x+y}}$的最大值.
\item 已知正常数$a,b$和正变数$x,y$满足$a+b=10$, $\dfrac ax+\dfrac by=1$, $x+y$的最小值为$18$, 求$a,b$的值.
\item 已知$x^2+y^2=1$, 求$(1+xy)(1-xy)$的最大值和最小值.
\item 已知$x^2+y^2=3$, $a^2+b^2=4$, 求$ax+by$的最大值和最小值.
\item 已知$\sqrt{1-y^2}+y\sqrt{1-x^2}=1$, 求$x+y$的最大值和最小值.
\item 已知函数$f(x)=\dfrac{2^{x+3}}{{4^x}+8}$.\\
(1) 求$f(x)$的最大值;\\
(2) 对于任意实数$a,b$, 求证: $f(a)<b^2-4b+\dfrac{11}2$.
\item 若直角三角形的周长为$1$, 求它的面积的最大值.
\item 若直角三角形的内切圆半径为$1$, 求它的面积的最小值.
\item 若球半径为$R$, 试求它的内接圆柱的最大体积. 请指出下向解法的错误, 并给出正确的解答.\\
解: 设圆柱底面半径为$r$, 则$4r^2=4R^2-h^2$, 而$V_=\pi {r^2}h=\dfrac{\pi}4(4{R^2}-{h^2})h=\dfrac{\pi }4(2R+h)(2R-h)=\dfrac{\pi}8(2R+h)(4R-2h)h\le \dfrac{\pi}8(\dfrac{2R+h+4R-2h+h}3)^3=\dfrac{\pi}8(2R)^3=\pi R^3$. 所以所求最大体积为$\pi R^3$.
\item 在$\triangle ABC$中, 已知$BC=a$, $CA=b$, $AB=c$, $\angle ACB=\theta$. 现将$\triangle ABC$分别以$BC,CA,AB$所在直线为轴旋转一周, 设所得三个旋转体的体积依次为$V_1,V_2,V_3$.\\
(1) 设$T=\dfrac{V_3}{V_1+V_2}$, 试用$a,b,c$表示$T$;\\
(2) 若$\theta$为定值, 并令$\dfrac{a+b}c=x$, 将$T=\dfrac{V_3}{V_1+V_2}$表示为$x$的函数, 写出这个函数的定义域, 并求这个函数的最大值$M$;\\
(3) 若$\theta \in [\dfrac{\pi }3,\pi)$, 求(2)中$M$的最大值.
\item 已知$A(0,\sqrt 3a)$, $B(-a,0)$, $C(a,0)$是等边$\triangle ABC$的顶点, 点$M,N$分别在边$AB,BC$上, 且将$\triangle ABC$的面积两等分, 记$N$的横坐标为$x$, $|MN|=y$.\\
(1) 写出$y=f(x)$的表达式;\\
(2) 求$y=f(x)$的最小值.
\item 已知$\triangle ABC$内接于单位圆, 且$(1+\tan A)(1+\tan B)=2$.\\
(1) 求证: 内角$C$为定值;\\
(2) 求$\triangle ABC$面积的最大值.
\item 已知关于$x$的不等式$ax^2+bx+c>0$的解集是$\{x|\alpha<x<\beta\}$, 其中$0<\alpha<\beta$, 求$cx^2+bx+a<0$的解集.
\item 解不等式$(x+1)^2(x-1)(x-4)^3>0$.
\item 解不等式$\dfrac{3x^2-14x+14}{x^2-6x+8}\ge 1$.
\item 解不等式$\sqrt{x^2-3x+2}>x-3$.
\item 解不等式$\sqrt{2x-1}<x-2$.
\item 解不等式$|x^2-4|\le x+2$.
\item 解不等式$|x^2-\dfrac 12|>2x$.
\item 解关于$x$的不等式$|\log_ax|<|\log_a(ax^2)|-2$($0<a<1$).
\item 若关于$x$的不等式$2x-1>a(x-2)$的解集是$\mathbf{R}$, 则实数$a$的取值范围是\bracket{20}.
\fourch{$a>2$}{$a=2$}{$a<2$}{$a$不存在}
\item 若关于$x$的不等式$ax^2+bx-2>0$的解集是$(-\infty ,-\dfrac 12)\cup (\dfrac 13,+\infty)$, 则$ab$等于\bracket{20}.
\fourch{$-24$}{$24$}{$14$}{$-14$}
\item 若关于$x$的不等式$(a-2)x^2+2(a-2)x-4<0$对一切实数$x$恒成立, 则实数$a$的取值范围是\bracket{20}.
\fourch{$(-\infty ,2]$}{$(-\infty,-2)$}{$(-2,2]$}{$(-2,2)$}
\item 若$q<0<p$, 则不等式$q<\dfrac 1x<p$的解集为\bracket{20}.
\twoch{$\{x|\dfrac 1q<x<\dfrac 1p,\  x\ne 0\}$}{$\{x|x<\dfrac 1q\text{或}x>\dfrac 1p\}$}{$\{x|-\dfrac 1p<x<-\dfrac 1q, \ x\ne 0\}$}{$\{x|\dfrac 1p<x<-\dfrac 1q\}$}
\item 若关于$x$的不等式$(a+b)x+2a-3b<0$的解集是$\{x|x<-\dfrac 13\}$, 则$(a-3b)x+b-2a>0$的解集是\blank{50}.
\item 若不等式$\dfrac{2x^2+2kx+k}{4x^2+6x+3}<1$对一切$x\in \mathbf{R}$恒成立, 则实数$k$的取值范围是\blank{50}.
\item 若关于$x$的不等式$ax^2+bx+c>0$的解集是$\{x|3<x<5\}$, 则不等式$cx^2+bx+a<0$的解集是\blank{50}.
\item 若关于$x$的不等式$\dfrac{x-a}{x^2-3x+2}\ge 0$的解集是$\{x|1<x\le ax>2\}$, 则实数$a$的取值范围是\blank{50}.
\item 不等式$(x+2)(x+1)^2(x-1)^3(x-3)>0$的解集为:\blank{50}.
\item 不等式$\dfrac{(x-1)^2(x+2)}{(x-3)(x-4)}\le 0$的解集为:\blank{50}.
\item 不等式$x+1\le \dfrac 4{x+1}$的解集为:\blank{50}.
\item 若不等式$f(x)\ge 0$的解集为$[1,2]$, 不等式$g(x)\ge 0$的解集为$\varnothing$, 则不等式$\dfrac{f(x)}{g(x)}$的解集是\bracket{20}.
\fourch{$\varnothing$}{$(-\infty ,1)\cup (2,+\infty)$}{$[1,2)$}{$\mathbf{R}$}
\item 若关于$x$的不等式$ax^2-bx+c<0$的解集为$(-\infty ,\alpha)\cup (\beta ,+\infty)$, 其中$\alpha <\beta <0$, 则不等式$cx^2+bx+a>0$的解集为\bracket{20}.
\fourch{$(\dfrac 1{\beta},\dfrac 1{\alpha})$}{$(\dfrac 1{\alpha},\dfrac 1{\beta})$}{$(-\dfrac 1{\beta},-\dfrac 1{\alpha})$}{$(-\dfrac 1{\alpha},-\dfrac 1{\beta})$}
\item 解关于$x$的不等式: $m^2x-1<x+m$.
\item 解关于$x$的不等式: $x^2-ax-2a^2<0$.
\item 已知关于$x$的不等式$\sqrt x>ax+\dfrac 32$的解集是$\{x|4<x<b\}$, 求$a,b$的值.
\item 已知$x=3$是不等式$ax>b$解集中的元素, 求实数$a,b$应满足的条件.
\item 已知集合$\{x|x<-2\text{或}x>3\}$是集合$\{x|2ax^2+(2-ab)x-b>0\}$的子集, 求实数$a,b$的取值范围.
\item 已知集合$A=\{x|\dfrac{2x-1}{x^2+3x+2}>0\}$, $B=\{x|x^2+ax+b\le 0\}$, 且$A\cap B=\{x|\dfrac 12<x\le 3\}$, 求实数$a,b$的取值范围.
\item 已知集合$A=\{x|(x+2)(x+1)(2x-1)>0\}$, $B=\{x|x^2+ax+b\le 0\}$, 且$A\cup B=\{x|x+2 >0\}$, $A\cap B=\{x|\dfrac 12<x\le 3\}$, 求实数$a,b$的值.
\item 已知关于$x$的不等式$x^2-ax-6a\le 0$有解, 且解$x_1,x_2$满足$|x_1-x_2|\le 5$, 求实数$a$的取值范围.
\item 已知关于$x$的方程$3x^2+x\log_{\frac 12}^2a+2\log_{\frac 12}a=0$的两根$x_1,x_2$满足条件$-1<x_1<0<x_2<1$, 求实数$a$的取值范围.
\item 已知关于$x$的方程$x^2+(m^2-1)x+m-2=0$的一个根比$-1$小, 另一个根比$1$大, 求参数$m$的取值范围.
\item 已知集合$A=\{x|x-a>0\}$, $B=\{x|x^2-2ax-3a^2<0\}$, 求$A\cap B$与$A\cup B$.
\item 不等式$\sqrt{x+3}>-1$的解集是\bracket{20}.
\fourch{$\{x|x>-2\}$}{$\{x|x\ge -3\}$}{$\varnothing$}{$\mathbf{R}$}
\item 不等式$(x-1)\sqrt{x+2}\ge 0$的解集是\bracket{20}.
\fourch{$\{x|x>1\}$}{$\{x|x\ge 1\}$}{$\{x|x\ge 1\text{或}x=-2\}$}{$\{x|x>1\text{或}x=-2\}$}
\item 与不等式$\sqrt{(x-4)(x+3)}\le 1$的解完全相同的不等式是\bracket{20}.
\fourch{$|(x-4)(x+3)|\le 1$}{$(x-4)(x+3)\le 1$}{$\lg [ (x-4)(x+3) ]\le 0$}{$0\le (x-4)(x+3)\le 1$}
\item 解不等式: $\sqrt{x-5}+4x-3>3x+1+\sqrt{x-5}$.
\item 解不等式: $\sqrt{x^2+1}>\sqrt{x^2-x+3}$.
\item 解不等式: $(x-4)\sqrt{x^2-3x-4}\ge 0$.
\item 解不等式: $\dfrac{x+1}{x+4}\sqrt{\dfrac{x+3}{1-x}}<0$.
\item 解不等式: $\sqrt{x+2}+\sqrt{x-5}\ge \sqrt{5-x}$.
\item 解不等式: $\sqrt{x-6}+\sqrt{x-3}\ge \sqrt{3-x}$.
\item 解不等式: $\sqrt{2-x}<x$.
\item 解不等式: $\sqrt{4-x^2}<x+1$.
\item 解不等式: $\sqrt{3-2x}>x$.
\item 解不等式: $\sqrt{(x-1)(2-x)}>4-3x$.
\item 不等式$\sqrt{4-x^2}+\dfrac{|x|}x\ge 0$的解集是\bracket{20}.
\fourch{$[-2,2]$}{$[-\sqrt 3,0)\cup (0,2]$}{$[-2,0]\cup (0,2]$}{$[-\sqrt 3,0)\cup (0,\sqrt 3]$}
\item 已知关于$x$的不等式$\sqrt{2x-x^2}>kx$的解集是$\{x|0<x\le 2\}$, 则实数$k$的取值范围是\bracket{20}.
\fourch{$k<0$}{$k\ge 0$}{$0<k<2$}{$-\dfrac 12<k<0$}
\item 解不等式: $\sqrt{2x-4}-\sqrt{x+5}<1$.
\item 解不等式: $\sqrt{x^2-5x-6}<|x-3|$.
\item 解不等式: $|2\sqrt{x+3}-x+1|<1$.
\item 解关于$x$的不等式: $\sqrt{a(a-x)}>a-2x$($a>0$).
\item 解关于$x$的不等式: $\sqrt{4x-x^2}>ax$($a<0$).
\item 解关于$x$的不等式: $\sqrt{1-ax}<x-1$($a>0$).
\item 解关于$x$的不等式: $\sqrt{a^2-x^2}>2x-a$.
\item $\lg x^2<2$的解集是\bracket{20}.
\twoch{$\{x|-10<x<0\text{或}0<x<10\}$}{$\{x|x<10\}$}{$\{x|0<x<10\}$}{$\{x|-10<x<10\}$}
\item 若$f(x)=\log_2x$, 则不等式$[f(x)]^2>f(x^2)$的解集是\bracket{20}.
\fourch{$\{x|0<x<\dfrac 14\}$}{$\{x|\dfrac 14<x<1\}$}{$\{x|0<x<1\text{或}x>4\}$}{$\{x|\dfrac 14<x<4\}$}
\item 若$a,b$都是小于$1$的正数, 且$a^{\log_b(x-5)}<1$, 则$x$的取值范围是\bracket{20}.
\fourch{$x>5$}{$x<6$}{$5<x<6$}{$x<5$或$x>6$}
\item 不等式$\log_x\dfrac 45<1$的解集是\bracket{20}.
\twoch{$\{x|0<x<\dfrac 45\}$}{$\{x|x>\dfrac 45\}$}{$\{x|\dfrac 45<x<1\}$}{$\{x|0<x<\dfrac 45\}\cup \{x|x>1\}$}
\item 若函数$f(x)=\log_{a^2-1}(2x+1)$在区间$(-\dfrac 12,0)$内恒有$f(x)>0$, 则实数$a$的取值范围是\bracket{20}.
\twoch{$0<a<1$}{$a>1$}{$-\sqrt 2<a<-1$或$1<a<\sqrt 2$}{$a<-\sqrt 2$或$a>\sqrt 2$}
\item 若不等式$\log_a(x^2-2x+3)\le -1$对一切实数都成立, 则$a$的取值范围是\bracket{20}.
\fourch{$a\ge 2$}{$1<a\le 2$}{$\dfrac 12\le a<1$}{$0<a\le \dfrac 12$}
\item 解关于$x$的不等式: $\log_{\frac 12}(3x-2)>\log_{\frac 12}(x+1)$.
\item 解关于$x$的不等式: $\log_{\frac 13}(x^2-x-2)>\log_{\frac 13}(2x^2-7x+3)$.
\item 解关于$x$的不等式: $\log_x\dfrac 12<1$.
\item 解关于$x$的不等式: $\lg (x-\dfrac 1x)<0$.
\item 解关于$x$的不等式: $\log_2|x-\dfrac 12|<-1$.
\item 已知集合$M=\{x|\log_3(x-m)>1\}$与$P=\{x|3^{5-3x} \ge \dfrac 13\}$满足$M\cap P\ne \varnothing$, 求实数$m$的取值范围.
\item 解不等式: $\log_8(2-x)+\log_{64}(x+1)\ge \log_4x$.
\item 解不等式: $\log_{0.5}(x+13)<\log_{0.5}(x^2-2x-15)$.
\item 解不等式: $\log_x(3\sqrt{x-1}-1)>1$.
\item 解不等式: $\log_{x-1}(6-x-x^2)>2$.
\item 解不等式: $\dfrac 1{\log_2(x-1)}<\dfrac 1{\log_2\sqrt{x+1}}$.
\item 解不等式: $\dfrac{\log_3(1-\dfrac{3x}2)}{\log_9(2x)}\ge 1$.
\item 解不等式: $\log_{0.5}({2^x}-1)\cdot \log_{0.5}({2^{x-1}}-\dfrac 12)\le 2$.
\item 解关于$x$的不等式, 其中$a>0$, $a\ne 1$: $\log_a(x+1-a)>1$.
\item 解关于$x$的不等式, 其中$a>0$, $a\ne 1$: $\log_a(1-\dfrac 1x)>1$.
\item 解关于$x$的不等式, 其中$a>0$, $a\ne 1$: $\log_a(2x-1)>\log_a(x-1)$.
\item 解关于$x$的不等式, 其中$a>0$, $a\ne 1$: $\log_a^2x<\log_x^2a$.
\item 解关于$x$的不等式, 其中$a>0$, $a\ne 1$: ${x^{\log_ax}}>\dfrac{x^4\cdot \sqrt x}{a^2}$.
\item 解关于$x$的不等式, 其中$a>0$, $a\ne 1$: $\sqrt{\log_ax-1}>3-\log_ax$.
\item 已知$x$满足不等式$(\dfrac 12)^{2x-4}-(\dfrac 12)^x-(\dfrac 12)^{x-2}+\dfrac 14\le 0$, 且$y=\log_{\frac 1a}(a^2x)\cdot \log_{\frac 1{a^2}}(ax)$的最大值是$0$, 最小值是$-\dfrac 18$, 求实数$a$的值.
\item 已知关于$x$的方程$x^2-5x\log_ak+6\log _a^2k=0$的两根中$(k>1)$, 仅较小的根在区间$(1,2)$内, 试用$a$表示$k$的取值范围($a>0$且$a\ne 1$).
\item 设$x\in \mathbf{R}$, 则$(1-|x|)(1+x)>0$成立的充要条件是\bracket{20}.
\fourch{$|x|<1$}{$x<1$}{$|x|>1$}{$x<1$且$x\ne 1$}
\item 若函数$f(x)=\sqrt{x^2-2x-8}$的定义域为$M$, $g(x)=\dfrac 1{\sqrt{1-|x-a|}}$的定义域为$N$, 则使$M\cap N=\varnothing$的实数$a$的取值范围为\bracket{20}.
\fourch{$-1<a<3$}{$-1\le a\le 3$}{$-2<a<4$}{$-2\le a\le 4$}
\item 设$a,b$是满足$ab<0$的实数, 则下列不等式中正确的一个是\bracket{20}.
\fourch{$|a+b|>|a-b|$}{$|a+b|<|a-b|$}{$|a-b|<||a|-|b||$}{$|a-b|<|a|+|b|$}
\item 不等式$|x|<\dfrac 1x$的解集为\bracket{20}.
\fourch{$\varnothing$}{$\{x|x<0\}$}{$\{x|0<x<1\}$}{$\{x|x<0\text{或}x\ge 1\}$}
\item 若$|a+b|<-c$, 则在\textcircled{1} $a<-b-c$; \textcircled{2} $a+b>c$; \textcircled{3} $a+c<b$; \textcircled{4} $|a|+c<|b|$; \textcircled{5} $|a|+|b|<-c$这五个式子中, 一定成立的个数是\bracket{20}.
\fourch{$1$}{$2$}{$3$}{$4$}
\item 若实数$a,b$满足$ab>0$, 则在\textcircled{1} $|a+b|>|a|$; \textcircled{2} $|a+b|<|b|$; \textcircled{3} $|a+b|<|a-b|$; \textcircled{4} $|a+b|>|a-b|$这四个式子中, 正确的是\bracket{20}.
\fourch{\textcircled{1}\textcircled{2}}{\textcircled{1}\textcircled{3}}{\textcircled{1}\textcircled{4}}{\textcircled{2}\textcircled{4}}
\item 不等式$|\dfrac x{1+x}|>\dfrac x{1+x}$的解集是\bracket{20}.
\fourch{$\{x|x\ne -1\}$}{$\{x|x>-1\}$}{$\{x|x<0\text{且}x\ne -1\}$}{$\{x|-1<x<0\}$}
\item 解不等式: $x^2+|x|-6<0$.
\item 解不等式: $x^2-2|x|-15>0$.
\item 解不等式: $4<|1-3x|\le 7$.
\item 解不等式: $|x-3|<x-1$
\item 解不等式: $\log_2|x-\dfrac 12|<-1$.
\item 若函数$y=\log_ax$在$x\in [2,+\infty)$上恒有$|y|>1$, 则实数$a$的取值范围是\blank{50}.
\item 解不等式: $|x^2-5x+10|>x^2-8$.
\item 解不等式: $|x^2-4|\le x+2$.
\item 解不等式: $|x+1|<\dfrac 1{x-1}$.
\item 解不等式: $|x+2|-|x-3|<4$.
\item 解不等式: $|x+3|-|2x-1|<\dfrac x2+1$.
\item 已知当$|x-2|<a$成立时, $|x^2-4|<1$必定成立, 求正数$a$的取值范围.
\item 已知关于$x$的不等式$|x-4|+|x-3|<a$在实数集$\mathbf{R}$上的解集不是空集, 求正数$a$的取值范围.
\item 解不等式: $\log_{\frac 14}|x|<\log_{\frac 12}|x+1|$.
\item 解不等式: $|\lg (1-x)|>|\lg (1+x)|$.
\item 解不等式: $|\log_{\frac 13}x|+|\log_{\frac 13}\dfrac 1{3-x}|\ge 1$.
\item 求函数$f(x)=|x-\dfrac 12|-|x+\dfrac 12|$的最大值.
\item 已知$|\lg x-\lg y|\le 1$, 则$\dfrac xy+\dfrac yx$的取值范围是\blank{50}.
\item 解关于$x$的不等式: $|\log_{\sqrt a}x-2|-|\log_ax-2|<2$.
\item 解关于$x$的不等式: $|\log_ax|<|\log_a(ax^2)|-2$.
\item 解关于$x$的不等式: $|3^x-3|+9^x-3>0$.	
\item 解关于$x$的不等式: $|a^x-1|+|a^{2x}-3|>2(a>0)$.
\item $\triangle ABC$三内角$A,B,C$对边长分别为$a,b,c$. 求证: $a^2+b^2+c^2\ge 2ab\cos C+2b\cos A+2ca\cos B$.
\item $\triangle ABC$三内角$A,B,C$对边长分别为$a,b,c$. 求证: $(a+b-c)(b+c-a)(c+a-b)\le abc$.
\item $\triangle ABC$三内角$A,B,C$对边长分别为$a,b,c$. 求证: $\dfrac 12(\dfrac 1a+\dfrac 1b+\dfrac 1c)\le \dfrac{\cos A}a+\dfrac{\cos B}b+\dfrac{\cos C}c<\dfrac 1a+\dfrac 1b+\dfrac 1c$.
\item $\triangle ABC$三内角$A,B,C$对边长分别为$a,b,c$, 外接圆半径记作$R$. 求证: $\dfrac 1{ab}+\dfrac 1{bc}+\dfrac 1{ca}\ge \dfrac 1{R^2}$.
\item 已知常数$a\in (0,1)$, 对任意$x>0$, $f(\log_ax)=\dfrac{a(x^2-1)}{x(a^2-1)}$.\\
(l) 求$f(x)$($x\in \mathbf{R}$)的表达式, 并判断它的单调性;\\
(2) 若$n\ge 2$, $n\in \mathbf{N}$, 求证: $f(n)>n$.
\item 若正数$a,b,c$满足$a+b>c$, 求证: $\dfrac a{1+a}+\dfrac b{1+b}>\dfrac c{1+c}$.
\item 求证: $\dfrac 12\cdot \dfrac 34\cdot \dfrac 56\cdot \dfrac 78\cdots \cdot \cdot \dfrac{99}{100}<\dfrac 1{10}$.
\item 求证: $(1+\dfrac 13)(1+\dfrac 15)\cdots (1+\dfrac 1{2n-1})>\dfrac 1{\sqrt 3}\sqrt{2n+1}$($n\in \mathbf{N}$, $n>1$).
\item 求证: $\dfrac{x_1^2}{x_2-1}+\dfrac{x_2^2}{x_3-1}+\cdots +\dfrac{x_{n-1}^2}{x_n-1}+\dfrac{x_n^2}{x_1-1}\ge n+x_1+x_2+\cdots +x_a$($x_1,x_2,\cdots,x_n$都是大于$1$的实数).
\item 若正数$a,b,c$满足$a+b+c=1$, 求证: $(1+a)(1+b)(1+c)\ge 8(1-a)(1-b)(1-c)$.
\item 若$0\le a\le 1$, $0\le b\le 1$, $0\le c\le 1$, 求证: $\dfrac a{1+b+c}+\dfrac b{1+c+a}+\dfrac c{1+a+b}+(1-a)(1-b)(1-c)\le 1$.
\item 已知三棱锥的三条侧棱两两互相垂直, 且六条棱之和为定值$m$, 求证: 它的体积$V\le \dfrac{5\sqrt 2-7}{162}m^3$.
\item 已知$a+b+c>0$, $ab+bc+ca>0$, $abc>0$求证: $a>0$, $b>0$, $c>0$.
\item 求证: 任何面积等于$1$的凸四边形的周长及两条对角线的长度之和不小于$4+\sqrt 8$.
\item 解不等式: $2^{x+1}+x>0$.
\item 解关于$x$的不等式: $\dfrac{a(x-1)}{x-2}>1$.
\item 解关于$x$的不等式: $x^2+(a-4)x+4-2a>0$, 其中$-1\le a\le 1$.
\item 解不等式: $\dfrac x{\sqrt{1+x^2}}+\dfrac{1-x^2}{1+x^2}>0$.
\item 解关于$x$的不等式: $\dfrac{cx}{a\cdot c^2+b}-\dfrac x{2\sqrt{ab}}>x^2$, 其中$a,b,c\in \mathbf{R}$, 且$a>0$, $b>0$.
\item 已知函数$f(x)=ax^2-c$满足$-4\le f(1)\le -1$, $-1\le f(2)\le 5$, 求证: $-1\le f(3)\le 20$.
\item 已知关于$x$的方程$a\sin^2x+\dfrac 12\cos x+\dfrac 12-a=0$在$0\le x<2\pi$内有两个相异的实根, 求实数$a$的取值范围.
\item 已知$|a|<1$, $|b|<1$, $|c|<1$, 求证: $|1-abc|>|ab-c|$.
\item 已知$|a|<1$, $|b|<1$, $|c|<1$, 求证: $a+b+c<abc+2$.
\end{enumerate}
\end{document}